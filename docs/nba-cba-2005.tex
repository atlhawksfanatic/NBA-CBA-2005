% Options for packages loaded elsewhere
\PassOptionsToPackage{unicode}{hyperref}
\PassOptionsToPackage{hyphens}{url}
%
\documentclass[
]{book}
\usepackage{amsmath,amssymb}
\usepackage{lmodern}
\usepackage{iftex}
\ifPDFTeX
  \usepackage[T1]{fontenc}
  \usepackage[utf8]{inputenc}
  \usepackage{textcomp} % provide euro and other symbols
\else % if luatex or xetex
  \usepackage{unicode-math}
  \defaultfontfeatures{Scale=MatchLowercase}
  \defaultfontfeatures[\rmfamily]{Ligatures=TeX,Scale=1}
\fi
% Use upquote if available, for straight quotes in verbatim environments
\IfFileExists{upquote.sty}{\usepackage{upquote}}{}
\IfFileExists{microtype.sty}{% use microtype if available
  \usepackage[]{microtype}
  \UseMicrotypeSet[protrusion]{basicmath} % disable protrusion for tt fonts
}{}
\makeatletter
\@ifundefined{KOMAClassName}{% if non-KOMA class
  \IfFileExists{parskip.sty}{%
    \usepackage{parskip}
  }{% else
    \setlength{\parindent}{0pt}
    \setlength{\parskip}{6pt plus 2pt minus 1pt}}
}{% if KOMA class
  \KOMAoptions{parskip=half}}
\makeatother
\usepackage{xcolor}
\usepackage{longtable,booktabs,array}
\usepackage{calc} % for calculating minipage widths
% Correct order of tables after \paragraph or \subparagraph
\usepackage{etoolbox}
\makeatletter
\patchcmd\longtable{\par}{\if@noskipsec\mbox{}\fi\par}{}{}
\makeatother
% Allow footnotes in longtable head/foot
\IfFileExists{footnotehyper.sty}{\usepackage{footnotehyper}}{\usepackage{footnote}}
\makesavenoteenv{longtable}
\usepackage{graphicx}
\makeatletter
\def\maxwidth{\ifdim\Gin@nat@width>\linewidth\linewidth\else\Gin@nat@width\fi}
\def\maxheight{\ifdim\Gin@nat@height>\textheight\textheight\else\Gin@nat@height\fi}
\makeatother
% Scale images if necessary, so that they will not overflow the page
% margins by default, and it is still possible to overwrite the defaults
% using explicit options in \includegraphics[width, height, ...]{}
\setkeys{Gin}{width=\maxwidth,height=\maxheight,keepaspectratio}
% Set default figure placement to htbp
\makeatletter
\def\fps@figure{htbp}
\makeatother
\setlength{\emergencystretch}{3em} % prevent overfull lines
\providecommand{\tightlist}{%
  \setlength{\itemsep}{0pt}\setlength{\parskip}{0pt}}
\setcounter{secnumdepth}{5}
\usepackage[margin=1in]{geometry}
\usepackage{booktabs}
\usepackage{enumitem}

% Without these you get ! LaTeX Error: Too deeply nested.
\setlistdepth{10}
\renewlist{enumerate}{enumerate}{10}
\setlist[itemize]{labelsep=.5em}

% Correct for the way articles/sections are defined and pretty the TOC
\usepackage{fancyhdr}
\usepackage{tocbasic}

\DeclareTOCStyleEntry[%
  entryformat=\bfseries,
  pagenumberformat=\bfseries,
]{tocline}{chapter}
\DeclareTOCStyleEntries[
  pagenumberbox=\hbox,
  dynnumwidth
]{tocline}{%
  chapter,section,subsection,subsubsection,paragraph,subparagraph,%
  figure,table
}
\DeclareTOCStyleEntries[
  dynindent
]{tocline}{subsection,subsubsection,subparagraph}

\renewcommand{\chaptername}{Article}
\renewcommand{\thechapter}{\Roman{chapter}}
\renewcommand{\appendixname}{Exhibit}
\ifLuaTeX
  \usepackage{selnolig}  % disable illegal ligatures
\fi
\usepackage[]{natbib}
\bibliographystyle{plainnat}
\IfFileExists{bookmark.sty}{\usepackage{bookmark}}{\usepackage{hyperref}}
\IfFileExists{xurl.sty}{\usepackage{xurl}}{} % add URL line breaks if available
\urlstyle{same} % disable monospaced font for URLs
\hypersetup{
  pdftitle={NBA Collective Bargaining Agreement - 2005},
  pdfauthor={Robert},
  hidelinks,
  pdfcreator={LaTeX via pandoc}}

\title{NBA Collective Bargaining Agreement - 2005}
\author{Robert}
\date{2023-04-01}

\begin{document}
\maketitle

{
\setcounter{tocdepth}{1}
\tableofcontents
}
\hypertarget{preface}{%
\chapter*{Preface}\label{preface}}
\addcontentsline{toc}{chapter}{Preface}

\emph{If you are only interested in a pdf or epub of the CBA, then please click on the download icon at the top left (it is next to the ``A'') and select the format you wish to download.}

This is the 2005 NBA's Collective Bargaining Agreement (CBA) which has been converted to a text file, then broken up by Article and Exhibit to .Rmd files which then serves as the basis for this bookdown site. The original version of the 2005 CBA can be found on my Github website \href{https://atlhawksfanatic.github.io/}{atlhawksfanatic.github.io} (\href{https://github.com/atlhawksfanatic/atlhawksfanatic.github.io/raw/master/research/CBA/2005-NBA-NBPA-Collective-Bargaining-Agreement.pdf}{here})

The purpose of this site is three-fold:

\begin{enumerate}
\def\labelenumi{\arabic{enumi}.}
\tightlist
\item
  to historically document collective bargaining agreements of the NBA;
\item
  to provide easier navigation of the CBA through a structured format of each Article and Section; and
\item
  for me to better understand the CBA through this exercise.
\end{enumerate}

I do not own any rights to the CBA and am simply redistributing it in a different format. This is not the official version of the CBA and I am not responsible for any errors that might be present in this document. If you believe you have found an error, please let me know and I will correct it.

Contact: \href{atlhawksfanatic@gmail.com}{via email}, \href{https://github.com/atlhawksfanatic}{through Github}, or \href{https://twitter.com/atlhawksfanatic}{on Twitter}

\hypertarget{definitions}{%
\chapter{DEFINITIONS}\label{definitions}}

\hypertarget{definitions.}{%
\section{Definitions.}\label{definitions.}}

As used in this Agreement, the following terms shall have the following meanings:

\begin{enumerate}
\def\labelenumi{\arabic{enumi}.}
\tightlist
\item
  ``Active List'' means the list of players, maintained by the NBA, who have signed Player Contracts with a Team and are otherwise eligible to participate in a Regular Season game.
\item
  ``Agreement'' means this Collective Bargaining Agreement entered into as of July 29, 2005.
\item
  ``Audit Report'' or ``final Audit Report'' means the audit report prepared in accordance with Article VII, Section 10.
\item
  ``Average Player Salary'' means, with respect to any Salary Cap Year, Total Salaries divided by an amount equal to the product of the number of Teams in the NBA (other than Expansion Teams during their first two (2) Salary Cap Years) multiplied by 13.2.
\item
  ``Base Compensation'' means the component of Compensation other than bonuses of any kind.
\item
  ``Base Year Compensation'' means an amount used to calculate the Exception that results from the trade of certain Player Contracts, as determined in accordance with Article VII, Section 6(h)(4).
\item
  ``Basketball Related Income'' or ``BRI'' means basketball related income as defined in Article VII, Section 1(a) and (b).
\item
  ``Benefits'' or ``Total Benefits'' means the sum of all amounts paid or to be paid on an accrual basis during any Salary Cap Year by the NBA or NBA Teams, other than Expansion Teams during their first two Salary Cap Years, for the specific benefits set forth in Article IV.
\item
  ``Commissioner'' means the Commissioner of the NBA.
\item
  ``Compensation'' means the compensation that is or could be earned by, or is paid or payable to, an NBA player (including players whose Player Contracts have been terminated) in accordance with a Player Contract (whether such payment is sent to the player directly or to a person or entity designated by a player).
\item
  ``Contract'' (see ``Uniform Player Contract'').
\item
  ``Current Base Compensation'' means the component of Base Compensation other than Deferred Base Compensation.
\item
  ``Current Compensation'' means the component of Compensation other than Deferred Compensation.
\item
  ``Deferred Base Compensation'' means the component of Deferred Compensation other than bonuses of any kind.
\item
  ``Deferred Compensation'' means the component of Compensation payable to a player during the period commencing after the term covered by the Player Contract, in accordance with the rules set forth in Article VII. The determination of whether Compensation is Deferred Compensation will be based upon the time set by the Player Contract for the player to receive the Compensation, without regard to whether the obligation is funded currently or secured in any fashion.
\item
  ``Draft'' or ``NBA Draft'' means the NBA's annual draft of Rookie basketball players.
\item
  ``Early Qualifying Veteran Free Agent'' means a Veteran Free Agent who, prior to becoming a Veteran Free Agent, played under one or more Player Contracts covering some or all of each of the two (2) preceding Seasons, and who either exclusively played with his Prior Team during such two Seasons, or, if he played for more than one Team during such period, changed Teams only (i) by means of trade, or (ii) by signing with his Prior Team during the first of the two (2) Seasons.
\item
  ``Early Termination Option'' (or ``ETO'') means an option in favor of a player to shorten the stated number of years covered by a Player Contract in accordance with Article XII.
\item
  ``Effective Season'' means, with respect to an Early Termination Option, the first Season covered by the Early Termination Option. (For example, if a Contract were to contain an Early Termination Option exercisable following the 2009-2010 Season, the Effective Season would be the 2010-11 Season.)
\item
  ``Estimated Average Player Salary'' means, for a particular Salary Cap Year, 108\% of the prior Salary Cap Year's Average Player Salary.
\item
  ``Exception'' means an exception to the rule that a Team's Team Salary may not exceed the Salary Cap.
\item
  ``Expansion Team'' means the Charlotte Bobcats and any other Team that becomes a member of the NBA through expansion following the date of this Agreement and commences play during the term of this Agreement.
\item
  ``Extension'' means an amendment to a Player Contract lengthening the term of the Contract for a specified period of years.
\item
  ``First Round Pick'' means a player selected by a Team in the first round of the Draft.
\item
  ``Free Agent'' means: (i) a Veteran Free Agent; (ii) a Rookie Free Agent; (iii) a Veteran whose Player Contract has been terminated in accordance with the NBA waiver procedure; or (iv) a player whose last Player Contract was a 10-Day Contract and who either completed the Contract by rendering the playing services called for thereunder or was released early from such Contract.
\item
  ``Generally Recognized League Honors'' means the following NBA league honors awarded to players: NBA Most Valuable Player; NBA Finals Most Valuable Player; NBA Defensive Player of the Year; NBA Sixth Man Award; NBA Most Improved Player; All-NBA Team (First, Second, or Third); NBA All-Defensive Team (First or Second); and All-Star Team Selection.
\item
  ``Inactive List'' means the list of players, maintained by the NBA, who have signed Player Contracts with a Team and are otherwise ineligible to participate in a Regular Season game.
\item
  ``Incentive Compensation'' means the component of Compensation consisting of one or more bonuses described in Article II, Sections 3(b)(iii) and (iv) and 3(c).
\item
  ``Likely Bonus'' means Incentive Compensation included in a player's Salary in accordance with Article VII, Section 3(d).
\item
  ``Member'' or ``Team'' means any team that is a member of the NBA.
\item
  ``Minimum Annual Salary'' means the minimum Salary that must be included in a Player Contract that covers the entire Regular Season in accordance with Article II, Section 6(a).
\item
  ``Minimum Annual Salary Scale'' means the scale annexed hereto as Exhibit C.
\item
  ``Minimum Player Salary'' means: (i) with respect to a Contract that covers the entire Regular Season, the Minimum Annual Salary called for under Article II, Section 6(a); (ii) with respect to a Rest-of-Season Contract, the Minimum Annual Salary called for under Article II, Section 6(a) multiplied by a fraction, the numerator of which is the number of days remaining in the NBA Regular Season as of the date such Rest-of-Season Contract is entered into, and the denominator of which is the total number of days of that NBA Regular Season; and (iii) with respect to a 10-Day Contract, the Minimum Annual Salary called for under Article II, Section 6(a) multiplied by a fraction, the numerator of which is the number of days covered by the Contract and the denominator of which is the total number of days of that NBA Regular Season.
\item
  ``Minimum Team Salary'' means the minimum amount in Salary obligations to, or on behalf of, players with respect to an NBA Season that each Team must incur or pay.
\item
  ``Moratorium Period'' means:

  \begin{enumerate}
  \def\labelenumii{(\roman{enumii})}
  \tightlist
  \item
    With respect to the 2006-07 Salary Cap Year, the period July 1, 2006 through July 11, 2006.
  \item
    With respect to the 2007-08 Salary Cap Year, the period July 1, 2007 through July 10, 2007.
  \item
    With respect to the 2008-09 Salary Cap Year, the period July 1, 2008 through July 8, 2008.
  \item
    With respect to the 2009-10 Salary Cap Year, the period July 1, 2009 through July 7, 2009.
  \item
    With respect to the 2010-11 Salary Cap Year, the period July 1, 2010 through July 7, 2010.
  \item
    With respect to the 2011-12 Salary Cap Year (if the NBA exercises its option to extend this Agreement pursuant to Article XXXIX), the period July 1, 2011 through July 7, 2011.
  \end{enumerate}
\item
  The term ``negotiate'' means, with respect to a player or his representatives on the one hand, and a Team or its representatives on the other hand, to engage in any written or oral communication relating to the possible employment, or terms of employment, of such player by such Team as a basketball player, regardless of who initiates such communication.
\item
  ``Non-Qualifying Veteran Free Agent'' means a Veteran Free Agent who is not a Qualifying Veteran Free Agent or an Early Qualifying Veteran Free Agent.
\item
  ``Option'' means an option in a Player Contract in favor of a Team or player to extend such Contract beyond its stated term.
\item
  ``Option Buy-Out Amount'' means any amount payable to a player in connection with either the exercise of an Early Termination Option or the non-exercise of an Option.
\item
  ``Option Year'' means the year that would be added to a Player Contract if an Option were exercised.
\item
  ``Performance Bonus'' means any Incentive Compensation described in Article II, Section 3(b)(iii).
\item
  ``Player Contract'' (see ``Uniform Player Contract'').
\item
  ``Prior Team'' means the Team for which a player was last under Contract prior to becoming a Qualifying Veteran Free Agent, Early Qualifying Veteran Free Agent or a Non-Qualifying Veteran Free Agent.
\item
  ``Qualifying Offer'' means an offer of a Uniform Player Contract, signed by the Team, that: (i) is either personally delivered to the player or his representative or sent by pre-paid certified, registered or overnight mail to the last known address of the player or his representative; (ii) is for a period of one year; and (iii) provides for: (A) for First Round Picks finishing their Rookie Scale Contracts, Salary (excluding Incentive Compensation), Likely Bonuses and Unlikely Bonuses equal to the Salary (excluding Incentive Compensation), Likely Bonuses and Unlikely Bonuses, respectively, provided in the fourth Salary Cap Year of the Rookie Scale Contract increased by the percentage called for in Exhibit B hereto; and (B) for all other players subject to a Right of First Refusal in accordance with Article XI, the greater of (x) 125\% of the player's Salary (excluding Incentive Compensation), Likely Bonuses and Unlikely Bonuses, respectively, for the last Salary Cap Year covered by the player's prior Contract, or (y) Base Compensation equal to the sum of the Minimum Annual Salary applicable to the player (for the Season covered by the Qualifying Offer) plus \$175,000 (with no bonuses of any kind). All other terms and conditions in the Qualifying Offer must be unchanged from those that applied to the last year of the player's prior Contract (including, but not limited to, the percentage of Base Compensation that is protected), provided that such terms and conditions are allowable amendments under this Agreement at the time the Qualifying Offer is made. In addition, a Team shall be permitted to include in any Qualifying Offer an Exhibit 6 to the UPC requiring that the player, if he signs the Qualifying Offer, pass a physical examination to be performed by a physician designated by the Team as a condition precedent to the validity of the Contract.
\item
  ``Qualifying Veteran Free Agent'' means a Veteran Free Agent who, prior to becoming a Veteran Free Agent, played under one or more Player Contracts covering some or all of each of the three preceding Seasons and either played exclusively with his Prior Team during such three Seasons, or, if he played with more than one Team during such period, changed Teams only (i) by means of trade, or (ii) by signing with his Prior Team during the first of the three Seasons.
\item
  ``Regular Salary'' means a player's Salary, less any component thereof that is a signing bonus (or deemed a signing bonus in accordance with Article VII) and any component thereof that is Incentive Compensation.
\item
  ``Regular Season'' means, with respect to any Season, the period beginning on the first day and ending on the last day of regularly scheduled (as opposed to exhibition or playoff) competition between NBA Teams.
\item
  ``Renegotiation,'' ``renegotiate,'' or ``renegotiated'' means a Contract amendment that provides for an increase in Salary and/or Unlikely Bonuses.
\item
  ``Replacement Player'' means, where appropriate, either a player who is acquired by a Team pursuant to the Traded Player Exception, or a player who is signed or acquired by a Team pursuant to the Disabled Player Exception.
\item
  ``Required Tender'' means an offer of a Uniform Player Contract to a Draft Rookie, signed by the Team, that: (i) is either personally delivered to the player or his representative or sent by pre-paid certified, registered, or overnight mail to the last known address of the player or his representative; (ii) with respect to a First Round Pick, (A) affords the player until at least the first day of the following Regular Season to accept, and (B) satisfies the requirements of a Rookie Scale Contract set forth in Article VIII, Section 1 or 2; and (iii) with respect to a Second Round Pick, (A) affords the player until at least the immediately following October 15 to accept, (B) has a stated term of one (1) Season, and (C) calls for at least the Minimum Annual Salary then applicable to the player. In addition, a Team shall be permitted to include in any Required Tender an Exhibit 6 to the UPC requiring that the player, if he signs the Required Tender, pass a physical examination to be performed by a physician designated by the Team as a condition precedent to the validity of the Contract.
\item
  ``Restricted Free Agent'' means a Veteran Free Agent who is subject to a Team's right of first refusal in accordance with Article XI.
\item
  ``Rookie'' means a person who has never signed a Player Contract with an NBA Team.

  \begin{enumerate}
  \def\labelenumii{(\roman{enumii})}
  \tightlist
  \item
    ``Draft Rookie'' means a Rookie who is selected in the NBA Draft.
  \item
    ``Non-Draft Rookie'' means a Rookie who is not selected in the NBA Draft for which he is first eligible.
  \end{enumerate}
\item
  ``Rookie Free Agent'' means: (i) a Draft Rookie who, pursuant to the provisions of Article VIII, Section 3 or Article X, is no longer subject to the exclusive negotiating rights of any Team, and who may be signed by any Team; or (ii) a Non-Draft Rookie.
\item
  ``Rookie Salary Scales'' means the tables annexed hereto as Exhibit B.
\item
  ``Rookie Scale Amounts'' means the amounts set forth in the tables annexed hereto as Exhibit B.
\item
  ``Rookie Scale Contract'' means the initial Uniform Player Contract entered into, in accordance with Article VIII, Section 1 or 2, between a First Round Pick and the Team that holds his draft rights.
\item
  ``Room'' means the extent to which: (i) a Team's then-current Team Salary is less than the Salary Cap; or (ii) a Team is entitled to use one of the Salary Cap Exceptions set forth in Article VII, Section 6(c), (d), (e) and (h) (Disabled Player, Bi-annual, Mid-Level Salary and Traded Player Exceptions).
\item
  ``Salary'' means, with respect to a Salary Cap Year, a player's Compensation with respect to the Season covered by such Salary Cap Year, plus any other amount that is deemed to constitute Salary in accordance with the terms of this Agreement, not including Unlikely Bonuses, any benefits the player received in accordance with the terms of this Agreement (including, e.g., the benefits provided for by Article IV, per diem, and moving expenses), and any portion of the player's Compensation that is attributable to another Salary Cap Year inaccordance with this Agreement. Salary also includes any consideration received by a retired player that is deemed to constitute Salary in accordance with the terms of Article XIII.
\item
  ``Salary Cap'' means the maximum allowable Team Salary for each Team for a Salary Cap Year, subject to the rules and exceptions set forth in this Agreement.
\item
  ``Salary Cap Year'' means the period from July 1 through the following June 30.
\item
  ``Season'' or ``NBA Season'' means the period beginning on the first day of training camp and ending immediately after the last game of the NBA Finals.
\item
  ``Second Round Pick'' means a player selected by a Team in the second round of the Draft.
\item
  ``Team'' or ``NBA Team'' (see ``Member'').
\item
  ``Team Affiliate'' means:

  \begin{enumerate}
  \def\labelenumii{(\roman{enumii})}
  \tightlist
  \item
    any individual or entity who or which (directly or indirectly) holds an ownership interest in a Team (other than ownership of publicly-traded securities constituting less than 5\% of the ownership interests in a Team);
  \item
    any individual or entity who or which (directly or indirectly) controls, is controlled by or is under common control with, or who or which is an entity affiliated with or an individual related to, a Team;
  \item
    any individual or entity who or which (directly or indirectly) controls, is controlled by or is under common control with, or who or which is an entity affiliated with or an individual related to, an individual or entity described in Section 1 (zzz)(i) or (ii) above; or
  \item
    any entity in which 10\% or more of the ownership interests are held (directly or indirectly) by an individual or entity who or which holds (directly or indirectly) 10\% or more of the ownership interests in a Team or in an entity described in Section 1(zzz)(ii) above.
  \end{enumerate}

  For the purposes of this Section 1(lll): an individual shall only be deemed to be ``related to'' a Team or another individual or entity if such individual is an officer, director or executive employee of such Team or entity, or is a member of such individual's immediate family; and ``controls'' or ``is controlled by'' shall include (without limitation) the circumstance in which an individual or a Team or entity has or can exercise effective control.
\item
  ``Team Salary'' means, with respect to a Salary Cap Year, the sum of all Salaries attributable to a Team's active and former players plus other amounts as computed in accordance with Article VII, less applicable credit amounts as computed in accordance with Article VII.
\item
  ``Total Salaries'' means the total Salaries included in the Team Salary of all NBA Teams for or with respect to a Salary Cap Year in accordance with this Agreement, other than the Salaries included in the Team Salary of Expansion Teams during their first two Salary Cap Years, as determined in accordance with Article VII. For purposes of this definition: (i) Total Salaries shall include all Incentive Compensation excluded from Salaries in accordance with Article VII, Section 3(d) but actually earned by NBA players during such Salary Cap Year, and shall exclude all Incentive Compensation included in Salaries in accordance with Article VII,Section 3(d) but not actually earned by NBA players during such Salary Cap Year; (ii) Total Salaries shall include any amounts paid by the NBA to the Players Association for distribution to NBA players pursuant to Article XXIX, Section 3(c); (iii) Total Salaries shall include the aggregate Salaries, if any, that are excluded from Team Salaries pursuant to Article VII, Section 4(h); and (iv) Total Salaries shall include any consideration received by a retired player that is included in Team Salary in accordance with the terms of Article XIII.
\item
  ``Total Salaries and Benefits'' means the sum of Total Salaries plus Total Benefits.
  68.''Traded Player'' means a player whose Player Contract is assigned by one Team to another Team other than by means of the NBA waiver procedure.
\item
  ``Uniform Player Contract'' or ``Player Contract'' or ``Contract'' means the standard form of written agreement between a person and a Team required for use in the NBA by Article II, pursuant to which such person is employed by such Team as a professional basketball player.
\item
  ``Unlikely Bonus'' means Incentive Compensation excluded from a player's Salary in accordance with Article VII, Section 3(d).
\item
  ``Unrestricted Free Agent'' means a Free Agent who is not subject to a Team's right of first refusal.
\item
  ``Veteran'' or ``Veteran Player'' means a person who has signed at least one Player Contract with an NBA Team.
\item
  ``Veteran Free Agent'' means a Veteran who completed his Player Contract (other than a 10-Day Contract) by rendering the playing services called for thereunder.
\item
  ``Years of Service'' means the number of years of NBA service credited to a player in accordance with the following: a player will be credited with one (1) year of NBA service for each year that he is on an NBA Active List or Inactive List for one (1) or more days during the Regular Season. Notwithstanding the above, a player will not receive credit for a Year of Service for any year in which he: (i) withholds playing services called for by a Player Contract or this Agreement for more than thirty (30) days after the Season begins, or (ii) is a Restricted Free Agent, has been tendered a Qualifying Offer by his Prior Team that has been expressly left open by that Team until at least March 1, and has not signed a Player Contract with any Team by March 1. In addition, notwithstanding the above, a player will not receive credit for a Year of Service for being on an NBA Active List or Inactive List as a result of signing a Player Contract that is disapproved by the Commissioner. In no event can a player be credited with more than one (1) Year of Service with respect to any one NBA Season. A Year of Service will be credited to a player on the June 30 following the Season with respect to which it is being credited. Under no circumstances shall the definition of Years of Service herein be used for purposes of determining a player's years of credited service under the NBA Players' Pension Plan. Players shall be credited with Years of Service pursuant to this Section 1(vvv) only in respect of Seasons covered by this Agreement. Years of Service credit for prior Seasons shall be determined in accordance with the provisions of the 1999 NBA/NBPA Collective Bargaining Agreement.
\end{enumerate}

\hypertarget{uniform-player-contract}{%
\chapter{UNIFORM PLAYER CONTRACT}\label{uniform-player-contract}}

\hypertarget{required-form.}{%
\section{Required Form.}\label{required-form.}}

The Player Contract to be entered into by each player and the Team by which he is employed shall be a Uniform Player Contract in the form annexed hereto as Exhibit A.

\hypertarget{limitation-on-amendments.}{%
\section{Limitation on Amendments.}\label{limitation-on-amendments.}}

\begin{enumerate}
\def\labelenumi{(\alph{enumi})}
\tightlist
\item
  Except as provided in Sections 3, 6, 7(d), 9, 10 and 11 of this Article, and in Article VII, Section 7 (Extensions, Renegotiations and Other Amendments) or Article XII (Option Clauses), no amendments to the form of Uniform Player Contract provided for by Section 1 of this Article shall be permitted.
\item
  If a Team and a player enter into (i) a Uniform Player Contract containing an amendment not specifically permitted by this Agreement or (ii) a subsequent amendment to an existing Player Contract where such amendment is not specifically permitted by this Agreement, then such Contract or subsequent amendment, as the case may be, shall be disapproved by the Commissioner and, consequently, rendered null and void.
\end{enumerate}

\hypertarget{allowable-amendments.}{%
\section{Allowable Amendments.}\label{allowable-amendments.}}

In their individual contract negotiations, a player and a Team may amend the provisions of a Uniform Player Contract, but only in the following respects:

\begin{enumerate}
\def\labelenumi{(\alph{enumi})}
\tightlist
\item
  By agreeing upon provisions (to be set forth in Exhibit 1 to a Uniform Player Contract) setting forth the Compensation to be paid or amounts to be loaned to the player for each Season of the Contract for rendering the services and performing the obligations described in such Contract.
\item
  By agreeing upon provisions (to be set forth in Exhibit 1 to a Uniform Player Contract) setting forth lump sum bonuses, and the payment date for each such bonus, to be paid as a result of: (i) the player's execution of a Uniform Player Contract or Extension (a ``signing bonus''); (ii) the exercise or non-exercise of an option pursuant to Articles VII and XII; (iii) the player's achievement of agreed-upon benchmarks relating to his performance as a player or the Team's performance during a particular NBA Season, subject to the limitations imposed by paragraph 3(c) of the Uniform Player Contract and Section 11(c) below; or (iv) the player's achievement of agreed-upon benchmarks relating to his physical condition or academic achievement, including the player's attendance at and participation in an off-season summer league and/or an off-season skill and/or conditioning program upon terms and conditions agreed upon by the Team and player (subject to the provisions of Section 11(b) below). Any amendment agreed upon pursuant to subsections (iii) or (iv) of this Section 3(b) must be structured so as to provide an incentive for positive achievement by the player and/or the Team; and any amendment agreed upon pursuant to subsection (iii) must be based upon specific numerical benchmarks or Generally Recognized League Honors. By way of example and not limitation, an amendment agreed upon pursuant to subsection (iii) may provide for the player to receive a bonus if his free-throw percentage exceeds 80\%, but may not provide for the player to receive a bonus if his free-throw percentage improves over his previous Season's percentage.
\item
  By agreeing upon provisions (to be set forth in Exhibit 1 to a Uniform Player Contract) with respect to extra promotional appearances to be performed by the player (in addition to those required by paragraph 13 of such Contract) and the Compensation therefor.
\item
  By agreeing upon a Compensation payment schedule (to be set forth in Exhibit 1 to a Uniform Player Contract) different from that provided for by paragraph 3(a) of the Uniform Player Contract; provided, however, that such amendment shall comply with the provisions of Section 3(b) above (relating to bonus payments) and Section 12(e) below and that the only such amendment that shall be permitted with respect to any Season in which the player's Compensation is not greater than the Minimum Player Salary shall be as described in Section 6(h) below.
\item
  By agreeing upon provisions (to be set forth in Exhibit 2 to a Uniform Player Contract) stating that the Base Compensation provided for by a Uniform Player Contract (as described in Exhibit 1 to such Contract) shall be, in whole or in part, and subject to any conditions or limitations, protected or insured (as provided for by, and in accordance with the definitions set forth in, Section 4 below) in the event that such Contract is terminated by the Team by reason of the player's:

  \begin{enumerate}
  \def\labelenumii{(\roman{enumii})}
  \tightlist
  \item
    lack of skill;
  \item
    death not covered by an insurance policy procured by a Team for the player's benefit (``non-insured death''), or death covered by an insurance policy procured by a Team for the player's benefit (``insured death''), provided that a Contract can contain protection or insurance for only one of the two categories set forth in this Section 3(e)(ii);
  \item
    disability or unfitness to play skilled basketball resulting from a basketball-related injury not covered by an insurance policy procured by a Team for the player's benefit (``non-insured basketball-related injury''), disability or unfitness to play skilled basketball resulting from any injury or illness not covered by an insurance policy procured by a Team for the player's benefit (``non-insured injury or illness''), or disability or unfitness to play skilled basketball resulting from an injury or illness covered by an insurance policy procured by a Team for the player's benefit (``insured injury or illness''), provided that a Contract can contain protection or insurance in only one of the three categories set forth in this Section 3(e)(iii); and/or
  \item
    mental disability not covered by an insurance policy procured by a Team for the player's benefit (``non-insured mental disability''), or mental disability covered by an insurance policy procured by a Team for the player's benefit (``insured mental disability''), provided that a Contract can contain protection or insurance in only one of the two categories set forth in this Section 3(e)(iv).
  \end{enumerate}
\item
  By agreeing upon provisions (to be set forth in Exhibit 3 to a Uniform Player Contract) limiting or eliminating the player's right to receive his Base Compensation (in accordance with paragraphs 7(c), 16(a)(iii), and 16(b) of the Uniform Player Contract) when the player's disability or unfitness to play skilled basketball is caused by the re-injury of an injury sustained prior to, or by the aggravation of a condition that existed prior to, the execution of the Uniform Player Contract providing for such Base Compensation.
\item
  By agreeing upon provisions (to be set forth in Exhibit 4 to a Uniform Player Contract) (i) entitling a player to earn Compensation if such player's Uniform Player Contract is traded to another NBA team, or (ii) prohibiting or limiting the Team's right to trade such player's Contract to another Team, subject, however, in either case (i) or (ii) to the provisions of Article XXIV.
\item
  By agreeing upon provisions (to be set forth in Exhibit 5 to a Uniform Player Contract) permitting the player to participate or engage in some or all of the activities otherwise prohibited by paragraph 12 of the Uniform Player Contract; provided, however, that no amendment to paragraph 12 of the Uniform Player Contract shall permit a player to participate in any public game or public exhibition of basketball not approved in accordance with Article XXIII of this Agreement.
\item
  By agreeing upon provisions (to be set forth in Exhibit 6 to a Uniform Player Contract) establishing that the player must report for and submit to a physical examination to be performed by a physician designated by the Team, subject to the provisions of Section 12(h) below.
\item
  By agreeing to delete paragraph 7(b) of the Uniform Player Contract in its entirety and substituting therefor the provision set forth in Exhibit 7 to a Uniform Player Contract.
\item
  By agreeing either (i) to delete paragraph 13(b) of the Uniform Player Contract in its entirety, or (ii) to delete the last sixteen words of the first sentence of paragraph 13(b) of such Contract.
\item
  By agreeing upon provisions for the purpose of terminating an already-existing Uniform Player Contract prior to the expiration of its stated term, stating as follows: (i) the Team will request waivers on the player in accordance with paragraph 16 of the Contract immediately following the Commissioner's approval of such amendment; and (ii) should the player clear waivers and his Contract thereupon be terminated (x) the amount of any Compensation protection or insurance contained in the Contract will immediately be reduced or eliminated, (y) as a result of the termination of the Contract, the payment schedule for any Compensation remaining to be paid will be accelerated over a shorter period or stretched over a longer period (subject, however, to Sections 12(e) and 12(f) below), and/or (z) the Team's right of set-off under Article XXVII of this Agreement will be modified or eliminated.
\item
  By agreeing upon provisions (to be set forth in Exhibit 8 to a Uniform Player Contract) stating that the Contract will be traded to another team within forty-eight (48) hours of its execution or amendment, such trade and the consummation of such trade to be conditions precedent to the validity of the Contract or an amendment thereto; provided, however, that any such sign-and-trade transaction must comply with Article VII, Section 8(e).
\end{enumerate}

\hypertarget{compensation-protection-or-insurance.}{%
\section{Compensation Protection or Insurance.}\label{compensation-protection-or-insurance.}}

\begin{enumerate}
\def\labelenumi{(\alph{enumi})}
\tightlist
\item
  \textbf{Lack of Skill.} When a Team agrees to protect, in whole or in part, the Base Compensation provided for by a Uniform Player Contract in the event such Contract is terminated by the Team, pursuant to paragraph 16(a)(iii) thereof, by reason of the player's lack of skill, such agreement shall mean that, subject to any conditions or limitations set forth in Exhibit 2 to the Uniform Player Contract or expressly set forth elsewhere in this Agreement, notwithstanding the provisions of paragraphs 16(a)(iii), 16(d), 16(e), and 16(g) of such Contract,the termination of such Contract by the Team on account of the player's failure to exhibit sufficient skill or competitive ability shall in no way affect the player's right to receive, in whole or in part, the Base Compensation payable pursuant to Exhibit 1 to such Contract in the amounts and at the times called for by such Exhibit.
\item
  \textbf{Non-Insured Death.} When a Team agrees to protect, in whole or in part, the Base Compensation provided for by a Uniform Player Contract in the event such Contract is terminated by the Team, pursuant to paragraph 16(a)(iv) thereof, by reason of the player's failure to render his services thereunder, if such failure has been caused by the player's non-insured death, such agreement shall mean that, subject to any conditions or limitations set forth in this Section 4(b), Exhibit 2 to the Uniform Player Contract, or expressly set forth elsewhere in this Agreement, notwithstanding the provisions of paragraphs 16(a), 16(b), 16(c), 16(d), 16(e), and 16(g) of such Contract, the termination of such Contract by the Team shall in no way affect the player's (or his estate's or duly appointed beneficiary's) right to receive, in whole or in part, the Base Compensation payable pursuant to Exhibit 1 to such Contract in the amounts and at the times called for by such Exhibit; provided, however, that: (i) such death does not result from the player's participation in activities prohibited by paragraph 12 of the Uniform Player Contract (as such paragraph may be modified by Exhibit 5 to the Player Contract), suicide, the abuse of alcohol, or the use of any Prohibited Substance or controlled substance; (ii) at the time of the player's failure to render playing services, the player is not in material breach of such Contract; (iii) if the Team, for its own benefit, seeks to procure an insurance policy covering the player's death, the player cooperates with the Team in procuring such an insurance policy, including by, among other things, supplying all information requested of him, and submitting to all examinations and tests requested of him by or on behalf of the insurance company in connection with the Team's efforts to procure such policy; and (iv) if the Team, for its own benefit, has procured such an insurance policy, the player's estate and/or duly appointed beneficiary cooperates with the Team and insurance company in the processing of the Team's claim under such policy.
\item
  \textbf{Insured Death.} When a Team agrees to insure, in whole or in part, the Base Compensation provided for by a Uniform Player Contract in the event such Contract is terminated by the Team, pursuant to paragraph 16(a)(iv) thereof, by reason of the player's failure to render his services thereunder, if such failure has been caused by the player's insured death, such agreement shall mean that, subject to any conditions set forth in this Section 4(c), Exhibit 2 to the Uniform Player Contract, or expressly set forth elsewhere in this Agreement, the Team has procured an insurance policy (specifically designated in Exhibit 2 to such Contract) for the benefit of the player or his estate or beneficiary that, subject to the conditions and limitations contained in the policy, would pay a benefit in the event of the player's death in an amount equal to or less than the Base Compensation remaining to be paid to the player under Exhibit 1 of his Player Contract at the time of his death; provided, however, that (i) such death does not result from the player's participation in activities prohibited by paragraph 12 of the Uniform Player Contract (as such paragraph may be modified by Exhibit 5 to the Player Contract), suicide, the abuse of alcohol, or the use of any Prohibited Substance or controlled substance, and (ii) at the time of the player's failure to render playing services, the player is not in material breach of such Contract.
\item
  \textbf{Non-Insured Basketball-Related Injury.} When a Team agrees to protect, in whole or in part, the Base Compensation provided for by a Uniform Player Contract in the event such Contract is terminated by the Team, pursuant to paragraph 16(a)(iv) thereof, by reason of the player's failure to render his services thereunder, if such failure has been caused by the player's disability and/or unfitness to play skilled basketball as a direct result of an injury sustained while participating in any basketball practice or game played for the Team, such agreement shall mean that, subject to any conditions or limitations set forth in this Section 4(d), Exhibit 2 to the Uniform Player Contract, or expressly set forth elsewhere in this Agreement, notwithstanding the provisions of paragraphs 7(b), 7(c), 16(a)(iii), 16(b), 16(c), 16(d), and 16(g) of such Contract, the termination of such Contract by the Team shall in no way affect the player's right to receive, in whole or in part, the Base Compensation payable pursuant to Exhibit 1 to such Contract in the amounts and at the times called for by such Exhibit; provided, however, that: (i) such injury does not result from an attempted suicide, the abuse of alcohol, or the use of any Prohibited Substance or controlled substance; (ii) at the time of the player's termination, the player is not in material breach of such Contract; (iii) if the Team, for its own benefit, seeks to procure an insurance policy covering the player's injury, the player cooperates with the Team in procuring such an insurance policy, including by, among other things, supplying all information requested of him, and submitting to all examinations and tests requested of him by or on behalf of the insurance company in connection with the Team's efforts to procure such policy; and (iv) if the Team, for its own benefit, has procured such an insurance policy, the player cooperates with the Team and the insurance company in the processing of the Team's claim under such policy.
\item
  \textbf{Non-Insured Injury or Illness.} When a Team agrees to protect, in whole or in part, the Base Compensation provided for by a Uniform Player Contract in the event such contract is terminated by the Team, pursuant to paragraph 16(a)(iv) thereof, by reason of the player's failure to render his services thereunder, if such failure has been caused by a non-insured injury, illness, or disability suffered or sustained by the player, such agreement shall mean that, subject to any conditions or limitations set forth in this Section 4(e), Exhibit 2 to the Uniform Player Contract, or expressly set forth elsewhere in this Agreement, notwithstanding the provisions of paragraphs 7(b), 7(c), 16(a)(iii), 16(b), 16(c), 16(d), and 16(g) of such Contract, the termination of such Contract by the Team shall in no way affect the player's right to receive, in whole or in part, the Base Compensation payable pursuant to Exhibit 1 to such Contract in the amounts and at the times called for by such Exhibit; provided, however, that: (i) such injury, illness, or disability does not result from the player's participation in activities prohibited by paragraph 12 of the Uniform Player Contract (as such paragraph may be modified in Exhibit 5 to the Player Contract), attempted suicide, the abuse of alcohol, or the use of any Prohibited Substance or controlled substance; (ii) at the time of such injury, illness, or disability the player is not in material breach of such Contract; (iii) if the Team, for its own benefit, seeks to procure an insurance policy covering the player's injury and/or illness, the player cooperates with the Team in procuring such an insurance policy, including by, among other things, supplying all information requested of him, and submitting to all examinations and tests requested of him by or on behalf of the insurance company in connection with the Team's efforts to procure such policy; and (iv) if the Team, for its own benefit, has procured such an insurance policy, the player cooperates with the Team and insurance company in the processing of the Team's claim under such policy.
\item
  \textbf{Insured Injury or Illness.} When a Team agrees to insure, in whole or in part, the Base Compensation provided for by a Uniform Player Contract in the event such Contract is terminated by the Team, pursuant to paragraph 16(a)(iv) thereof, by reason of the player's failure to render his services thereunder, if such failure has been caused by the player's disability or \textbf{\emph{{[}sic{]}}}
\item
  \textbf{Non-Insured Mental Disability.} When a Team agrees to protect, in whole or in part, the Base Compensation provided for by a Uniform Player Contract in the event such Contract is terminated by the Team, pursuant to paragraph 16(a)(iv) thereof, by reason of the player's failure to render his services thereunder, if such failure has been caused by the player's non-insured mental disability, such agreement shall mean that, subject to any conditions or limitations set forth in this Section 4(g), Exhibit 2 to the Uniform Player Contract, or expressly set forth elsewhere in this Agreement, notwithstanding the provisions of paragraphs 16(a), 16(b), 16(c), 16(d), 16(e), and 16(g) of such Contract, the termination of such Contract by the Team shall in no way affect the player's (or his duly appointed legal representative's) right to receive, in whole or in part, the Base Compensation payable pursuant to Exhibit 1 to such Contract in the amounts and at the times called for by such Exhibit; provided, however, that: (i) such mental disability does not result from the player's attempted suicide, or the use of any Prohibited Substance or controlled substance; (ii) at the time of the player's failure to render playing services, the player is not in material breach of such Contract; (iii) if the Team, for its own benefit, seeks to procure an insurance policy covering the player's mental disability, the player (and/or his duly appointed legal representative) cooperates with the Team in procuring such an insurance policy, including by, among other things, supplying all information requested of him, and submitting to all examinations and tests requested of him by the insurance company in connection with the Team's efforts to procure such policy; and (iv) if the Team, for its own benefit, has procured such an insurance policy, the player (and/or his duly appointed legal representative) cooperates with the Team and insurance company in the processing of the Team's claim under such policy.
\item
  \textbf{Insured Mental Disability.} When a Team agrees to insure, in whole or in part, the Base Compensation provided for by a Uniform Player Contract in the event such Contract is terminated by the Team, pursuant to paragraph 16(a)(iv) thereof, by reason of the player's failure to render his services thereunder, if such failure has been caused by the player's insured mental disability, such agreement shall mean that, subject to any conditions or limitations set forth in this Section 4(h), Exhibit 2 to the Uniform Player Contract, or expressly set forth elsewhere in this Agreement, the Team has procured an insurance policy (specifically designated in Exhibit 2 to such Contract) for the benefit of the player or his estate or beneficiary that, subject to the conditions and limitations contained in the policy, would pay a benefit in the event of the player's mental disability in an amount equal to or less than the Base Compensation remaining to be paid to the player under Exhibit 1 of his Player Contract at the time of his termination;provided, however, that: (i) such mental disability does not result from the player's participation in activities prohibited by paragraph 12 of the Uniform Player Contract (as such paragraph may be modified by Exhibit 4 to the Player Contract), attempted suicide, or the use of any Prohibited Substance or controlled substance; and (ii) at the time of the player's termination, the player is not in material breach of such Contract.
\item
  No agreement by a Team to protect or insure, in whole or in part, the Base Compensation provided for by a Uniform Player Contract shall require (or be construed as requiring) such Team to continue to employ the player (whether on the Active List, Inactive List, or otherwise); nor shall any such agreement afford the player any right to be employed, or to be deemed as having been employed, by such Team for any purpose.
\item
  Notwithstanding any other provision of this Agreement, when a Team agrees to protect or insure, in whole or in part, the Base Compensation provided for by a Uniform Player Contract, and such protection or insurance is contingent on the satisfaction of a condition expressly set forth in Exhibit 2 to that Contract, such protection or insurance shall be applicable and effective only if the Player Contract has not previously been terminated at the time such condition is satisfied.
\item
  Notwithstanding any other provision of this Agreement, when a Team agrees to protect or insure, in whole or in part, the Base Compensation provided for in any Option Year in favor of the Team included in a Uniform Player Contract, such protection or insurance shall be applicable and effective only if the option to extend the term provided for in the Contract was exercised by the Team prior to the termination of the Contract. When a Team agrees to protect or insure, in whole or in part, the Base Compensation provided for in any Option Year in favor of the player, the applicability of such protection or insurance in the circumstances in which the Option has not been exercised by the player shall be governed by the provisions of Article XII, Section 2(a).
\item
  During the term of a Player Contract, the percentage of protected or insured Base Compensation for any future Season shall not exceed the percentage of unearned protected or insured Base Compensation for any prior Season.
\item
  With respect to Player Contracts entered into or extended on or after the date of this Agreement:

  \begin{enumerate}
  \def\labelenumii{(\roman{enumii})}
  \tightlist
  \item
    The maximum amount of aggregate Base Compensation that can be protected for non-insured death is thirty million dollars (\$30,000,000).
  \item
    If a player elects to purchase term life insurance for his benefit, his Team shall be permitted to reimburse him each Season for the premiums paid for such insurance with respect to such Season and any other future Season(s); provided, however, that (A) the amount of coverage for which premiums are reimbursed by the Team in any Season shall not exceed the lesser of (x) the aggregate amount of the player's unearned Base Compensation for such Season and each remaining Season (excluding an Option Year if not yet exercised) that is not protected for non-insured death, and (y) forty million dollars (\$40,000,000) and (B) any such premium reimbursement shall not exceed the cost for 10-year guaranteed term coverage at preferred rates.
  \item
    If a Contract contains non-insured death protection covering ten million dollars (\$10,000,000) or more of Base Compensation, the player shall be precluded from purchasing life insurance for a period of ninety (90) days following the execution of the Contract or until such earlier time as the Team notifies the player in writing that it is no longer attempting to purchase life insurance coverage on the player (up to the amount of the player's Base Compensation protection for non-insured death) for the Team's benefit. During such ninety (90) day period or until such time as the Team issues the foregoing written notification to the player, the Team's efforts to purchase life insurance on the player for the Team's benefit shall be conducted diligently and in good faith.
  \end{enumerate}
\item
  Notwithstanding that a Team and a player are authorized under Article II, Section 4(a)--(h) to negotiate additional conditions or limitations applicable to the player's Compensation protection or insurance for such categories as the Team and player agree to protect or insure, no Player Contract entered into on or after the date of this Agreement may contain an additional condition or limitation relating to the player's use, possession, or distribution of any substance covered under the NBA Anti-Drug Program set forth in Article XXXIII.
\end{enumerate}

\hypertarget{conformity.}{%
\section{Conformity.}\label{conformity.}}

\begin{enumerate}
\def\labelenumi{(\alph{enumi})}
\tightlist
\item
  All currently effective Player Contracts, and all Player Contracts entered into following the execution of this Agreement that do not otherwise so provide, shall be deemed amended in such manner to require the parties to comply with all terms of this Agreement, including the terms of the Uniform Player Contract annexed hereto as Exhibit A. All Player Contracts shall be subject to the terms of this Agreement, which shall supersede the terms of any Player Contract inconsistent herewith. No Player Contract shall provide for the waiver by a player or a Team of any benefits or the sacrifice of any rights to which the player or the Team is entitled by virtue of a Uniform Player Contract or this Agreement.
\item
  Notwithstanding Section 5(a) above, the elimination from Section 3 above of any allowable amendment that was permitted under the 1999 NBA/NBPA Collective Bargaining Agreement (with the exception of the elimination of Article II, Section 3(k) of such Agreement), shall not affect the terms of any Player Contract entered into prior to the date of this Agreement. Nor shall any such Player Contracts be affected by any provisions of this Agreement expressly indicating that they apply only to Player Contracts entered into on or after the date of this Agreement.
\end{enumerate}

\hypertarget{minimum-player-salary.}{%
\section{Minimum Player Salary.}\label{minimum-player-salary.}}

\begin{enumerate}
\def\labelenumi{(\alph{enumi})}
\item
  Except with respect to 10-Day Contracts provided for in Section 9 below, and Rest-of-Season Contracts provided for in Section 10 below, no Player Contract shall provide for a Salary of less than the applicable scale amount contained in the Minimum Annual Salary Scale set forth as Exhibit C hereto.
\item
  No 10-Day Contract or Rest-of-Season Contract (as those terms are defined in Sections 9 and 10 below) shall provide for a Salary of less than the Minimum Player Salary applicable to that player.
\item
  In determining whether a Player Contract provides for a Salary of no less than the Minimum Player Salary applicable to that player, the allocation of a deemed signing bonus in respect of an ``international player payment'' in excess of \$500,000 (but no other bonuses) shall \textbf{\emph{{[}sic{]}}}
\item
  On July 1 of each Salary Cap Year, any Player Contract (whether entered into before or after the date of this Agreement) that provides for a Salary for the upcoming Season that is less than the applicable Minimum Player Salary shall be deemed amended to provide for the applicable Minimum Player Salary.
\item
  Nothing in this Section 6 shall alter the respective rights and liabilities of a player and a Team, as provided for in the Uniform Player Contract or in this Agreement, with respect to the termination of a Player Contract.
\item
  Every Contract entered into between a player and Team that is intended to provide for Compensation equal to the Minimum Player Salary (with no bonuses of any kind) for each Season must contain the following sentence in Exhibit 1A of such Contract and shall be deemed amended in the manner described in such sentence: ``This Contract is intended to provide for Compensation for the \_\_\_\_\_\_\_\_\_\_\_\_ Season(s) equal to the Minimum Player Salary for such Season(s) (with no bonuses of any kind) and shall be deemed amended to the extent necessary to so provide.''
\item
  Every Contract entered into between a player and a Team that covers the 2011-12 Season and one or more subsequent Seasons, and that is intended to provide for Compensation equal to the Minimum Player Salary (with no bonuses of any kind) for such Season(s), (i) must state in Exhibit 1A of such Contract that the Compensation for such Season(s) is the ``Minimum Player Salary,'' and (ii) shall comply with and be subject to the provisions of Article II, Section6(f) above. The player's Salary for each such Season shall equal his then-applicable Minimum Player Salary.
\item
  A Uniform Player Contract that provides in any Season for the player to earn Compensation not greater than his applicable Minimum Player Salary (with no bonuses of any kind) that, at the time the Contract is signed, is fully or partially protected for lack of skill and non-insured injury or illness may be amended to provide for the player to be paid a portion of his Compensation for such Season (the ``Advance''), up to the Maximum Advance Amount as defined below, prior to November 15 of such Season. The Maximum Advance Amount for a Season shall equal the lesser of (i) 80\% of the amount of the player's Compensation for such Season that is protected for lack of skill and non-insured injury or illness, or (ii) 7.5\% of the player's Base Compensation for such Season. Any Advance paid to a player for a Season pursuant to the foregoing must be deducted in full from the first (i.e., November 15) installment of Base Compensation for such Season that the player would have received pursuant to paragraph 3(a) of the Contract had there been no such Advance. To effectuate the requirement set forth in the preceding sentence, every such Contract that provides for an Advance must contain the following language (and only such language) under the ``Payment Schedule'' heading (next to the ``Current'' sub-heading) in Exhibit 1A with respect to each applicable Season:

  \begin{quote}
  ``Player's Current Base Compensation with respect to the \_\_\_\_\_\_\_\_\_ Season(s) shall be paid in accordance with paragraph 3(a), except that the November 15 installment of such Current Base Compensation shall be reduced by \${[}amount of Advance{]}, which amount shall be paid to Player in advance on {[}date{]}.''
  \end{quote}
\end{enumerate}

\hypertarget{maximum-annual-salary.}{%
\section{Maximum Annual Salary.}\label{maximum-annual-salary.}}

\begin{enumerate}
\def\labelenumi{(\alph{enumi})}
\tightlist
\item
  Notwithstanding any other provision of this Agreement, no Player Contract entered into after the date of this Agreement may provide for a Salary plus Unlikely Bonuses in the first Season covered by the Contract that exceeds the following amounts:

  \begin{enumerate}
  \def\labelenumii{(\roman{enumii})}
  \tightlist
  \item
    for any player who has completed fewer than seven (7) Years of Service, the greater of (x) 25\% of the Salary Cap in effect at the time the Contract is executed, (y) 105\% of the Salary for the final Season of the player's prior Contract, or (z) \$9 million;
  \item
    for any player who has completed at least seven (7) but fewer than ten (10) Years of Service, the greater of (x) 30\% of the Salary Cap in effect at the time the Contract is executed, (y) 105\% of the Salary for the final Season of the player's prior Contract, or (z) \$11 million; or
  \item
    for any player who has completed ten (10) or more Years of Service, the greater of (x) 35\% of the Salary Cap in effect at the time the Contract is executed, (y) 105\% of the Salary for the final Season of the player's prior Contract, or (z) \$14 million.
  \end{enumerate}
\item
  Notwithstanding any other provision of this Agreement, no Renegotiation entered into after the date of this Agreement may provide for a Salary plus Unlikely Bonuses in the Renegotiation Season (as defined in Article VII, Section 7(c)) that exceeds the following amounts:

  \begin{enumerate}
  \def\labelenumii{(\roman{enumii})}
  \tightlist
  \item
    for any player who has completed fewer than seven (7) Years of Service, the greater of (x) 25\% of the Salary Cap in effect at the time the Renegotiation is executed, (y) 105\% of the Salary for the Season prior to the Renegotiation Season, or (z) \$9 million;
  \item
    for any player who has completed at least seven (7) but fewer than ten (10) Years of Service, the greater of (x) 30\% of the Salary Cap in effect at the time the Renegotiation is executed, (y) 105\% of the Salary for the Season prior to the Renegotiation Season, or (z) \$11 million; or
  \item
    for any player who has completed ten (10) or more Years of Service, the greater of (x) 35\% of the Salary Cap in effect at the time the Renegotiation is executed, (y) 105\% of the Salary for the Season prior to the Renegotiation Season, or (z) \$14 million.
  \end{enumerate}
\item
  The parties recognize that it may not be possible to ascertain at the time an Extension is executed whether the Salary plus Unlikely Bonuses called for in the first Season of the extended term will exceed the Maximum Annual Salary set forth in this Section 7. Accordingly, and notwithstanding any other provision of this Agreement, the following rule shall apply to an Extension entered into in accordance with Article VII, Section 7(a) or a Rookie Scale Extension entered into in accordance with Article VII, Section 7(b) after the date of this Agreement: if, on the day following the last day of the Moratorium Period of the Salary Cap Year encompassing the first Season of the extended term of such Extension, the Salary plus Unlikely Bonuses provided for in such Season exceeds the following amounts:

  \begin{enumerate}
  \def\labelenumii{(\roman{enumii})}
  \tightlist
  \item
    for any player who has completed fewer than seven (7) Years of Service, the greater of (x) 25\% of the Salary Cap in effect on the day following the last day of the Moratorium Period, (y) 105\% of the Salary provided for in the final Season of the original term of the Contract, or (z) \$9 million;
  \item
    for any player who has completed at least seven (7) but fewer than ten (10) Years of Service, the greater of (x) 30\% of the Salary Cap in effect on the day following the last day of the Moratorium Period, (y) 105\% of the Salary provided for in the final Season of the original term of the Contract, or (z) \$11 million; or
  \item
    for any player who has completed ten (10) or more Years of Service, the greater of (x) 35\% of the Salary Cap in effect on the day following the last day of the Moratorium Period, (y) 105\% of the Salary provided for in the final Season of the original term of the Contract, or (z) \$14 million;
  \end{enumerate}

  then such Salary plus Unlikely Bonuses shall immediately be deemed amended to provide for the maximum amount allowed by the applicable subsection (c)(i), (c)(ii), or (c)(iii) set forth above. In such circumstance, Salaries plus Unlikely Bonuses in subsequent Seasons of the extended term shall also immediately be deemed amended to provide for increases or decreases over the amended Salary plus Unlikely Bonuses in the first Season of the extended term in accordance with Article VII, Section 5(c).
\item
  A player and a Team may provide in a Rookie Scale Extension that the player's Salary (in the first Season of the extended term) will equal ``the Maximum Annual Salary applicable to such player in the first Season of the extended term,'' and that the Salaries in any subsequent Seasons of the extended term will be increased or decreased based on percentages specified by the parties that comply with Article VII, Section 5(c). Any such Rookie Scale Extension shall be deemed amended on the day following the last day of the Moratorium Period of the Salary Cap Year covering the first Season of the extended term to provide for specific Salaries for each Season of the extended term, based on the Maximum Annual Salary applicable to such player on the day following the last day of the Moratorium Period. A Rookie Scale Extension entered into pursuant to this subsection may not include any Incentive Compensation.
\item
  Notwithstanding any other provision of this Agreement, if a trade of a Uniform Player Contract entered into or extended after the date of this Agreement would, by reason of a trade bonus contained in such Contract, cause the player's Salary plus Unlikely Bonuses for the Salary Cap Year in which such trade occurs to exceed the following amounts:

  \begin{enumerate}
  \def\labelenumii{(\roman{enumii})}
  \tightlist
  \item
    for any player who has completed fewer than seven (7) Years of Service, the greater of (x) 25\% of the Salary Cap in effect at the time the trade bonus is earned, (y) 105\% of the player's Salary for the Season prior to the Season in which the trade bonus is earned, or (z) \$9 million;
  \item
    for any player who has at least seven (7) but fewer than ten (10) Years of Service, the greater of (x) 30\% of the Salary Cap in effect at the time the trade bonus is earned, (y) 105\% of the player's Salary for the Season prior to the Season in which the trade bonus is earned, or (z) \$11 million; or
  \item
    for any player who has completed ten (10) or more Years of Service, the greater of (x) 35\% of the Salary Cap in effect at the time the trade bonus is earned, (y) 105\% of the player's Salary for the Season prior to the Season in which the trade bonus is earned, or (z) \$14 million; then such player's trade bonus shall be deemed amended to the extent necessary to reduce the player's Salary plus Unlikely Bonuses to the maximum amount allowed by the applicable subsection (e)(i), (e)(ii), or (e)(iii) set forth above.
  \end{enumerate}
\item
  For purposes of this Section 7 only, the Salary Cap shall be calculated in accordance with Article VII, Section 2, except that the percentage of Projected BRI to be utilized for such calculation shall be 48.04\% for all Salary Cap Years (including 2011-12, if the NBA exercises its option to extend this Agreement pursuant to Article XXXIX) during the term of this Agreement.
\end{enumerate}

\hypertarget{promotional-activities.}{%
\section{Promotional Activities.}\label{promotional-activities.}}

\begin{enumerate}
\def\labelenumi{(\alph{enumi})}
\tightlist
\item
  A player's obligation (pursuant to paragraph 13(d) of a Uniform Player Contract) to participate, upon request, in all other reasonable promotional activities of the Team and the NBA shall be deemed satisfied if:

  \begin{enumerate}
  \def\labelenumii{(\roman{enumii})}
  \tightlist
  \item
    during each year of the period covered by such Contract, the Player makes seven (7) individual personal appearances (at least two (2) of which shall be in connection with season ticketholder events) and five (5) group appearances for or on behalf of or at the request of the Team (or Team Affiliate) by which he is employed and/or the NBA. Up to two (2) of these twelve (12) appearances may be assigned by the Team and/or the NBA in any year to NBA Properties. The Player shall be reimbursed for the actual expenses incurred in connection with any such appearance, provided that such expenses result directly from the appearance and are ordinary and reasonable. The Player shall also receive compensation from the Team by which he is employed of at least \$2,500, in accordance with paragraph 13(d) of the Uniform Player Contract, for each promotional appearance he makes for a commercial sponsor of such Team. Any personal or group appearance required under this subsection (a) must:

    \begin{enumerate}
    \def\labelenumiii{(\Alph{enumiii})}
    \tightlist
    \item
      take place during (1) the period from the first day of a Season through the day of the NBA Draft following such Season, or (2) the off-season, provided that no player may be required to make more than one off-season appearance in any year covered by his Contract and no player may be required to make such an off- season appearance unless he resides in or is otherwise located in the area where the appearance is to take place;
    \item
      occur in the home city (or geographic vicinity thereof) of the player's Team (subject to Section 8(a)(i)(A)(2) above) or in a city (or geographic vicinity thereof) to which the player has traveled to play in a scheduled NBA game;
    \item
      not occur at a time that would interfere with a player's reasonable preparation to play on the day of a Team game;
    \item
      not occur at a time that would interfere with a player's ability to attend and participate fully in any practice session conducted by the Team, taking into account the commuting time from the practice to the appearance;
    \item
      be scheduled with the player at least fourteen (14) days in advance (by providing written notice to the player of the time, nature, location, and expected duration of the appearance) and called to his attention again seven (7) days prior to the appearance;
    \item
      not exceed a reasonable period of time; and
    \item
      not require the player to sign autographs as the primary purpose of the appearance, and
    \end{enumerate}
  \item
    The player participates in reasonable fan appreciation activities before and after home games, including but not limited to signing autographs for fans, greeting fans, and participating in merchandise giveaways to fans; provided, however, that no player shall be required to participate in more than four (4) such activities per Season.
  \end{enumerate}
\item
  Upon request by the Team, the NBA, or a League-related entity, and upon the Player's consent, and subject to the conditions and limitations set forth below, the Player shall wear a wireless microphone during any game or practice, including warm-up periods and going to and from the locker room to the playing floor. The rights in any audio captured by such microphone shall belong to the NBA or a League-related entity and may be used in any manner for publicity or promotional purposes.

  \begin{enumerate}
  \def\labelenumii{(\roman{enumii})}
  \tightlist
  \item
    The NBA or a League-related entity will be responsible for providing the audio equipment and for the placement of the microphone on the player.
  \item
    The player may remove the microphone at any time.
  \item
    The audio captured by the wireless microphone worn by the player (``Player Audio'') will be screened and approved prior to airing by the telecast producer and an NBA representative, and no such audio will be aired live.
  \item
    A game telecast will not include any Player Audio that contains profanity or that could reasonably be considered prejudicial or detrimental to the player or other players.
  \item
    All audio tapes containing Player Audio will be returned by the telecaster to the NBA and archived.
  \item
    At the request of the player or the Players Association, the NBA shall make available a copy of the Player Audio.
  \item
    In the event a player believes that any Player Audio excerpt would be prejudicial or detrimental to him if replayed in any non-game programming (e.g., home videos) or other publicity or promotional content, and notifies the NBA to that effect in writing within one hundred twenty (120) hours of the recording of such audio, then neither the NBA nor any League-related entity, following receipt of such notice from the player, shall incorporate, or license others to incorporate, such excerpt into any such content.
  \item
    The player will receive advance written notice of the conditions and limitations set forth in Section 8(b)(i)-(vii) above.
  \end{enumerate}
\item
  Each player shall be required to participate each Season, upon request, in promotional activities for the benefit of the NBA's television partners, provided that such participation does not exceed one (1) hour per player per Season and that the player is reimbursed for any reasonable expenses he incurs in connection with such participation.
\end{enumerate}

\hypertarget{day-contracts.}{%
\section{10-Day Contracts.}\label{day-contracts.}}

\begin{enumerate}
\def\labelenumi{(\alph{enumi})}
\tightlist
\item
  Beginning on January 5 (if a business day) or the first business day following January 5 of any NBA Season, a Team may enter into a Player Contract with a player for the longer of (i) ten (10) days, or (ii) a period encompassing three (3) games played by such Team (a ``10-Day Contract'').
\item
  The Salary provided for by a 10-Day Contract shall not be less than the Minimum Player Salary.
\item
  No Team may enter into a 10-Day Contract with the same player more than twice during the course of any one Season. No Team may be a party at any one time to more 10-Day Contracts than the number of players on such Team's Inactive List.
\item
  No Team may enter into a 10-Day Contract if the length of such Contract, in accordance with the first sentence of this Section 10(a), would extend to or past the date of the Team's last Regular Season game for such Season.
\item
  Notwithstanding anything to the contrary contained in a Uniform Player Contract, a 10-Day Contract shall be terminated simply by providing written notice to the player (and not by following the waiver procedure set forth in paragraph 16 of the Uniform Player Contract) and paying only such sums as are set forth in Exhibit 1 of such Contract.
\end{enumerate}

\hypertarget{rest-of-season-contracts.}{%
\section{Rest-of-Season Contracts.}\label{rest-of-season-contracts.}}

\begin{enumerate}
\def\labelenumi{(\alph{enumi})}
\tightlist
\item
  At any time after the first day of an NBA Regular Season, a Team may enter into a Player Contract that may provide Compensation to a player only for the remainder of that Season (a ``Rest-of-Season Contract'').
\item
  The Salary provided for in a Rest-of-Season Contract shall not be less than the Minimum Player Salary.
\end{enumerate}

\hypertarget{bonuses.}{%
\section{Bonuses.}\label{bonuses.}}

\begin{enumerate}
\def\labelenumi{(\alph{enumi})}
\item
  \begin{enumerate}
  \def\labelenumii{(\roman{enumii})}
  \tightlist
  \item
    Notwithstanding any other provision of this Agreement, (A) no Uniform Player Contract entered into or extended on or after the date of this Agreement may provide for a signing bonus that exceeds twenty (20) percent of the Compensation (excluding Incentive Compensation) called for by the Contract (or, in the case of an Extension, in the extended term of the Extension), and (B) no Offer Sheet may provide for a signing bonus that exceeds seventeen and one-half (17½) percent (or, with respect to \textbf{\emph{{[}sic{]}}}

    \begin{enumerate}
    \def\labelenumiii{(\roman{enumiii})}
    \setcounter{enumiii}{1}
    \tightlist
    \item
      If a player's Contract provides for a signing bonus and the player is suspended for the intentional failure or refusal to render the services required under his Contract, the Team shall be entitled to a return from the player of an amount equal to the product of the signing bonus multiplied by a fraction, the numerator of which is the number of Regular Season games that the player is suspended as a result of his failure or refusal to render such services and the denominator of which is the total number of Regular Season games to be played by the Team during the term of the Contract (excluding any Option Year). The foregoing shall not limit any other rights or remedies a Team may have under the Contract or by law.
    \end{enumerate}
  \end{enumerate}
\item
  \begin{enumerate}
  \def\labelenumii{(\roman{enumii})}
  \tightlist
  \item
    No Uniform Player Contract may provide for the player's attendance at and participation in an off-season skill and/or conditioning program that exceeds two (2) weeks in length.

    \begin{enumerate}
    \def\labelenumiii{(\roman{enumiii})}
    \setcounter{enumiii}{1}
    \tightlist
    \item
      A Uniform Player Contract that contains a bonus to be paid as a result of the player's attendance at and participation in an off-season summer league and/or an off-season skill and/or conditioning program may also contain a provision providing that such bonus will be paid if: (A) the Team elects in writing to waive the requirement that the player perform the specified services; (B) the player, in lieu of providing the specified services, participates in training and/or plays games with his national team during the off-season; and/or (C) the player has an injury, illness or other medical condition that renders the player unable to participate in such summer league and/or skill and conditioning program. If a Contract contains a provision of the type described in (A) above and the Team exercises its right to waive the requirement that the player perform the specified services with respect to one or more off-seasons, the amounts paid to the player shall continue to be treated as a bonus for the player's participation in an off-season summer league or off-season skill and conditioning program and shall continue to be subject to the rules in this Agreement relating to such bonuses.
    \item
      If a Uniform Player Contract contains a bonus to be paid as a result of the player's attendance at and participation in an off-season summer league and/or an off-season skill and/or conditioning program, the Team shall be required to provide the player with a reasonable opportunity to earn the bonus by, for example, providing the player with the dates, times and location(s) at which the specified services are to be performed. A Team's failure to comply with this requirement with respect to any off-season shall be deemed to constitute a waiver of the requirement that the player perform the specified services for such off-season.
    \item
      No Player Contract entered into or extended after the date of this Agreement may provide for bonuses for any Season to be paid as a result of the player's attendance at and participation in an off-season summerleague and/or an off-season skill and/or conditioning program that exceed 20\% of the player's Base Compensation for such Season.
    \end{enumerate}
  \end{enumerate}
\item
  No Uniform Player Contract entered into or extended on or after the date of this Agreement may contain a bonus for the player being on a Team's Roster as of a specified date or for a specified duration, or for the player dressing in uniform for or being eligible to play in a specified number of games.
\item
  If a Player Contract contains Incentive Compensation, a Team and player shall not be permitted at any time to amend the Contract to modify the conditions that the player must satisfy in order to earn all or any portion of such Incentive Compensation.
\end{enumerate}

\hypertarget{general.}{%
\section{General.}\label{general.}}

\begin{enumerate}
\def\labelenumi{(\alph{enumi})}
\item
  \begin{enumerate}
  \def\labelenumii{(\roman{enumii})}
  \tightlist
  \item
    Subject to Section 14 below, any oral or written agreement between a player and a Team concerning terms and conditions of employment shall be reduced to writing in the form of a Uniform Player Contract or an amendment thereto as soon as practicable. Immediately upon the consummation of any such oral or written agreement, the Team shall notify the NBA by facsimile or e-mail and provide the NBA with all economic terms of such agreement. Upon its receipt of such notice, the NBA shall promptly provide the same notice to the Players Association.

    \begin{enumerate}
    \def\labelenumiii{(\roman{enumiii})}
    \setcounter{enumiii}{1}
    \tightlist
    \item
      Notwithstanding subsection (a)(i) above, neither the NBA, any Team, nor the Players Association or any player, shall contend that any agreement concerning terms and conditions of employment is binding upon the player or the Team until a Player Contract embodying such terms and conditions has been duly executed by the parties. Nothing herein is intended to affect (A) any authority of the Commissioner to approve or disapprove Player Contracts, or (B) the effect of the Commissioner's approval or disapproval on the validity of such Player Contracts.
    \item
      A violation of the first sentence of subsection (a)(i) above may be considered evidence of a violation of Article XIII.
    \end{enumerate}
  \end{enumerate}
\item
  No player shall attend the regular training camp of any Team, or participate in games or organized practices with the Team at any time, unless he is a party to a Player Contract then in effect. For purposes of this Section 12(b), a player shall be considered to be a party to a Player Contract then in effect if such Contract has been extended in accordance with an Option permitted by this Agreement.
\item
  The only form of Compensation that a Team may pay a player under his Uniform Player Contract is cash via a check made payable to the player or via a direct deposit made to the player's bank account. Compensation of any other kind is prohibited.
\item
  No Team shall make any direct or indirect payment of any money, property, investments, loans, or anything else of value for fees or otherwise to an agent, attorney, or representative of a player (for or in connection with such person's representation of such player); nor shall any Player Contract provide for such payment. No player shall assign or otherwise transfer to any third party his right to receive Compensation from the Team under his Uniform Player Contract. Nothing in this subsection (d), however, shall prevent a Team from sending a player's regular paycheck to a player's agent, attorney, or representative if so instructed in writing by the player.
\item
  Every Uniform Player Contract entered into or extended after the date of this Agreement must provide that for each Season of such Contract, the player will be paid at least twenty (20) percent of his Salary for such Season, excluding Likely Bonuses and any portion of the player's Salary attributable to a trade bonus, in Current Base Compensation in accordance with the payment schedule provided in paragraph 3 of the Contract.
\item
  No Uniform Player Contract may provide for the payment of any Compensation earned for a Season prior to the July 1 immediately preceding such Season.
\item
  A Team's termination of a Uniform Player Contract by reason of the player's ``lack of skill'' (under paragraph 16(a)(iii) of the Uniform Player Contract) shall be interpreted to include a termination based on the Team's determination that, in view of the player's level of skill (in the sole opinion of the Team), the Compensation paid (or to be paid) to the player is no longer commensurate with the Team's financial plans or needs. The foregoing sentence shall not affect any post-termination obligation to pay Compensation that may result from Compensation protection provisions included in a Uniform Player Contract.
\item
  The following provisions shall govern an agreement (to be set forth in Exhibit 6 to a Uniform Player Contract) establishing that the player must report for and submit to a physical examination to be performed by a physician designated by the Team:

  \begin{enumerate}
  \def\labelenumii{(\roman{enumii})}
  \tightlist
  \item
    The player must report for such physical examination at the time designated by the Team (which shall be no later than the third business day following the execution of the Contract), and must, upon reporting, supply all information reasonably requested of him, provide complete and truthful answers to all questions posed to him, and submit to all examinations and tests requested of him. The determination of whether the player has passed the physical examination shall be made by the Team in its sole discretion. If the player does not pass the physical examination, the Team shall so notify the player no later than the sixth business day following the execution of the Contract.
  \item
    The Team's determination that the player has passed the physical examination by the player shall be a condition precedent to the validity of the Contract. Accordingly, and without limiting the generality of the preceding sentence, until such time as a player has passed the physical examination, the prohibitions set forth in Section 12(b) above shall continue to apply to the Team and player.
  \item
    A Required Tender or a Qualifying Offer may contain an Exhibit 6. If a player accepts such a Required Tender or Qualifying Offer but does not pass the required physical examination, the Required Tender or Qualifying Offer shall be deemed to have been withdrawn, which shall have the consequences described in Article X, Section 4 or Article XI, Section 4, as the case may be.
  \end{enumerate}
\item
  A player who knows he has an injury, illness, or condition that renders, or he knows will likely render, him physically unable to perform the playing services required under a Player Contract may not validly enter into such Contract without prior written disclosure of such injury, illness, or condition to the Team.
\end{enumerate}

\hypertarget{void-contracts.}{%
\section{Void Contracts.}\label{void-contracts.}}

If a Player Contract fails to take effect or becomes void as a result of a Commissioner disapproval, the player's failure to pass a physical examination conducted pursuant to Exhibit 6 to such Contract, or the rescission of a trade conducted pursuant to Article VII, Section 8(e), then, in each such case:

\begin{enumerate}
\def\labelenumi{(\alph{enumi})}
\tightlist
\item
  the Team shall continue to possess such rights with respect to the player as the Team possessed at the time of the execution of the Contract, including, without limitation, any such rights that the Team possessed pursuant to Article VII, Section 6(b), Article X and Article XI;
\item
  any Required Tender or Qualifying Offer that was outstanding at the time the Contract was executed shall continue in effect as if the Contract had not been executed (including if the original deadline for accepting the Required Tender or Qualifying Offer expired following the execution of the Contract), but for no fewer than six (6) business days following the Commissioner's disapproval, the Team's issuance of notice to the player that he did not pass the physical examination, or the rescission of such trade, as the case may be; and
\item
  in the case of a player who does not pass a physical examination pursuant to Exhibit 6: (i) the player shall not be permitted to accept such Required Tender or Qualifying Offer for a period of two (2) business days following his receipt of notice from the Team that he did not pass his physical examination, during which period the Team may elect to withdraw the Required Tender or Qualifying Offer, which shall have the consequences described in Article X,Section 4 or Article XI, Section 4, as the case may be; and (ii) if the Required Tender or Qualifying Offer is not withdrawn by the Team during this period, the Required Tender or Qualifying Offer shall thereafter be deemed amended so as to eliminate any Exhibit 6 that may be contained therein.
\end{enumerate}

\hypertarget{moratorium-period.}{%
\section{Moratorium Period.}\label{moratorium-period.}}

Except as permitted in the next sentence, notwithstanding any other provision of this Agreement, no player and Team may enter into any oral or written agreement concerning terms and conditions of the player's employment, or reduce any such agreement to writing in the form of a Uniform Player Contract or amendment, during the Moratorium Period. The following shall be permitted during the Moratorium Period: (i) a player may accept any Required Tender, Qualifying Offer, or ``Maximum Qualifying Offer'' (as defined in Article XI, Section 4(a)(ii)) that is outstanding; (ii) a player and a Team may negotiate over the terms and conditions of a Player Contract or Offer Sheet that may be entered into following the conclusion of the Moratorium Period; (iii) a First Round Pick and the Team that holds his draft rights may enter into a Rookie Scale Contract; and (iv) a player and a Team may enter into a Player Contract, not to exceed two (2) Seasons in length, that provides for a Salary for each Salary Cap Year equal to the Minimum Player Salary applicable to the player (with no Unlikely Bonuses).

\hypertarget{player-expenses}{%
\chapter{PLAYER EXPENSES}\label{player-expenses}}

\hypertarget{moving-expenses.}{%
\section{Moving Expenses.}\label{moving-expenses.}}

\begin{enumerate}
\def\labelenumi{(\alph{enumi})}
\tightlist
\item
  A Team's obligation to reimburse a player for ``reasonable'' expenses related to the assignment of a Player Contract from one Team to another (in accordance with paragraph 10 of a Uniform Player Contract) shall extend to the reimbursement of the actual expenses incurred by such player in moving to the home territory of his new Team, provided that such expenses result directly from the assignment and are ordinary and reasonable, and provided further that, prior to his actually incurring such expenses, the player consults with the Team to which his Contract has been assigned (furnishing a written estimate of such proposed expenses, if requested by the Team), so as to afford such assignee-Team an opportunity to make reasonably comparable alternative arrangements for the move of the player. In the event that the assignee-Team requests an estimate of such proposed expenses, the player shall furnish such estimate to the Team within a reasonable time following the notice of the assignment of the Player Contract. Upon receipt of such estimate from the player, the Team shall, within ten (10) days, either agree to reimburse the player for the expenses set forth in such estimate or make alternative arrangements (at the Team's expense) for the move of the player.
\item
  A player whose Contract is assigned from one Team to another shall be reimbursed by the assignee-Team for the cost of a hotel room in a hotel (comparable to that in which such Team's players are lodged while ``on the road'') in the assignee-Team's home city for up to forty-five (45) days following the assignment.
\item
  A player whose Contract is assigned from one Team to another shall be reimbursed by the assignee-Team for the cost of his living quarters (either rent or mortgage expense) in the city from which he is assigned, for a period of three months after the date of the assignment; provided, however, that such payment shall be made only if and to the extent that the player is legally obligated for such costs, and shall not exceed \$4,000 per month.
\item
  Prior to reimbursing an assigned player as provided in this Section, an assignee-Team may require satisfactory proof that the player has paid the amounts for which he seeks reimbursement, and, in the case of housing costs reimbursements, satisfactory proof that the player is legally obligated to pay such housing costs and the amount thereof. Upon notice to the player, the assignee-Team may, as an alternative to reimbursement, pay the expenses incurred upon assignment (in accordance with the foregoing provisions of this Section) directly to the persons, firms, or corporations involved.
\item
  So as to minimize the potential liability of NBA Teams under this Section, a player who does not establish permanent or year-round residence in the home city (or geographic vicinity thereof) of the Team by which he is employed shall use his best efforts (i) to obtain a short-term lease on the living quarters he selects, and (ii) to procure lease provisions authorizing him to sublet such premises and/or granting such Team the option to take over such lease in the event the Contract of such player is assigned to another NBA Team.
\end{enumerate}

\hypertarget{meal-expense-allowance.}{%
\section{Meal Expense Allowance.}\label{meal-expense-allowance.}}

\begin{enumerate}
\def\labelenumi{(\alph{enumi})}
\tightlist
\item
  The meal expense allowance, provided for in paragraph 4 of a Uniform Player Contract, shall be as follows:

  \begin{enumerate}
  \def\labelenumii{(\roman{enumii})}
  \tightlist
  \item
    For the 2005-06 Season: \$102 per day.
  \item
    For each subsequent Season of this Agreement: \$102 plus a cost of living adjustment (which shall be calculated by applying to \$102 the percentage increase in the national Consumer Price Index from the June 1 through the May 31 immediately preceding such Season, and which shall be rounded off to the nearest whole dollar) per day.
  \end{enumerate}
\item
  When a Team is ``on the road'' for less than a full day, a partial meal expense shall be paid based upon the time of departure from or time of arrival in the Team's home city, in accordance with the following:

  \begin{enumerate}
  \def\labelenumii{(\roman{enumii})}
  \tightlist
  \item
    Departure after 9:00 a.m. or arrival before 7:00 a.m., no meal expense allowance for breakfast.
  \item
    Departure after 1:00 p.m. or arrival before 11:30 a.m., no meal expense allowance for lunch.
  \item
    Departure after 7:00 p.m. or arrival before 5:30 p.m., no meal expense allowance for dinner.
    For purposes of this Section 2(b), the meal expense allowance for breakfast shall be deemed to be 18\% of the applicable daily meal expense allowance (rounded off to the nearest whole dollar); the meal expense allowance for lunch shall be deemed to be 28\% of the applicable daily meal expense allowance (rounded off to the nearest whole dollar); and the meal expense allowance for dinner shall be deemed to be 54\% of the applicable daily meal expense allowance (rounded off to the nearest whole dollar).
  \end{enumerate}
\item
  For purposes of this Agreement and paragraph 4 of the Uniform Player Contract, the ``home city'' of an NBA Team shall be deemed to include only the city in which the facility regularly used by the Team for home games is located and any other location at which such home games are played, provided that such other location(s) is not more than 75 miles from such city.
\end{enumerate}

\hypertarget{benefits}{%
\chapter{BENEFITS}\label{benefits}}

\hypertarget{player-pension-benefits.}{%
\section{Player Pension Benefits.}\label{player-pension-benefits.}}

Except as set forth below in this Section 1, effective with the date of this Agreement, and continuing for the duration thereof, the NBA shall provide the following pension benefits to NBA players and former NBA players:

\begin{enumerate}
\def\labelenumi{(\alph{enumi})}
\tightlist
\item
  Subject to the provisions of Section 1(d) below, the NBA shall provide pension benefits in accordance with the terms of the National Basketball Association Players' Pension Plan, as restated effective February 2, 1997, and as amended by the First, Second, Third and Fourth Amendments thereto (the ``Pension Plan''). In accordance with the September 1995 and January 1999 Collective Bargaining Agreements between the parties, the ``Normal Retirement Pension'' (as defined under the Pension Plan) payable to a player under the Pension Plan is the maximum monthly amount permitted by the applicable benefit limitations under the Internal Revenue Code of 1986, as amended (the ``Code''), as in effect immediately prior to the enactment of the Economic Growth and Tax Relief Reconciliation Act of 2001 (``EGTRRA''), to be paid to the player at his ``Normal Retirement Date'' (as defined under the Pension Plan) under the Pension Plan, based upon a Social Security Retirement Age of 65 (the ``Maximum Monthly Benefit''). The Maximum Monthly Benefit shall be increased only as specifically provided for in
  this Section 1(a).
  Effective only for the duration of this Agreement, the Maximum Monthly Benefit shall, except as otherwise provided herein, be adjusted for increases in the cost of living in the manner provided for under Section 415(d)(2) of the Code. In no event, however, shall the adjusted Maximum Monthly Benefit for a Plan Year exceed an amount that would require the actuarially-determined scheduled contributions (to be made to the Pension Plan to fund for such adjusted benefit for the Plan Year) to exceed, by more than five (5) percent, the actuarially-determined scheduled contributions that would be made to the Pension Plan for that Plan Year using the Maximum Monthly Benefit in effect for the immediately preceding Plan Year. The parties agree that the determinations described in the preceding sentence, including any actuarial assumptions and projections related thereto, shall be made by the actuaries of the Pension Plan and that any such determinations shall be binding and conclusive. Any increase in the Maximum Monthly Benefit hereunder shall be effective as of the first day of the month following the beginning of the Plan Year of the Pension Plan to which the increase relates (the ``Benefit Increase Commencement Date''), shall apply only with respect to any benefit payment or payments to be made on or after the Benefit Increase Commencement Date, and shall not require the recalculation of any benefit payment or payments made prior to the Benefit Increase Commencement Date.
  Notwithstanding the foregoing:

  \begin{enumerate}
  \def\labelenumii{(\arabic{enumii})}
  \tightlist
  \item
    Except as may otherwise be required under the Code, the benefit payable to any player or beneficiary under the Pension Plan shall in no event exceed the limitations on benefits under the Code, as in effect immediately prior to the enactment of EGTRRA.
  \item
    If all or any portion of the actuarially-determined scheduled contributions to be made to the Pension Plan will not be fully deductible under the Code when paid, the Maximum Monthly Benefit shall not exceed the amount which would result in all of such contributions being fully deductible when paid. The Players Association shall be given written notice of any such determination. The parties agree that the determinations described in this subsection (a)(2), including any actuarial assumptions and projections related thereto, shall be reasonable and shall be made by the actuaries of the Pension Plan. Any such determinations shall be binding and conclusive.
  \item
    The ``Normal Retirement Benefit'' (as defined under the Pension Plan) payable to a ``Pre-1965 Player'' (as defined under the Pension Plan) under the Pension Plan shall continue to be \$200 per month for each ``Year of Pre-1965 Credited Service'' (as defined under the Pension Plan). Any benefits that are unable to be paid to Pre-1965 Players under the Pension Plan because of the benefit limitations imposed by Section 415 of the Code shall be paid to such Pre-1965 Players pursuant to the National Basketball Association Excess Benefit Plan for Pre-1965 Players.
  \item
    The benefit payable to any player or beneficiary under the Pension Plan for a Plan Year shall in no event exceed the maximum benefit that may be paid to such player or beneficiary under the applicable benefit limitations under the Code, as in effect for that Plan Year.
  \item
    The Maximum Monthly Benefit for a Plan Year shall in no event exceed the maximum monthly amount permitted by the applicable benefit limitations under the Code, as in effect for that Plan Year.
  \end{enumerate}
\item
  Notwithstanding the provisions of Section 1(a) above, the NBA and the Players Association agree as follows:

  \begin{enumerate}
  \def\labelenumii{(\arabic{enumii})}
  \tightlist
  \item
    The NBA, with the assistance of the Players Association, will seek a private letter ruling from the Internal Revenue Service (``IRS'') determining that if the assets and liabilities under the Pension Plan attributable to former players who were never eligible to participate in the NBA-NBPA 401(k) Savings Plan (``Pre-1999 Players'') are transferred to a newly-established defined benefit pension plan (which shall be substantially similar to the Pension Plan) covering only Pre-1999 Players (the ``New Pension Plan''), the combined deduction limitation provisions of Section 404(a)(7) of the Code shall not apply to the New Pension Plan and contributions made by the NBA Teams to the New Pension Plan shall be subject only to the deduction limitations of Section 404(a)(1) of the Code.
  \item
    The NBA, with the assistance of the Players Association, shall file with the Pension Benefit Guaranty Corporation (``PBGC''), in accordance with Section 4231 of the Employee Retirement Income Security Act of 1974, as amended (``ERISA''), and the regulations promulgated thereunder, notice of their intent to transfer assets and liabilities attributable to Pre-1999 Players from the Pension Plan to the New Pension Plan (the ``Spinoff Transaction''). The NBA shall also file with the PBGC a request for a determination that the Spinoff Transaction complies with the requirements of Section 4231 of ERISA as soon as reasonably feasible following the date that notice of the Spinoff Transaction is provided to the PBGC. The NBA and the Players Association shall cooperate in satisfying any action or condition that the PBGC may establish in order for the NBA to obtain a favorable determination with respect to the Spinoff Transaction.
  \item
    If the NBA receives the private letter ruling described in Section 1(b)(1) above, the NBA shall, to the extent permitted by applicable law, establish the New Pension Plan and effectuate the Spinoff Transaction, subject to the approval of the IRS. The effective date of the New Pension Plan and the Spinoff Transaction shall be no earlier than the later of either (i) 120 days after receipt by the PBGC of the notice of the Spinoff Transaction from the NBA or (ii) the receipt from the PBGC of the determination requested pursuant to Section 1(b)(2) above. Upon the establishment of the New Pension Plan and the consummation of the Spinoff Transaction, pension benefits to Pre-1999 Players shall be provided solely under the New Pension Plan. The Pension Plan shall be amended to reflect the Spinoff Transaction.
  \item
    If the NBA receives the private letter ruling described in Section 1(b)(1) above, following the completion of the actions described in Section 1(b)(2) and (b)(3) above, subject to approval by the IRS and to the extent permitted by applicable law, the Pension Plan shall be amended to provide, and the New Pension Plan shall provide (or if applicable, shall be amended to provide), that the Normal Retirement Pension payable to a player under the Pension Plan and New Pension Plan is the maximum monthly amount permitted by the applicable benefit limitations under the Code, as in effect for the year in which this Agreement is executed, to be paid to the player at his Normal Retirement Date under the Pension Plan and New Pension Plan (the ``New Maximum Monthly Benefit''). The New Maximum Monthly Benefit shall be increased only as specifically provided for in this Section 1(b)(4).
    Effective only for the duration of this Agreement, the New Maximum Monthly Benefit shall, except as otherwise provided herein, be adjusted for increases in the cost of living in the same manner as the cost of living adjustment for the dollar limitation under Section 415(d)(1) of the Code, as in effect for the year in which the Agreement is executed (or, if applicable, in the manner provided for under Section 415(d)(2) of the Code). In no event, however, shall the adjusted New Maximum Monthly Benefit for a Plan Year exceed an amount that would require the actuarially-determined scheduled contributions (to be made in the aggregate to the Pension Plan and New Pension Plan to fund for such adjusted benefit for the Plan Year) to exceed, by more than five (5) percent, the actuarially-determined scheduled contributions that would be made in the aggregate to the Pension Plan and New Pension Plan for that Plan Year using the New Maximum Monthly Benefit in effect for the immediately preceding Plan Year. The parties agree that the determinations described in the preceding sentence, including any actuarial assumptions and projections related thereto, shall be made by the actuaries of the Pension Plan and New Pension Plan and that any such determinations shall be binding and conclusive. Any increase in the New Maximum Monthly Benefit hereunder shall be effective as of the first day of the month following the beginning of the Plan Year of the Pension Plan and New Pension Plan to which the increase relates (the ``New Benefit Increase Commencement Date''), shall apply only with respect to any benefit payment or payments to be made on or after the New Benefit Increase Commencement Date, and shall not require the recalculation of any benefit payment or payments made prior to the New Benefit Increase Commencement Date.
    Notwithstanding the foregoing:

    \begin{enumerate}
    \def\labelenumiii{(\roman{enumiii})}
    \tightlist
    \item
      The benefits payable to any player or beneficiary under the Pension Plan and New Pension Plan shall in no event exceed the limitations on benefits under the Code, as in effect for the year in which this Agreement is executed.
    \item
      If all or any portion of the actuarially-determined scheduled contributions to be made to the Pension Plan and New Pension Plan will not be fully deductible under the Code when paid, the New Maximum Monthly Benefit shall not exceed the amount which would result in all of such contributions being fully deductible when paid. The Players Association shall be given written notice of any such determination. The parties agree that the determinations described in this Section 1(b)(4)(ii), including any actuarial assumptions and projections related thereto, shall be reasonable and shall be made by the actuaries of the Pension Plan and the New Pension Plan. Any such determinations shall be binding and conclusive.
    \item
      The Normal Retirement Benefit payable to a Pre-1965 Player under the New Pension Plan shall continue to be \$200 per month for each Year of Pre-1965 Credited Service. Any benefits that are unable to be paid to Pre-1965 Players under the Pension Plan because of the benefit limitations imposed by Section 415 of the Code shall be paid to such Pre-1965 Players pursuant to the National Basketball Association Excess Benefit Plan for Pre-1965 Players.
    \item
      The benefit payable to any player or beneficiary under the Pension Plan or the New Pension Plan for a Plan Year shall in no event exceed the maximum benefit that may be paid to such player or beneficiary under the applicable benefit limitations under the Code as in effect for that Plan Year.
    \item
      The applicable provisions of this Section 1(b) shall become effective only after the adoption of both the New Pension Plan and the amendment to the Pension Plan as provided for under this Section 1(b)(4). Any additional benefits that may be payable to a player or beneficiary under the Pension Plan and the New Pension Plan as a result of the adoption of both the New Pension Plan and the amendment to the Pension Plan in order to implement the provisions of this Section 1(b) shall apply only to those players or beneficiaries who had not yet received or begun to receive a benefit under the Pension Plan as of July 1, 2005, and to those players or beneficiaries who were receiving monthly benefits under the Pension Plan as of July 1, 2005; and the provisions of this Section 1(b) shall apply only with respect to any benefit payment or payments made on or after July 1, 2005, and shall not require the recalculation of any benefit payment or payments made prior to July 1, 2005. Except as may otherwise be required by law, under no circumstances will a player or beneficiary, who has received or begun to receive payment of his benefit under the Pension Plan and who is entitled to receive an additional benefit under the Pension Plan or the New Pension Plan pursuant to the provisions of this Section 1(b)(4)(v), be entitled to receive any additional amounts in respect of such additional benefit, including, but not limited to, any interest or actuarial increase relating to the additional benefit.
    \item
      The increase in the amount of the actuarially-determined scheduled contributions that in the aggregate are required to be made for a Plan Year to the Pension Plan and the New Pension Plan in order to fund for the benefit described in this Section 1(b) (including any contributions that may be required in order to satisfy Section 4231 of ERISA and the regulations promulgated thereunder, and/or to receive the PBGC determination described in Section 1(b)(2) above), over the amount of the actuarially-determined scheduled contributions that in the aggregate would be required to be made for that Plan Year to the Pension Plan and the New Pension Plan in order to fund for the benefit described in Section 1(a) had this Section 1(b) never been in effect (the ``EGTRRA Cost Increase''), shall be applied against the New Benefit Amount provided for by Section 7 below; provided, however, that the amount to be applied against the New Benefit Amount pursuant to this Section 1 (b)(4)(vi) shall not exceed an amount equal to, for that Plan Year, fifty (50) percent of the EGTRRA Cost Increase or, if greater, the EGTRRA Cost Increase minus \$5 Million. For purposes of the preceding sentence, the determination of the amount of the actuarially-determined scheduled contributions that would be required to be made for a Plan Year to the Pension Plan and the New Pension Plan in order to fund for the benefit described in Section 1(a) had this Section 1(b) never been in effect shall be made without regard to the five (5) percent limitation on cost of living adjustments to the Maximum Monthly Benefit provided in Section 1(a) above. For purposes of determining the EGTRRA Cost Increase for a Plan Year, the scheduled contributions shall be determined based on the law in effect for that Plan Year, taking into account (A) any new law or change or amendment made to the Code, ERISA, any other applicable law, or to any regulations(whether final, temporary or proposed) or rulings issued thereunder or (B) any regulations (whether final, temporary or proposed) or rulings issued under the Code or ERISA. The parties agree that the determinations described in this Section 1(b)(4)(vi), including any actuarial assumptions and projections related thereto, shall be reasonable and shall be made by the actuaries of the Pension Plan and the New Pension Plan. Any such determinations shall be binding and conclusive.
    \item
      The New Maximum Monthly Benefit for a Plan Year shall in no event exceed the maximum monthly amount permitted by the applicable benefit limitations under the Code, as in effect for that Plan Year.
    \item
      The Players Association shall have the right, exercisable no later than 90 days after the NBA's receipt of both the private letter ruling described in Section 1(b)(1) above and the completion of the actions described in Sections 1(b)(2) and (b)(3) above, to direct the NBA either (A) to not establish the New Pension Plan and not effectuate the Spinoff Transaction (in which case the provisions of this Section 1(b) shall cease to apply under this Agreement and the provisions of Section 1(a) above shall continue to apply to the pension benefits being provided under the Pension Plan for the duration of this Agreement) or (B) to establish the New Pension Plan and effectuate the Spinoff Transaction and to provide that the New Maximum Monthly Benefit shall, in lieu of the amount specified in the first sentence of this Section 1(b)(4), be an amount specified by the Players Association that is less than the maximum monthly amount permitted by the applicable benefit limitations under the Code, as in effect for the year in which this Agreement is executed.
    \end{enumerate}
  \item
    All reasonable costs, including the cost of professional fees (e.g., attorneys, accountants, actuaries and consultants), that are incurred in connection with (i) requesting the private letter ruling described in Section 1(b)(1) above, (ii) the PBGC filing and request for the PBGC determination described in Section 1(b)(2) above, (iii) establishing the New Pension Plan, and (iv) effectuating the Spinoff Transaction, shall be paid by the Teams and shall be applied against the New Benefit Amount provided for by Section 7 below.
  \item
    After submission to the IRS of the private letter ruling request described in Section 1(b)(1) above and the filing with the PBGC of the notice and determination request described in Section 1(b)(2) above, if, for any reason, the NBA fails to obtain either the rulings from the IRS described in Section 1(b)(1) above or the determination from the PBGC described in Section 1(b)(2) above, the provisions of this Section 1(b) shall cease to apply under this Agreement, the NBA shall have no further obligation to continue to seek any such rulings or determination, and the provisions of Section 1(a) above shall continue to apply to the pension benefits being provided under the Pension Plan for the duration of this Agreement.
  \end{enumerate}
\item
  Notwithstanding anything else in this Agreement: (1) if any change or amendment made to the Code, ERISA, or to any regulations (whether final, temporary or proposed) or rulings issued thereunder; (2) if any interpretation, application or enforcement (or any proposed interpretation, application or enforcement), by a court of competent jurisdiction in the United States or by the IRS, of the Code, ERISA, or any regulations or rulings issued thereunder; or (3) if any regulations (whether final, temporary or proposed) or rulings issued by the IRS under the Code or ERISA; or (4) if any provisions of this Agreement, including any of the amendments or benefit increases to be provided under the Pension Plan and/or the New Pension Plan pursuant to this Section 1, would result in the Pension Plan and/or the New Pension Plan no longer being a tax-qualified plan under Section 401(a) of the Code, or would require NBA Teams to incur costs over and above any costs required to be incurred to implement the provisions of this Agreement or any prior collective bargaining agreement in order for the Pension Plan and/or the New Pension Plan to maintain its tax-qualified status under Section 401(a) of the Code (provided, however, that such additional costs are incurred solely in connection with the provision of pension benefits to their non-player employees or to non-player employees of affiliates (within the meaning of Sections 414(b), (c) or (m) of the Code) of such Teams), then any obligation to maintain and/or make contributions to the Pension Plan and/or the New Pension Plan pursuant to this Agreement or pursuant to any prior collective bargaining agreement shall terminate; provided, however, that any such termination shall not impair the legally binding effect of any other provision of this Agreement or the legally binding effect (if any) of any other provision of any prior collective bargaining agreement, nor shall it create any right (i) to unilaterally implement during the term of this Agreement any terms concerning the provision of pension benefits to the players, (ii) to lockout, or (iii) to strike. In the event of such termination, the NBA Teams shall provide alternative benefits to the players, at an annual cost (as determined on an after-tax basis) to NBA Teams equal to the annual cost that such Teams would have incurred under the Pension Plan and/or the New Pension Plan to fund the benefit described in Section 1(a) above, commencing on the date of termination. The NBA and the Players Association shall agree upon the type(s) of alternative benefits to be provided.
\item
  Players employed by Maple Leaf Sports \& Entertainment Ltd.~(or any successor thereto) (``Toronto'') or by an NBA Team located in any other country other than the United States shall receive pension benefits of comparable value. Players employed by Toronto (``Toronto Players'') shall receive such benefits by means of the Pension Plan and a separate pension plan maintained by Toronto (the ``Toronto Plan''); provided, however, that (1) if the provision of pension benefits under the Pension Plan to the Toronto Players (or, if applicable, the provision of pension benefits under the New Pension Plan to players formerly employed by Toronto or any predecessor thereto who were never eligible to participate in the NBA-NBPA 401(k) Savings Plan (``Pre-1999 Toronto Players'')) would, at any time, result in the Pension Plan or the New Pension Plan being subject to Canadian Provincial Pension Legislation and/or Canadian Federal Tax Laws (to the extent that the application of such tax laws would result in adverse tax consequences to the Pension Plan, the New Pension Plan, the NBA Teams, the Toronto Players, and/or the Pre-1999 Toronto Players), (2) if the Toronto Plan would not, at any future time, either satisfy United States tax qualifications or be able to be registered under Canadian Provincial Pension Legislation and/or Canadian Federal Tax Laws, then any obligation to establish, maintain and/or make contributions to the Pension Plan with respect to Toronto Players, the New Pension Plan with respect to Pre-1999 Toronto Players, and the Toronto Plan pursuant to this Agreement or pursuant to any prior collective bargaining agreement, shall terminate. In the event of such termination, Toronto shall provide alternative benefits to the Toronto Players and/or the Pre-1999 Toronto Players and at an annual cost (as determined on an after-tax basis) to Toronto equal to the annual cost that Toronto would have incurred under the Pension Plan, the New Pension Plan and the Toronto Plan commencing on the date of termination. The NBA and the Players Association shall agree upon the type(s) of alternative benefits to be provided.
\end{enumerate}

\hypertarget{player-401k-benefits.}{%
\section{Player 401(k) Benefits.}\label{player-401k-benefits.}}

Except as set forth below in this Section 2, effective with the date of this Agreement, and continuing for the duration thereof, the NBA shall provide the following 401(k) benefits to NBA players:

\begin{enumerate}
\def\labelenumi{(\alph{enumi})}
\tightlist
\item
  Benefits in accordance with the NBA-NBPA Players' 401(k) Savings Plan established effective November 1, 1999 by the NBA and the Players Association pursuant to the 1999 NBA/NBPA Collective Bargaining Agreement, and as amended by the First, Second, Third, Fourth, Fifth and Sixth Amendments thereto (the ``401(k) Plan''). The 401(k) Plan shall continue to provide for (1) deferrals by players pursuant to Section 401(k) of the Code and (2) except as may be limited below, Team matching contributions in respect of player deferrals for a Salary Cap Year to be determined by the Players Association. Team matching contributions and deferrals shall be subject to all applicable limitations under the Code. The cost of funding all matching contributions made by Teams under the 401(k) Plan (i) shall be applied against the New Benefit Amount provided for by Section 7 below, and (ii) shall be limited to the portion of the New Benefit Amount, if any, that is available for this purpose pursuant to Section 7(b)(2) below. Any matching contributions to be made to the 401(k) Plan in respect of each Season shall be made no later than thirty (30) days following the completion of the BRI Audit Report for the Salary Cap Year covering such Season.
  Notwithstanding the foregoing, the total amount of the deferral contributions to be made by players to the 401(k) Plan and the Team matching contributions to be made to the 401(k) Plan in respect of deferral contributions made by players under the 401(k) Plan shall be limited to an amount that (1) after first taking into account the contributions made by Teams to the Pension Plan (and, to the extent that any ruling described in Section 1(b)(1) above that may be obtained from the IRS no longer applies, the New Pension Plan) and the contributions made by Toronto to the Toronto Plan and (2) taking into account only compensation paid to current players by the Teams, would result in all of such deferral contributions and matching contributions being fully deductible under the Code (and, where applicable, Canadian income tax laws) when paid to the 401(k) Plan.
\item
  The terms of the 401(k) Plan shall continue to permit participation by Toronto Players on a tax-effective basis under Canadian income tax laws. If the NBA and the Players Association should determine that the 401(k) Plan cannot continue to be provided to Toronto Players on a tax-effective basis under Canadian income tax laws, an alternative arrangement, which is acceptable to both the NBA and the Players Association, shall be established for Toronto Players in lieu of the 401(k) Plan. The cost to Toronto of funding for any such alternative arrangement (1) shall be applied against the New Benefit Amount provided for by Section 7 below, and (2) shall be limited to the portion of the New Benefit Amount, if any, that is available for this purpose pursuant to Section 7(b)(2).
\item
  Notwithstanding anything else in this Agreement: (1) if any change or amendment made to the Code, ERISA, or to any regulations (whether final, temporary or proposed) or rulings issued thereunder; or (2) if any interpretation, application or enforcement (or any proposed interpretation, application or enforcement), by a court of competent jurisdiction in the United States or by the IRS, of the Code, ERISA, or any regulations or rulings issued thereunder; or (3) if any regulations (whether final, temporary or proposed) or rulings issued by the IRS under the Code or ERISA; or (4) if any provisions of this Agreement would result in the 401(k) Plan no longer being a tax-qualified plan under Section 401(a) of the Code, or would require NBA Teams to incur costs over and above any costs required to be incurred to implement the provisions of this Agreement or any prior collective bargaining agreement in order for the 401(k) Plan to maintain its tax-qualified status under Section 401(a) of the Code (provided, however, that such additional costs are incurred solely in connection with the provision of benefits to their non-player employees or to non-player employees of affiliates (within the meaning of Sections 414(b), (c) or (m) of the Code) of such Teams), then any obligation to maintain and/or make contributions to the 401(k) Plan pursuant to this Agreement or pursuant to any prior collective bargaining agreement shall terminate; provided, however, that any such termination shall not impair the legally binding effect of any other provision of this Agreement or the legally binding effect (if any) of any other provision of any prior collective bargaining agreement, nor shall it create any right (i) to unilaterally implement during the term of this Agreement any terms concerning the provision of 401(k) benefits to the players, (ii) to lockout, or (iii) to strike. In the event of such termination, the NBA Teams shall provide alternative benefits to the players, at an annual cost (as determined on an after-tax basis) to NBA Teams equal to the annual cost that such Teams would have incurred under the 401(k) Plan with respect to Team matching contributions commencing on the date of termination. The NBA and the Players Association shall agree upon the type(s) of alternative benefits to be provided. The costs of funding of any such alternative benefits (A) shall be applied against the New Benefit Amount provided for by Section 7 below, and (B) shall be limited to the portion of the New Benefit Amount, if any, that is available for this purpose pursuant to Section 7(b)(2) below.
\end{enumerate}

\hypertarget{player-supplemental-medical-benefits.}{%
\section{Player Supplemental Medical Benefits.}\label{player-supplemental-medical-benefits.}}

Except as set forth below in this Section 3, effective with the date of this Agreement, and continuing for the duration thereof, the NBA shall provide the following supplemental medical benefits to NBA players:

\begin{enumerate}
\def\labelenumi{(\alph{enumi})}
\tightlist
\item
  Benefits in accordance with the terms of the NBPA-NBA Supplemental Benefit Plan established, effective August 1, 2004, and as amended from time to time (the ``Supplemental Benefit Plan'') and the Agreement and Declaration of Trust of the NBPA-NBA Supplemental Benefit Plan established, July 22, 2004 (the Trust created thereunder to be referred to hereinafter as the ``SBP Trust''). The SBP Trust shall continue to be jointly operated and administered by the NBA and Players Association in accordance with Section 302(c)(5) of the Labor Management Relations Act of 1947, as amended, and the provisions of the SBP Trust and the Supplemental Benefit Plan. It is intended by the NBA and Players Association that the SBP Trust shall constitute a collectively-bargained voluntary employees' beneficiary association (``VEBA'') that is tax exempt pursuant to Section 501(c)(9) of the Code, and the parties shall continue to cooperate in seeking qualification of the SBP Trust as such a tax-exempt association.
\item
  The SBP Trust and the Supplemental Benefit Plan shall continue to be operated and administered for the purpose of providing certain health-related benefits to players who played in the NBA during and/or after the 2000-2001 Season in accordance with provisions set forth in the Supplemental Benefit Plan.
\item
  The costs of funding the Supplemental Benefit Plan and the costs attributable to the operation and administration of the SBP Trust and the Supplemental Benefit Plan, including the reasonable cost of professional fees (e.g., attorneys, accountants, actuaries and consultants)incurred in connection with the administration of the SBP Trust and the Supplemental Benefit Plan, shall be paid by the Teams, and (1) shall be applied against the New Benefit Amount provided for by Section 7 below, and (2) shall be limited to the portion of the New Benefit Amount, if any, that is available for this purpose pursuant to Section 7(b)(6) below. Any contributions to fund the Supplemental Benefit Plan in respect of each Salary Cap Year shall be made no later than thirty (30) days following the completion of the BRI Audit Report for such Salary Cap Year.
\item
  The daily operations of the Supplemental Benefit Plan shall continue to be administered by an independent third-party administrator, as selected by the trustees of the SBP Trust, at the administrator's office. In the exercise of its responsibilities, the independent third-party administrator shall be required to comply with ERISA and all other applicable laws and to act in a manner that is consistent with the provisions of the SBP Trust and the Supplemental Benefit Plan.
\item
  The SBP Trust and the Supplemental Benefit Plan shall be operated and administered in a manner that will result in all contributions by the Teams being fully deductible under the Code (and, where applicable, Canadian income tax laws) when paid to the SBP Trust. If any Team is disallowed a deduction (in whole or in part) for such contributions, and unless the NBA determines otherwise, the obligation to maintain the Supplemental Benefit Plan and to make further contributions to the SBP Trust shall immediately terminate; provided, however, that any such termination shall not impair the legally binding effect of any other provision of this Agreement, and shall not create any right (1) to unilaterally implement, during the term of this Agreement, any terms concerning the provision of benefits provided or to be provided by the Supplemental Benefit Plan, (2) to lockout, or (3) to strike.
\item
  In the event of any termination pursuant to Section 3(e) above, the parties agree to bargain in good faith with respect to an alternative arrangement designed to provide the benefits described in the Supplemental Benefit Plan. Such alternative arrangement shall, to the extent permitted by applicable law, be funded by such monies as may then remain in the SBP Trust and, if the monies remaining in the SBP Trust may not lawfully be used for, or are insufficient for, such purpose, such alternative arrangement shall be funded, by the NBA Teams; provided, however, that the annual cost incurred by the Teams in connection with such alternative arrangement (as determined on an after-tax basis) shall not exceed the annual cost that such Teams would have incurred to fund the Supplemental Benefit Plan commencing on the date of termination. Any such alternative arrangement shall be operated and administered in a manner that will result in all contributions by the Teams being fully deductible under the Code (and, where applicable, Canadian income tax laws) when paid; and, no matter how funded, the costs of funding for any alternative to the Supplemental Benefit Plan shall be applied against the New Benefit Amount provided for by Section 7 below, shall be limited to the portion of the New Benefit Amount, if any, that is available for this purpose pursuant to Section 7(b)(6) below, and shall be subject to the limitations set forth in this Agreement. If despite good faith negotiations, the NBA and the Players Association fail to agree with respect to an alternative arrangement as described above, such failure to agree shall not create any right (1) to unilaterally implement, during the term of this Agreement, any terms concerning the provision of benefits provided or to be provided by the Supplemental Benefit Plan, (2) to lockout, or (3) to strike.
\item
  The terms of the Supplemental Benefit Plan shall continue to permit participation by Toronto Players in accordance with the terms of the July 22, 2004 letter agreement between the NBA and the Players Association regarding the implementation of the Supplemental Benefit Plan with respect to Toronto Players, so that Toronto Players receive benefits that are substantially equivalent on an after-tax basis to the benefits received by players employed by Teams located in the United States. If the NBA and the Players Association determine that the Supplemental Benefit Plan cannot continue to provide benefits to Toronto Players that are substantially equivalent on an after-tax basis to the benefits provided to players employed by Teams located in the United States, an alternative arrangement relating to Toronto Players, which is acceptable to both the NBA and the Players Association, shall be established in lieu thereof. The cost to Toronto of funding for any such alternative arrangement shall be applied against the New Benefit Amount provided for by Section 7 below, shall be limited to the portion of the New Benefit Amount, if any, that is available for this purpose pursuant to Section 7(b)(6) below, and shall be subject to the limitations set forth in this Agreement. If despite good faith negotiations, the NBA and the Players Association fail to agree with respect to an alternative arrangement as described above, such failure to agree shall not create any right to lockout or to strike.
\end{enumerate}

\hypertarget{labor-management-cooperation-and-education-trust.}{%
\section{Labor-Management Cooperation and Education Trust.}\label{labor-management-cooperation-and-education-trust.}}

\begin{enumerate}
\def\labelenumi{(\alph{enumi})}
\item
  Except as set forth below in this Section 4, effective with the date of this Agreement, and continuing for the duration thereof, the National Basketball Players Association/National Basketball Association Labor-Management Cooperation and Education Trust (the ``Education Trust'') shall continue to be jointly operated and administered by the NBA and the Players Association in accordance with the provisions of the Agreement and Declaration of Trust Establishing the National Basketball Players Association/National Basketball Association Labor-Management Cooperation and Education Trust (the ``Education Trust Agreement''). It is intended by the NBA and the Players Association that, at all times, the Education Trust shall comply with the provisions of Section 302(c)(9) of the Labor Management Relations Act of 1947, as amended, and shall qualify as an exempt organization under the provisions of Section 501(c)(5) or 501(c)(3) of the Code.
\item
  The Education Trust shall continue to be operated and administered for the purpose of establishing and providing (i) HIV/AIDS education programs and (ii) education and career counseling programs designed to assist the NBA, NBA Teams and NBA players in solving problems of mutual concern not susceptible to resolution within the collective bargaining process and to enhance the involvement of NBA players in making decisions that affect their working lives.
\item
  \begin{enumerate}
  \def\labelenumii{(\arabic{enumii})}
  \tightlist
  \item
    Except as provided in Section 4(c)(2) below, the costs of funding the Education Trust and the costs attributable to the operation and administration of the Education Trust, including the cost of professional fees (e.g., attorneys, accountants, actuaries and consultants) incurred in connection with the administration of the Education Trust, shall be paid by the Teams, and (i) shall be applied against the New Benefit Amount, provided for by Section 7 below, and (ii) shall be limited to the portion of the New Benefit Amount, if any, that is available for this purpose pursuant to Section 7(b)(5) below. Payment of the amount necessary to fund the Education Trust in respect of each Salary Cap Year shall be made within 30 days following the completion of the BRI Audit Report for such Salary Cap Year. The parties agree that, subject to the limitations set forth in this Section 4, the amount to be paid by the Teams to fund the education and career counseling programs to be operated and administered by the Education Trust for the 2005-2006 Salary Cap Year shall be no greater than \$840,000 and such maximum funding amount shall be increased by five (5) percent for each subsequent Salary Cap Year.
  \item
    The costs of funding the HIV/AIDS education programs to be operated and administered by the Education Trust (or any programs that, pursuant to Section 4(f) below, are substituted for the HIV/AIDS education programs) shall be paid by the Teams. Payment of the amount necessary to fund such programs in respect of each Salary Cap Year shall be made within 30 days following the completion of the BRI Audit Report for such Salary Cap Year. The parties agree that, subject to the limitations set forth in this Section 4, the amount to be paid by the Teams to fund the HIV/AIDS education programs (or any programs that, pursuant to Section 4(f) below, are substituted for the HIV/AIDS education programs) to be operated and administered by the Education Trust for the 2005-2006 Salary Cap Year shall be no greater than \$300,000 and such maximum funding amount shall be increased by five (5) percent for each subsequent Salary Cap Year.
  \end{enumerate}
\item
  The Education Trust shall be operated and administered in a manner that will result in all contributions by the Teams being fully deductible under the Code (and, where applicable, Canadian income tax laws) when paid. If any Team is disallowed a deduction (in whole or in part) for such contributions, and unless the NBA determines otherwise, the obligation to maintain the Education Trust and to make further contributions to the Education Trust shall immediately terminate; provided, however, that any such termination shall not impair the legally binding effect of any other provision of this Agreement, and shall not create any right (1) to unilaterally implement, during the term of this Agreement, any terms concerning the provision of education programs provided or to be provided by the Education Trust, (2) to lockout, or (3) to strike.
\item
  In the event of any termination pursuant to Section 4(d) above, the parties agree to bargain in good faith with respect to an alternative arrangement designed to provide the programs described in the Education Trust Agreement. Such alternative arrangement shall, to the extent permitted by applicable law, be funded by such monies as may then remain in the Education Trust and, if the monies remaining in the Education Trust may not lawfully be used for, or are insufficient for, such purpose, such alternative arrangement shall be funded, by the NBA Teams; provided, however, that the annual cost incurred by the Teams in connection with such alternative arrangement (as determined on an after-tax basis) shall not exceed the annual cost that such Teams would have incurred to fund the Education Trust commencing on the date of termination. Any such alternative arrangement shall be operated and administered in a manner that will result in all contributions by the Teams being fully deductible under the Code (and, where applicable, Canadian income tax laws) when paid; and, no matter how funded, the costs of funding for any alternative to the Education Trust shall be applied against the New Benefit Amount provided for by Section 7 below, shall be limited to the portion of the New Benefit Amount, if any, that is available for this purpose pursuant to Section 7(b)(5) below, and shall be subject to the limitations set forth in this Agreement. If despite good faith negotiations, the NBA and the Players Association fail to agree with respect to an alternative arrangement as described above, such failure to agree shall not create any right (1) to unilaterally implement, during the term of this Agreement, any terms concerning the provision of programs provided or to be provided by the Education Trust, (2) to lockout, or (3) to strike.
\item
  Upon written notice delivered to the NBA at least six (6) months prior to the commencement of any Salary Cap Year, the Players Association may elect to terminate the programs currently provided by the Education Trust and substitute alternative programs; provided, however, that the NBA consents to such substitution, which such consent shall not be unreasonably withheld; and provided, further, that any new programs shall comply with the provisions of Section 302(c)(9) of the Labor Management Relations Act of 1947, as amended.
\end{enumerate}

\hypertarget{additional-player-benefits.}{%
\section{Additional Player Benefits.}\label{additional-player-benefits.}}

Except as set forth below, effective with the date of this Agreement, and continuing for the duration thereof, the NBA shall provide the following additional benefits to NBA players:

\begin{enumerate}
\def\labelenumi{(\alph{enumi})}
\item
  Life insurance and accidental death and dismemberment benefits, as set forth in the U.S. Life Insurance Company Policy No.~G-245667 (the ``U.S. Life Policy''); provided, however, that if all or a portion of the New Benefit Amount funding specified in Section 7(b)(3) below is not available to be used for this purpose, the benefits provided by the U.S. Life Policy shall be reduced commensurate with the reduced funding (but in no event shall such benefits be less than those provided pursuant to Article IV, Section 1(b) of the 1999 NBA/NBPA Collective Bargaining Agreement).
\item
  Disability insurance benefits, as set forth in the Houston Casualty Company Policy No.~05/700131.
\item
  Workers' compensation benefits in accordance with applicable statutes.
\item
  Medical and dental insurance benefits in accordance with the terms of the CIGNA HealthCare Policy No.~3211244 (the ``CIGNA Policy''). With respect to a Veteran Free Agent, such medical and dental benefits shall remain in effect until the August 31 following the last Season of the player's Contract.
  The CIGNA Policy or any subsequent policy or plan providing medical and dental benefits shall be modified or replaced effective as of the commencement of the 2006-2007Season or as of the commencement of any subsequent Season covered by this Agreement, as requested in writing by the Players Association (a ``Player Change''), provided such written request is delivered to the NBA on or before the March 1 preceding such Season. Any Player Change shall be subject to the approval of the NBA, which approval shall not be unreasonably withheld. Any additional aggregate costs that might be incurred by the Teams for medical and dental benefits with respect to the 2006-2007 or any subsequent Season as a result of a Player Change, over the aggregate cost of medical and dental benefits that otherwise would have been incurred by the Teams for the players under the CIGNA Policy with respect to such Season, absent any Player Change: (1) shall be applied against the New Benefit Amount provided for by Section 7 below; and (2) shall be limited to the portion of the New Benefit Amount, if any, that is available for this purpose pursuant to Section 7(b)(4) below.
\item
  Vision benefits in accordance with the Davis Vision Policy No.~500153.
\item
  Funding for the annual Players Association High School Basketball Camp (or any substitute program mutually agreed upon by the parties) in the amount of \$433,000 for the 2005-2006 Season, increasing by 7.5\% per Season thereafter for the term of this Agreement.
\item
  Player Playoff Pool amounts, as follows:

  \begin{longtable}[]{@{}
    >{\raggedright\arraybackslash}p{(\columnwidth - 2\tabcolsep) * \real{0.5312}}
    >{\centering\arraybackslash}p{(\columnwidth - 2\tabcolsep) * \real{0.4688}}@{}}
  \toprule()
  \endhead
  2005-2006 Season & \$10 million \\
  2006-2007 Season & \$10 million \\
  2007-2008 Season & \$11 million \\
  2008-2009 Season & \$11 million \\
  2009-2010 Season & \$12 million \\
  2010-2011 Season & \$12 million \\
  2011-2012 Season (if the NBA exercises its option to extend this Agreement pursuant to Article XXXIX) & \$12 million \\
  \bottomrule()
  \end{longtable}

  If the NBA increases the number of Teams participating in the playoffs, the Player Playoff Pool shall be increased by \$558,000 for each Team added with respect to the 2005-06 and 2006-07 Seasons; by \$586,000 with respect to the 2007-08 and 2008-09 Seasons; and \$615,000 with respect to each subsequent Season. The NBA will consult with the Players Association with respect to the method of allocation of the Player Playoff Pool.
\item
  The employer's portion of payroll taxes.
\item
  The Players Association's one-half share of the payment of fees and expenses to the Accountants (as defined in Article VII, Section 10(a) below) in connection with any audit conducted under this Agreement, and the Player Association's one-half share of the payment of fees and expenses payable with respect to the TV Expert (as defined in Article VII, Section 1(a)(7)(iii) below) and any expert selected in accordance with Article VII, Section 1(a)(7)(i).
\item
  The Players Association's share of the costs of the Anti-Drug Program as provided for by Article XXXIII.
\item
  \begin{enumerate}
  \def\labelenumii{(\arabic{enumii})}
  \tightlist
  \item
    The sum of the Compensation paid to each player with three (3) or more Years of Service who signs a one-year, 10-Day or Rest-of-Season Contract for the Minimum Player Salary during a Season, less, for each such player, the Minimum Player Salary for a player with two (2) Years of Service.
  \item
    The Compensation paid to any player with three (3) or more Years of Service who signs a one-year, 10-Day or Rest-of-Season Contract for the Minimum Player Salary in excess of the Minimum Player Salary for a player with two (2) Years of Service shall be paid by the player's Team pursuant to the terms of such player's Uniform Player Contract, and then reimbursed to the Team out of a League-wide fund created and maintained by the NBA. Such reimbursement shall be made at the conclusion of the Season covered by the Contract.
  \end{enumerate}
\item
  The benefits funded by the New Benefit Amount set forth in Section 7 below.
\end{enumerate}

\hypertarget{insurance-carriers.}{%
\section{Insurance Carriers.}\label{insurance-carriers.}}

At any time during the term of this Agreement, the NBA may change the carrier of any of the foregoing insurance programs, subject to the Players Association's prior written approval, which approval shall not be unreasonably withheld. In no event shall any change in insurance carrier result in a change in the types or levels of any of the benefits provided for above, except as otherwise requested by the Players Association under Section 5(d) above. In the event that a type of or level of benefit is not commercially available, the NBA may substitute a type of or level of benefit of comparable value, subject to the Players Association's approval, which approval shall not be unreasonably withheld.

\hypertarget{new-benefits-funding.}{%
\section{New Benefits Funding.}\label{new-benefits-funding.}}

\begin{enumerate}
\def\labelenumi{(\alph{enumi})}
\item
  \begin{enumerate}
  \def\labelenumii{(\arabic{enumii})}
  \tightlist
  \item
    For each Salary Cap Year during the term of this Agreement, an aggregate amount (the ``New Benefit Amount'') equal to \$1.1 million multiplied by the number of Teams in the NBA during the Covered Season shall be provided by the Teams to fund the benefits described in Section 7(b) below, unless the Players Association designates a lesser amount with respect to a Salary Cap Year, by notice in writing to the NBA delivered on or before the March 15 prior to the commencement of such Salary Cap Year.
  \item
    Notwithstanding subsection (a)(1) above, the New Benefit Amount for each Salary Cap Year shall be subject to reduction pursuant to Article VII, Section 12(b)(1). For purposes of all calculations called for under the CBA of, or relating to Benefits (including, but not limited to, for purposes of (i) preparing the Audit Report, Interim Audit Report, or Interim Escrow Audit Report, and calculating Total Benefits, Total Salaries and Benefits, and Projected Benefits), the amount to be included with respect to the New Benefit Amount shall be the full New Benefit Amount specified in Section 7(a)(1) above and not the reduced New Benefit Amount provided for under Article VII, Section 12(b)(1).
  \end{enumerate}
\item
  Subject to Section 7(c) below, the New Benefit Amount, after taking into account the reduction provided for in Section 7(a)(2) above, shall be utilized in the following manner for each Salary Cap Year:

  \begin{enumerate}
  \def\labelenumii{(\arabic{enumii})}
  \tightlist
  \item
    The New Benefit Amount shall first be utilized, to the extent necessary, to fund any pension contribution increases described in Article IV, Section 1(b)(4)(vi) and to pay the costs described in Article IV, Section 1(b)(5). If the New Benefit Amount is insufficient for these purposes, the shortfall, over as short a period of time as is reasonably possible, shall be offset against: (i) the amounts owed to the Players Association pursuant to the Group License Agreement; (ii) the New Benefit Amount for the next Salary Cap Year; and/or (iii) the NBA's obligation to provide Benefits (other than Benefits funded via the New Benefit Amount) under this Article IV. The determination of the allocation of and type(s) of offset(s) to be applied (as among (i), (ii) and/or (iii) above) shall be made by the Players Association, subject to the NBA's consent, which shall not be unreasonably withheld.
  \item
    Subject to the provisions of Section 2 above, and after taking into account the expenditure described in Section 7(b)(1) above, the remainder of the New Benefit Amount, if any, shall be utilized, to the extent necessary, to fund the cost of matching contributions with respect to players under the 401(k) Plan (and, if applicable, to fund the cost of any alternative arrangement described in Sections 2(b) and (c) above).
  \item
    After taking into account the expenditures described in Section 7(b)(1) and (2) above, the remainder of the New Benefit Amount, if any, shall be utilized, to the extent necessary, to fund \$582,000 of the cost of the life insurance and accidental death and dismemberment benefits described in Section 5(a) above.
  \item
    After taking into account the expenditures described in Section 7(b)(1) -- (3) above, the remainder of the New Benefit Amount, if any, shall be utilized, to the extent necessary, to fund an incremental cost of changes in the medical and dental benefits made pursuant to a Player Change in accordance with provisions of Section 5(d) above.
  \item
    After taking into account the expenditures described in Section 7(b)(1) -- (4) above, the remainder of the New Benefit Amount, if any, shall be utilized, to the extent necessary, to fund the education and career counseling programs to be operated and administered by the Education Trust (or any programs that, pursuant to Section 4(f) above, are substituted for such education and career counseling programs) described in Section 4 above and to pay the costs described in Section 4(c) above.
  \item
    After taking into account the expenditures described in Section 7(b)(1) -- (5) above, the remainder of the New Benefit Amount, if any, shall be utilized to fund the Supplemental Benefit Plan (and, if applicable, to fund the cost of any alternative arrangement described in Sections 3(f) and (g) above) in accordance with the provisions of Section 3 above and to pay the costs described in Section 3(c) above.
  \end{enumerate}
\item
  Notwithstanding anything to the contrary in this Article IV:

  \begin{enumerate}
  \def\labelenumii{(\arabic{enumii})}
  \tightlist
  \item
    In no event shall the Teams (or the NBA) pay amounts for any Salary Cap Year with respect to the benefits described in Section 7(b) above in excess of the New Benefit Amount for such Salary Cap Year.
  \item
    Until the final Audit Report (or, if applicable, the Interim Escrow Audit Report) for a Salary Cap Year is completed, the NBA shall not be required to spend, or commit to spend, any portion of the New Benefit Amount for such Salary Cap Year.
  \end{enumerate}
\end{enumerate}

\hypertarget{projected-benefits.}{%
\section{Projected Benefits.}\label{projected-benefits.}}

\begin{enumerate}
\def\labelenumi{(\alph{enumi})}
\tightlist
\item
  For purposes of computing the Tax Level, Salary Cap and Minimum Team Salary in accordance with Article VII, ``Projected Benefits'' shall mean the projected amounts, as estimated by the NBA in good faith, to be paid or accrued by the NBA or the Teams, other than Expansion Teams during their first two Salary Cap Years, for the upcoming Salary Cap Year with respect to the benefits to be provided for such Salary Cap Year. In the event that the amount of any benefit for the upcoming Salary Cap Year is not reasonably calculable, then, for purposes of computing Projected Benefits, such amount shall be projected to be 104.5\% of the amount attributable to the same benefit for the prior Salary Cap Year.
\item
  For purposes of computing Projected Benefits, the amount to be included with respect to players with three (3) or more Years of Service who receive the Minimum Player Salary shall be the same amount included in Benefits with respect to such players for the immediately preceding Season, except that with respect to the 2005-06 Salary Cap Year, the amount to be included with respect to such players shall be \$7,868,066.
\item
  For purposes of computing Projected Benefits with respect to a Salary Cap Year, there shall be taken into account any reduction in the New Benefit Amount with respect to a Salary Cap Year as designated by the Players Association, by notice in writing to the NBA delivered on or before the March 15 immediately preceding the commencement of such Salary Cap Year.
\item
  Projected Benefits for the 2005-06 Salary Cap Year shall be deemed to be \$112 million.
\end{enumerate}

\hypertarget{compensation-and-expenses-in-connection-with-military-duty}{%
\chapter{COMPENSATION AND EXPENSES IN CONNECTION WITH MILITARY DUTY}\label{compensation-and-expenses-in-connection-with-military-duty}}

\chaptermark{COMPENSATION AND EXPENSES \ldots}

\hypertarget{salary.}{%
\section{Salary.}\label{salary.}}

A player drafted into military service during the Season, or a player serving on active duty with a reserve unit during the Season, shall be compensated for so long as the player remains on the Active or Inactive List of the Team in such amount as may be negotiated between the player and the Team by which he is employed, subject to the provisions of this Agreement.

\hypertarget{travel-expenses.}{%
\section{Travel Expenses.}\label{travel-expenses.}}

\begin{enumerate}
\def\labelenumi{(\alph{enumi})}
\tightlist
\item
  A player serving on military weekend duty with a reserve unit during the Season shall be entitled to reimbursement for any net out-of-pocket expenses incurred by such player in traveling to and from his place of duty to enable him to join his Team for purposes of participating in a Regular Season game.
\item
  In the event that the Player Contract of a player who is required to serve on military weekend duty with a reserve unit is assigned to another Team, the player shall be entitled to reimbursement for any out-of-pocket expenses incurred by such player in traveling during the off-season to and from his home and his place of military weekend duty with a reserve unit; provided that (i) the player makes reasonable efforts to change his reserve unit location to one located reasonably close to his home, and (ii) such obligation to reimburse the player shall cease six (6) months from the date that such player's Contract is assigned.
\end{enumerate}

\hypertarget{player-conduct}{%
\chapter{PLAYER CONDUCT}\label{player-conduct}}

\hypertarget{general.-1}{%
\section{General.}\label{general.-1}}

In addition to any other rights a Team or the NBA may have by contract (including but not limited to the rights set forth in paragraphs 9 and 16 of the Uniform Player Contract) or by law, when a player fails or refuses, without proper and reasonable cause or excuse, to render the services required by a Player Contract or this Agreement, or when a player is, for proper cause, suspended by his Team or the NBA in accordance with the terms of such Contract or this Agreement, the Current Base Compensation payable to the player for the year of the Contract during which such refusal or failure and/or suspension occurs may be reduced (or, in the case of a suspension, shall be reduced) by 1/110th of the player's Base Compensation for each missed Exhibition, Regular Season or Playoff game.

\hypertarget{practices.}{%
\section{Practices.}\label{practices.}}

\begin{enumerate}
\def\labelenumi{(\alph{enumi})}
\tightlist
\item
  When a player, without proper and reasonable excuse, fails to attend a practice session scheduled by his Team, he shall be subject to the following discipline: (i) for the first missed practice during a Season -- \$2,500; (ii) for the second missed practice during such Season -- \$5,000; (iii) for the third missed practice during such Season -- \$7,500; and (iv) for the fourth (or any additional) missed practice during such Season -- such discipline as is reasonable under
  the circumstances.
\item
  Notwithstanding Section 2(a) above, when a player, without proper and reasonable excuse, refuses or intentionally fails to attend any practice session scheduled by his Team, he shall be subject to such discipline as is reasonable under the circumstances.
\end{enumerate}

\hypertarget{promotional-appearances.}{%
\section{Promotional Appearances.}\label{promotional-appearances.}}

When a player, without proper and reasonable excuse, fails or refuses to attend a promotional appearance required by and in accordance with Article II, Section 8 and Paragraph 13(d) of the Uniform Player Contract, he shall be fined \$20,000.

\hypertarget{mandatory-programs.}{%
\section{Mandatory Programs.}\label{mandatory-programs.}}

\begin{enumerate}
\def\labelenumi{(\alph{enumi})}
\tightlist
\item
  NBA players shall be required to attend and participate in educational and life skills programs designated as ``mandatory programs'' by the NBA and the Players Association. Such ``mandatory programs,'' which shall be jointly administered by the NBA and the Players Association, shall include a Rookie Transition Program (for rookies only), Team Awareness Meetings (which shall cover, among other things, substance abuse awareness, HIV awareness, and gambling awareness), and such other programs as the NBA and the Players Association shall jointly designate as mandatory.
\item
  When a player, without proper and reasonable excuse, fails or refuses to attend a ``mandatory program,'' he shall be fined \$20,000 by the NBA; provided, however, that if the player misses the Rookie Transition Program, he shall be suspended for five (5) games.
\end{enumerate}

\hypertarget{media-training-and-business-of-basketball.}{%
\section{Media Training and Business of Basketball.}\label{media-training-and-business-of-basketball.}}

\begin{enumerate}
\def\labelenumi{(\alph{enumi})}
\tightlist
\item
  All players shall be required each Season to attend and participate in one media training session conducted by their Teams and/or the NBA. If a player, without proper and reasonable excuse, fails or refuses to attend a media training session, he shall be fined \$20,000.
\item
  All players shall be required to attend and participate each Season in one ``business of basketball'' program conducted by their Teams and/or the NBA. If a player, without proper and reasonable excuse, fails or refuses to attend such program, he shall be fined \$5,000.
\end{enumerate}

\hypertarget{charitable-contributions.}{%
\section{Charitable Contributions.}\label{charitable-contributions.}}

\begin{enumerate}
\def\labelenumi{(\alph{enumi})}
\tightlist
\item
  In the event that (i) a fine or suspension is imposed on a player, (ii) such fine or suspension-related Compensation amount is collected by the League, and (iii) the fine or suspension is not grieved pursuant to Article XXXI, then the NBA shall remit fifty percent (50\%) of the amount collected to the National Basketball Players Association Foundation (the ``NBPA Foundation'') or such other charitable organization selected by the Players Association that qualifies for treatment under Section 501(c)(3) of the Internal Revenue Code of 1986, as now in effect or as it may hereafter be amended (a ``Section 501(c)(3) Organization''), and that is approved by the NBA (which approval shall not be unreasonably withheld) (both hereinafter, the ``NBPA-Selected Charitable Organization''). The NBA shall remit the remaining fifty percent (50\%) of the amount collected to a Section 501(c)(3) organization selected by the NBA and approved by the Players Association, which approval shall not be unreasonably withheld. For purposes of this Section 6(a), and with respect to any suspension imposed on a player by the NBA of five (5) games or more, the NBA shall be required to collect a suspension-related Compensation amount equal to at least five (5) games of such suspension.
\item
  The remittances made by the NBA pursuant to this Section 6 shall be made annually, thirty (30) days following the Accountants' (as defined in Article VII, Section 10(a)) submission to the NBA and the Players Association of a final Audit Report or an Interim Escrow Audit Report (as defined in Article VII, Section 10(a)) for the Salary Cap Year covering the Season during which the fines and suspension-related Compensation amounts are collected by the NBA. For purposes of this Article and all other provisions of this Agreement, any money remitted or paid to the National Basketball Players Association Foundation by the NBA shall be used for charitable purposes only, and not, for example, for any salaries of Foundation employees or administrative expenses.
\item
  If a timely Grievance is filed under Article XXXI challenging a fine or suspension of the kind designated in Section 6(a) above, and, following the disposition of the Grievance, the Grievance Arbitrator determines that all or part of the fine or suspension-related amount (plus any accrued interest thereon) is payable by the player to the League, then the League shall remit the amount collected by the League (plus any interest) in accordance with the provisions of Sections 6(a) and (b) above.
\end{enumerate}

\hypertarget{unlawful-violence.}{%
\section{Unlawful Violence.}\label{unlawful-violence.}}

When a player is convicted of (including a plea of guilty, no contest, or nolo contendere to) a violent felony, he shall immediately be suspended by the NBA for a minimum of ten (10) games.

\hypertarget{counseling-for-violent-misconduct.}{%
\section{Counseling for Violent Misconduct.}\label{counseling-for-violent-misconduct.}}

\begin{enumerate}
\def\labelenumi{(\alph{enumi})}
\tightlist
\item
  In addition to any other rights a Team or the NBA may have by contract or law, when the NBA and the Players Association agree that there is reasonable cause to believe that a player has engaged in any type of off-court violent conduct, the player will (if the NBA and the Players Association so agree) be required to undergo a clinical evaluation by a neutral expert and, if deemed necessary by such expert, appropriate counseling, with such evaluation and counseling program to be developed and supervised by the NBA and the Players Association.For purposes of this paragraph, ``violent conduct'' shall include, but not be limited to, sexual assault and acts of domestic violence.
\item
  Any player who, after being notified in writing by the NBA that he is required to undergo the clinical evaluation and/or counseling program authorized by Section 8(a) above, refuses or fails, without a reasonable explanation, to attend or participate in such evaluation and counseling program within seventy-two (72) hours following such notice, shall be fined by the NBA in the amount of \$10,000 for each day following such seventy-two (72) hours that the player refuses or fails to participate in such program.
\end{enumerate}

\hypertarget{firearms.}{%
\section{Firearms.}\label{firearms.}}

\begin{enumerate}
\def\labelenumi{(\alph{enumi})}
\tightlist
\item
  Whenever a player is physically present at a facility or venue owned, operated, or being used by a Team, the NBA, or any League-related entity, and whenever a player is traveling on any NBA-related business, whether on behalf of the player's Team, the NBA, or any League-related entity, such player shall not possess a firearm of any kind. For purposes of the foregoing, ``a facility or venue'' includes, but is not limited to: an arena; a practice facility; a Team or League office or facility; an All-Star or NBA Playoff venue; and the site of a promotional or charitable appearance.
\item
  Any violation of Section 9(a) above shall be considered conduct prejudicial to the NBA under Article 35(d) of the NBA Constitution and By-Laws, and shall therefore subject the player to discipline by the NBA in accordance with such Article.
\end{enumerate}

\hypertarget{one-penalty.}{%
\section{One Penalty.}\label{one-penalty.}}

\begin{enumerate}
\def\labelenumi{(\alph{enumi})}
\tightlist
\item
  The NBA and a Team shall not discipline a player for the same act or conduct. The NBA's disciplinary action will preclude or supersede disciplinary action by any Team for the same act or conduct.
\item
  Notwithstanding anything to the contrary contained in Section 10(a), (i) the same act or conduct by a player may result in both a termination of the player's Uniform Player Contract by his Team and the suspension of the player by the NBA if the egregious nature of the act or conduct is so lacking in justification as to warrant such double penalty, and (ii) both the NBA and the Team to which a player is traded may impose discipline for a player's failure to report for a trade in accordance with paragraph 10(d) of the Uniform Player Contract.
\end{enumerate}

\hypertarget{league-investigations.}{%
\section{League Investigations.}\label{league-investigations.}}

\begin{enumerate}
\def\labelenumi{(\alph{enumi})}
\tightlist
\item
  Players are required to cooperate with investigations of alleged player misconduct conducted by the NBA. Failure to so cooperate, in the absence of a reasonable apprehension of criminal prosecution, will subject the player to reasonable fines and/or suspensions imposed by
  the NBA.
\item
  Except as set forth in Section 11(c) below, the NBA shall provide the Players Association with such advance notice as is reasonable in the circumstances of any interview or meeting to be held (in person or by telephone) between an NBA representative and a player under investigation by the NBA for alleged misconduct, and shall invite a representative of the Players Association to participate or attend. The failure or inability of a Players Association representative to participate in or attend the interview or meeting, however, shall not prevent the interview or meeting from proceeding as scheduled. A willful disregard by the NBA of its obligation to notify the Players Association as provided for by this Section 11(b) shall bar the NBA from using as evidence against the player in a proceeding involving such alleged misconduct any statements made by the player in the interview or meeting conducted by the NBA representative.
\item
  The provisions of Section 11(b) above shall not apply to interviews or meetings: (i) held by the NBA as part of an investigation with respect to alleged player misconduct that occurred at the site of a game; and (ii) which take place during the course of, or immediately preceding or following, such game. With respect to any such interview or meeting, the NBA's only obligation shall be to provide notice to the Players Association that the NBA will be conducting an investigation and holding an interview or meeting in connection therewith. Such notice may be given by telephone at a telephone number, pager number or message-recording number to be designated in writing by the Players Association.
\end{enumerate}

\hypertarget{on-court-conduct.}{%
\section{On-Court Conduct.}\label{on-court-conduct.}}

In addition to its authority under paragraph 5 of the Uniform Player Contract, the NBA is entitled to promulgate and enforce reasonable rules governing the conduct of players on the playing court (as that term is defined in Article XXXI, Section 8(c)) that do not violate the provisions of this Agreement. Prior to the date on which any new rule promulgated by the NBA becomes effective, the NBA shall provide notice of such new rule to the Players Association and consult with the Players Association with respect thereto.

\hypertarget{motor-vehicles.}{%
\section{Motor Vehicles.}\label{motor-vehicles.}}

At the commencement of each Season, and if the player owns or operates any motor vehicle, the player will provide the Team with proof that the player possesses a valid driver's license, registration documents, and insurance for any such vehicle.

\hypertarget{basketball-related-income-salary-cap-minimum-team-salary-and-escrow-arrangement}{%
\chapter{BASKETBALL RELATED INCOME, SALARY CAP, MINIMUM TEAM SALARY, AND ESCROW ARRANGEMENT}\label{basketball-related-income-salary-cap-minimum-team-salary-and-escrow-arrangement}}

\chaptermark{BASKETBALL RELATED INCOME \ldots}

\hypertarget{definitions.-1}{%
\section{Definitions.}\label{definitions.-1}}

For purposes of this Agreement, the following terms shall have the meanings set forth below:

\begin{enumerate}
\def\labelenumi{(\alph{enumi})}
\tightlist
\item
  \textbf{Basketball Related Income.}

  \begin{enumerate}
  \def\labelenumii{(\arabic{enumii})}
  \item
    ``Basketball Related Income'' (``BRI'') for a Salary Cap Year means the aggregate operating revenues (including the value of any property or services received in any barter transactions), accounted for in accordance with Section 1(b)(1) below, received or to be received for or with respect to such Salary Cap Year by the NBA, NBA Properties, Inc., including any of its subsidiaries whether now in existence or created in the future (hereinafter, ``Properties''), NBA Media Ventures LLC (``Media Ventures''), any other entity which is controlled by, or in which the NBA, Properties, Media Ventures, and/or a group of NBA Teams owns at least 50\% of the issued and outstanding ownership interests (hereinafter, ``League-related entity'') (but excluding the amount of such League-related entity's revenues equal to the portion of its total revenues that is proportionate to the share of the entity's profits to which ownership interests not owned by the NBA, Properties, Media Ventures and/or a group of NBA Teams are entitled), all NBA Teams other than Expansion Teams during their first two (2) Salary Cap Years (but including the Expansion Teams' shares of national television, radio, cable and other broadcast revenues, and any other League-wide revenues shared by the Expansion Teams, provided such revenues are otherwise included in BRI) and Related Parties (in accordance with Section 1(a)(7)(i) below), from all sources, whether known or unknown, whether now in existence or created in the future, to the extent derived from, relating to, or arising directly or indirectly out of, the performance of Players in NBA basketball games or in NBA-related activities. For purposes of this definition of BRI: (x) ``operating revenues'' shall include, but not be limited to, any type of revenue included in BRI for the 1995-96 and 1996-97 Salary Cap Years (without regard to whether such type of revenue is received on a lump-sum, non-recurring or extraordinary basis, but subject to any specific rules set forth in this Article VII relating to the recognition or amortization of such amounts); and (y) ``Player'' means a person: who is under a Player Contract to an NBA Team; who completed the playing services called for under a Player Contract with an NBA Team at the conclusion of the prior Season; or who was under a Player Contract with an NBA Team during (but not at the conclusion of) the prior Season, but only with respect to the period for which he was under such Contract. Subject to the foregoing, BRI shall include, but not be limited to, the following revenues:

    \begin{enumerate}
    \def\labelenumiii{(\roman{enumiii})}
    \tightlist
    \item
      Regular Season gate receipts, net of applicable taxes, surcharges, imposts, and other charges (including, without limitation, charges related to arena financings) imposed by governmental or quasi-governmental agencies other than income taxes (collectively, ``Taxes''), including, without limitation, gate receipts received or to be received by a Related Party in accordance with Section 1(a)(7)(i) below, including: (A) the value (determined on the basis of the price of the ticket) of all tickets traded by a Team for goods or services; and (B) the value (determined on the basis of the League-wide average ticket price for non-Season tickets) of all tickets for Regular Season games provided by a Team on a complimentary basis, without monetary or other compensation to a Team;provided, however, that (x) the value of the ``Excluded Complimentary Tickets'' with respect to all Regular Season games in a Season shall be excluded from BRI, and (y) in addition, tickets provided as part of sponsorships and other transactions, where the proceeds from such transactions have been included in BRI, shall not be included in determining the number of complimentary tickets in any Season. For purposes of the foregoing, ``Excluded Complimentary Tickets'' shall mean: 1.35 million tickets for the 2005-06 Season; 1.40 million tickets for the 2006-07 Season; 1.45 million tickets for the 2007-08 Season; 1.50 million tickets for the 2008-09 Season; 1.55 million tickets for the 2009-10 Season; 1.6 million tickets for the 2010-2011 Season; and, in the event the NBA exercises its option to extend the Agreement pursuant to Article XXXIX of this Agreement, 1.6 million tickets for the 2011-2012 Season;
    \item
      all proceeds of any kind, net of reasonable and customary expenses related thereto, from the broadcast or exhibition of, or the sale, license or other conveyance or exploitation of the right to broadcast or exhibit, NBA preseason, Regular Season and Playoff games and summer league and other NBA-related off-season games involving Players, highlights or portions of such games, and non-game NBA programming, on any and all forms of radio, television, telephone, internet, and any other communications media, forms of reproduction and other technologies, whether presently existing or not, anywhere in the world, whether live or on any form of delay, including, without limitation, network, local, cable, direct broadcast satellite and any form of pay television, and all other means of distribution and exploitation, whether presently existing or not and whether now known or hereafter developed, including, without limitation, such proceeds received or to be received by a Related Party (in accordance with Section 1(a)(7)(i) below), but not including the value of any broadcast, cablecast or telecast time provided as part of any such transaction that is used solely: (A) to promote or advertise the NBA, its Teams, Players, the NBA Development League (the ``NBADL'') (except to the extent the value of such time exceeds \$5 million), or the sport of basketball (but not the value of time used to promote or advertise the Women's National Basketball Association (the ``WNBA'') which shall be included in BRI); (B) to promote or advertise products, programming, merchandise, services or events that produce revenues that are includable in BRI or are receivable by Properties pursuant to the Group License Agreement (as defined in Article XXXVII, Section 1; (C) to promote or advertise charitable, not-for-profit or governmental organizations or agencies; or (D) for public service announcements;
    \item
      all Exhibition game proceeds of any kind, net of Taxes and all reasonable and customary game, pre-season and training camp expenses, including, without limitation, such proceeds received or to be received by a Related Party (in accordance with Section 1(a)(7)(i) below);
    \item
      all playoff gate receipts of any kind, net of Taxes, arena rentals to the extent reasonable and customary, and all other reasonable and customary expenses, except the player playoff pool, including, without limitation, such proceeds received or to be received by a Related Party (in accordance with Section 1(a)(7)(i) below);
    \item
      all proceeds of any kind, net of reasonable and customary expenses (including Taxes) related thereto, subject to the provisions of Section 1(a)(6) below, from in-arena sales of novelties and concessions, sales of novelties in team-identified stores located within such radius of the Team's home arena as is permitted by the NBA, NBA game parking and programs, Team sponsorships (whether or not the proceeds are directly or indirectly donated to charity), Team promotions, temporary arena signage, arena club revenues, summer camps, non-NBA basketball tournaments, mascot and dance team appearances, the sale of the right to pour beverages or (except as provided in Section 1(a)(2)(xx) below) to provide concessions, in each case, to the extent that such proceeds are related to the performance of Players in NBA basketball games or NBA-related activities, including, without limitation, such proceeds received or to be received by a Related Party (in accordance with Section 1(a)(7)(i) below);
    \item
      forty (40) percent of the gross proceeds, net of Taxes, from the sale of fixed arena signage within or outside of the arena in which an NBA Team plays more than one-half of its Regular Season home games, including, without limitation, such proceeds received or to be received by a Related Party (in accordance with Section 1(a)(7)(i) below);
    \item
      forty (40) percent of the gross proceeds of any kind, net of Taxes, from the sale, lease or licensing of luxury suites calculated on the basis of the actual proceeds received by the entity, including, without limitation, proceeds received or to be received by a Related Party (in accordance with Section 1(a)(7)(i) below), that sold, leased, or licensed such luxury suites; provided, however, that, other than the additional amounts paid by luxury suite holders to the Team for tickets pursuant to arrangements in which admission to games is not part of the agreement to buy, lease or license the luxury suite, thereby requiring the luxury suiteholder to make a separate payment for such admission, if any, this amount shall be the only amount included in BRI for the sale, lease or licensing of luxury suites and that, to the extent that the sale, lease or licensing of the luxury suite grants rights to the luxury suite for a period of more than one (1) year, for purposes of calculating the amount includable in BRI for any Salary Cap Year, the proceeds shall be determined on the basis of the annual fee or charge provided for in any such transaction and, if payments are made in addition to or in the absence of such an annual fee or charge, the value of such payments shall be amortized over the period of the sale, lease or license, unless such period exceeds twenty (20) years, in which event an amortization period of twenty (20) years shall be used;
    \item
      forty-five (45) percent for the 2005-06 through 2007-08 Seasons, and fifty (50) percent for the 2008-09 through 2010-11 Seasons (and, in the event that the NBA exercises its option to extend the Agreement pursuant to Article XXXIX, fifty (50) percent for the 2011-12 Season) of the gross proceeds, net of Taxes, from arena naming rights agreements with respect to arenas in which an NBA Team plays more than one-half of its Regular Season home games, including, without limitation, such proceeds received or to be received by a Related Party (in accordance with Section 1(a)(7)(i) below);
    \item
      except as provided in Section 1(a)(2) below, proceeds received by Properties, net of Taxes and actual expenses that are directly attributable to the generation of such proceeds, as long as those expenses are consistent with the types and categories of expenses incurred by Properties as reflected in the audited financial reports for Properties for the year ended July 31, 1994 (or, in the case of new sources of proceeds or new types of expenses, as long as the expenses are reasonable and customary in the opinion of the Accountants (as defined in Section 10(a) below), subject to the provisions of Section 1(a)(6) below), including proceeds derived from the following categories (defined in the same manner as was used in those audited financial reports): (A) international television; (B) sponsorships; (C) NBA-related revenues from NBA Entertainment; (D) the All-Star Game; (E) other NBA special events; and (F) all other sources of revenue received by Properties other than those specifically excluded under Section 1(a)(2) below;
    \item
      proceeds from premium seat licenses (other than licenses of luxury suites, which are governed by Section 1(a)(1)(vii) above), net of Taxes, attributable to NBA-related events amortized over the period of the license (including, without limitation, such proceeds received or to be received by a Related Party (in accordance with Section 1(a)(7)(i) below), unless such period exceeds twenty (20) years, in which event an amortization period of twenty years shall be used; and
    \item
      if the right to receive revenues included in BRI is sold or transferred to an entity other than an entity referred to in Section 1(a)(1) above (such that those revenues would not be included in BRI pursuant to that subsection), then BRI shall be deemed to include the amount of revenues that would have been received by the seller or transferor and would have been included in BRI in such Salary Cap Year (subject to any applicable allocations provided for above), absent such sale or transfer, provided that a pledge, hypothecation, collateral assignment or other similar transaction involving such revenues shall not be considered a sale or transfer within the meaning of this Section 1(a)(1)(xi).
    \end{enumerate}
  \item
    Notwithstanding anything to the contrary in Section (a)(1) above, it is understood that the following is a non-exclusive list of examples of revenues that are or may be received by the NBA, Properties, Media Ventures, other League-related entities, NBA Teams and Related Parties (the foregoing persons or entities, beginning with ``NBA,'' collectively referred to in this Section 1(a)(2) only as ``NBA-related entities'') that are not derived from, and do not relate to or arise out of, the performance of Players in NBA basketball games or in NBA-related activities or are otherwise expressly excluded from the definition of BRI:

    \begin{enumerate}
    \def\labelenumiii{(\roman{enumiii})}
    \tightlist
    \item
      needs from the assignment of Player Contracts;
    \item
      proceeds (A) from the sale, transfer or other disposition of any of the assets or property (excluding ordinary course sales of inventory and the revenues (if any) deemed to be included in BRI pursuant to Section 1(a)(1)(xi) above) of, or ownership interests in, any NBA-related entity, or (B) from loans or other financing transactions;
    \item
      proceeds from the grant of Expansion Teams;
    \item
      dues;
    \item
      capital contributions received by an NBA-related entity from one of its owners, shareholders, members or partners;
    \item
      fines and compensation withheld in connection with suspensions;
    \item
      revenue sharing (by means of revenue transfers or otherwise) among Teams;
    \item
      interest income;
    \item
      insurance recoveries, except where, and only to the extent that, such recoveries are in respect of lost revenues that would have otherwise been included in BRI, in which event such recoveries shall be included in BRI in the Salary Cap Year in which they are received;
    \item
      proceeds from the sale or rental of real estate;
    \item
      any thing of value received in connection with the design or construction of a new or renovated arena or other team facility including, but not limited to, receipt of title to or a leasehold interest in real property or improvements, reimbursement of project-related expenses, benefits from project-related infrastructure improvements, or tax abatements, unless (and only to the extent that) such value is being provided to the Team or a Related Party in lieu of payments that the Team or Related Party would have otherwise received pursuant to an arena lease or other instrument concerning a Team's use of an arena (``lease'') and would have constituted BRI if paid to the Team or a Related Party; provided, however, that the determination of the amount, if any, to be included in BRI with respect to the value of any of the foregoing shall be made either (A) in accordance with the provisions of Section 1(a)(4) below or (B) based upon direct evidence that the Team or Related Party, after proposing that it would receive certain revenues constituting arena-generated BRI, subsequently agreed specifically to forego such revenues indirect exchange for a thing of value (as described above in this Section 1(a)(2)(xi)) with the consequence that the arena-generated BRI revenues received or to be received by the Team or Related Party were or would be (in the opinion of the Accountants) less than the fair market value of arena-generated BRI revenues received or to be received by other NBA Teams in similar transactions, or (C) based upon direct evidence that the parties to the transaction had agreed that certain revenues constituting arena-generated BRI would be paid to the Team or Related Party and that such revenues were subsequently foregone by the Team or the Related Party in direct exchange for a thing of value (as described above in this Section 1(a)(2)(xi)); and provided further that, when a determination is made pursuant to clause (B) or clause (C) of this Section 1(a)(2), the amount(s), if any, to be included in BRI shall be allocated (with an appropriate interest adjustment to reflect the time value of money where the thing of value received by the Team or Related Party is in the form of cash or a cash equivalent, such as a check or wire transfer) over the Salary Cap Years in which the arena-generated BRI revenues foregone would have been received by the Team or Related Party (up to a maximum of twenty (20) Salary Cap Years) and not on a lump-sum basis;
    \item
      any thing of value that induces or is intended to induce a Team either to relocate to or remain in a particular geographic location, unless (and only to the extent that) such value is being provided to the Team or a Related Party in lieu of payments that the Team or Related Party would have otherwise received pursuant to an arena lease and that would have constituted BRI had they been paid to the Team or a Related Party; provided, however, that the determination of the amount, if any, to be included in BRI shall be made either (A) in accordance with the provisions of Section 1(a)(4) below or (B) based upon direct evidence that the parties to the transaction had agreed that certain revenues constituting arena-generated BRI would be foregone by the Team or Related Party, in direct exchange for a thing of value as described above in this Section 1(a)(2)(xii), and provided, further that, when a determination is made pursuant to clause (B) of this Section 1(a)(2)(xii), the amount(s), if any, to be included in BRI shall be allocated (with an appropriate interest adjustment to reflect the time value of money where the thing of value received by the Team or Related Party is in the form of cash or a cash equivalent, such as a check or wire transfer) over the Salary Cap Years in which the arena-generated BRI revenues foregone would have been received by the Team or Related Party (up to a maximum of fifteen (15) Salary Cap Years) and not on a lump-sum basis;
    \item
      payments made to Teams or to the NBA pursuant to the provisions of Article VII, Section 12 (Escrow/Tax Arrangement) below;
    \item
      distributions, dividends or royalties paid by any NBA-related entity to owners, shareholders, members or partners;
    \item
      any category or source of revenue or proceeds that was expressly identified in any BRI Report (as defined in Section 10(b) below) or in any document or written communication (including debriefing memos) authored by the Accountants and provided to the Players Association and the NBA (but excluding any underlying work papers) in connection with the Audit Reports for any of the 1995-96 through 2004-05 Salary Cap Years that was not included in BRI for such Salary Cap Years, unless such category or source was included on the ``open issues'' list prepared by the Accountants in connection with any of the Audit Reports for the 1999-2000 through 2004-05 Salary Cap Years, in which case such category or source shall be included in or excluded from BRI, as the case may be, in accordance with the other terms of this Article;
    \item
      proceeds received by (A) Properties pursuant to the Group License Agreement (including, but not limited to, proceeds received pursuant to the license of ``fantasy games,'' which proceeds are to be included in the computation of Player Merchandise Revenues in accordance with the Group License Agreement), or (B) by a League-related entity relating to the following categories defined in the same manner as was used in the audited financial reports for Properties for the year ended July 31, 1997: (x) licensing; and/or (y) a League-related entity's representation of, and services performed for, third parties. For purposes of the foregoing sentence, ``third parties'' refers to persons or entities that are not owned or controlled by persons or entities that own a majority interest in or otherwise control an NBA Team or, if such third party is a Related Party, proceeds received by the League-related entity shall not be included in BRI if representation of such Related Party does not relate either to such entity's NBA ownership or NBA Players;
    \item
      monies collected from team-related fundraising for charitable purposes or other charitable activities, other than monies paid pursuant to Team sponsorship agreements that are included in BRI pursuant to Section 1(a)(1)(v) above; and
    \item
      proceeds solely related to the NBADL;
    \item
      proceeds from the leasing or use of any Team physical assets (e.g., a Team plane); and
    \item
      any thing of value received from a concessionaire, food service vendor or other third party equipment or service provider that, if received in kind, is installed in an NBA arena or, if received in cash, is directed to defraying the costs of the construction or substantial renovation of an NBA arena.
    \end{enumerate}
  \item
    The parties agree that (i) in determining whether a category or source of revenue or proceeds constitutes BRI: (A) consideration shall be given to whether such category or source is more similar in kind or nature to the included categories and sources listed in Section 1(a)(1)(i) through (xi) above, on the one hand, or to the excluded categories and sources listed in Section 1(a)(2)(i) through (xx) above, on the other; and, (B) no inference may be drawn from the fact that such category or source was not included in the categories and sources listed in Section 1(a)(1)(i) through (x) above, or the fact that such category or source was not included in the categories and sources listed in Sections 1(a)(2)(i) through (xx) above; and (ii) in any proceeding involving a dispute over (A) the includability or categorization of any revenue or expense item for BRI purposes; (B) the amount to be included in or deducted from BRI with respect to any revenue or expense item; or (C) the accounting methodology used by the Accountants in connection with any audit of BRI, the parties may refer to the past practice of the parties or the Accountants in connection with the Audit Reports for any of the 1999-2000 through 2004-05 Salary Cap Years; provided, however, that no reference may be made to the past practice of the parties or the Accountants with respect to any source or category of revenue or expense that was included on the ``open issues'' list prepared by the Accountants in connection with any of such Audit Reports.
  \item
    The parties agree that, with respect to any lease entered into after the date of this Agreement between a Team (or a Related Party) and an arena that is not a Related Party, the Accountants may attribute to the Team (or a Related Party) for purposes of computing BRI for a Salary Cap Year portions of arena revenues received by the arena or its related entities that would be included in BRI if received by the Team (or a Related Party) to the following extent: in the event of a renewal, extension or renegotiation of a lease between the same parties,or a new lease entered into by a Team (or a Related Party) with an arena that is not a Related Party, the Team will be deemed to receive in the first Salary Cap Year covered by the new lease or by the renewal, extension or renegotiation of the existing lease (as the case may be) the greater of (i) the amount of such revenues that the Team or the Related Party in fact receives under the lease or, (ii) if in the opinion of the Accountants, the Team (and/or the Related Party) is receiving substantially less than fair market value as determined by the Accountants (taking into account factors such as the rent paid by the Team or the Related Party, the number and identity of other major tenants in the arena, market conditions, the extent to which arena revenues are used to fund construction or renovations of the arena, and comparable lease arrangements in the NBA), an amount determined by the Accountants to constitute the fair market value of the revenues that a tenant, in the same circumstances as the Team or Related Party, would receive for such Salary Cap Year. In either of the preceding cases, the Accountants will also determine the amount to be included in BRI for Salary Cap Years beyond the first Salary Cap Year.
  \item
    \begin{enumerate}
    \def\labelenumiii{(\roman{enumiii})}
    \tightlist
    \item
      In no event shall the same revenues be included in BRI, directly or indirectly, more than once (including as a result of changes in accounting methods or practices), the purpose of this provision being to preclude the double-counting of revenues, whether in the same or in multiple Salary Cap Years.
    \item
      In no event shall the same expenses be deducted from BRI, directly or indirectly, more than once (including as a result of changes in accounting methods or practices), the purpose of this provision being to preclude the double-counting of expenses, whether in the same or in multiple Salary Cap Years.
    \end{enumerate}
  \item
    Subject to Section 11 below (Players Association Audit Rights):

    \begin{enumerate}
    \def\labelenumiii{(\roman{enumiii})}
    \tightlist
    \item
      With respect to expenses incurred in connection with all proceeds coming within Section 1(a)(1)(v) and (ix) above, all reported expenses shall be conclusively presumed to be reasonable and customary (other than expenses related to sources of revenues that were not reflected in the audited financial report for Properties for the year ended July 31, 1994), and such expenses shall not be the subject of the accounting procedures set forth in Section 10 below. Such expenses shall be disallowed, however, to the extent that they exceed the ratio of League-wide reported expenses to League-wide reported revenues (the ``Expense Ratio'') for that category of revenues set forth in Exhibit D hereto.
    \item
      With respect to the NBA Store (the ``Store'') and any other new venture undertaken by the NBA, Properties, Media Ventures, or any other League-related entity requiring significant capital investment or start-up costs (``New Venture''), reasonable and customary expenses shall include, but not be limited to, cost of goods sold, sales tax, all reasonable operating expenses of the Store or New Venture (including, but not limited to, salaries and benefits directly related to the operations of the Store or New Venture, promotional and advertising costs, rent, direct overhead, general and administrative expenses of the Store or New Venture), reasonable financing costs and amortization of capital improvements and start-up costs; provided, however, that in no event shall the expenses attributable to the Store or New Venture cause the amount included in BRI for the Store or New Venture to be less than zero (0) for any Salary Cap Year.
    \item
      With respect to new categories of revenue that may be included in BRI during the term of this Agreement (other than revenues attributable to the Store or a New Venture), the NBA, Properties, Media Ventures, other League-related entities, NBA Teams and Related Parties shall be able to deduct all expenses that the parties agree (or, in the absence of such agreement, that the Accountants determine) are reasonable and customary, provided, however, that if a new category of revenue is substantially similar to the type of revenues described in Section 1(a)(1)(i) and (iv) above, the expenses attributable to such new category of revenue shall be deductible only to the extent contemplated by such subsections.
    \end{enumerate}
  \item
    It is acknowledged by the parties hereto that for purposes of determining BRI:

    \begin{enumerate}
    \def\labelenumiii{(\roman{enumiii})}
    \tightlist
    \item
      Some NBA Teams have engaged or may engage in transactions with third parties that control, or own at least 50\% of, the NBA Team or that are controlled or owned at least 50\% by the persons or entities controlling or owning at least 50\% of the NBA Team (such third parties are referred to in this Agreement as a ``Related Party''), and Related Parties themselves engage in transactions with third parties that may result in a Related Party's receipt of revenues that constitute BRI. (Any entity that was an ``entity related to an NBA team'' as defined by Article VII, Section 1(a)(4)(i) of the September 18, 1995 Collective Bargaining Agreement between the NBA and the Players Association (the ``1995 CBA'') shall be deemed a Related Party under this Agreement for so long as such entity continues to be an entity related to an NBA Team within the meaning of the 1995 CBA.) As provided in Section 1(a)(1) above, the relevant proceeds received by any Related Party that come within such subsection and that relate to such Related Party's Team shall be included in BRI. However, with respect to any such revenues or proceeds retained or received by a Related Party (other than arena revenues that relate to such Related Party's Team including, but not limited to, in-arena sales of novelties and concessions, NBA game parking, arena club revenues, suite and seat revenues and fixed and temporary in- \textbf{\emph{{[}sic{]}}} subscriber fees earned by a Related Party television network that relate, directly or indirectly, to the telecast of NBA games licensed to the television network by a Team).
    \item
      In the event that, following the execution of this Agreement, a Team (other than the New York Knicks (``Knicks'')) enters into a local or regional telecast agreement with a Related Party, a copy of such agreement shall be provided to the Players Association within ten (10) days of approval of such agreement by the NBA. The Players Association and the NBA shall each have the right, not later than ten (10) days following the date on which the Players Association receives a copy of such agreement, to submit such agreement to a jointly-selected television valuation expert or (in the absence of such agreement) determined in accordance with the procedure set forth in this subsection (``TV Expert'') for the limited purpose set forth in this Section 1(a)(7)(ii). In the event that the parties have not jointly selected a TV Expert within twenty (20) days following the date on which the Players Association receives a copy of such agreement, each party shall appoint its own television valuation designee and the two designees so appointed shall within ten (10) days of their appointment, jointly select a third party to serve as the TV Expert. Such TV Expert shall review such agreement to determine if the aggregate amount to be paid to the Team by the Related Party for the rights to telecast the Team's games pursuant to such agreement is more than 15\% above or more than 15\% below the fair market value of such rights over the term of such agreement. In the event that the TV Expert determines that such aggregate amount is more than 15\% above or below fair market value, the TV Expert shall be instructed to submit to the parties the amount for each Season of such agreement that he determines reflects the fair market value of such rights and such amounts, and no other amounts, shall be included in BRI with respect to such agreement for each Salary Cap Year covered by such agreement. Any determination made by the TV Expert pursuant to either of the preceding two sentences shall be submitted to the parties no later than twenty (20) days from the date on which such agreement was submitted to the TV Expert for his review. Any fees or costs associated with the retention or determination of the TV Expert shall be borne equally by the Players Association and NBA. The Players Association and the TV Expert shall maintain the confidentiality of any such agreement (and any determination made by the TV expert in accordance with this Section 1(a)(7)(ii)) pursuant to the terms of Section 11(c) below relating to confidentiality of BRI Audits.
    \item
      With respect to the transactions listed below in this Section 1(a)(7)(iii), the parties agree that, because the proceeds attributable to these transactions cannot be accurately ascertained, the following procedures shall be used for each NBA Season in which MSG Network is a Related Party of the Knicks (in the case of Section 1(a)(7)(iii)(A) below) and the Madison Square Garden arena is a Related Party of the Knicks (in the case of Section 1(a)(7)(iii)(B) below):

      \begin{enumerate}
      \def\labelenumiv{(\Alph{enumiv})}
      \tightlist
      \item
        New York Knicks transaction with MSG Network regarding the sale of local media rights: BRI for the Knicks for each NBA Season covered by this Agreement shall include an amount equal to the net proceeds included in BRI attributable to the Los Angeles Lakers' sale, license or other conveyance of all local media rights (including, but not limited to, broadcast and cable television and radio) for such NBA season.
      \item
        New York Knicks transactions with Related Parties involving signage: BRI for the Knicks for the 1999-2000 NBA Season shall include \$3,750,000 for signage. In each subsequent Season covered by this Agreement, this amount shall be increased (or decreased, as the case may be) by the League-wide percentage increase (or decrease) in signage as determined in accordance with Section 1(a)(1)(v) and (a)(1)(vi) above.
      \end{enumerate}

      At such time as the MSG Network and/or the Madison Square Garden Arena are no longer Related Parties, BRI for the New York Knicks in the categories described in Section 1(a)(7)(iii)(A) and/or (B) above, as the case may be, shall not be determined in accordance with the foregoing and will instead be determined by the applicable provisions of Section 1(a)(1) and (a)(7)(ii) above.
    \end{enumerate}
  \item
    In the event that, pursuant to the NBA's national broadcast, national telecast and network cable television agreements, NBA Teams receive revenue sharing proceeds that are attributable to NBA game telecasts in more than one Salary Cap Year, such proceeds shall be allocated over the same number of Salary Cap Years (beginning with first Salary Cap Year after the Salary Cap Year in which such proceeds are actually received) as the number of Salary Cap Years in which such games were televised. Any other contingent payments received by the NBA pursuant to such agreements shall be included in BRI to the extent and in a manner agreed upon by the parties, or, if the parties cannot agree, in a reasonable manner determined by the Accountants.
  \item
    The NBA and each NBA Team shall in good faith act and use their best efforts to maximize BRI for each Salary Cap Year during the term of this Agreement. In the exercise of such best efforts, the NBA and each NBA Team shall be entitled to act in a manner consistent with their sound business judgment and shall not take any action intended to benefit, at the expense of BRI, other commercial activities (such as the WNBA and the NBADL) unrelated to the performance of Players in NBA basketball games or in NBA-related activities. Without limiting the generality of the foregoing, the parties agree that it is within the sound business judgment of the NBA and each NBA Team to enter into, terminate or modify commercial arrangements or transactions, in good faith, in response to market exigencies, the acts or needs of unrelated third-party business partners, and/or the best interests of NBA fans.
  \item
    The parties agree that upon a finding by the System Arbitrator (which, if appealed, is affirmed by the Appeals Panel) that the NBA or an NBA Team (or a Related Party) has willfully failed to provide to the Accountants information concerning revenues or expenses material to the Accountants' preparation of an Audit Report, and that such failure to provide information resulted in an understatement of BRI of more than \$2.5 million with respect to the 2005-06 Salary Cap Year (increasing by 4.5\% for each subsequent Salary Cap Year of this Agreement, beginning in the 2006-07 Salary Cap Year), then the amount by which BRI was understated shall be included in BRI in the Salary Cap Year in which such finding is made, with interest accruing from the date of the Audit Report for the Salary Cap Year in which such amount would have been included but for such understatement, with interest (at a rate equal to the one year Treasury Bill rate as published in the Wall Street Journal on the date of the issuance of such Audit Report). In addition, if any Team, or if the NBA, violates the foregoing, it shall be fined \$1 million for its first violation during the term of this Agreement and an additional \$1 million for each additional violation. (For example, if a Team violates the foregoing for the first time, it shall be fined \$1 million; if such Team violates the foregoing a second time, it shall be fined \$2 million; and if such Team violates the foregoing a third time, it shall be fined \$3million.) Fifty percent (50\%) of any such fine amounts shall be remitted by the NBA to an NBPA-Selected Charitable Organization (as defined in Article VI, Section 6 above) and 50\% shall be remitted by the NBA to a Section 501(c)(3) organization selected by the NBA.
  \item
    Neither the NBA or a League-related entity nor a Team or a Related Party will enter into any lease or other agreement providing for the receipt of revenues includable in BRI that contains provisions that purport to limit access of the Accountants to the books and records of the NBA, such League-related entity, such Team, or such Related Party in a manner inconsistent with the terms of this Agreement or that would preclude the calculation of revenues (if any) to be included in BRI pursuant to the provisions of Section 1(a)(1)(xi) above.
  \item
    Premium payments made by a Team for any insurance that, if paid, would be includable in BRI pursuant to Section 1(a)(2)(ix) above, shall be deducted from such Team's BRI for the Salary Cap Year in which any such insurance recovery is received.
  \end{enumerate}
\item
  \textbf{Accounting Methods/Lump Sum Payments.}

  \begin{enumerate}
  \def\labelenumii{(\arabic{enumii})}
  \tightlist
  \item
    Subject to Section 1(b)(2) and (b)(3) below, BRI for each Salary Cap Year shall be calculated exclusively pursuant to the accrual method of financial accounting (and not, for any purpose, the cash method of financial accounting) and in accordance with United States Generally Accepted Accounting Principles. By way of example, and not limitation, in the event a team receives a signing bonus in consideration for its agreement to enter into a five (5) year contract for the local telecast of its games, such signing bonus shall be amortized in equal annual amounts over the five (5) Salary Cap Years covered by such television contract.
  \item
    Except as otherwise provided in the case of luxury suites and premium seat licenses, in no event shall the amortization period for any lump sum payment exceed seven (7) years.
  \item
    Any payments that constitute BRI and that are subject to being repaid to the payor under certain circumstances (the ``Contingencies'') shall constitute BRI in the Salary Cap Year in which such payments would have been earned but for the Contingencies unless, at the time of such payments, the Contingencies under which the payments would be repaid are likely to occur, in which case the payments will not be included in BRI unless and until such time as the Contingencies under which such repayments would be made do not occur or are not likely to occur. In the event that a payment that has been included in BRI is subsequently repaid, BRI shall be reduced by the amount of such repayment in the Salary Cap Year in which such repayment is made. In any proceeding commenced before the System Arbitrator relating to the terms of this Section 1(b)(3), the NBA will bear the burden of demonstrating that the applicable Contingencies are likely to occur.
  \end{enumerate}
\item
  \textbf{``Projected BRI''} for a Salary Cap Year means the sum of amounts determined in accordance with the following:

  \begin{enumerate}
  \def\labelenumii{(\arabic{enumii})}
  \tightlist
  \item
    With respect to BRI sources other than national broadcast, national telecast or network cable television contracts, Projected BRI shall include BRI for the preceding Salary Cap Year, increased by 4.5\%. For purposes of this Section 1(c)(1), a contract between or among any League-related entities and/or Teams shall not be considered national broadcast, national telecast or network cable television contract.
  \item
    With respect to national broadcast, national telecast or network cable television contracts including the NBA/ABC agreement dated January 17, 2002 (``NBA/ABC Agreement'') (a copy of which has been provided to the Players Association) and the NBA/TBS agreement, dated January 18, 2002 (``NBA/TBS Agreement'') (a copy of which has been provided to the Players Association), and national broadcast, national telecast or network cable television contracts covering Seasons that succeed the Seasons covered by the NBA/ABC and NBA/TBS Agreements (``Successor Agreements'') (copies of which shall be provided to the Players Association within ten (10) days of execution), Projected BRI for a Salary Cap Year shall include (i) the rights fees or other non-contingent payments stated in such contracts with respect to the Season covered by such Salary Cap Year (as such rights fees or non-contingent payments may be adjusted by agreement of the parties to such contracts); (ii) the amounts of revenue sharing proceeds, if any, that are includable in BRI for such Salary Cap Year pursuant to Section 1(a)(8) above; (iii) the amounts with respect to contingent payments (other than revenue sharing proceeds), if any, attributable to Salary Cap Years covered by this Agreement in Successor Agreements as such amounts are agreed upon by the parties, or if the parties do not reach agreement, by the Accountants; and (iv) the amount included in BRI for the preceding Salary Cap Year with respect to the value of advertising or promotional time provided to the NBA as part of the NBA/ABC and NBA/TBS Agreements (or any Successor Agreements) that is used to promote the WNBA or for any purpose other than those listed in Section 1(a)(1)(ii)(A)-(D).
  \item
    Projected BRI for the 2005-06 Salary Cap Year shall be deemed to be \$3,120,159,000 (``2005-06 Projected BRI'').
  \end{enumerate}
\item
  \textbf{``Local Expansion Team BRI''} means the BRI of the Expansion Teams during their first two (2) Seasons, but not including the Expansion Teams' share of League-wide revenues that are otherwise included in BRI (including, but not limited to, their share of national television, cable, radio and other broadcast revenues).
\item
  \textbf{``Projected Local Expansion Team BRI''} means Local Expansion Team BRI for the immediately preceding Season, increased by 4.5\%.
\item
  \textbf{``Interim Projected BRI''} means a projection of BRI for a Salary Cap Year using Estimated BRI in place of BRI for the previous Salary Cap Year.
\item
  \textbf{``Barter''} means to trade by exchanging one commodity, service or other non-cash item for another.
\item
  \textbf{``Estimated Total Benefits''} means the estimate of Total Benefits for a Salary Cap Year as set forth in the Interim Audit Report (as defined in Section 10(a) below) for such Salary Cap Year.
\item
  \textbf{``Estimated Total Salaries''} means the estimate of Total Salaries for a Salary Cap Year as set forth in the Interim Audit Report for such Salary Cap Year.
\item
  \textbf{``Estimated Total Salaries and Benefits''} means the sum of Estimated Total Benefits and Estimated Total Salaries for a Salary Cap Year as set forth in the Interim Audit Report for such Salary Cap Year.
\item
  \textbf{``Estimated BRI''} means the estimate of BRI for a Salary Cap Year as set forth in the Interim Audit Report for such Salary Cap Year.
\end{enumerate}

\hypertarget{calculation-of-salary-cap-and-minimum-team-salary.}{%
\section{Calculation of Salary Cap and Minimum Team Salary.}\label{calculation-of-salary-cap-and-minimum-team-salary.}}

\begin{enumerate}
\def\labelenumi{(\alph{enumi})}
\tightlist
\item
  \textbf{Salary Cap.}

  \begin{enumerate}
  \def\labelenumii{(\arabic{enumii})}
  \item
    For each Salary Cap Year during the term of this Agreement, there shall be a Salary Cap. Subject to the adjustments set forth in Section 2(d) below, the Salary Cap for each Salary Cap Year will equal the percentage of Projected BRI for such Salary Cap Year set forth below, less Projected Benefits (as defined in Article IV, Section 8) for such Salary Cap Year, divided by the number of Teams scheduled to play in the NBA during such Salary Cap Year, other than Expansion Teams during their first two Salary Cap Years in the NBA:

    \begin{longtable}[]{@{}cc@{}}
    \toprule()
    Salary Cap Year & Projected BRI \% \\
    \midrule()
    \endhead
    2005-06 & 49.5\% \\
    2006-07 & 51\% \\
    2007-08 & 51\% \\
    2008-09 & 51\% \\
    2009-10 & 51\% \\
    2010-11 & 51\% \\
    2011-12 & 51\% \\
    \bottomrule()
    \end{longtable}

    (if the NBA exercises its option to extend this Agreement pursuant to Article XXXIX)
  \item
    Notwithstanding Section 2(a)(1) above, in the event that, subject to the adjustments set forth in Section (d) below, Projected BRI for any Salary Cap Year, in which one or more Expansion Teams is scheduled to play its second Season, plus Projected Local Expansion Team BRI for such Salary Cap Year, multiplied by the applicable percentage of Projected BRI set forth in Section 2(a)(1) above, less Projected Benefits for such Salary Cap Year (including for the Expansion Team(s)), divided by the number of Teams scheduled to play in the NBA during such Salary Cap Year (including the Expansion Team(s)), exceeds the Salary Cap calculated in accordance with Section 2(a)(1) above, the Salary Cap shall equal the amount calculated pursuant to this Section 2(a)(2).
  \item
    The Salary Cap for the 2005-06 Salary Cap Year for all Teams other than the Charlotte Bobcats shall be deemed to be \$49.5 million. The Salary Cap for the 2005-06 Salary Cap Year for the Charlotte Bobcats shall be deemed to be \$37.125 million.
  \item
    The Salary Cap for a Salary Cap Year will be in effect commencing on the day following the last day of the Moratorium Period in such Salary Cap Year and shall continue through and including the last day of the Moratorium Period in the next Salary Cap Year.
  \item
    In the event that the Audit Report for a Salary Cap Year, beginning with the 2005-06 Salary Cap Year, has not been completed as of the last day of the Moratorium Period immediately following the end of such Salary Cap Year, then the Salary Cap for the Salary Cap Year that commenced on the immediately preceding July 1 will be calculated pursuant to Section 2(a)(1)-(2) above, except that Interim Projected BRI shall be utilized instead of Projected BRI, Estimated BRI shall be utilized instead of BRI and Estimated Total Salaries and Benefits shall be utilized instead of Total Salaries and Benefits, for all purposes under this Section 2 including, without limitation, the adjustments set forth in Section 2(d) below. In the event that the Interim Audit Report for a Salary Cap Year, beginning with the 2005-06 Salary Cap Year, has not been completed as of the last day of the Moratorium Period immediately following the end of such Salary Cap Year, then the Salary Cap for the Salary Cap Year that commenced on the immediately preceding July 1 shall, until such Interim Audit Report is completed, be an amount that would have been the Salary Cap for the preceding Salary Cap Year (except that if the preceding Salary Cap Year was the 2005-06 Salary Cap Year, then an amount that would have been the Salary Cap for the 2005-06 Salary Cap Year had it been computed based on 51\% of Projected BRI rather than 49.5\% of Projected BRI) had Projected BRI or Interim Projected BRI, as the case may be, for such preceding Salary Cap Year included, with respect to the NBA's national broadcast, national telecast or network cable television contracts, the rights fees or other non-contingent payments stated in such contracts for the Season following the Season covered by such preceding Salary Cap Year instead of for the Season covered by such preceding Salary Cap Year.
  \end{enumerate}
\item
  \textbf{Minimum Team Salary.}

  \begin{enumerate}
  \def\labelenumii{(\arabic{enumii})}
  \tightlist
  \item
    For each Salary Cap Year during the term of this Agreement, there shall be a Minimum Team Salary equal to 75\% of the Salary Cap for such Salary Cap Year. The Minimum Team Salary for the 2005-06 Salary Cap Year for all Teams other than the Charlotte Bobcats shall be deemed to be \$37.125 million. The Minimum Team Salary for the 2005-06 Salary Cap Year for the Charlotte Bobcats shall be deemed to be \$27.844 million.
  \item
    In the event that, by the conclusion of a Salary Cap Year, a Team has failed to make aggregate Salary payments and/or incur aggregate Salary obligations equal to or greater than the applicable Minimum Team Salary for that Salary Cap Year, the NBA shall cause such Team to make payments equal to the shortfall (to be disbursed to the players on such Team pro rata or in accordance with such other formula as may be reasonably determined by the Players Association).
  \item
    Nothing contained herein shall preclude a Team from having a Team Salary in excess of the Minimum Team Salary, provided that the Team's Team Salary does not exceed the Salary Cap plus any additional amounts authorized pursuant to the Exceptions set forth in this Article VII.
  \end{enumerate}
\item
  \textbf{Expansion Team Salary Caps and Minimum Team Salaries.} Expansion Teams shall have the same Salary Caps and Minimum Team Salaries as all other Teams, except as follows:

  \begin{enumerate}
  \def\labelenumii{(\arabic{enumii})}
  \tightlist
  \item
    During the first Salary Cap Year in which it begins play, an Expansion Team shall have a Salary Cap equal to 66 and 2/3\% of the Salary Cap applicable to all other Teams (the ``First Year Expansion Team Salary Cap''); and shall have a Minimum Team Salary equal to 75\% of the First Year Expansion Team Salary Cap.
  \item
    During the second Salary Cap Year in which it begins play, an Expansion Team shall have a Salary Cap equal to 75\% of the Salary Cap applicable to all other Teams (the ``Second Year Expansion Team Salary Cap''); and shall have a Minimum Team Salary equal to 75\% of the Second Year Expansion Team Salary Cap.
  \end{enumerate}
\item
  \textbf{Adjustments to Salary Cap and Minimum Team Salary.}

  \begin{enumerate}
  \def\labelenumii{(\arabic{enumii})}
  \item
    In the event that (i) 49.5\% of BRI for the 2005-06 Salary Cap Year is less than 49.5\% of 2005-06 Projected BRI; or (ii) 51\% of BRI for any other Salary Cap Year is less than 51\% of Projected BRI for such Salary Cap Year, then, in either case, for purposes of calculating the Salary Cap for the subsequent Salary Cap Year, the difference shall be deducted from 51\% of Projected BRI for such subsequent Salary Cap Year.
  \item
    \begin{enumerate}
    \def\labelenumiii{(\roman{enumiii})}
    \tightlist
    \item
      In the event that Total Salaries and Benefits paid with respect to the 2005-06 Salary Cap Year is less than 49.5\% of BRI for such Salary Cap Year, then for purposes of calculating the Salary Cap for the 2006-07 Salary Cap Year, the amount of such shortfall shall be added to 51\% of Projected BRI for the 2006-07 Salary Cap Year.
    \item
      In the event that Total Salaries and Benefits paid with respect to any Salary Cap Year beginning with the 2006-07 Salary Cap Year is less than 51\% of BRI for such Salary Cap Year (plus, if applicable, any shortfall against the applicable percentage of BRI for the prior Salary Cap Year (i.e., 49.5 \% of BRI for the 2005-06 Salary Cap Year and 51\% of BRI for each subsequent Salary Cap Year)), then for purposes of calculating the Salary Cap for the subsequent Salary Cap Year, the amount of such shortfall shall be added to 51\% of Projected BRI for the subsequent Salary Cap Year.
    \end{enumerate}
  \item
    In the event that Benefits for any Salary Cap Year exceeds Projected Benefits for such Salary Cap Year, then for purposes of calculating the Salary Cap for the subsequent Salary Cap Year, the difference shall be added to Projected Benefits for such subsequent Salary Cap Year.
  \item
    In the event that Benefits for any Salary Cap Year is less than Projected Benefits for such Salary Cap Year, then for purposes of calculating the Salary Cap for the subsequent Salary Cap Year, the difference shall be deducted from Projected Benefits for the subsequent Salary Cap Year.
  \item
    In the event that the Salary Cap for a Salary Cap Year, beginning with the 2006-07 Salary Cap Year, is calculated in accordance with Section 2(a)(5) above (i.e., is based upon an Interim Audit Report for the prior Salary Cap Year) and BRI and Total Salaries and Benefits as set forth in the Audit Report for the prior Salary Cap Year are different from those in the Interim Audit Report such that the Salary Cap would have been different from that based upon the Interim Audit Report, any such difference in the Salary Cap shall be debited or credited, as the case may be, to the Salary Cap for the subsequent Salary Cap Year, except that, with respect to the 2010-11 Salary Cap Year (or, in the alternative, if the NBA exercises its option pursuant to Article XXXIX, the 2011-12 Salary Cap Year) any such differences shall be debited or credited, as the case may be, to the Salary Cap for the then current Salary Cap Year, in all such cases with interest (at a rate equal to the one year Treasury Bill rate as published in The Wall Street Journal on the date of the issuance of the Interim Audit Report).
  \end{enumerate}
\item
  \textbf{Guarantee.}

  \begin{enumerate}
  \def\labelenumii{(\arabic{enumii})}
  \tightlist
  \item
    In the event that for any Salary Cap Year Total Salaries and Benefits is less than 57\% of BRI, the difference shall be paid by the NBA to the Players Association no later than thirty (30) days following the completion of the Audit Report for such Salary Cap Year for distribution to all NBA players who were on an NBA roster during the Season covered by such Salary Cap Year on such proportional basis as may be reasonably determined by the Players Association.
  \end{enumerate}
\end{enumerate}

\hypertarget{determination-of-salary.}{%
\section{Determination of Salary.}\label{determination-of-salary.}}

For the purposes of determining a player's Salary with respect to an NBA Season, the following rules shall apply:

\begin{enumerate}
\def\labelenumi{(\alph{enumi})}
\tightlist
\item
  \textbf{Deferred Compensation.}

  \begin{enumerate}
  \def\labelenumii{(\arabic{enumii})}
  \tightlist
  \item
    \emph{General Rules:}

    \begin{enumerate}
    \def\labelenumiii{(\roman{enumiii})}
    \tightlist
    \item
      All Player Contracts entered into, extended or renegotiated after the date of this Agreement shall specify the Season(s) in which any Deferred Compensation is earned. Deferred Compensation shall be included in a player's Salary in the Season in which such Deferred Compensation is earned.
    \item
      Notwithstanding Section 3(a)(1)(i) above, for purposes of an annuity compensation arrangement included in a Player Contract in accordance with Article XXV, Section 3 of this Agreement only, Deferred Compensation shall include only the portion of the cost of the annuity instrument to be paid by the Team after the playing term covered by the Contract, if any, and shall not include any Compensation that the player is scheduled to receive after the term of the Contract pursuant to such annuity compensation arrangement. The portion of the cost of the annuity paid by the Team while the player is required to render playing services under the Player Contract shall be included in Salary for the Salary Cap Year in which such cost is paid.
    \end{enumerate}
  \item
    \emph{Over 36 Rule:} The following provisions shall apply to any Player Contract entered into, extended, or renegotiated that, beginning with the date such Contract, Extension or Renegotiation is signed, covers four (4) or more Seasons, including one (1) or more Seasons commencing after such player will reach or has reached age thirty-six (36) (an ``Over 36 Contract'') (except that any Player Contract signed prior to November 1, 2005 shall not be governed by the following provisions and shall instead be governed by the provisions of Article VII, Section 3(a)(2) of the 1999 NBA/NBPA Collective Bargaining Agreement):

    \begin{enumerate}
    \def\labelenumiii{(\roman{enumiii})}
    \tightlist
    \item
      Except as provided in Section 3(a)(2)(ii)-(iv) below, the aggregate Salaries in an Over 36 Contract for Salary Cap Years commencing with the fourth Salary Cap Year of such Over 36 Contract or the first Salary Cap Year that covers a Season that follows the player's 36th birthday, whichever is later, shall be attributed to the prior Salary Cap Years pro rata on the basis of the Salaries for such prior Salary Cap Years.
    \item
      If a Qualifying Veteran Free Agent who is age 33 or 34 enters into an Over 36 Contract with his Prior Team covering more than four (4) Seasons, the aggregate Salaries in such Over 36 Contract for Salary Cap Years commencing with the fifth Salary Cap Year shall be attributed to the prior Salary Cap Years pro rata on the basis of the Salaries for such prior Salary Cap Years.
    \item
      If a player who has played for his current Team for at least ten (10) consecutive Seasons enters into an Over 36 Contract that is an Extension and that, beginning with the date the Extension is signed, covers more than four (4) Seasons, the aggregate Salaries in such Over 36 Contract for Salary Cap Years commencing with the fifth Salary Cap Year shall be attributed to the prior Salary Cap Years pro rata on the basis of the Salaries for such prior Salary Cap Years.
    \item
      For each Salary Cap Year of an Over 36 Contract beginning with the second Salary Cap Year prior to the First Zero Year (as defined in Section 3(a)(2)(viii) below), if the player has not been placed on waivers as of the July 1 of such Salary Cap Year, then the Salaries of the player for such Salary Cap Year and the subsequent two (2) or fewer Salary Cap Years covered \textbf{\emph{{[}sic{]}}}
    \item
      Notwithstanding Section 3(a)(2)(i) above, there shall be no re-allocation of Salaries pursuant to this Section 3(a)(2) for:

      \begin{enumerate}
      \def\labelenumiv{(\Alph{enumiv})}
      \tightlist
      \item
        Any contact between a Qualifying Veteran Free Agent and his Prior Team covering four (4) or fewer Seasons entered into by a player at age 33 or 34; and
      \item
        any Extension that, beginning with the date the Extension is signed, covers four (4) or fewer Seasons and is entered into by a player who has played for his current Team for at least ten 910) consecutive Seasons.
      \end{enumerate}
    \item
      For purposes of determining whether a Contract is an Over 36 Contract pursuant to this Section 3(a)(2) only, Seasons shall be deemed to commence on October 1 and conclude on the last day of the Salary Cap Year.
    \item
      ``Zero Year'' means, with respect to an Over 36 Contract, any Salary Cap Year in which the Salary called for under the Contract has been attributed, in accordance with Section 3(a)(2)(i), (ii), (iii) or (iv) above, to prior Salary Cap Years of the Contract. ``First Zero Year'' means, with respect to an Over 36 Contract, the earliest Salary Cap Year in which the Salary called for under the Contract has been attributed, in accordance with Section 3(a)(2)(i), (ii), (iii) or (iv) above, to prior Salary Cap Years of the Contract.
    \item
      For purposes of this subsection (a)(2), a player (A) whose birthday is on a date during the Moratorium Period and (B) who signs a Contract, Extension or Renegotiation on or before the fifth day following the conclusion of the Moratorium Period shall be treated as if his age, at the time of such signing, was his age on the immediately preceding June 30.
    \end{enumerate}
  \end{enumerate}
\item
  \textbf{Signing Bonuses.}

  \begin{enumerate}
  \def\labelenumii{(\arabic{enumii})}
  \tightlist
  \item
    \emph{Amounts Treated as Signing Bonuses:} For purposes of determining a player's Salary, the term ``signing bonus'' shall include:

    \begin{enumerate}
    \def\labelenumiii{(\roman{enumiii})}
    \tightlist
    \item
      any amount provided for in a Player Contract that is earned upon the signing of such Contract;
    \item
      any Option Buy-out Amount;
    \item
      at the time of a trade of a Player Contract, any amount that, under the terms of the Contract, is earned in the form of a bonus upon the trade of the Contract; and
    \item
      payments in excess of \$500,000 with respect to foreign players, in accordance with Section 3(e) below.
    \end{enumerate}
  \item
    \emph{Proration:} Any signing bonus contained in a Player Contract shall be allocated over the number of Salary Cap Years (or over the then-current and any remaining Salary Cap Years in the case of a signing bonus described in Section 3(b)(1)(iii) above) covered by such Contract in proportion to the percentage of Base Compensation in each such Salary Cap Year that, at the time of allocation, is protected for lack of skill; provided, however, that if the Player Contract provides for an Early Termination Option (``ETO''), the foregoing allocation shall be performed only over Salary Cap Years that precede the Effective Season of such ETO. In the event that, at the time of allocation, none of the Base Compensation provided for by a Player Contract (or none of the then-current or remaining Base Compensation in the case of a signing bonus described in Section 3(b)(1)(iii) above) is protected for lack of skill, then the entire amount of the signing bonus shall be allocated to the first Salary Cap Year of the Contract (or, in the case of a signing bonus described in Section 3(b)(1)(iii) above, the Salary Cap Year during which the player's Contract is traded).
  \item
    \emph{Signing Bonus Credits:} Upon the occurrence of an event that determines that a player shall not be entitled to receive an Option Buy-Out Amount (the ``non-payment determination''):

    \begin{enumerate}
    \def\labelenumiii{(\roman{enumiii})}
    \tightlist
    \item
      all amounts that were included in the player's Salary pursuant to Section 3(b)(1)(ii) above for Salary Cap Years up to and including the Salary Cap Year in which the non-payment determination is made (the ``unpaid amounts'') shall be deducted from the calculation of Total Salaries and Benefits for the Salary Cap Year in which the non-payment determination is made;
    \item
      all amounts that were included in the player's Salary pursuant to Section 3(b)(1)(ii) above for Salary Cap Years following the Salary Cap Year in which the non-payment determination is made shall be deducted from the player's Salary for such Salary Cap Years; and
    \item
      the unpaid amounts shall be deducted from the Team's Team Salary, in accordance with the following:

      \begin{enumerate}
      \def\labelenumiv{(\Alph{enumiv})}
      \tightlist
      \item
        The total amount available to be deducted from the Team's Team Salary (the ``credit amount'') will equal the aggregate of the unpaid amounts less, for each Season in which a portion of the unpaid amounts was included in the player's Salary and in which his Team's Team Salary did not fall below the Salary Cap, the smallest amount by which his Team's Team Salary exceeded the Salary Cap during such Salary Cap Year.
      \item
        The credit amount shall be allocated, in equal parts, over the same number of Salary Cap Years over which the unpaid amounts were allocated, beginning with the first Salary Cap Year following the non-payment determination, plus, for each Salary Cap Year following the first Salary Cap Year of such allocation, 10\% of the amount allocated to the first Salary Cap Year.
      \item
        If, during the course of any Salary Cap Year in which a credit allocation has been made, the Team's Team Salary does not fall below the Salary Cap, the full credit allocation for such Salary Cap Year will be carried forward to a subsequent Salary Cap Year. If, during the course of a Salary Cap Year in which a credit allocation has been made, the Team's Team Salary does fall below the Salary Cap, the amount carried forward, if any, will equal the amount of the credit allocation for such Salary Cap Year less the largest amount by which the Team's Team Salary fell below the Salary Cap during such Salary Cap Year. In the event a credit allocation is carried forward pursuant to this subsection, such amount shall be deducted from Team Salary in the Salary Cap Year immediately following the last Salary Cap Year in which a portion of the credit amount is then currently being allocated, subject to the terms of this Section 3(b)(3)(iii)(C).
      \end{enumerate}
    \end{enumerate}
  \item
    \emph{Extensions:}

    \begin{enumerate}
    \def\labelenumiii{(\roman{enumiii})}
    \tightlist
    \item
      In the event that a Team with a Team Salary at or over the Salary Cap enters into an Extension that calls for or contains a signing bonus, such signing bonus shall be paid no sooner than the first day of the first Salary Cap Year covered by the extended term and shall be allocated, in equal parts, over the number of Salary Cap Years covered by the extended term in proportion to the percentage of Base Compensation in each such Salary Cap Year that, at the time of allocation, is protected for lack of skill. In the event that, at the time of the allocation, none of the Base Compensation provided for during the extended term is protected for lack of skill, then the entire amount of the signing bonus shall be allocated to the first Salary Cap Year of the extended term.
    \item
      A Team with a Team Salary below the Salary Cap may enter into an Extension that calls for or contains a signing bonus to be paid at any time during the Contract's original or extended term. In the event that a Team with a Team Salary below the Salary Cap enters into an Extension that calls for or contains a signing bonus to be paid no sooner than the first day of the Salary Cap Year covered by such extended term, the bonus shall be allocated in accordance with the proration rules set forth in Section 3(b)(4)(i) above. In the event a Team with a Team Salary below the Salary Cap enters into an Extension that calls for or contains a signing bonus to bepaid prior to the first day of the first Salary Cap Year covered by the extended term, the following rules shall apply:

      \begin{enumerate}
      \def\labelenumiv{(\Alph{enumiv})}
      \tightlist
      \item
        The signing bonus shall be allocated over the remaining Salary Cap Years (including the then-current Salary Cap Year) under the original term of the Contract and the extended term in proportion to the percentage of Base Compensation in each such Salary Cap Year that, at the time of allocation, is protected for lack of skill. In the event that, at the time of allocation, none of the Base Compensation provided for during the then-current and any remaining Salary Cap Years under the original term of the Contract or during the extended term is protected for lack of skill, then the entire amount of the signing bonus shall be allocated to the Salary Cap Year during which the Extension is signed; and
      \item
        The Extension shall be deemed a Renegotiation and shall be subject to the rules governing Renegotiations set forth in Section 7 below.
      \end{enumerate}
    \end{enumerate}
  \end{enumerate}
\item
  \textbf{Loans to Players.} The following rules shall apply to any loan made by any Team to a player:

  \begin{enumerate}
  \def\labelenumii{(\arabic{enumii})}
  \tightlist
  \item
    If any such loan bears no interest (or annual interest at an effective rate lower than the ``Target Rate'' (as defined below)), then the difference between the Target Rate and the actual rate of interest to be paid by the player shall be imputed to the outstanding balance and included in the player's Salary. The ``Target Rate'' means the ``Prime Rate'' (as defined below) plus 1\% as of the date the loan is agreed upon, except that the ``Target Rate'' shall be no lower than 7\% or greater than 9\%. For purposes of this Section 3(c)(1), ``Prime Rate'' means the prime rate reported in the ``Money Rates'' column or any successor column of the Wall Street Journal.
  \item
    No loan made to a player may (along with other outstanding loans to the player) exceed the amount of the player's Salary for the then-current Salary Cap Year that is protected for lack of skill. All loans must be repaid through deductions from the player's remaining Current Base Compensation over the years of the Contract that, at the time the loan is agreed upon, provide for Base Compensation that is fully protected for lack of skill (prior to the Effective Season of any ETO) in equal annual amounts (the ``annual allocable repayment amounts''). If a loan is made at a time when the remaining Current Base Compensation due for the relevant Season that is fully protected for lack of skill is less than the annual allocable repayment amount that would be owed on a loan for the full amount of the player's Current Base Compensation that is fully protected for lack of skill for the relevant Season (the ``maximum annual allocable repayment amount''), the maximum loan amount for that Season shall be reduced by the amount by which the maximum annual allocable repayment amount exceeds the amount of remaining Current Base Compensation that is fully protected for lack of skill. (For example, if a Player has \$1 million in Current Base Compensation (fully protected for lack of skill) in the first Season of a five-year Contract, and a loan is made during that Season at a time when the Player has already received his Current Base Compensation for that Season, the loan may not exceed \$800,000.)
  \item
    In addition to the restrictions set forth in Section 3(c)(2) above: (i) no loan may be made that would result in a violation of Article II, Section 12(e); and (ii) no loan may be made to a player whose Contract provides for Base Compensation equal to the Minimum Player Salary.
  \item
    Any forgiveness by a Team of a loan to a player shall be deemed a Renegotiation in the Salary Cap Year of such forgiveness and shall be subject to the rules governing Renegotiations set forth in Section 7 below.
  \end{enumerate}
\item
  \textbf{Incentive Compensation.}

  \begin{enumerate}
  \def\labelenumii{(\arabic{enumii})}
  \tightlist
  \item
    For purposes of determining a player's Salary each Salary Cap Year, except as provided in Section 3(d)(2)-(4) below, any Performance Bonus (provided such Performance Bonus may be included in a Player Contract in accordance with Section 5(d) below), shall be included in Salary only if such Performance Bonus would be earned if the Team's or player's performance were identical to the performance in the immediately preceding Salary Cap Year.
  \item
    Notwithstanding Section 3(d)(1) above, in the event that, at the time of the signing of a Contract, Renegotiation or Extension, the NBA or the Players Association believes that the performance of a player and/or his team during the immediately preceding Salary Cap Year does not fairly predict the likelihood of the player earning a Performance Bonus during any Salary Cap Year covered by the Contract, Renegotiation or extended term of the Extension (as the case may be), the NBA or the Players Association may request that a jointly selected basketball expert (``Expert'') determine whether (i) in the case of an NBA challenge, it is very likely that the bonus will be earned, or (ii) in the case of a Players Association challenge, it is very likely that the bonus will not be earned. The party initiating a proceeding before the Expert shall carry the burden of proof. The Expert shall conduct a hearing within five (5) business days after the initiation of the proceeding, and shall render a determination within five (5) business days after the hearing. Notwithstanding anything to the contrary in this Section 3(d)(2), no party may, in connection with any proceeding before the Expert, refer to the facts that, absent a challenge pursuant to this Section 3(d)(2), a Performance Bonus would or would not be included in a player's Salary pursuant to Section 3(d)(1) above, or would be termed ``Likely'' or ``Unlikely'' pursuant to Article I, Section 1(cc) or (vvv). If, following an NBA challenge, the Expert determines that a Performance Bonus is very likely to be earned, the bonus shall be included in the player's Salary. If, following a Players Association challenge, the Expert determines that a Performance Bonus is very likely not to be earned, the bonus shall be excluded from the player's Salary. The Expert's determination that a Performance Bonus is very likely to be earned or very likely not to be earned shall be final, binding and unappealable. The fees and costs of the Expert in connection with any proceeding brought pursuant to this Section 3(d)(2) shall be borne equally by the parties.
  \item
    In the case of a Rookie or a Veteran who did not play during the immediately preceding Salary Cap Year who signs a Contract containing a Performance Bonus, or in the case of a player signed or acquired by an Expansion Team whose Contract contains a Performance Bonus to be paid as a result of, in whole or in part, the player's achievement of agreed-upon benchmarks relating to the Team's performance during its first Salary Cap Year, such Performance Bonus will be included in Salary if it is likely to be earned. In the event that the NBA and the Players Association cannot agree as to whether a Performance Bonus is likely to be earned, such dispute will be referred to the Expert, who will determine whether the bonus is likely to be earned or not likely to be earned. The Expert shall conduct a hearing within five (5) business days after the initiation of the proceeding, and shall render a determination within five (5) business days after the hearing. The Expert's determination that a Performance Bonus is likely to be earned or not likely to be earned shall be final, binding and unappealable. The fees and costs of the Expert in connection with any proceeding brought pursuant to this Section 3(d)(3) shall be borne equally by the parties.
  \item
    In the event that either party initiates a proceeding pursuant to Section 3(d)(2) or (3) above, the player's Salary plus the full amount of any disputed bonuses shall be included in Team Salary during the pendency of the proceeding.
  \item
    In the event the NBA and the Players Association cannot agree on an Expert, any challenge pursuant to Section 3(d)(2) and (3) above may be filed with the Grievance Arbitrator in accordance with Article XXXI, Sections 2-6 and 14.
  \item
    All Incentive Compensation described in Article II, Sections 3(b)(iv) and 3(c) shall be included in Salary.
  \end{enumerate}
\item
  \textbf{International Player Payments.}

  \begin{enumerate}
  \def\labelenumii{(\arabic{enumii})}
  \tightlist
  \item
    Any amount in excess of \$500,000 paid or to be paid by or at the direction of any NBA Team to (i) any basketball team other than an NBA Team, or (ii) any other entity, organization, representative or person, for the purpose of inducing an international player (as defined in Article X, Section 1(c)) to enter into a Player Contract or in connection with securing the right to enter into a Player Contract with an international player shall be deemed Salary (in the form of a signing bonus) to the player.
  \item
    Subject to Article XIII, any payment of \$500,000 or less paid by or at the direction of any NBA Team pursuant to Section 3(e)(1) above (the ``\$500,000 exclusion''), shall not be deemed Salary to the player.
  \item
    The \$500,000 exclusion may be paid in a single installment or in multiple installments. The \$500,000 exclusion, whether used in whole or in part, may be used by an NBA Team whenever it signs an international player to a new Player Contract, except that the \$500,000 exclusion may not be used, in whole or in part, more than once in any three-Season period with respect to the same international player.
  \item
    The \$500,000 exclusion, or any part of it, shall be deemed to have been used as of the date of the Player Contract to which it applies, regardless of when it is actually paid. A schedule of payments relating to the \$500,000 exclusion, or any part of it, agreed upon at the time of the signing of the Player Contract to which it applies, shall not be deemed a multiple use of the \$500,000 exclusion.
  \item
    Notwithstanding Section 3(e)(1) above, no amount paid or to be paid pursuant to this Section 3(e) shall be counted toward the Minimum Team Salary obligation of a Team in accordance with Section 2(b) or (c) above.
  \end{enumerate}
\item
  \textbf{One-Year Minimum Contracts.} Except where otherwise stated in this Agreement, the Salary of every player who, after the date of this Agreement, signs a one-year, 10-Day or Rest-of-Season Contract for the Minimum Player Salary applicable to such player shall be the lesser of (1) such Minimum Player Salary, or (2) the portion of such Minimum Player Salary that is not reimbursed out of the League-wide benefits fund described in Article IV, Section 5(k)(2).
\item
  \textbf{Insurance Premium Reimbursement.} If a Team reimburses a player for life insurance premiums pursuant to Article II, Section 4(m)(ii), such premium reimbursement shall not be included in the computation of the player's Salary.
\item
  \textbf{Averaging.} In accordance with Article XI, Section 5(c)(iii), a player's Salary for each Salary Cap Year covered by his Contract shall be deemed in certain circumstances to be the average of the aggregate Salaries for each such Salary Cap Year.
\item
  \textbf{Existing Contracts.} A player's Salary with respect to any Salary Cap Year covered by a Contract entered into prior to the date of this Agreement shall continue to be calculated in accordance with the Salary Cap rules that were in existence at the time the Contract was entered into. In no event shall the preceding sentence apply to the calculation of Salary with respect to any Contract, Extension, Renegotiation, transaction, or event entered into or occurring on or after the date of this Agreement.
\end{enumerate}

\hypertarget{determination-of-team-salary.}{%
\section{Determination of Team Salary.}\label{determination-of-team-salary.}}

\begin{enumerate}
\def\labelenumi{(\alph{enumi})}
\tightlist
\item
  \textbf{Computation.} For purposes of computing Team Salary under this Agreement, all of the following amounts shall be included:

  \begin{enumerate}
  \def\labelenumii{(\arabic{enumii})}
  \item
    Subject to the rules set forth in this Article VII, the aggregate Salaries of all active players (and former players to the extent provided by the terms of this Agreement) attributable to a particular Salary Cap Year, including, without limitation:

    \begin{enumerate}
    \def\labelenumiii{(\roman{enumiii})}
    \tightlist
    \item
      Salaries paid or to be paid to players whose Player Contracts have been terminated pursuant to the NBA's waiver procedure (without regard to any revised payment schedule that might be provided for in the terminated Player Contracts).
    \item
      Any amount called for in a retired player's Player Contract paid or to be paid to the player. When a player retires and the Team continues to pay such amounts, then, for purposes of computing the player's Salary for the then-current and any remaining Salary Cap Year covered by the Contract, the aggregate of such amounts, notwithstanding the payment schedule, shall be allocated pro rata over the then-current and each remaining Salary Cap Year on the basis of the remaining unearned protected Compensation in each such Salary Cap Year at the time of retirement.
    \item
      Amounts paid or to be paid pursuant to awards for, or settlements of, grievances between a player and a Team concerning Compensation obligations under a Player Contract in accordance with the following rules (which, except for purposes of Section 4(a)(1)(iii)(D) below, shall be applied with respect to each Season for which there is any Compensation in dispute, as if the grievance relates only to such Season):

      \begin{enumerate}
      \def\labelenumiv{(\Alph{enumiv})}
      \item
        \begin{enumerate}
        \def\labelenumv{(\arabic{enumv})}
        \tightlist
        \item
          When a player initiates a Grievance (as defined in Article XXXI) against a Team seeking the payment of Compensation for a Season covered by the current or any future Salary Cap Year that the Team asserts is not owed, 50\% of the disputed amount shall be included in Team Salary for the Salary Cap Year to which the grievance relates. If the Grievance is resolved during or prior to the Salary Cap Year to which it relates, following resolution of the Grievance, whether by award or settlement, the disputed amount payable by the Team in excess of the 50\% allocation shall be included in Team Salary for the Salary Cap Year to which the Grievance relates, or, alternatively, the amount by which the 50\% allocation exceeds the disputed amount payable by the Team shall be subtracted from Team Salary for the Salary Cap Year to which the Grievance relates.
        \item
          If a Grievance described in the first sentence of Section 4(a)(1)(iii)(A)(1) above is resolved after the conclusion of the Salary Cap Year to which it relates, the disputed amount payable by the Team related to such Salary Cap Year in excess of the 50\% allocation shall be included in Team Salary for the Salary Cap Year in which the Grievance is resolved, or, alternatively, the amount by which the 50\% allocation exceeds the disputed amount payable by the Team related to such Salary Cap Year shall be subtracted from Team Salary for the Salary Cap Year in which the grievance is resolved. Notwithstanding the preceding sentence, a Team shall be required to pay additional tax to the NBA if and to the extent that, due to the operation of this Section 4(a)(1)(iii)(A)(2), the aggregate tax it pays to the NBA pursuant to Section 12(f) below for the two Salary Cap Years in question (the Salary Cap Year for which the 50\% allocation was made and the subsequent Salary Cap Year in which the Grievance was resolved) is less than it would have been had the Grievance been resolved during the Salary Cap Year to which it related; a Team shall be entitled to a tax refund from the NBA if and to the extent that, due to the operation of this Section 4(a)(1)(iii)(A)(2), the aggregate tax it pays to the NBA pursuant to Section 12(f) below for the two Salary Cap Years in question is greater than it would have been had the grievance been resolved during the Salary Cap Year to which it related.
        \end{enumerate}
      \item
        When a player initiates a Grievance against a Team seeking the payment of Compensation for a Season covered by a prior Salary Cap Year that the Team asserts is not owed, following resolution of the Grievance, whether by award or settlement, the disputed amount payable by the Team, if any, shall be included in Team Salary for the Salary Cap Year in which the Grievance is resolved (but only to the extent that it had been previously excluded from Team Salary). Notwithstanding the preceding sentence, a Team shall be required to pay additional tax to the NBA if and to the extent that, due to the operation of this Section 4(a)(1)(iii)(B), the aggregate tax it pays to the NBA pursuant to Section 12(f) below for the two (2) Salary Cap Years in question (the Salary Cap Year to which the Grievance related and the subsequent Salary Cap Year in which the Grievance was resolved) is less than it would have been had the disputed amount payable by the Team been included in Team Salary during the Salary Cap Year to which it related; a Team shall be entitled to a tax refund from the NBA if and to the extent that, due to the operation of this Section 4(a)(1)(iii)(B), the aggregate tax it pays to the NBA pursuant to Section 12(f) below for the two (2) Salary Cap Years in question is greater than it would have been had the disputed amount payable by the Team been included in Team Salary during the Salary Cap Year to which it related.
      \item
        If a Grievance relates to a player's Compensation for more than one (1) Season, for purposes of determining the disputed amount payable by the Team with respect to each such Season following the resolution of the Grievance, the aggregate amounts payable to the player for all Seasons pursuant to the resolution of the grievance, whether by award or settlement, shall be allocated to each such Season in proportion to the amount of Compensation that was in dispute for such Season, unless, in the case of an award, the Grievance Arbitrator allocates the amounts payable to the player to specific Seasons.
      \item
        Immediately upon reaching any agreement (oral or written) to resolve a Grievance relating to a player's Compensation, a Team shall notify the NBA by facsimile or e-mail and provide the NBA with the terms of such agreement. A Team's failure to comply with the preceding sentence may be considered evidence of a violation of Article XIII. If a Team delays or attempts to delay in any manner the processing or resolution of a Grievance relating to a player's Compensation for the purpose of creating or increasing its Room in any Salary Cap Year or for the purpose of reducing or deferring a tax payment to the NBA, such conduct shall constitute a violation of Article XIII.
      \end{enumerate}
    \item
      Salaries anticipated to be included in Team Salary based upon any agreement disclosed to the NBA pursuant to Article II, Section 12(a)(i) (including, without limitation, any executed Player Contract whose validity is conditional on the passage of a physical examination by the player or on the assignment of the Contract), except to the extent that any such Salary is less than a player's Free Agent Amount (as defined in Section 4(d) below).
    \end{enumerate}
  \item
    \begin{enumerate}
    \def\labelenumiii{(\roman{enumiii})}
    \tightlist
    \item
      With respect to each Veteran Free Agent who last played for a Team who is an Unrestricted Free Agent, the Free Agent Amount (as defined in Section 4(d) below) attributable to such Veteran Free Agent.
    \item
      With respect to each Veteran Free Agent who last played for a Team who is a Restricted Free Agent, the greater of (A) the Free Agent Amount (as defined in Section 4(d) below) attributable to such Veteran Free Agent, (B) the Salary called for in any outstanding Qualifying Offer tendered to such Veteran Free Agent, or (C) the Salary called for in any First Refusal Exercise Notice (as defined in Article XI, Section 5(c)) issued with respect to such Veteran Free Agent.
    \end{enumerate}
  \item
    The aggregate Salaries called for under all outstanding Offer Sheets (as defined in Article XI, Section 5(b)).
  \item
    An amount with respect to a Team's unsigned First Round Pick, if any, as determined in accordance with Section 4(e) below.
  \item
    An amount with respect to the number of players fewer than twelve (12) included in a Team's Team Salary, as determined in accordance with Section 4(f) below.
  \item
    Value or consideration received by retired players that is determined to be includable in Team Salary in accordance with Article XIII, Section 5.(7) The amount of any Salary Cap Exception that is deemed included in Team Salary in accordance with Section 6(k)(2) below.
  \end{enumerate}
\item
  \textbf{Expansion.} The Salary of any player selected by an Expansion Team in an expansion draft and terminated in accordance with the NBA waiver procedure before the first day of the Expansion Team's first Season shall not be included in the Expansion Team's Team Salary, except, to the extent such Salary is paid, for purposes of determining whether the Expansion Team has satisfied its Minimum Team Salary obligation for such Season.
\item
  \textbf{Assigned Contracts.} For purposes of calculating Team Salary, with respect to any Player Contract that is assigned, the assignee Team shall, upon assignment, have included in its Team Salary the entire Salary for the then-current Salary Cap Year and for all future Salary Cap Years.
\item
  \textbf{Free Agents.} Subject to Section 4(a)(2)(ii) above, until a Team's Veteran Free Agent re-signs with his Team, signs with another NBA Team, or is renounced, he will be included in his Prior Team's Team Salary at one of the following amounts (``Free Agent Amounts''):

  \begin{enumerate}
  \def\labelenumii{(\arabic{enumii})}
  \item
    \begin{enumerate}
    \def\labelenumiii{(\roman{enumiii})}
    \tightlist
    \item
      A Qualifying Veteran Free Agent, other than a Qualifying Veteran Free Agent described in Section 4(a)(1)(ii) or (iii) below, will be included at 150\% of his prior Salary if it was equal to or greater than the Estimated \textbf{\emph{{[}sic{]}}}
    \item
      A Qualifying Veteran Free Agent following the second Option Year of his Rookie Scale Contract will be included at 250\% of the player's prior Salary if it was equal to or greater than the Estimated Average Player Salary, and 300\% of his prior Salary if it was less than the Estimated Average Player Salary.
    \item
      A Qualifying Veteran Free Agent following the first Option Year of his Rookie Scale Contract will be included at an amount equal to the maximum Salary that the Team may pay the player using the Qualifying Veteran Free Agent Exception applicable to such player pursuant to Section 6(b)(1) below.
    \end{enumerate}
  \item
    An Early Qualifying Veteran Free Agent will be included at 130\% of his prior Salary, except that an Early Qualifying Veteran Free Agent following the second Season of his Rookie Scale Contract will be included at an amount equal to the maximum Salary that the Team may pay the player using the Early Qualifying Veteran Free Agent Exception applicable to such player pursuant to Section 6(b)(3) below; provided, however, that the player's prior Team may, by written notice to the NBA, renounce its rights to sign the player pursuant to the Early Qualifying Veteran Free Agent Exception, in which case the player will be deemed a Non-Qualifying Veteran Free Agent for purposes of this Section 4(d) and Sections 6(b) and 6(h)(4) below.
  \item
    A Non-Qualifying Veteran Free Agent will be included at 120\% of his prior Salary.
  \item
    Notwithstanding Section 4(d)(1)-(3) above, if the player's prior Salary was equal to or less than the Minimum Player Salary applicable to such player, he will be included at the portion of the then-current Minimum Annual Salary applicable to such player that would not be reimbursed out of the League-wide benefits fund described in Article IV, Section 5(k).
  \item
    Notwithstanding Section 4(d)(1)-(3) above, at no time shall a player's Free Agent Amount exceed the Maximum Player Salary applicable to such player or be less than the portion of the Minimum Annual Salary applicable to such player that would not be reimbursed out of the league-wide benefits fund described in Article IV, Section 5(k).
  \item
    For purposes of this Section 4(d) only, a player's ``prior Salary'' means his Regular Salary for the prior Season plus any signing bonus allocation and the amount of any Incentive Compensation actually earned for such Season.
  \item
    For purposes of this Section 4(d) only, in the event that a Veteran Free Agent's prior Contract provides for an increase or decrease in Salary between the second-to-last and last Seasons covered by the Contract of greater than \$4 million, such player's prior Salary shall be deemed to be equal to the average of the Salaries for the last two (2) Seasons of the Contract.
  \end{enumerate}
\item
  \textbf{First Round Picks.}

  \begin{enumerate}
  \def\labelenumii{(\arabic{enumii})}
  \tightlist
  \item
    A First Round Pick, immediately upon selection in the Draft, shall be included in the Team Salary of the Team that holds his draft rights at 100\% of his applicable Rookie Scale Amount, and, subject to Section 4(e)(2) below, shall continue to be included in the Team Salary of any Team that holds his draft rights (including any Team to which the player's draft rights are assigned) until such time as the player signs with such Team or until the Team loses or assigns its exclusive draft rights to the player.
  \item
    In the event that a First Round Pick signs with a non-NBA team, the player's applicable Rookie Scale Amount shall be excluded from the Team Salary of the Team that holds his draft rights, beginning on the date he signs such non-NBA contract or the first day of the Regular Season, whichever is later, and shall be included again in his Team's Team Salary at the applicable Rookie Scale Amount on the following July 1 or the date the player's contract ends (or the player is released from his non-NBA contractual obligations), whichever is earlier, unless the Team renounces its exclusive rights to the player in accordance with Article X, Section 4(f). If, after such following July 1, or any subsequent July 1, the player signs another, or remains under, contract with a non-NBA team, the player's applicable Rookie Scale Amount will again be excluded from Team Salary beginning on the date of the contract signing or the first day of the Regular Season commencing after such July 1, whichever is later, and will again be included in Team Salary at the applicable Rookie Scale Amount on the following July 1 or the date the player's contract ends (or the player is released from his non-NBA contractual obligations), whichever is earlier, unless the Team renounces its exclusive rights to the player in accordance with Article X, Section 4(f).
  \item
    For purposes of this Section 4(e), in the event that a First Round Pick does not sign a Contract with the Team that holds his draft rights during the Salary Cap Year immediately following the Draft in which he was selected (or during the same Salary Cap Year in which he was drafted if the Draft occurs on or after July 1), the ``applicable Rookie Scale Amount'' for such First Round Pick means, with respect to any subsequent Salary Cap Year, the Rookie Scale Amount that would apply if the player were drafted in the Draft immediately preceding such Salary Cap Year at the same draft position at which he was actually selected.
  \end{enumerate}
\item
  \textbf{Incomplete Rosters.}

  \begin{enumerate}
  \def\labelenumii{(\arabic{enumii})}
  \tightlist
  \item
    If at any time from July 1 through the day prior to the first day of the Regular Season a Team has fewer than twelve (12) players, determined in accordance with Section 4(f)(2) below, included in its Team Salary, then the Team's Team Salary shall be increased by an amount calculated as follows:\\
    STEP 1: Subtract from twelve (12) the number of players included in Team Salary.\\
    STEP 2: If the result in Step 1 is a positive number, multiply the result in Step 1 by the Minimum Annual Salary applicable to players with zero (0) Years of Service for that Salary Cap Year.
  \item
    In determining whether a Team has fewer than twelve (12) players included in its Team Salary for purposes of Section 4(f)(1) above only, the only players who shall be counted are (i) players under Contract with the Team who are included in Team Salary, (ii) Free Agents who are included in Team Salary pursuant to Section 4 (a)(2) above, (iii) players to whom Offer Sheets have been given, and (iv) unsigned First Round Picks who are included in Team Salary pursuant to Section 4(e) above.
  \end{enumerate}
\item
  \textbf{Renouncing.}

  \begin{enumerate}
  \def\labelenumii{(\arabic{enumii})}
  \tightlist
  \item
    To renounce a Veteran Free Agent, a Team must provide the NBA with an express, written statement renouncing its right to re-sign the player, effective no earlier than the July 1 following the last Season covered by the player's Contract. (The NBA shall notify the Players Association of any such renunciation by fax or e-mail within two (2) business days following receipt of notice of such renunciation.) If a Team renounces a Veteran Free Agent, the player will no longer qualify as a Qualifying Veteran Free Agent, Early Qualifying Veteran Free Agent, or Non-Qualifying Veteran Free Agent, as the case may be, and the Team will only be permitted to re-sign such player with Room (i.e., the Team cannot sign such player pursuant to Section 6(b) below) or pursuant to the Minimum Player Salary Exception. Notwithstanding the foregoing, in the event a Team renounces one or more players in order to create Room for an Offer Sheet, and the offeree-player's Prior Team subsequently matches the Offer Sheet and enters into a Contract with that player, the Team may rescind the renunciation(s) within two (2) business days of the date the Offer Sheet is matched, whereupon any such ``unrenounced'' player may again sign a Player Contract with the Team as a Qualifying Veteran Free Agent, Early Qualifying Veteran Free Agent, or Non-Qualifying Veteran Free Agent, as the case may be, and will again be included in his Prior Team's Team Salary at his applicable Free Agent Amount. Notwithstanding the foregoing, a Team may not rescind the renunciation of a player if (i) at the time the player was renounced the Team's Team Salary was at or below the Salary Cap and ``unrenouncing'' the player would cause the Team's Team Salary to exceed the Salary Cap, or (ii) at the time the player was renounced the Team's Team Salary was above the Salary Cap and ``unrenouncing'' the player would cause the Team's Team Salary to exceed the Salary Cap by more than the amount by which Team Salary exceeded the Salary Cap prior to the renunciation.
  \item
    A Team cannot renounce any player to whom the Team has made a Qualifying Offer until such time as the Qualifying Offer is no longer in effect.
  \end{enumerate}
\item
  \textbf{Long-Term Injuries.} Any player who suffers a career-ending injury or illness, and whose contract is terminated by the Team in accordance with the NBA waiver procedure, will be excluded from his Team's Team Salary as follows:

  \begin{enumerate}
  \def\labelenumii{(\arabic{enumii})}
  \tightlist
  \item
    Beginning on the first anniversary of the injury or illness, the Team may apply to the NBA to have the player's Salary for each remaining Salary Cap Year covered by the Contract excluded from Team Salary.
  \item
    The determination of whether a player has suffered a career-ending injury or illness shall be made by a physician selected jointly by the NBA and the Players Association.
  \item
    Notwithstanding Section 4(h)(1) and (2) above, the career-ending injury or illness of a player who plays in more than ten (10) games in any Season shall not be deemed to have occurred prior to the last game in which the player played in such Season.
  \item
    Notwithstanding Section 4(h)(1) and (2) above, if after a player's Salary is excluded from Team Salary in accordance with this Section 4(h), the player plays in ten (10) NBA games in any Season, the excluded Salary for the Salary Cap Year covering such Season and each subsequent Salary Cap Year shall thereupon be included in Team Salary (and if the tenth game played is a playoff game, then the excluded Salary shall be included in Salary retroactively as of the start of the Team's last Regular Season game). After a player's Salary forone (1) or more Salary Cap Years has been included in Team Salary in accordance with this Section 4(h)(4), the player's Team shall be permitted at the appropriate time to re-apply to have the player's Salary (for each Salary Cap Year remaining at the time of the re-application) excluded from Team Salary in accordance with the rules set forth in this Section 4(h).
  \item
    If a Team applies to have a player's Salary excluded from its Team Salary pursuant to this Section 4(h), the player shall cooperate in the processing of the application, including by appearing at the reasonably scheduled place and time for examination by the jointly-selected physician.
  \item
    Only the Team with which the player was under Contract at the time his career-ending injury or illness became known or reasonably should have become known shall be permitted to apply to have the player's Salary excluded from Team Salary pursuant to this Section 4(h).
  \end{enumerate}
\item
  \textbf{Summer Contracts.}

  \begin{enumerate}
  \def\labelenumii{(\arabic{enumii})}
  \tightlist
  \item
    Except as provided in Section 4(i)(2) below and subject to Article II, Section 14, from July 1 until the day prior to the first day of the next Regular Season, a Team may enter into Player Contracts that will not be included in Team Salary until the first day of such Regular Season (i.e., the player will be deemed not to have any Salary until the first day of such Regular Season), provided that such Contracts satisfy the requirements of this Section 4(i) (a ``Summer Contract''). Except as set forth in the following sentence, no Summer Contract may provide for (i) Compensation of any kind that is or may be paid or earned prior to the first day of the next Regular Season, or (ii) Compensation protection or insurance of any kind pursuant to Article II, Section 3(e) or 4. The only consideration that may be provided to a player signed to a Summer Contract, prior to the start of the Regular Season, is per diem, lodging, transportation, compensation in accordance with paragraph 3(b) of the Uniform Player Contract, and a disability insurance policy covering disabilities incurred while such player participates in summer leagues or rookie camps for the Team. A Team that has entered into one or more Summer Contracts must terminate such Contracts no later than the day prior to the first day of a Regular Season, except to the extent the Team has Room for such Contracts.
  \item
    A Team may not enter into a Summer Contract with a Veteran Free Agent who last played for the Team unless the Contract is for one (1) Season only and provides for no more than the Minimum Player Salary applicable to such player.
  \end{enumerate}
\item
  \textbf{Team Salary Summaries.}

  \begin{enumerate}
  \def\labelenumii{(\arabic{enumii})}
  \tightlist
  \item
    The NBA shall provide the Players Association with Team Salary summaries and a list of current Exceptions and Base Year Compensations twice a month during the Regular Season and once every week during the off-season.
  \item
    In the event that the NBA fails to provide the Players Association with any Team Salary summary or list of Exceptions or Base Year Compensations as provided for in Section 4(j)(1) above, the Players Association shall notify the NBA of such failure, and the NBA, upon receipt of such notice, shall as soon as reasonably possible, but in no event later than two business days following receipt of such notice, provide the Players Association with any such summary or list that should have been provided pursuant to Section 4(j)(1) above.
  \end{enumerate}
\end{enumerate}

\hypertarget{operation-of-salary-cap.}{%
\section{Operation of Salary Cap.}\label{operation-of-salary-cap.}}

\begin{enumerate}
\def\labelenumi{(\alph{enumi})}
\tightlist
\item
  \textbf{Basic Rule.} A Team's Team Salary may not exceed the Salary Cap at any time unless the Team is using one of the Exceptions set forth in Section 6 below.
\item
  \textbf{Room.} Subject to the other provisions of this Agreement, including without limitation Article II, Section 7, any Team with Room may enter into a Player Contract that calls for a Salary in the first Salary Cap Year covered by such Contract that would not exceed the Team's then-current Room.
\item
  \textbf{Annual Salary Increases and Decreases.}

  \begin{enumerate}
  \def\labelenumii{(\arabic{enumii})}
  \tightlist
  \item
    The following rules apply to all Player Contracts other than Contracts between Qualifying Veteran Free Agents or Early Qualifying Veteran Free Agents and their Prior Team:

    \begin{enumerate}
    \def\labelenumiii{(\roman{enumiii})}
    \tightlist
    \item
      For each Salary Cap Year covered by a Player Contract after the first Salary Cap Year, the player's Salary, excluding Incentive Compensation, may increase or decrease in relation to the previous Salary Cap Year's Salary, excluding Incentive Compensation, by no more than 8\% of the Regular Salary for the first Salary Cap Year covered by the Contract.
    \item
      In the event that the first Salary Cap Year covered by a Contract provides for Incentive Compensation, the total amount of Likely Bonuses in each subsequent Salary Cap Year covered by the Contract may increase or decrease by up to 8\% of the amount of Likely Bonuses in the first Salary Cap Year, and the total amount of Unlikely Bonuses in each subsequent Salary Cap Year may increase or decrease by up to 8\% of the amount of Unlikely Bonuses in the first Salary Cap Year.
    \end{enumerate}
  \item
    The following rules apply to all Players Contracts between Qualifying Veteran Free Agents or Early Qualifying Veteran Free Agents and their Prior Team (except any such Contracts signed pursuant to Section 6(d)(3) or Section 6(e)(2) below, which shall be governed by Section 5(c)(1) above):

    \begin{enumerate}
    \def\labelenumiii{(\roman{enumiii})}
    \tightlist
    \item
      For each Salary Cap Year covered by a Player Contract after the first Salary Cap Year, the player's Salary, excluding Incentive Compensation, may increase or decrease in relation to the previous Salary Cap Year's Salary, excluding Incentive Compensation, by no more than 10.5\% of the Regular Salary for the first Salary Cap Year covered by the Contract.
    \item
      In the event that the first Salary Cap Year covered by a Contract provides for Incentive Compensation, the total amount of Likely Bonuses in each subsequent Salary Cap Year covered by the Contract may increase or decrease by up to 10.5\% of the amount of Likely Bonuses in the first Salary Cap Year, and the total amount of Unlikely Bonuses in each subsequent Salary Cap Year may increase or decrease by up to 10.5\% of the amount of Unlikely Bonuses in the first Salary Cap Year.
    \end{enumerate}
  \item
    The following rules apply to Extensions other than Extensions of Rookie Scale Contracts:

    \begin{enumerate}
    \def\labelenumiii{(\roman{enumiii})}
    \tightlist
    \item
      For each Salary Cap Year covered by an Extension after the first Salary Cap Year covered by the extended term, the player's Salary, excluding Incentive Compensation, may increase or decrease in relation to the previous Salary Cap Year's Salary, excluding Incentive Compensation, by no more than 10.5\% of the Regular Salary for the last Salary Cap Year covered by the original term of the Contract.
    \item
      In the event that the last Salary Cap Year covered by the original term of the Contract provides for Incentive Compensation, the amount of Likely Bonuses and Unlikely Bonuses in each Salary Cap Year covered by the Extension after the first Salary Cap Year covered by the extended term may increase or decrease by up to 10.5\% of the amount of Likely Bonuses and Unlikely Bonuses, respectively, in the last Salary Cap Year covered by the original term.
    \end{enumerate}
  \item
    The following rules apply to Extensions of Rookie Scale Contracts:

    \begin{enumerate}
    \def\labelenumiii{(\roman{enumiii})}
    \tightlist
    \item
      For each Salary Cap Year covered by an Extension of a Rookie Scale Contract after the first Salary Cap Year covered by the extended term, the Player's Salary, excluding Incentive Compensation, may increase or decrease in relation to the previous Salary Cap Year's Salary, excluding Incentive Compensation, by no more than 10.5\% of the Regular Salary for the first Salary Cap Year covered by the extended term of the Contract.
    \item
      In the event that the first Salary Cap Year covered by the extended term of the Contract provides for Incentive Compensation, the amount of Likely Bonuses and Unlikely Bonuses in each Salary Cap Year covered by the Extension after the first Salary Cap Year covered by the extended term may increase or decrease by up to 10.5\% of the amount of Likely Bonuses and Unlikely Bonuses, respectively, in the first Salary Cap Year covered by the extended term.
    \end{enumerate}
  \item
    For purposes of this Article VII, Section 5(c) only, the amount of any bonuses that a player may receive pursuant to Article II, Sections 3(b)(iv) and 3(c) shall be added to the player's Regular Salary and excluded from his Incentive Compensation.
  \end{enumerate}
\item
  \textbf{Performance Bonuses.}

  \begin{enumerate}
  \def\labelenumii{(\arabic{enumii})}
  \tightlist
  \item
    Notwithstanding any other provision of this Agreement, no Player Contract may provide for Unlikely Bonuses in any Salary Cap Year that exceed 25\% of the player's Regular Salary for such Salary Cap Year.
  \item
    No Player Contract may provide for any Unlikely Bonus for the first Salary Cap Year covered by the Contract that, if included in the player's Salary for such Salary Cap Year, would result in the Team's Team Salary exceeding the Room under which it is signing the Contract. For the sole purpose of determining whether a Team has Room for a new Unlikely Bonus, the Team's Room shall be deemed reduced by all Unlikely Bonuses in Contracts approved by the Commissioner that may be paid to all of the Team's players that entered into Player Contracts (including Renegotiations) during that Salary Cap Year.
  \end{enumerate}
\item
  \textbf{No Futures Contracts.} Subject to Section 5(e)(4) below, but notwithstanding any other provision in this Agreement:

  \begin{enumerate}
  \def\labelenumii{(\arabic{enumii})}
  \item
    Every Player Contract must cover at least the then-current Season (or the upcoming Season in the case of a Contract entered into from July 1 through the day prior to the first day of the Season).
  \item
    No Team and player may enter into a Player Contract from the commencement of the Team's last game of the Regular Season through the following June 30. The preceding sentence shall not prohibit a Team and player from entering into an amendment to an existing Player Contract during such period if such amendment would otherwise be permitted under this Agreement.
  \item
    A Player Contract that covers more than one (1) Season must be for a consecutive period of Seasons.
  \item
    \begin{enumerate}
    \def\labelenumiii{(\roman{enumiii})}
    \tightlist
    \item
      A player who receives a Required Tender or a Qualifying Offer during the month of June may accept such Required Tender or Qualifying Offer beginning on the date he receives it.
    \item
      From February 1 through May 31 of any Salary Cap Year, a First Round Pick may enter into a Rookie Scale Contract commencing with the following Season, provided that as of or at any point following the first day of the then-current Regular Season (or the preceding Regular Season in the case of a Contract signed from the day following the last day of the Regular Season through May 31) the player was a party to a player contract with a professional basketball team not in the NBA covering such Regular Season.
    \end{enumerate}
  \end{enumerate}
\end{enumerate}

\hypertarget{exceptions-to-the-salary-cap.}{%
\section{Exceptions to the Salary Cap.}\label{exceptions-to-the-salary-cap.}}

There shall be the following exceptions to the rule that a Team's Team Salary may not exceed the Salary Cap:

\begin{enumerate}
\def\labelenumi{(\alph{enumi})}
\tightlist
\item
  \textbf{Existing Contracts.} A Team may exceed the Salary Cap to the extent of its current contractual commitments, provided that such contracts satisfied the provisions of this Agreement when entered into or were entered into prior to the date of this Agreement in accordance with the rules then in effect.
\item
  \textbf{Veteran Free Agent Exception.} Beginning on the day following the last day of the Moratorium Period following the last Season covered by a Veteran Free Agent's Player Contract, such player may enter into a new Player Contract with his Prior Team (or, in the case of a player selected in an Expansion Draft that year, with the Team that selected such player in an Expansion Draft) as follows:
\end{enumerate}

\begin{enumerate}
\def\labelenumi{(\arabic{enumi})}
\item
  If the player is a Qualifying Veteran Free Agent, the new Player Contract may provide for Salary and Unlikely Bonuses in the first Salary Cap Year totaling up to the maximum amount provided for in Article II, Section 7. Notwithstanding the preceding sentence, if the player is a Qualifying Veteran Free Agent whose last Contract was his Rookie Scale Contract and whose Prior Team did not exercise the second Option Year to extend such Contract for a fourth Season, the new Player Contract may provide for Regular Salary, Likely Bonuses and Unlikely Bonuses in the first Salary Cap Year of up to the Regular Salary, Likely Bonuses and Unlikely Bonuses, respectively, that the player would have received for such Salary Cap Year had his Prior Team exercised its second Option Year. Annual increases and decreases in Salary and Unlikely Bonuses shall be governed by Section 5(c)(2) above.
\item
  the player is a Non-Qualifying Veteran Free Agent, then, subject to Article II, Section 7, the new Player Contract may provide in the first Salary Cap Year up to the greater of: (i) 120\% of the Regular Salary for the final Salary Cap Year of the player's prior Contract, plus 120\% of any Likely Bonuses and Unlikely Bonuses, respectively, called for in the final Salary Cap Year covered by the player's prior Contract; (ii) Salary plus Unlikely Bonuses totaling 120\% of the then-current Minimum Annual Salary applicable to the player; or (iii) in the case of a Contract between a Team and its Restricted Free Agent, the Salary and Unlikely Bonuses required to be provided in a Qualifying Offer. Annual increases and decreases in Salary and Unlikely Bonuses shall be governed by Section 5(c)(1) above.
\item
  \begin{enumerate}
  \def\labelenumii{(\roman{enumii})}
  \tightlist
  \item
    If the player is an Early Qualifying Veteran Free Agent, the new Player Contract must cover at least two (2) Seasons (not including a Season covered by an Option Year) and, subject to Article II, Section 7, may provide in the first Salary Cap Year up to the greater of: (A) 175\% of the Regular Salary for the final Salary Cap Year covered by his prior Contract, plus 175\% of any Likely Bonuses and Unlikely Bonuses, respectively, called for in the final Salary Cap Year covered by the player's prior Contract, or (B) Salary plus Unlikely Bonuses totaling 108\% of the Average Player Salary for the prior Salary Cap Year (or if the Audit Report for the prior Salary Cap Year has not been completed, 108\% of the Average Player Salary for the prior Salary Cap Year as computed by substituting Estimated Total Salaries (as defined in Article VII, Section 1(i)) for Total Salaries). Notwithstanding the preceding sentence, if the player is an Early Qualifying Veteran Free Agent whose last Contract was his Rookie Scale Contract and whose Prior Team did not exercise its Option to extend such Contract for a third Season,the new Player Contract may provide for Regular Salary, Likely Bonuses and Unlikely Bonuses in the first Salary Cap Year of up to the Regular Salary, Likely Bonuses and Unlikely Bonuses, respectively, that the player would have received for such Salary Cap Year had his Prior Team exercised such Option. Annual increases and decreases in Salary and Unlikely Bonuses shall be governed by Section 5(c)(2) above.
  \item
    Notwithstanding anything to the contrary in Section 5(b)(2) above, if an Early Qualifying Veteran Free Agent with two (2) Years of Service receives an Offer Sheet in accordance with the provisions of Article XI, Section 5(c)(ii), the player's Prior Team may use the Early Qualifying Veteran Free Agent Exception to match the Offer Sheet.
  \end{enumerate}
\end{enumerate}

\begin{enumerate}
\def\labelenumi{(\alph{enumi})}
\setcounter{enumi}{2}
\tightlist
\item
  \textbf{Disabled Player Exception.}

  \begin{enumerate}
  \def\labelenumii{(\arabic{enumii})}
  \tightlist
  \item
    Subject to the rules set forth in Section 6(k) below, a Team may, in accordance with the rules set forth in this Section 6(c), sign or acquire one Replacement Player to replace a player who, as a result of a Disabling Injury or Illness (as defined below), is unable to render playing services (the ``Disabled Player'').

    \begin{enumerate}
    \def\labelenumiii{(\roman{enumiii})}
    \tightlist
    \item
      If a team wishes to sign a Replacement Player pursuant to this Section 6(c), such Replacement Player's Contract may provide Salary and Unlikely Bonuses for the first Salary Cap Year totaling up to the lesser of (A) 50\% of the Disabled Player's Salary at the time the Disabling Injury or Illness occurred, or (B) 108\% of the Average Player Salary for the prior Salary Cap Year (or, if the Audit Report for the prior Salary Cap Year has not been completed, 108\% of the Average Player Salary for the prior Salary Cap Year as computed by substituting Estimated Total Salaries (as defined in Article VII, Section 1(i) for Total Salaries). Annual increases and decreases in Salary and Unlikely Bonuses shall be governed by Section 5(c)(1) above.
    \item
      If a Team wishes to acquire a Replacement Player pursuant to this Section 6(c), the Replacement Player's post-assignment Salary for the Season in which the Replacement Player is acquired may be up to the lesser of the amount described in Section 6(c)(1)(i)(A) above or the amount described in Section 6c)(1)(i)(B) above, plus, in either case, \$100,000.
    \end{enumerate}
  \item
    For purposes of this Section 6(c), Disabling Injury or Illness means:

    \begin{enumerate}
    \def\labelenumiii{(\roman{enumiii})}
    \tightlist
    \item
      for the period July 1 through the immediately following November 30, any injury or illness that, in the opinion of the physician described in subsection (c)(5) below, will render a player unable to play all (or the remainder) of the Season immediately following such July 1; and
    \item
      for the period December 1 through the immediately following June 30, any injury or illness that, in the opinion of the physician described in subsection (c)(5) below, will render a player unable to play all of the following Season.
    \end{enumerate}
  \item
    The Exception for a Disabling Injury or Illness that occurs during the period July 1 through the immediately following November 30 shall arise on the earlier of (i) forty-five (45) days prior to the last day of the Regular Season immediately following such July 1, or (ii) the date the Team knew or reasonably should have known that the injury or illness would cause the player to miss the Season immediately following such July 1, and shall expire 45 days from the date the Exception arises.
  \item
    The Exception for a Disabling Injury or Illness that occurs during the period December 1 through the immediately following June 30 shall arise on the earlier of (i) forty-five (45) days prior to the October 1 immediately following the date on which the Disabling Injury or Illness occurs, or (ii) the date the Team knew or reasonably should have known that the injury or illness would cause the player to miss all of the following Season; provided, however, that if the Team knew or reasonably should have known prior to the July 1 immediately following the injury or illness that the injury or illness would cause the player to miss all of the following Season, and if the Team does not use the Exception prior to such July 1, then the Exception shall be deemed to arise on the day following the last day of the Moratorium Period. The Exception for a Disabling Injury or Illness that occurs during the period December 1 through the immediately following June 30 shall expire on the October 1 immediately following the date on which the Exception arises.
  \item
    The determination of whether a player has suffered a Disabling Injury or Illness shall be made by a physician designated by the NBA, who shall review the relevant medical information and, if the physician deems it appropriate, examine the player. The NBA shall advise the Players Association of the determination of its physician within one (1) business day of such determination. In the event the Players Association disputes the NBA physician's determination, the parties will immediately refer the matter to a neutral physician (to be selected by the parties at the commencement of each Salary Cap Year) to review the relevant medical information and, if the neutral physician deems it appropriate, examine the player. Within three (3) business days of his receipt of such information (and examination of the player, if requested), the neutral physician shall make a final determination, which will be final, binding and unappealable. The cost of the NBA physician will be borne by the NBA. The cost of the neutral physician will be borne jointly by the NBA and the Players Association.
  \item
    If a Team requests an Exception pursuant to this Section 6(c), the player with respect to whom the request is made shall cooperate in the processing of the request, including by appearing at the scheduled place and time for examination by the NBA-appointed physician and, if necessary, the neutral physician.
  \item
    Notwithstanding a Team's receipt of an Exception in respect of a Disabled Player pursuant to this Section 6(c), such player, upon recovering from his injury or illness, may resume playing for the Team. If the player resumes playing for the Team, or is traded, prior to the Team's use of its Exception, the Exception shall be extinguished.
  \item
    In no event may a Team enter into a Contract with a Replacement Player pursuant to Section 6(c)(4) above, unless the Disabled Player's Contract covers the Season following the Season in which the Disabling Injury or Illness occurs.
  \item
    The Disabled Player Exception is available only to the Team with which the player was under Contract at the time his Disabling Injury or Illness occurred.
  \item
    If a Team makes a request for an Exception to replace a Disabled Player pursuant to this Section 6(c) and such request is denied, the Team shall not be permitted to make any subsequent request for an Exception to replace the same player unless ninety (90) days have passed since the first request was denied and the Team establishes that the subsequent request is based on a new injury or an aggravation of the same injury.
  \end{enumerate}
\item
  \textbf{Bi-annual Exception} Subject to the rules set forth in Section 6(k) below:

  \begin{enumerate}
  \def\labelenumii{(\arabic{enumii})}
  \tightlist
  \item
    During each Salary Cap Year in which a Team is permitted to use the Bi-annual Exception, a Team may sign one (1) or more Player Contracts, not to exceed two (2) Seasons in length, that, in the aggregate, provide for Salaries and Unlikely Bonuses in the first Salary Cap Year totaling up to the amounts set forth below:

    \begin{enumerate}
    \def\labelenumiii{(\roman{enumiii})}
    \tightlist
    \item
      For the 2005-06 Salary Cap Year: \$1.67 million
    \item
      For the 2006-07 Salary Cap Year: \$1.75 million
    \item
      For the 2007-08 Salary Cap Year: \$1.83 million
    \item
      For the 2008-09 Salary Cap Year: \$1.91 million
    \item
      For the 2009-10 Salary Cap Year: \$1.99 million
    \item
      For the 2010-11 Salary Cap Year: \$2.08 million
    \item
      For the 2011-12 Salary Cap Year: \$2.18 million
    \end{enumerate}

    (if the NBA exercises its option to extend this Agreement pursuant to Article XXXIX)
  \item
    A Team may use all or any portion of the Bi-annual Exception to sign one (1) or more new Player Contracts during no more than three (3) separate Salary Cap Years during the term of this Agreement (or no more than four (4) separate Salary Cap Years if the NBA exercises its option to extend this Agreement pursuant to Article XXXIX); provided, however, that the Bi-annual Exception or any portion thereof may not be used in any two (2) consecutive Salary Cap Years. The prohibition in the preceding sentence against using the Bi-annual Exception or any portion thereof in any two (2) consecutive Salary Cap Years shall apply to the 2004-05 Salary Cap Year (i.e., if a Team used all or any portion of the Bi-annual Exception during the 2004-05 Salary Cap Year, that Team shall not be permitted to use all or any portion of the Bi-annual Exception during the 2005-06 Salary Cap Year).
  \item
    Player Contracts signed pursuant to the Bi-annual Exception covering two (2) Seasons may provide for an increase or decrease in Salary and Unlikely Bonuses for the second Season in accordance with Section 5(c)(1) above.
  \item
    The Bi-annual Exception, if applicable, arises on the day following the last day of the Moratorium Period of each Salary Cap Year and expires on the last day of the Team's Regular Season during that Salary Cap Year.
  \end{enumerate}
\item
  \textbf{Mid-Level Salary Exception.} Subject to the rules set forth in Section 6(k) below:

  \begin{enumerate}
  \def\labelenumii{(\arabic{enumii})}
  \tightlist
  \item
    A Team may sign one (1) or more Player Contracts during each Salary Cap Year, not to exceed five (5) Seasons in length, that, in the aggregate, provide for Salaries and Unlikely Bonuses in the first Salary Cap Year totaling up to 108\% of the Average Player Salary for the prior Salary Cap Year (or, if the Audit Report for the prior Salary Cap Year has not been completed, 108\% of the Average Player Salary for the prior Salary Cap Year as computed by substituting Estimated Total Salaries (as defined in Article VII, Section 1(i)) for Total Salaries).
  \item
    Player Contracts signed pursuant to the Mid-Level Salary Exception may provide for annual increases and decreases in Salary and Unlikely Bonuses in accordance with Section 5(c)(1) above.
  \item
    Notwithstanding anything to the contrary in Section 6(e)(2) above, if a Veteran Free Agent with one (1) or two (2) Years of Service receives an Offer Sheet in accordance with the provisions of Article XI, Section 5(c)(ii), the player's Prior Team may use the Mid-Level Exception to match the Offer Sheet.
  \item
    The Mid-Level Salary Exception shall arise on the day following the last day of the Moratorium Period of each Salary Cap Year and shall expire on the last day of the Team's Regular Season during that Salary Cap Year.
  \end{enumerate}
\item
  \textbf{Rookie Exception.} A Team may enter into a Rookie Scale Contract in accordance with Article VIII, Section 1.
\item
  \textbf{Minimum Player Salary Exception.} A Team may sign a player to, or acquire by assignment, a Player Contract, not to exceed two (2) Seasons in length, that provides for a Salary for the first Season equal to the Minimum Player Salary applicable to that player (with no Unlikely Bonuses). A Player Contract signed or acquired pursuant to the Minimum Player Salary Exception covering two (2) Seasons must provide for a Salary for the second Season equal to the Minimum Player Salary applicable to the player for such Season (with no Unlikely Bonuses).
\item
  \textbf{Traded Player Exception.}

  \begin{enumerate}
  \def\labelenumii{(\arabic{enumii})}
  \item
    Subject to the rules set forth in Section 6(k) below, a Team may, for a period of one year following the date of the trade of a Player Contract to another Team, replace the Traded Player with one (1) or more players acquired by assignment as follows:

    \begin{enumerate}
    \def\labelenumiii{(\roman{enumiii})}
    \tightlist
    \item
      A Team may replace a Traded Player with one (1) or more Replacement Players whose Player Contracts are acquired simultaneously and whose post-assignment Salaries for the then-current Salary Cap Year, in the aggregate, are no more than an amount equal to 125\% of the pre-trade Salary (or Base Year Compensation, if applicable) of the Traded Player, plus \$100,000.
    \item
      If a Team's trade of a player and acquisition of one (1) or more Replacement Players do not occur simultaneously, then the post-assignment Salary or aggregate Salaries of the Replacement Player(s) for the Salary Cap Year in which the Replacement Player(s) are acquired may not exceed 100\% of the pre-trade Salary (or Base Year Compensation, if applicable) of the Traded Player at the time the Traded Player's Contract was traded, plus \$100,000.
    \item
      A Team may aggregate the pre-trade Salaries in two (2) or more Player Contracts for the purpose of acquiring in a simultaneous trade one (1) or more Replacement Players whose post-trade Salaries, in the aggregate, are no more than an amount equal to 125\% of the pre-trade aggregated Salaries (or Base Year Compensations, if applicable) of the Traded Players, plus \$100,000. Notwithstanding the preceding sentence, no Player Contract acquired pursuant to an Exception may give rise to an aggregated trade exception for a period of two (2) months from the date the Player Contract is acquired.
    \end{enumerate}
  \item
    Except as provided in Section 6 (h)(3) below, and notwithstanding Section 6(k) below, a Team with a Team Salary below the Salary Cap may acquire one (1) or more players by assignment whose post-assignment Salaries, in the aggregate, are no more than an amount equal to the Team's Room plus \$100,000.
  \item
    In lieu of conducting the trade in accordance with Section 6(h)(2) above, and notwithstanding Section 6(k) below, a Team with a Team Salary below the Salary Cap may (i) replace a Traded Player with one (1) or more Replacement Players whose Player Contracts are acquired simultaneously and whose post-trade Salaries for the then-current Season, in the aggregate, are no more than an amount equal to 125\% of the pre-trade Salary of the Traded Player, plus \$100,000, or (ii) aggregate the pre-trade Salaries in two (2) or more Player Contracts for the purpose of acquiring in a simultaneous trade one (1) or more Replacement Players whose post-trade Salaries, in the aggregate, are no more than an amount equal to 125\% of the pre-trade aggregated Salaries of the Traded Players, plus \$100,000. Notwithstanding the preceding sentence, no Player Contract acquired pursuant to an Exception may be traded by a Team in accordance with this Section 6(h)(3) for a period of two (2) months from the date the Player Contract is acquired.
  \item
    \begin{enumerate}
    \def\labelenumiii{(\roman{enumiii})}
    \tightlist
    \item
      For purposes of the Traded Player Exception, a player shall be subject to a Base Year Compensation in the event that the Team Salary of the player's Team is at or above the Salary Cap and the player:

      \begin{enumerate}
      \def\labelenumiv{(\Alph{enumiv})}
      \tightlist
      \item
        is a Qualifying Veteran Free Agent or Early Qualifying Veteran Free Agent who, in accordance with Section 6(b)(1) or (3) above, enters into a new Player Contract with his prior Team that provides for a Salary for the first Season of such new Contract greater than 120\% of the Salary for the last Season of the player's immediately prior Contract (except that such a player shall not be subject to a Base Year Compensation if the new Contract provides for Salary equal to the Minimum Player Salary with no Unlikely Bonuses);
      \item
        is a First Round Pick who, in accordance with Section 7(b) below, enters into an Extension of his Rookie Scale Contract that provides for a Salary for the first Season of the extended term greater than 120\% of the Salary for the last Season of the original term of the Contract; or
      \item
        will be subject to a Base Year Compensation on some future date based upon an Extension entered into prior to the date of this Agreement. For purposes of Section 6(h)(4)(i)(A) above, if the player's immediately prior Contract was a one-year Contract that provided for Salary equal to the Minimum Player Salary (with no Unlikely Bonuses), the player's prior Salary shall include the portion of the Minimum Player Salary, if any, that was reimbursed out of the League-wide benefits fund described in Article IV, Section 5(k).
      \end{enumerate}
    \item
      A player's Base Year Compensation shall equal the greater of (A) the Salary for the last Season of his preceding Contract or, in the case of an Extension, the last Season of the original term of the Contract (the preceding amount hereinafter referred to as the ``Base Year Salary''), or (B) 50\% of the Salary for the first Season of his new Contract (or extended term, if applicable). A player's Base Year Compensation shall go into effect on the date the new contract is entered into or, in the case of an Extension, the day following the last day of the Moratorium Period that precedes the first Season of the extended term and shall expire and be of no further effect on the later of (x) the following June 30, or (y) six (6) months following the date the Base Year Compensation goes into effect.
    \item
      If a player has a Base Year Compensation that, pursuant to Section 6(h)(4)(ii) above, expires six (6) months after it goes into effect and during such six (6) month period the player signs a new Contract with his prior Team, the player shall continue to be subject to a Base Year Compensation in his new Contract until the conclusion of such six (6) month period. For purposes of computing such Base Year Compensation during the new Contract, the player's Base Year Salary shall equal the Base Year Compensation applicable under his prior Contract. Notwithstanding the foregoing, in the event the player is a Qualifying Veteran Free Agent or an Early Qualifying Veteran Free Agent, and the new Contract would itself subject the player to a Base Year Compensation in accordance with Section 6(h)(4)(i)(A) above, then the player shall be subject to a new Base Year Compensation for, the amount and duration of which shall be determined in accordance with Section 6(h)(4)(ii) above.
    \item
      A player's Base Year Compensation shall be extinguished upon any of the following:

      \begin{enumerate}
      \def\labelenumiv{(\Alph{enumiv})}
      \tightlist
      \item
        The Team Salary of the player's Team falls below the Salary Cap, unless this occurs prior to the beginning of an extended term described in Section 6(h)(4)(i)(B) or (C) above;
      \item
        The player signs a Contract with a Team other than his Prior Team; or
      \item
        The player is traded, unless the trade occurs prior to the beginning of an extended term described in Section 6(h)(4)(i)(B) or (C) above.
      \end{enumerate}
    \end{enumerate}
  \end{enumerate}
\item
  \textbf{Reinstatement.} If a player has been disqualified from further association with the NBA and subsequently reinstated pursuant to Article XXXIII (Anti-Drug Agreement), the Team for which the player last played may enter into a Player Contract with such player in accordance with the applicable rules set forth in Article XXXIII, Section 12(e) or (f), even if the Team has a Team Salary at or above the Salary Cap or such Player Contract causes the Team to have a Team Salary above the Salary Cap. If, in accordance with the preceding sentence, a Team and a player enter into a Player Contract and such Contract covers more than one Season, annual increases and decreases in Salary and Unlikely Bonuses shall be governed by Section 5(c)(1) above.
\item
  \textbf{Non-Aggregation.} Other than in accordance with Section 6(h) above, a Team may not aggregate or combine any of the Exceptions set forth above in order to sign or acquire one or more players at Salariesgreater than that permitted by any one of the Exceptions. If a Team has more than one (1) Exception available at the same time, the Team shall have the right to choose which Exception it wishes to use to sign or acquire a player.
\item
  \textbf{Other Rules.}

  \begin{enumerate}
  \def\labelenumii{(\arabic{enumii})}
  \tightlist
  \item
    A Team shall be entitled to use the Disabled Player, Bi-annual, Mid-Level Salary, and Assigned Player Exceptions set forth in Section 6(c), (d), (e) and (h) above, respectively, except as set forth in Section 6(h)(2) and (3) above, only if, at the time any such Exception would arise and at all times until it is used, the Team's Team Salary, excluding the amount(s) of such Exception and any other Exception that would be included in Team Salary pursuant to Section 6(k)(2) below, is (i) at or above the Salary Cap, or (ii) below the Salary Cap by less than the amount(s) of the Team's Exception(s).
  \item
    In the event that when a Disabled Player Exception, Bi-annual Exception, Mid-Level Salary Exception and/or Assigned Player Exception arises, the Team's Team Salary is below the Salary Cap (or in the event that, prior to the expiration of any such Exceptions, the Team's Team Salary falls below the Salary Cap) by less than the amount of such Exceptions, then (i) the Team's Team Salary shall include, until the Exceptions are actually used or until the Team no longer is entitled to use the Exceptions, the amount of the Exceptions (or any unused portion of the Exceptions), and (ii) the amount by which the Team's Team Salary is less than the Salary Cap shall thereby be extinguished. When the Disabled Player Exception is used to sign or acquire a player, the Replacement Player's Salary for the first Season of his Contract, instead of the amount of the Exception, shall be included in Team Salary. When a Bi-annual Exception or Mid-Level Salary Exception is used to sign a player, or when an Assigned Player Exception is used to acquire a player, the Salary for the first Season of the signed or acquired Contract plus any then-unused portion of the Exception, instead of the full amount of the Exception, shall be included in Team Salary. A Team may at any time renounce its rights to use an Exception, in which case the Exception (or any unused portion of the Exception) will no longer be included in Team Salary.
  \item
    Beginning on January 10 of each Season, each unused Exception, other than the Traded Player Exception, the Minimum Player Salary Exception (which is governed by Section 6(g) above and Article I, Section (gg)) and the Disabled Player Exception, shall be reduced daily throughout the end of the Regular Season by the product of the amount of the unused Exception as of January 10 multiplied by a fraction, the numerator of which is one (1) and the denominator of which is the total number of days in such Regular Season.
  \end{enumerate}
\end{enumerate}

\hypertarget{extensions-renegotiations-and-other-amendments.}{%
\section{Extensions, Renegotiations and Other Amendments.}\label{extensions-renegotiations-and-other-amendments.}}

\begin{enumerate}
\def\labelenumi{(\alph{enumi})}
\tightlist
\item
  \textbf{Veteran Extensions.} No Player Contract, other than a Rookie Scale Contract, may be extended except in accordance with the following:

  \begin{enumerate}
  \def\labelenumii{(\arabic{enumii})}
  \item
    \begin{enumerate}
    \def\labelenumiii{(\roman{enumiii})}
    \tightlist
    \item
      With respect to Contracts entered into prior to the date of this Agreement, subject to the rules set forth in Section 7(a)(2) below, a Player Contract covering a term of six (6) or seven (7) Seasons may be extended no sooner than the fourth anniversary of the signing of the Contract, and a Player Contract with a term of four (4) or five (5) Seasons may be extended no sooner than the third anniversary of the signing of the Contract.
    \item
      With respect to Contracts entered into after the date of this Agreement, subject to the rules set forth in Section 7(a)(2) below, a Player Contract covering a term of four (4), five (5) or six (6) Seasons may be extended no sooner than the third anniversary of the signing of the Contract.
    \end{enumerate}
  \item
    \begin{enumerate}
    \def\labelenumiii{(\roman{enumiii})}
    \tightlist
    \item
      A Player Contract that has been extended, or that has been renegotiated to provide for an increase in Salary in any Salary Cap Year covered by the Contract of more than 10\%, may not subsequently be extended until the third anniversary of such Extension or Renegotiation.
    \item
      A Team and a player shall not be permitted to extend any Player Contract whose term has been shortened as a result of the player's exercise of an Early Termination Option.
    \item
      Subject to the rules set forth in this Section 7(a), a Contract may be extended following the non-exercise of an Option by a player or Team only if:

      \begin{enumerate}
      \def\labelenumiv{(\Alph{enumiv})}
      \tightlist
      \item
        The extended term covers a minimum of two (2) Seasons (excluding any new Option Year); and
      \item
        The player's Regular Salary, Likely Bonuses and Unlikely Bonuses in the first Salary Cap Year covered by the extended term are no less than the Regular Salary, Likely Bonuses and Unlikely Bonuses, respectively, that the player would have received for such Salary Cap Year had the Option been exercised.
        In order to effectuate an Extension of the type described in this Section 7(a)(2)(iii), a Team and player may amend a Contract to provide simultaneously for the non-exercise of the Option and the Extension.
      \end{enumerate}
    \end{enumerate}
  \item
    Subject to Article II, Section 7, a Player Contract extended in accordance with this Section 7(a) may, in the first Salary Cap Year covered by the extended term, provide for a Salary, excluding Incentive Compensation, of up to 110.5\% of the Regular Salary in the last Salary Cap Year covered by the original term of the Contract. In the event that the last Salary Cap Year covered by the original term of the Contract provides for Incentive Compensation, the first Salary Cap Year covered by the extended term may provide for Likely Bonuses and Unlikely Bonuses of up to 110.5\% of the Likely Bonuses and Unlikely Bonuses, respectively, in the last Salary Cap Year covered by the original term. Annual increases and decreases in Salary and Unlikely Bonuses shall be governed by Section 5(c)(3) above.
  \item
    Notwithstanding Section 7(a)(3) above and subject to Article II, Section 7, any Player Contract of a player who has played for his current Team for at least ten (10) Seasons and whose Salary in the last Salary Cap Year covered by the original term of the Contract is less than the Salary in the second-to-last Salary Cap Year covered by such Contract may, in the first Salary Cap Year covered by an extended term, provide for a Salary equal to 110.5\% of the greater of (1i the average of the Regular Salaries for each Salary Cap Year covered by the original Contract beginning with the Salary Cap Year in which such Contract was entered into, or previously extended, as the case may be, or (ii) the Regular Salary in the last Salary Cap Year covered by his original Contract. In the event that the last Salary Cap Year covered by the original term of the Contract provides for Incentive Compensation, the first Salary Cap Year covered by the extended term may provide for Likely Bonuses and Unlikely Bonuses of up to 110.5\% of the Likely Bonuses and Unlikely Bonuses, respectively, in the last Salary Cap Year covered by the original term. Annual increases and decreases in Salary and Unlikely Bonuses shall be governed by Section 5(c)(3) above, except that, for purposes of Section 7(a)(4)(ii) only,the phrase ``Regular Salary'' in Section 5(c)(3) above shall be deemed to mean the greater of (A) the average of the Regular Salaries for each Salary Cap Year covered by the original Contract beginning with the Salary Cap Year in which such Contract was entered into, or previously extended, as the case may be, or (B) the Regular Salary in the last Salary Cap Year covered by the original Contract.
  \end{enumerate}
\item
  \textbf{Rookie Scale Extensions.} No Rookie Scale Contract may be extended except in accordance with the following:

  \begin{enumerate}
  \def\labelenumii{(\arabic{enumii})}
  \tightlist
  \item
    A First Round Pick who entered into a Rookie Scale Contract that commenced with a Season prior to the 2005-06 Season may enter into an Extension of such Rookie Scale Contract during the period from the day following the last day of the Moratorium Period through October 31 of the Option Year provided for in such Contract (assuming the Team exercises such Option).
  \item
    A First Round Pick who enters into a Rookie Scale Contract that commences with the 2005-06 Season or a subsequent Season may enter into an Extension of such Rookie Scale Contract during the period from the day following the last day of the Moratorium Period through October 31 of the second Option Year provided for in such Contract (assuming the Team exercises such Option).
  \item
    An Extension of a Rookie Scale Contract may provide for Salary and Unlikely Bonuses in the first Salary Cap Year covered by the extended term totaling no more than the maximum amount provided for in Article II, Section 7. Annual increases and decreases in Salary and Unlikely Bonuses shall be governed by Section 5(c)(4) above.
  \end{enumerate}
\item
  \textbf{Renegotiations.} No Player Contract may be renegotiated except in accordance with the following:

  \begin{enumerate}
  \def\labelenumii{(\arabic{enumii})}
  \tightlist
  \item
    Subject to Section 7(c)(2) and (3) below, a Player Contract covering a term of four (4) or more Seasons may be renegotiated no sooner than the third anniversary of the signing of the Contract.
  \item
    Subject to Section 7(c)(3) below, any Player Contract that has been renegotiated in accordance with Section 7(c)(1) above to provide for an increase in Salary or Incentive Compensation in any Salary Cap Year covered by the Contract of more than 8\%, or extended in accordance with Section 7(a) or (b) above, may not subsequently be renegotiated until the third anniversary of such Extension or Renegotiation.
  \item
    Assuming Section 7(c) (1) or (2) above are satisfied, a Team with a Team Salary below the Salary Cap may renegotiate a Player Contract in accordance with the following rules:

    \begin{enumerate}
    \def\labelenumiii{(\roman{enumiii})}
    \tightlist
    \item
      Subject to Article II, Section 7, the Renegotiation may provide for additional Regular Salary, Likely Bonuses and/or Unlikely Bonuses for the then-current Salary Cap Year covered by the Contract (the ``Renegotiation Season'') that, in the aggregate, would not exceed the Team's Room at the time of the Renegotiation.
    \item
      Every category (Regular Salary, Likely Bonuses and Unlikely Bonuses, respectively) that is increased for the Renegotiation Season must also be increased for each of the remaining Seasons of the Contract. For each Season of the Contract after the Renegotiation Season, the player's additional Regular Salary may increase or decrease over the previous Season's additional Regular Salary by no more than 10.5\% of the additional Regular Salary provided for in the Renegotiation Season. In the event that the Renegotiation Season provides for additional Incentive Compensation, the amount of additional Likely Bonuses and Unlikely Bonuses provided for in each Season after the Renegotiation Season may increase or decrease by up to 10.5\% of the amount of additional Likely Bonuses and Unlikely Bonuses, respectively, provided for in the Renegotiation Season.
    \item
      No Renegotiation may contain a signing bonus, unless the Renegotiation is accompanied by an Extension and the signing bonus would otherwise be permitted under the rules governing the inclusion of signing bonuses in Extensions.
    \end{enumerate}
  \item
    In no event may a Team with a Team Salary at or above the Salary Cap renegotiate a Player Contract.
  \item
    In no event may a Team and a player renegotiate a Player Contract from March 1 through June 30 of any Salary Cap Year.
  \end{enumerate}
\item
  \textbf{Other.}

  \begin{enumerate}
  \def\labelenumii{(\arabic{enumii})}
  \tightlist
  \item
    In no event shall a Team and player negotiate a decrease in Salary or in any Incentive Compensation for any Salary Cap Year covered by a Player Contract.
  \item
    A Player Contract that is extended pursuant to Section 7(a) above may be renegotiated simultaneously, but only if and to the extent permitted by the rules set forth in Section 7(c) above.
  \item
    For the sole purpose of enabling an assignee Team to acquire a Player Contract by trade, the player and the assignor Team may agree to waive all or any portion of a trade bonus, but only to the extent necessary to make the trade permissible in accordance with the rules set forth in Section 6(h) above. In the event that, in connection with a trade, a player's Contract is amended in accordance with this Section 7(d)(3), such Contract may not be subsequently extended or renegotiated until the later of (i) six (6) months from the date of the trade, or (ii) the first date on which the Contract could otherwise be extended or renegotiated pursuant to this Section 7.
  \item
    In the event that a Team and a player agree to amend a Player Contract in accordance with Article II, Section 3(l), then for purposes of calculating the player's Salary for the then-current and any remaining Salary Cap Year covered by the Contract, notwithstanding any stretch or acceleration of the player's protected Compensation payment schedule, the aggregate reduction in the player's protected Compensation shall be allocated pro rata over the then-current and each remaining Salary Cap Year on the basis of the remaining unearned protected Base Compensation in each such Salary Cap Year.
  \item
    In no event shall a Team and player amend a Contract for the purpose of terminating or shortening the term of the Contract, except in accordance with the NBA waiver procedure or Article XII, Section 2.
  \end{enumerate}
\end{enumerate}

\hypertarget{trade-rules.}{%
\section{Trade Rules.}\label{trade-rules.}}

\begin{enumerate}
\def\labelenumi{(\alph{enumi})}
\item
  A Team shall not be permitted to receive in connection with any trade, directly or indirectly, more than \$3 million in cash or other compensation, including cash or other compensation received as reimbursement for Compensation obligations to players who the Team is acquiring. For purposes of this Section 8(a), if a Contract is signed and then traded pursuant to Section 8(e)(1) below, and the Contract contains a signing bonus, the payment of all or any portion of such bonus by the Team that signed the Contract shall be treated as a reimbursement of a Compensation obligation of the assignee Team and shall be subject to this Section 8(a).
\item
  A player with a one-year Contract (excluding any Option Year) who would be a Qualifying Veteran Free Agent or an Early Qualifying Veteran Free Agent upon completing the playing services called for under his Contract cannot be traded without the player's consent. Should the player consent and be traded, then, for purposes of determining whether the player is a Qualifying Veteran Free Agent, Early Qualifying Veteran Free Agent or Non-Qualifying Veteran Free Agent at the conclusion of the Contract or any subsequent Contract between the player and the assignee Team, the player shall be considered as having changed Teams by means of signing a Contract with the assignee Team as a Free Agent (and not by means of trade).
\item
  A Team cannot trade any player after the NBA trade deadline occurring in the last Season of the player's Contract, or after the NBA trade deadline occurring in any Season that could be the last Season of the player's Contract based upon the exercise or non-exercise of an Option or Early Termination Option.
\item
  Except as set forth in Section 8(e) below: (1) no player who signs a Contract as a Free Agent may be traded before the later of (i) three (3) months following the date on which such Contract was signed or (ii) the December 15 of the Salary Cap Year in which such Contract was signed; and (2) no Draft Rookie who signs a Player Contract may be traded before thirty (30) days following the date on which the Contract is signed.
\item
  \begin{enumerate}
  \def\labelenumii{(\arabic{enumii})}
  \tightlist
  \item
    A Veteran Free Agent and his Prior Team may enter into a Player Contract pursuant to an agreement between the Prior Team and another Team concerning the signing and subsequent trade of such Contract, but only if (i) the Contract is for three (3) or more Seasons (excluding any Option Year), (ii) the Contract is not signed pursuant to the Mid-Level Salary Exception or the Disabled Player Exception, (iii) the first Season of the Contract is fully protected for lack of skill, and (iv) the acquiring Team has Room for the player's Salary plus any Unlikely Bonuses provided for in the first Season of the Contract.
  \item
    A player and his Team may amend a Player Contract (including by entering into an Extension) pursuant to an agreement between such Team and another Team concerning the signing of the amendment and subsequent trade of the amended Contract; provided, however, that no such agreement may be made during the period from the last day of the last Regular Season covered by the Contract (or the last day of any Regular Season that could be the last Regular Season covered by the Contract based upon the exercise or non-exercise of an Option or ETO) through the following June 30.
  \item
    A Player Contract or Extension entered into pursuant to Section 8(e)(1) or (2) above may not contain an Exhibit 6 thereto. However, the preceding sentence shall not prohibit the Teams involved in the trade from agreeing that the trade (and thus the validity of the Player Contract or Extension) will be conditional upon the passage of a physical examination to be performed by a physician designated by the assignee-Team in accordance with NBA procedures.
  \end{enumerate}
\item
  In the event a Rookie Scale Contract is extended pursuant to Section 7(b) above and a Team proposes to trade such Contract to another Team prior to the July 1 immediately \textbf{\emph{{[}sic{]}}}
\item
  If a Team trades a player and the assignee Team subsequently terminates the Player's Contract, the assignor Team shall not be permitted to sign the player to a new Contract until at least thirty (30) days have passed following the date all conditions to the trade were satisfied (or until at least twenty (20) days have passed in the case of a trade that occurs during the period from the last day of a Season through the first day of the following Regular Season).
\item
  Prior to the assignment of any Player Contract, the Team from which such Player Contract is to be assigned and the player whose Player Contract is to be assigned shall be required to divest themselves, on terms mutually agreeable to the player and the Team, of any pre-existing financial arrangements between such Team and such player. The foregoing shall not apply to Compensation earned by the player prior to the assignment or to loans.
\item
  Within one (1) week following each trade, the NBA shall send to the Players Association, by fax or e-mail, a summary of the principal terms of the trade; provided, however, that the NBA may omit from such summary any terms that the NBA or one (1) or more Teams involved in the trade reasonably deem confidential (other than such terms as may be necessary to verify the Teams' compliance with Section 8(a) above).
\end{enumerate}

\hypertarget{miscellaneous.}{%
\section{Miscellaneous.}\label{miscellaneous.}}

\begin{enumerate}
\def\labelenumi{(\alph{enumi})}
\tightlist
\item
  Except where this Agreement states otherwise, for purposes of any rule in this Agreement that limits, involves counting, or otherwise relates to, the number of Seasons covered by a Contract:

  \begin{enumerate}
  \def\labelenumii{(\arabic{enumii})}
  \tightlist
  \item
    If a Player Contract is signed after the beginning of a Season, the Season in which the Contract is signed shall be counted as one (1) full Season covered by the Contract.
  \item
    An Option Year shall be counted as one (1) Season covered by the Contract.
  \end{enumerate}
\item
  Except where this Agreement states otherwise, all of the rules in this Agreement that limit, affect the calculation of, or otherwise relate to, the Compensation or Salary provided for in a Player Contract shall apply to Option Years.
\end{enumerate}

\hypertarget{accounting-procedures.}{%
\section{Accounting Procedures.}\label{accounting-procedures.}}

\begin{enumerate}
\def\labelenumi{(\alph{enumi})}
\item
  \begin{enumerate}
  \def\labelenumii{(\arabic{enumii})}
  \tightlist
  \item
    The NBA and the Players Association shall jointly engage an independent auditor (the ``Accountants'') to provide the parties with an ``Audit Report'' (and a ``Draft Audit Report,'' and, if applicable, an ``Interim Audit Report'' and, if applicable, an ``Interim Escrow Audit Report'') setting forth BRI, and Total Salaries and Benefits for the immediately preceding Salary Cap Year and the information called for by Section 12 below (the ``Escrow Information''). The audit reports provided for by this Section 10(a)(1) are to be prepared in accordance with the provisions and definitions contained in this Agreement. The engagement of the Accountants shall be deemed to be renewed annually unless they are discharged by either party during the period from the submission of an Audit Report up to January 1 of the following year. The parties agree to share equally the costs incurred by the Accountants in preparing the audit reports provided for by this Section 10(a)(1).
  \item
    The Accountants shall submit a ``Draft Audit Report'' for each Salary Cap Year to the NBA and the Players Association, along with relevant supporting documentation, two (2) weeks prior to the scheduled issuance of the final Audit Report.
  \item
    The final Audit Report shall be submitted by the Accountants to the parties on or before the last day of the Moratorium Period following the conclusion of the Salary Cap Year. The Audit Report shall not be deemed final until the parties have confirmed in writing their agreement (in a form acceptable to the parties) with such Report. The NBA, the Players Association and the Teams shall use their best efforts to facilitate the Accountants' timely completion of the Audit Report.
  \item
    In the event that, for any reason, the Accountants fail to submit to the parties a final Audit Report by the last day of the Moratorium Period, the Accountants shall prepare an interim Audit Report (the ``Interim Audit Report'') by such date setting forth the Accountants' best estimate of BRI and Total Salaries and Benefits for the preceding Salary Cap Year and, based upon such best estimates, the Escrow Information. Such Interim Audit Report shall
    include:

    \begin{enumerate}
    \def\labelenumiii{(\roman{enumiii})}
    \setcounter{enumiii}{1}
    \tightlist
    \item
      \textbf{\emph{{[}sic{]}}} All amounts of BRI and Total Salaries and Benefits (or the portions thereof) and all Escrow Information (or the portions thereof) for such Salary Cap Year as to which the Accountants have completed their review and, by written agreement of the Players Association and the NBA (waiving their respective rights to dispute such amounts), are not in dispute.
    \item
      With respect to any amounts of BRI or Total Salaries and Benefits (or portions thereof) as to which the Accountants have not completed their review or which are the subject of a good faith dispute between the parties, the NBA's good faith proposal as to the proper amount, if any, that should be included in the Audit Report.
    \item
      With respect to any items of Escrow Information that are the subject of a good faith dispute between the parties, the Accountants' good faith determination as to such items, taking into account the provisions of Section 10(a)(2) (i) and (ii).
    \end{enumerate}
  \end{enumerate}

  As soon as practicable after the Interim Audit Report is submitted to the parties, the Accountants shall submit the final Audit Report, including a description of the differences, if any, from the Interim Audit Report. The Audit Report shall not be deemed final until the parties have confirmed in writing their agreement (in a form acceptable to the parties) with such Report or all disputes with respect to such Report have been finally resolved by means of the dispute-resolution procedures provided for by this Agreement.

  If, at the conclusion of the Audit Report Challenge Period (as defined by Section 12(b)(4) below), the Accountants have not submitted or are unable to submit a final Audit Report (because, by way of example but not limitation, there are disputes or claims that have been asserted pursuant to Article XXXII, Section 9(c) and which remain pending), the Accountants shall prepare and submit to the parties, within five (5) business days following the completion of the Audit Report Challenge Period, an Interim Escrow Audit Report that shall include the information set forth in the Interim Audit Report as adjusted or amended so as to reflect any final determinations made by the System Arbitrator or the Appeals Panel (as the case may be) in proceedings commenced pursuant to Article XXXII, Section 9(b) and involving disputes or claims with respect to such Interim Audit Report. The sole purpose for which any Interim Escrow Audit Report is to be used under this Agreement is to perform or form the basis for the calculations to be made pursuant to Article VII, Section 12 below.
\item
  For purposes of determining BRI, Total Salaries and Benefits and the Escrow Information, the Accountants shall perform at least such review procedures as shall be agreed upon by the parties. In connection with the preparation of Audit Reports for each Salary Cap Year, each Team and the NBA shall submit a report to the Accountants, the NBA and the Players Association setting forth BRI, Team Salaries and Benefits information for such Salary Cap Year, on forms agreed upon by the NBA, the Players Association and the Accountants (the ``BRI Reports''). The NBA and the Players Association shall agree upon such forms no later than April 1 of each Salary Cap Year.
\item
  The Accountants shall review the reasonableness of any estimates of revenues or expenses for a Salary Cap Year included in the Teams' and the NBA's BRI Reports for such Salary Cap Year and may make such adjustments in such estimates as they deem appropriate. To the extent the actual amounts of revenues received or expenses incurred for a Salary Cap Year differ from such estimates, adjustments shall be made in BRI for the following Salary Cap Year in accordance with the provisions of Section 10(f) below.
\item
  With respect to expenses deducted by the NBA or the Teams, the NBA and the Teams shall report in BRI Reports only those expenses that are reasonable and customary in accordance with the provisions of Section 1(a)(1) above. Subject to the terms of Section 1(a)(6) above and Section 11 below, all categories of expenses deducted in a BRI Report completed by the NBA or a Team shall be reviewed by the Accountants, but such categories shall be presumed to be reasonable and customary and the amount of the expenses deducted by the NBA or a Team that come within such expense categories shall also be presumed to be reasonable and customary, unless such categories or amounts are found by the Accountants to be either unrelated to the revenues involved or grossly excessive.
\item
  The Accountants shall notify designated representatives of the NBA and the Players Association: (1) if the Accountants have any questions concerning the amounts of revenues or expenses reported by the Teams and the NBA or any other information contained in the BRI Reports; or (2) if the Accountants propose that any adjustments be made to any revenue or expense item or any other information contained in the BRI Reports.
\item
  The Accountants shall indicate which amounts included in BRI for a Salary Cap Year, if any, represent estimates of revenues. With respect to any such estimated revenues, the Accountants shall, in preparing the Audit Report for the immediately succeeding Salary Cap Year (``Subsequent Audit Report''), or the Audit Report for the same Salary Cap Year in the event that an Interim Audit Report was previously issued for that Salary Cap Year, determine the actual revenues received for the prior Salary Cap Year and include as a credit or debit to BRI in such Subsequent Audit Report the amount of the aggregate difference, if any, between all such estimated revenues for the prior Salary Cap Year and the actual revenues received for such Salary Cap Year (the ``Estimated Revenue Adjustment'').
\item
  In the event that in the course of preparing an Audit Report for a Salary Cap Year the Accountants discover that they committed an error in computing BRI in the Audit Reports for either of the two previous Salary Cap Years, which error resulted in a material understatement or overstatement of BRI for either of such Salary Cap Years, and the parties agree that such error was committed and agree as to the amount of the resulting understatement or overstatement (or, if they do not agree, an error (and the amount of such error) is established pursuant to the dispute resolution procedures provided for in this Agreement) the amount of such understatement or overstatement of BRI shall be added to or subtracted from BRI, as the case may be, with interest (at a rate equal to the one year Treasury Bill rate as published in the Wall Street Journal on the date of the issuance of such Audit Report) accruing from the date of the Audit Report for the Salary Cap Year in which such understatement or overstatement occurred in equal annual amounts over the then-current and subsequent Salary Cap Year. Notwithstanding the foregoing, the parties will jointly instruct the Accountants that their audits shall not include procedures specifically designed to detect errors committed in prior audits.
\item
  In the event that there is an NHL players' strike or owners' lockout (``work stoppage'') resulting in the cancellation of all or part of any NHL season in any Salary Cap Year, and such work stoppage results in a refund being made to luxury suite-holders, premium seat license-holders or to purchasers of fixed arena signage and/or naming rights in arenas in which both an NBA Team and an NHL team plays its home games, then the revenues for luxury suites, premium seat licenses and fixed arena signage and/or naming rights in such arenas shall be determined as if such refunds were not made. If the work stoppage continues for a second year, then the NHL revenues shall be deemed to be the amount included for the prior year.
\item
  All disputes with respect to any Interim Audit Report shall be resolved exclusively in accordance with the procedures set forth in Article XXXII.
\end{enumerate}

\hypertarget{players-association-audit-rights.}{%
\section{Players Association Audit Rights.}\label{players-association-audit-rights.}}

\begin{enumerate}
\def\labelenumi{(\alph{enumi})}
\tightlist
\item
  \textbf{Team Audits.} The Players Association shall have the right as part of the annual review of BRI Reports to retain its own accountants (the ``Players Association's Accountants''), at its own expense, after the submission of each Audit Report under this Agreement (the ``First Audit''), to audit the books and records of five (5) NBA teams (of its choosing) and shall also have the right to review the books and records of the NBA League Office; provided, however, that such review shall be limited to (1) revenue items, and (2) expense items that appear or should have appeared in the BRI Reports. In the event that, in the opinion of the Players Association's Accountants, such audit indicates misallocations or miscategorizations of revenues or expenses (other than with respect to matters that constituted Disputed Adjustments in connection with the prior Audit Report) resulting in an understatement of BRI in excess of \$3 million, they shall submit to the NBA proposed adjustments to BRI consistent with their findings. In the event that the NBA disputes such proposed adjustments, such proposed adjustments shall be deemed to be ``Disputed Adjustments'' and shall be resolved in accordance with the procedures set forth in Article XXXII. In addition, in the event that First Audit Disputed Adjustments in excess of \$3 million are resolved in favor of the Players Association, the Players Association shall then have the right, that Season, to have the Players Association's Accountants audit an additional five (5) NBA teams, in accordance with the foregoing procedures (the ``Second Audit''). If, as a result of the Second Audit, additional Disputed Adjustments in excess of \$3 million are resolved in favor of the Players Association, the Players Association shall then have the right, that Season, to have the Players Association's Accountants audit all remaining NBA Teams. The amount of any and all Disputed Adjustments that are ultimately resolved in favor of the Players Association in accordance with this Section 11 (a) shall be added to BRI in the Season in which such resolution is reached.
\item
  \textbf{Expense Audit.} The Players Association shall have the right to retain the Players Association's Accountants to conduct one (1) audit, at its own expense, of the expenses incurred in connection with the proceeds that come within Article VII, Section 1(a)(1)(viii) above regardless of whether such expenses exceed the applicable Expense Ratios set forth in Exhibit D. In the event that in the opinion of the Players Association's Accountants, such audit indicates a misallocation or miscategorization of expenses resulting in an understatement of BRI, they shall submit proposed adjustments to the NBA consistent with their findings. In the event the NBA disputes such proposed adjustments, such proposed adjustments shall be deemed to be Disputed Adjustments and resolved in accordance with the procedures set forth in Article XXXII. The amount of any and all such Disputed Adjustments that are resolved in the Players Association's favor shall be included in BRI in the Season in which such resolution is reached. In addition, in the event that any such Disputed Adjustments are resolved in the Players Association's favor, the Accountants shall be directed to correct such expense misallocations and/or miscategorizations in the remaining Seasons covered by the Agreement.
\item
  \textbf{Confidentiality.} In connection with any audit conducted by the Players Association pursuant to this Section 11, the Players Association agrees to sign, and to cause its representatives to sign, a confidentiality agreement in the form annexed hereto as Exhibit J-1. The Players Association also agrees to sign, and to cause its representatives to sign, a similar confidentiality agreement with respect to information obtained in connection with the Accountants' audit pursuant to Section 10 above.
\end{enumerate}

\hypertarget{escrow-and-tax-arrangement.}{%
\section{Escrow and Tax Arrangement.}\label{escrow-and-tax-arrangement.}}

\begin{enumerate}
\def\labelenumi{(\alph{enumi})}
\tightlist
\item
  \textbf{Definitions.} As used in this Agreement, the following terms shall have the following meanings:

  \begin{enumerate}
  \def\labelenumii{(\arabic{enumii})}
  \item
    ``Adjustment Player'' means, with respect to a Salary Cap Year, every current or former player included in a Team's Team Salary for such Salary Cap Year and every player who is excluded from Team Salary pursuant to Section 4(h) above.
  \item
    ``Aggregate Compensation Adjustment Amount'' means, with respect to a Salary Cap Year, the lesser of (i) the Aggregate Salaries and Benefits Adjustment Amount, and (ii) a percentage of Total Salaries for such Salary Cap Year as follows:

    \begin{longtable}[]{@{}cc@{}}
    \toprule()
    Salary Cap Year & \% of Total Salaries \\
    \midrule()
    \endhead
    2005-06 & 10\% \\
    2006-07 & 9\% \\
    2007-08 & 9\% \\
    2008-09 & 9\% \\
    2009-10 & 9\% \\
    2010-11 & 8\% \\
    2011-12 & 8\% \\
    \bottomrule()
    \end{longtable}

    (if the NBA exercises its option to extend this Agreement pursuant to Article XXXIX)

    For purposes of clause (ii) above, Total Salaries shall be increased by the amount that is included in Benefits for such Salary Cap Year pursuant to Article IV, Section 5(k)(1) of this Agreement.
  \item
    ``Aggregate Compensation Adjustment Amount Shortfall'' means the amount by which the amount received by the NBA from the Escrow Agent with respect to a Salary Cap Year pursuant to Section 12(d)(1) below is less than the Aggregate Compensation Adjustment Amount for such Salary Cap Year.
  \item
    ``Aggregate Salaries and Benefits Adjustment Amount'' means, with respect to a Salary Cap Year, the lesser of (i) the Overage (as defined in Section 12(a)(14) below) for such Salary Cap Year, and (ii) a percentage of Total Salaries and Benefits for such Salary Cap Year as follows:

    \begin{longtable}[]{@{}cc@{}}
    \toprule()
    Salary Cap Year & \% of Total Salaries and Benefits \\
    \midrule()
    \endhead
    2005-06 & 10\% \\
    2006-07 & 9\% \\
    2007-08 & 9\% \\
    2008-09 & 9\% \\
    2009-10 & 9\% \\
    2010-11 & 8\% \\
    2011-12 & 8\% \\
    \bottomrule()
    \end{longtable}

    (if the NBA exercises its option to extend this Agreement pursuant to Article XXXIX)
  \item
    ``Audit Report Challenge Period'' means the period beginning with the date on which an Interim Audit Report is issued by the Accountants and ending on the last date by which all challenges thereto brought pursuant to Article XXXII, Section 9(b) are resolved.
  \item
    ``Deduction Date'' means each of the twelve (12) semi-monthly payment dates from November 15 through May 1 provided for under paragraph 3 of the Uniform Player Contract.
  \item
    ``Designated Percentage'' means, with respect to a Salary Cap Year, the percentage set forth in Section 12(b)(3) below.
  \item
    ``Escrow Agent'' means, for purposes of this Section 12, the escrow agent identified in the Salary Escrow Agreement.
  \item
    ``Escrow Amount'' means for an Adjustment Player, with respect to a Salary Cap Year, an amount equal to the sum of the player's ``Base Escrow amount'' (as defined below) plus the player's ``Performance Bonus Escrow Amount'' (as defined below) plus the player's ``Trade Bonus Escrow Amount'' (as defined below). ``Base Escrow Amount'' means for an Adjustment Player, with respect to a Salary Cap Year, an amount equal to a percentage of the Adjustment Player's Salary for such Salary Cap Year as follows:

    \begin{longtable}[]{@{}cc@{}}
    \toprule()
    Salary Cap Year & Percentage \\
    \midrule()
    \endhead
    2005-06 & 10\% \\
    2006-07 & 9\% \\
    2007-08 & 9\% \\
    2008-09 & 9\% \\
    2009-10 & 9\% \\
    2010-11 & 8\% \\
    2011-12 & 8\% \\
    \bottomrule()
    \end{longtable}

    (if the NBA exercises its option to extend this Agreement pursuant to Article XXXIX)

    For purposes of calculating an Adjustment Player's Base Escrow Amount in accordance with the preceding sentence: (i) a player's Salary shall exclude all Performance Bonuses included in Salary under Section 3(d) above; and (ii) the Salary of a player under a one-year Contract making the Minimum Player Salary shall include the portion of such Minimum Player Salary that is reimbursed out of the League-wide benefits fund described in Article IV, Section 5(k)(2). ``Performance Bonus Escrow Amount'' means for an Adjustment Player, with respect to a Salary Cap Year, an amount equal to a percentage of the Performance Bonuses earned by such player during such Salary Cap Year specified in the table above. ``Trade Bonus Escrow Amount'' means for an Adjustment Player whose Contract contains a trade bonus, with respect to a Salary Cap year in which the player's Contract is traded during the period following the conclusion of the Team's Season through June 30, an amount equal to a percentage of the portion of the trade bonus that is included in Salary for such Salary Cap year as specified in the table above.
  \item
    ``Escrow Schedules'' means the schedules prepared by the NBA with respect to each Salary Cap Year setting forth: the Base Escrow Amount; when calculable following the conclusion of the applicable Regular Season, the Performance Bonus Escrow Amount; when calculable, the Trade Bonus Escrow Amount; and the dates on which each Adjustment Player's Base Escrow Amount, Performance Bonus Escrow Amount and Trade Bonus Escrow Amount are to be deducted from the player's Compensation and/or delivered to the Escrow Agent.
  \item
    ``Individual Compensation Adjustment Amount'' means for an Adjustment Player, with respect to a Salary Cap Year, the amount calculated following the conclusion of the Salary Cap Year by multiplying the Aggregate Compensation Adjustment Amount for such Salary Cap Year by a fraction, the numerator of which is the Adjustment Player's Salary for such Salary Cap Year and the denominator of which is the sum of all Adjustment Players' Salaries for such Salary Cap Year. For purposes of calculating the fraction described in the preceding sentence: (i) a player's Salary shall include all Performance Bonuses excluded from Salary under Section 3(d) above but actually earned by the player during such Salary Cap Year, and shall exclude all Performance Bonuses included in Salary under Article VII, Section 3(d) above but not actually earned by the player during such Salary Cap Year; and (ii) the Salary of a player under a one-year Contract making the Minimum Player Salary shall include the portion of such Minimum Player Salary that is reimbursed out of the league-wide benefits fund described in Article IV, Section 5(k)(2).
  \item
    ``Individual Shortfall Adjustment Amount'' means, with respect to each Contract that is amended pursuant to Section 12(e)(1) below, the amount that the Compensation otherwise payable in accordance with that Contract shall be reduced pursuant to Section 12(e)(2) below.
  \item
    ``New Benefit Adjustment Amount'' means an amount equal to the difference between the Aggregate Salaries and Benefits Adjustment Amount and the Aggregate Compensation Adjustment Amount.
  \item
    ``Overage'' means the amount, if any, by which Total Salaries and Benefits for a Salary Cap Year exceed an amount equal to the Designated Percentage of BRI for such Salary Cap Year.
  \item
    ``Performance Bonus Deduction Date'' means, with respect to an Adjustment Player, the earlier of (i) the date that any Performance Bonus earned by the player during the applicable Salary Cap Year is paid to the player pursuant to his Contract, or (ii) the day following the date of the last game of the NBA Finals occurring during such Salary Cap Year.
  \item
    ``Salary Escrow Agreement'' means the escrow agreement in the form agreed upon by the parties (or such other form to which the parties may agree) to be entered into with the Escrow Agent.
  \item
    \begin{enumerate}
    \def\labelenumiii{(\roman{enumiii})}
    \tightlist
    \item
      ``Tax Level'' means, with respect to a Salary Cap Year, an amount determined by the following calculation:\\
      Step 1: Compute 61\% of Projected BRI (as defined in Section 1(c) above).\\
      Step 2: Subtract Projected Benefits (as defined in Article IV, Section 8) for such Salary Cap Year from the result in Step 1.\\
      Step 3: Divide the result in Step 2 by the number of Teams (excluding Expansion Teams during their first two (2) Salary Cap Years) in the NBA during such Salary Cap Year.
    \item
      In the event that the Audit Report for a Salary Cap Year has not been completed as of the last day of the Moratorium Period immediately following the end of such Salary Cap Year, then the Tax Level for the Salary Cap Year that commenced on the immediately preceding July 1 will be calculated using Interim Projected BRI instead of Projected BRI, Estimated BRI instead of BRI and Estimated Benefits instead of Benefits for all purposes under this Section 12(a)(17) including, without limitation, the adjustments set forth in Section 12(a)(17)(iv) and (v) below. In the event that the Interim Audit Report for a Salary Cap Year has not been completed as of the last day of the Moratorium Period immediately following the end of such Salary Cap Year, then the Tax Level for the Salary Cap Year that commenced on the immediately preceding July 1 shall, until such Interim Audit Report is completed, be an amount that would have been the Tax Level for the preceding Salary Cap Year had Projected BRI or Interim Projected BRI, as the case may be, for such preceding Salary Cap Year included, with respect to the NBA's national broadcast, national telecast or network cable television contracts, the rights fees or other non-contingentpayments stated in such contracts for the Season following the Season covered by such preceding Salary Cap Year instead of for the Season covered by such preceding Salary Cap Year.
    \item
      In the event that the Tax Level for a Salary Cap Year is calculated in accordance with Section 12(a)(17)(ii) above (i.e., is based upon an Interim Audit Report for the prior Salary Cap Year) and BRI and Benefits as set forth in the Audit Report for the prior Salary Cap Year are different from those in the Interim Audit Report such that the Tax Level would have been different from that based upon the Interim Audit Report, any such difference in the Tax Level shall be debited or credited, as the case may be, to the Tax Level for the subsequent Salary Cap Year, except that, with respect to the 2010-11 Salary Cap Year (or, in the alternative, if the NBA exercises its option pursuant to Article XXXIX, the 2011-12 Salary Cap Year) any such differences shall be debited or credited, as the case may be, to the Tax Level for the then-current Salary Cap Year, in all such cases with interest (at a rate equal to the one year Treasury Bill rate as published in The Wall Street Journal on the date of the issuance of the Interim Audit Report).
    \item
      In the event that 61\% of BRI for a Salary Cap Year, beginning with the 2005-06 Salary Cap Year, is greater or less than 61\% of Projected BRI for that Salary Cap Year, then for purposes of calculating the Tax Level for the subsequent Salary Cap Year: (A) if 61\% of BRI is greater than 61\% of Projected BRI, the difference shall be added to 61\% of Projected BRI for such subsequent Salary Cap Year; or (B) if 61\% of BRI is less than 61\% of Projected BRI, the difference shall be deducted from 61\% of Projected BRI for such subsequent Salary Cap Year.
    \item
      In the event that Benefits for a Salary Cap Year, beginning with the 2005-06 Salary Cap Year, is greater or less than Projected Benefits for that Salary Cap Year, then for purposes of calculating the Tax Level for the subsequent Salary Cap Year: (A) if Benefits is greater than Projected Benefits, the difference shall be added to Projected Benefits for such subsequent Salary Cap Year; or (B) if Benefits is less than Projected Benefits, the difference shall be deducted from Projected Benefits for such subsequent Salary Cap Year.
    \item
      The tax level with respect to the 2005-06 Salary Cap Year shall be deemed to be \$61.7 million.
    \end{enumerate}
  \item
    ``Trade Bonus Deduction Date'' means, with respect to an Adjustment Player, the earlier of (i) the date that any trade bonus earned by the player during the applicable Salary Cap Year is paid to the player pursuant to this Contract, or (ii) the June 30 of such Salary Cap Year.
  \end{enumerate}
\item
  \textbf{Compensation Adjustment Rules.}

  \begin{enumerate}
  \def\labelenumii{(\arabic{enumii})}
  \tightlist
  \item
    In the event that there is an Overage in any Salary Cap Year, (i) the Contracts of all Adjustment Players shall be amended by operation of this Agreement, such that the aggregate Compensation otherwise payable to all such Adjustment Players with respect to such Salary Cap Year shall be reduced by the Aggregate Compensation Adjustment Amount, and (ii) the New Benefit Amount for such Season as provided for by Article IV, Section 7(a)(1) (i.e., the New Benefit amount that is presently specified in Article IV, Section 7(a)(1) or such lesser New Benefit Amount as is designated by the Players Association, in accordance with the provisions of Article IV, Section 7(a)(1), by notice in writing to the NBA delivered on or before the March 15 prior to the commencement of the Salary Cap Year encompassing such Season) shall be reduced by the New Benefit Adjustment Amount.
  \item
    Subject to Section 12(d) and (e) below, to effectuate the aggregate Compensation reduction provided for in Section 12(b)(1)(i) above, the Compensation otherwise payable in accordance with each Adjustment Player's Contract shall be reduced by the player's respective Individual Compensation Adjustment Amount.
  \item
    The Designated Percentage for each Salary Cap Year shall be 57\%, except that:

    \begin{enumerate}
    \def\labelenumiii{(\roman{enumiii})}
    \tightlist
    \item
      if BRI for any Salary Cap Year exceeds BRI for the 2004-05 Salary Cap Year by more than 30\%, then the Designated Percentage for such Salary Cap Year shall be 57.5\%; and
    \item
      if BRI for any Salary Cap Year exceeds BRI for the 2004-05 Salary Cap Year by more than 60\%, then the Designated Percentage for such Salary Cap Year shall be 58\%.
    \end{enumerate}
  \end{enumerate}
\item
  \textbf{Escrow Procedure.}

  \begin{enumerate}
  \def\labelenumii{(\arabic{enumii})}
  \tightlist
  \item
    The following shall apply with respect to each Salary Cap Year (subject to Section 12(d) below regarding final reconciliation):

    \begin{enumerate}
    \def\labelenumiii{(\roman{enumiii})}
    \tightlist
    \item
      The Compensation otherwise payable to each Adjustment Player shall be reduced by the Escrow Amount applicable to such Adjustment Player; and
    \item
      Each Team shall deposit the Escrow Amount with respect to each of its Adjustment Players with the Escrow Agent.
    \end{enumerate}
  \item
    Except as set forth in Section 12(c)(4) below: (i) the Base Escrow Amount for each Adjustment Player shall be collected in twelve (12) equal installments from each of the player's semi-monthly Compensation payments on each Deduction Date; (ii) the Performance Bonus Escrow Amount for each Adjustment Player shall be collected in one (1) installment on the Performance Bonus Deduction Date; and (iii) the Trade Bonus Escrow Amount for each Adjustment Player shall be collected in one (1) installment on the Assignment Bonus Deduction Date.
  \item
    The procedure for deducting and depositing Escrow Amounts shall be as follows:

    \begin{enumerate}
    \def\labelenumiii{(\roman{enumiii})}
    \item
      The NBA will prepare and send to the Players Association the Escrow Schedules on or before November 8 of each Salary Cap Year, and periodically thereafter to reflect any new or adjusted Escrow Amounts calculated in accordance with Section 12 (c)(4) below.
    \item
      \begin{enumerate}
      \def\labelenumiv{(\Alph{enumiv})}
      \tightlist
      \item
        Within three (3) business days after each Deduction Date, each Team shall deliver to the Escrow Agent, in accordance with the Salary Escrow Agreement, the aggregate Base Escrow Amounts that the Team is obligated to deduct with respect to such Deduction Date for all of its Adjustment Players. Within three (3) business days following the date of the last game of the NBA Finals occurring during each Salary Cap, each Team shall deliver to the Escrow Agent, in accordance with the Salary Escrow Agreement, the aggregate Performance Bonus Escrow Amounts for all of its Adjustment Players. Within three (3) business days following an Adjustment Player's Trade Bonus Deduction Date, the player's Team shall deliver to the Escrow Agent, in accordance with the Salary Escrow Agreement, the player's Trade Bonus Escrow Amount. All amounts received by the Escrow Agent shall be invested and disbursed in accordance with the provisions of the Salary Escrow Agreement.
      \item
        It is the intention of the NBA and the Players Association that the provisions of this Section 12(c) result each Salary Cap Year in the delivery to the Escrow Agent, with respect to each Adjustment Player, of the player's full Escrow Amount. Accordingly, if for any reason the procedures in this Section 12(c) would result in less than the full Escrow Amount with respect to any Adjustment Player being delivered to the Escrow Agent, the NBA and the Players Association shall agree on such supplemental measures as are necessary to effectuate the deduction of the appropriate amount from the player's Compensation for delivery to the Escrow Agent.
      \end{enumerate}
    \end{enumerate}
  \item
    After November 8, the NBA shall periodically update the Escrow Schedules to add Escrow Amounts for players who enter into new Player Contracts and to make such adjustments as may be necessary to previously-listed Escrow Amounts (such as adjustments resulting from earned Performance Bonuses, Contract terminations, Renegotiations, etc.). Any portion of a Base Escrow Amount that has not been deducted as of the date any such updated Schedules are prepared shall be deducted in equal installments from each of the remaining semi-monthly Compensation payments to be made to the player from November 15 through May 1 of the applicable Salary Cap Year.
  \item
    Within seven (7) days after receiving any set of Escrow Schedules from the NBA, or within seven (7) days after any event that the Players Association believes warrants a change in any previously-issued Schedules, the Players Association may bring a proceeding before the System Arbitrator, in accordance with Article XXXII, Section 10, contesting the NBA's calculation of any player's Escrow Amount for such Salary Cap Year. Notwithstanding the commencement of any such proceeding, each Team shall commence and continue remitting to the Escrow Agent the total deductions due with respect to each Deduction Date and each Performance Bonus Deduction Date as set forth in the Schedules, and in no event shall any Team be prohibited from remitting to the Escrow Agent any such deduction prior to a final determination in any such proceeding.
  \item
    In the event that the NBA makes a determination in accordance with Section 12(c)(4) above, or a final determination is made in a proceeding in accordance with Section 12(c)(5) above, that an Escrow Amount was erroneously calculated by the NBA, the sole remedy with respect to any amounts erroneously deducted from the player's Salary shall be to modify, as soon as practicable, the deduction schedule applicable to such player so as to reduce, in equal amounts, all scheduled future deductions from post-determination payments of Compensation until the amount of any prior over-deduction is fully off-set; provided, however, that to the extent that reducing the player's future deductions would not fully offset the prior over-deductions, the parties shall instruct the Escrow Agent to pay the player as soon aspracticable, with interest, such additional amounts as are necessary to fully off-set such over- deductions.
  \end{enumerate}
\item
  \textbf{Reconciliation Procedures.}

  \begin{enumerate}
  \def\labelenumii{(\arabic{enumii})}
  \tightlist
  \item
    In the event of an Overage: (i) the NBA shall be entitled to receive from the Escrow Agent, with respect to each Adjustment Player, such player's Individual Compensation Adjustment Amount (or, in the event that the player's Escrow Amount is less than his Individual Compensation Adjustment Amount, a portion of his Individual Compensation Adjustment Amount equal to his Escrow Amount); and (ii) each Adjustment Player shall be entitled to receive from the Escrow Agent the amount, if any, by which the player's Escrow Amount exceeds his Individual Compensation Adjustment Amount. In the event that there is no Overage, each Adjustment Player shall be entitled to receive from the Escrow Agent his entire Escrow Amount.
  \item
    Any interest earned on Escrow Amounts remitted to the Escrow Agent shall be allocated among the Adjustment Players, collectively, and the NBA in proportion to the percentage of the aggregate Escrow Amounts that the Adjustment Players, collectively, and the NBA are to receive from the Escrow Agent in accordance with Section 12(d)(1) above. The Adjustment Players' collective share of interest shall be allocated among the individual players in proportion to each player's Escrow Amount.
  \item
    The parties shall cause the Accountants to include in the Interim Audit Report and the Audit Report (or, if no final Audit Report has been submitted at the conclusion of \textbf{\emph{{[}sic{]}}}

    \begin{enumerate}
    \def\labelenumiii{(\roman{enumiii})}
    \tightlist
    \item
      the amount of any Overage;
    \item
      the Aggregate Salaries and Benefits Adjustment Amount, if any;
    \item
      the Aggregate Compensation Adjustment Amount, if any;
    \item
      the New Benefits Adjustment Amount, if any;
    \item
      each Adjustment Player's Individual Compensation Adjustment Amount, if any;
    \item
      each Adjustment Player's Escrow Amount, if any, as set forth in the Escrow Schedules;
    \item
      a list of all Adjustment Players whose Individual Compensation Adjustment Amounts exceed their Escrow Amounts, which list shall also include (A) each such player's Individual Compensation Adjustment Amount, (B) each such player's Escrow Amount, (C) the amount by which each such player's Individual Compensation Adjustment Amount exceeds his Escrow Amount, (D) the sum of all such players' Escrow Amounts, (E) the sum of all such players' Individual Compensation Adjustment Amounts, and (F) the aggregate amount by which all such players' Individual Compensation Adjustment Amounts exceed their Escrow Amounts;
    \item
      a list of all Adjustment Players whose Individual Compensation Adjustment Amounts are equal to or less than their Escrow Amounts, which list shall also include (A) each such player's Individual Compensation Adjustment Amount, (B) each such player's Escrow Amount, (C) the amount, if any, by which each such player's Escrow Amount exceeds his Individual Compensation Adjustment Amount, (D) the sum of all such players' Escrow Amounts, (E) the sum of all such players' Individual Compensation Adjustment Amounts, and (F) the aggregate amount by which all such players' Escrow Amounts exceed their Individual Compensation Adjustment Amounts;
    \item
      in accordance with the provisions of Section 12(d)(1) and (d)(2) above, (A) the percentage of the interest earned on the Escrow Amounts to be allocated to the NBA, (B) the percentage of the interest earned on the Escrow Amounts to be allocated to the Adjustment Players collectively, and (C) the percentage of the interest earned on the Escrow Amounts to be allocated to each individual Adjustment Player;
    \item
      the Tax Level; and
    \item
      the amount, if any, by which each Team's Team Salary as completed in Section 12(f) below exceeds the TSO Level.
    \end{enumerate}
  \item
    In addition to the information described in Section 12(d)(3) above, the parties shall cause the Accountants to include in the Audit Report (or, if no final Audit Report has been submitted at the conclusion of the Audit Report Challenge Period, in the Interim Escrow Audit Report) a separate Notice to the NBA and the Players Association, in the form attached to the Salary Escrow Agreement, setting forth:

    \begin{enumerate}
    \def\labelenumiii{(\roman{enumiii})}
    \tightlist
    \item
      the amount, if any, by which each Team's Team Salary as computed in Section 12(f)(3) below exceeds the Tax Level.
    \item
      in the space designated in paragraph 1 of the Notice, the sum of the amounts described in Section 12(d)(3)(vii)(D) and Section 12(d)(3)(viii)(E) above, which sum is to be disbursed by the Escrow Agent to the NBA;
    \item
      in the space designated in paragraph 2 of the Notice, the amounts described in Section 12(d)(3)(viii)(C) above, which amounts are to be disbursed by the Escrow Agent to each respective Adjustment Player described in Section 12(d)(3)(viii) above;
    \item
      in the space designated in paragraph 3 of the Notice, the information described in Section 12(d)(3)(ix)(A) above, which information shall be the basis for the Escrow Agent's calculation of interest earned on the Escrow Amounts, which interest is to be disbursed by the Escrow Agent to the NBA; and
    \item
      in the space designated in paragraph (4) of the Notice, the information described in Section 12(d)(3)(ix)(C) above, which information shall be the basis for the Escrow Agent's calculation of interest earned on the Escrow Amounts, which interest is to be disbursed by the Escrow Agent to each Adjustment Player.
    \end{enumerate}
  \item
    No later than seven (7) business days after the earlier of (i) the completion of the Audit Report for the prior Salary Cap Year, or (ii) the completion of the Audit Report Challenge Period, the NBA and/or the Players Association shall deliver to the Escrow Agent the completed Notice to the NBA and the Players Association. As soon as practicable following receipt of such Notice, the Escrow Agent shall disburse the specified sums to the specified payees.
  \item
    Any amounts that the Escrow Agent is obligated to disburse to a player pursuant to this Section 12, including, if the Players Association so elects, the amounts described in Section 12(d)(4)(iv) above, shall be reduced by all amounts required to be withheld by federal, state, and local authorities, which withholdings shall be disbursed by the Escrow Agent to the Player's Team for remittance to the appropriate authorities. To assist the Escrow Agent in disbursing the appropriate amounts to each Adjustment Player and his respective Team, each Team, based on the information set forth in paragraph 2 (and, if applicable, paragraph 4) of the Notice to the NBA and the Players Association, shall promptly provide the Escrow Agent with a schedule for each of its Adjustment Players showing the exact withholding amount to be disbursed to the Team for remittance to the appropriate federal, state and local authorities. In no circumstance shall the employer's share of FICA, FUTA, or any other employer taxes be paid out of the amounts deposited in escrow or any interest or earnings thereon. Any such obligations shall remain with each player's individual employer.
  \end{enumerate}
\item
  \textbf{Aggregate Compensation Adjustment Amount Shortfalls.}

  \begin{enumerate}
  \def\labelenumii{(\arabic{enumii})}
  \tightlist
  \item
    If, with respect to any Salary Cap Year, there is an Aggregate Compensation Adjustment Amount Shortfall, then the Contract of each of the following Salary Cap Year's Adjustment Players shall be amended by operation of this Agreement, in accordance with Section 12(e)(2) below, such that the aggregate Compensation paid to all such players with respect to the Season covered by such following Salary Cap Year shall be reduced by the Aggregate Compensation Adjustment Amount Shortfall, which reduction shall be in addition to the full amount of any reduction for such following Salary Cap Year called for in Section 12(b)(1)-(2) above.
  \item
    The Individual Shortfall Adjustment Amount for each Adjustment Player whose Contract is amended in accordance with Section 12(e)(1) above shall be calculated by multiplying the Aggregate Compensation Adjustment Amount Shortfall for the prior Salary Cap Year by a fraction, the numerator of which is the player's then-current Salary, and the denominator of which is the sum of all such players' then-current Salaries. For purposes of calculating the fraction described in the preceding sentence, the Salary of a player making the Minimum Player Salary shall include the portion of such Minimum Player Salary that is reimbursed out of the league-wide benefits fund described in Article IV, Section 5(k).
  \item
    The Individual Shortfall Adjustment Amount for each Adjustment Player shall be deducted by the player's Team in four (4) equal installments from each of the player's first four (4) semi-monthly Cash Compensation payments following delivery to the Escrow Agent of the completed Notice to the NBA and the Players Association. All such deductions shall be promptly remitted by the Teams to the NBA.
  \end{enumerate}
\item
  \textbf{Team Payments.}

  \begin{enumerate}
  \def\labelenumii{(\arabic{enumii})}
  \tightlist
  \item
    Each Team whose Team Salary exceeds the Tax Level for any Salary Cap Year shall be required to pay a tax to the NBA equal to the amount by which the Team's Team Salary exceeds the Tax Level. For purposes of computing the amount of tax a Team owes:

    \begin{enumerate}
    \def\labelenumiii{(\roman{enumiii})}
    \tightlist
    \item
      a Team's Team Salary shall be the sum of: (A) its Team Salary as of the start of its last Regular Season game, plus all Performance Bonuses excluded from Salary under Article VII, Section 3(d) but actually earned by the player during such Salary Cap Year, less all Performance Bonuses included in Salary under Section 3(d) above but not actually earned by the player during such Salary Cap Year; plus (B) with respect to any trade that occurs following the conclusion ofthe Team's last Regular Season game, the portion of any trade bonus earned by a player that is included in the Team's Team Salary for such Salary Cap Year, plus (C) any amount that is added to the Team's Team Salary for such Salary Cap Year following the start of the Team's last Regular Season game pursuant to Section 4(a)(iii) above; and
    \item
      the Salary attributable to the Contract of any player with zero (0) Years of Service or one (1) Year of Service who signs for the Minimum Player Salary shall be deemed to equal the Minimum Player Salary that would be applicable to a player with two (2) Years of Service.
    \end{enumerate}
  \item
    Each Team that owes a tax shall make the required tax payment to the NBA no later than ten (10) business days following the earlier of (i) the completion of the Audit Report for the prior Salary Cap Year, or (ii) the completion of the Audit Report Challenge Period.
  \item
    For purposes of this Section 12(f) and subject to the provisions of Section 12(f)(1) above and Section 12(f)(4) below, Team Salary shall be calculated by the Accountants in the same manner as Team Salary is calculated by the Accountants for purposes of computing Total Salaries and Benefits in the Audit Report.
  \item
    The Salary of one (1) player who was signed or acquired by a Team prior to June 21, 2005, and whose Contract is terminated by such Team in accordance with the NBA waiver procedure following a request for waivers made by such Team on or before August 15, 2005 (or was terminated prior to the date of this Agreement), shall be excluded from such Team's Team Salary solely for the purpose of computing the amount of tax such Team owes, provided that the Team designates the player for such treatment by written notice to the NBA on or before August 15, 2005. If a Team so designates a player, the Team shall be prohibited from re-signing or re-acquiring the player prior to the conclusion of the term of the terminated Contract.
  \end{enumerate}
\item
  \textbf{Escrow and Tax Proceeds.} All amounts remitted to the NBA by the Escrow Agent or NBA Teams pursuant to this Section 12 shall be the exclusive property of the NBA, and the use and/or disposition of all such amounts, including the allocation or distribution of such amounts to one (1) or more NBA Teams, if any, shall be subject only to the following limitations:

  \begin{enumerate}
  \def\labelenumii{(\arabic{enumii})}
  \tightlist
  \item
    The Escrow Amounts remitted to the NBA by the Escrow Agent pursuant to Section 12(d) above with respect to each Salary Cap Year shall be used and/or distributed as follows:

    \begin{enumerate}
    \def\labelenumiii{(\roman{enumiii})}
    \tightlist
    \item
      the NBA may elect to distribute all or a portion of such amounts to NBA teams, provided that any such distribution pursuant to this Section 12(g)(1)(i) must be made to all Teams in equal shares; and
    \item
      amounts not distributed in accordance with Section 12(g)(1)(i) above shall be used for one (1) or more ``League purposes'' (as defined in Section 12(g)(3) below) selected by the NBA.
    \end{enumerate}
  \item
    The tax amounts remitted to the NBA by NBA Teams pursuant to Section 12(f) above for each Salary Cap Year shall be used and/or distributed as follows:

    \begin{enumerate}
    \def\labelenumiii{(\roman{enumiii})}
    \tightlist
    \item
      the NBA shall distribute to each Team that does not owe a tax for such Salary Cap Year an amount equal to the aggregate tax remitted to the NBA multiplied by a fraction, thenumerator of which is one (1) and the denominator of which is the number of Teams in the NBA during such Salary Cap Year; and
    \item
      amounts not distributed in accordance with Section 12(g)(2)(i) above shall be used and/or distributed in the manner described (with respect to the use and/or distribution of Escrow Amounts) in Section 12(g)(1)(i) and (ii) above.
    \end{enumerate}
  \item
    For purposes of this Section 12(g), the use of Escrow amounts or tax amounts for a ``League purpose'' shall mean the use of such amounts for any purpose, including, but not limited to, the distribution of such amounts to one (1) or more Teams; provided, however, that such amounts may not be distributed to a Team or expended for the benefit or detriment of a Team in a manner that is based, directly or indirectly, on the amount of the Team's Team Salary or on whether the Team is a taxpayer. By way of example and not limitation, a team-assistance plan adopted by the NBA and funded with Escrow amounts and/or tax amounts shall be considered a ``League purpose'' if, pursuant to the plan, (i) a Team's entitlement to an assistance payment and/or the amount of such payment is based, in whole or in part, on a pro forma profit, loss, and/or expenses computation determined by the NBA under which all Teams are assumed to have the same Team Salary, and/or (ii) a Team's assistance payment is limited to an amount not to exceed the Team's actual losses as determined by the NBA; provided, however, that (x) with respect to clause (ii) of this Section 12(g)(3), in determining whether a Team has sustained an actual loss, the only expense amount that may be used in respect of the Team's Compensation expenditures shall be an amount that is no less than its Team Salary, and (y) in order to qualify as a ``League purpose,'' such a plan may not otherwise base a Team's entitlement to assistance and/or the amount of such assistance on the amount of a Team's Team Salary or on whether the Team is a taxpayer. If the NBA adopts ateam assistance plan, it will advise the Players Association of the terms thereof, provided that the Players Association agrees to sign, and to counsel its representatives to sign, a confidentiality agreement similar to the confidentiality agreement annexed hereto as Exhibit J-1.
  \end{enumerate}
\item
  \textbf{Miscellaneous.}

  \begin{enumerate}
  \def\labelenumii{(\arabic{enumii})}
  \item
    Notwithstanding any other provision of this Agreement, the computation of an Adjustment Player's Salary shall for purposes of the rules set forth in this Agreement be made without regard to any reduction in such player's Compensation made pursuant to this Section 12.
  \item
    The NBA shall be permitted to assign to such designee, as the NBA may determine, any rights the NBA has to receive amounts from the Escrow Agent or NBA Teams pursuant to this Section 12.
  \item
    Consistent with Section 3(f) above (One-Year Minimum Contracts), except for purposes of calculating the amounts referred to in the definitions of Escrow Amount, Individual Compensation Adjustment Amount, and Individual Shortfall Adjustment Amount (set forth in Sections 12(a)(9), (b)(11), and (b)(12) above), and subject to Section 12(f)(1)(ii) above, the Salary of every player who signs a one-year Contract after the date of this Agreement for the Minimum Player Salary applicable to such player shall, for all other purposes in this Section 12, be the lesser of (i) such Minimum Player Salary, or (ii) the portion of such Minimum Player Salary that is not reimbursed out of the League-wide benefits fund described in Article IV, Section 5(k)(1).
  \item
    \begin{enumerate}
    \def\labelenumiii{(\roman{enumiii})}
    \tightlist
    \item
      For purposes of the computations made by the Accountants pursuant to Section 10 above and this Section 12, the Salary of a player who is suspended by the NBA -- but not by a Team -- shall be reduced (for the Salary Cap Year covering the Season during which the player is suspended) by an amount equal to fifty percent (50\%) of the suspension-related Compensation amount that is collected from the player and retained by the NBA at the time the computations of the Accountants are made. Other than as set forth in the preceding sentence, the computation of a player's Salary under the CBA shall be made without regard to any reduction in Compensation that results from the player's suspension by the NBA or his Team.
    \item
      When (A) a player has forfeited a portion of his Compensation for a Season as a result of a suspension imposed by the NBA or his Team and (B) the player's Compensation is later reduced pursuant to Section 12(b)(1) and (2) above, the player shall be entitled to a refund of a portion of the Compensation forfeited as a result of the suspension. The refund shall be in an amount equal to (x) the player's Individual Compensation Adjustment Amount for the Salary Cap Year to which the suspension related multiplied by a fraction, the numerator of which is the amount of the player's Compensation that was forfeited as a result of the suspension and the denominator of which is the player's Base Compensation forsuch Season as of the date(s) he served the suspension, less (y) all amounts required to be withheld by federal, state, and local authorities. For purposes of the calculation required in clause (x) above, a player's Individual Compensation Adjustment Amount shall be deemed to include only the portion of the player's Individual Compensation Adjustment Amount that relates to the player's Base Compensation earned under the Player Contract for the NBA team that the player was playing for while he was suspended. Such refund shall be made to the player within thirty (30) days following the Accountants' submission to the NBA and the Players Association of a final Audit Report or an Interim Escrow Audit Report for the Salary Cap Year covering the Season for which the suspension-related Compensation amount is collected.
    \end{enumerate}
  \item
    In the event that the Overage for any Salary Cap Year exceeds the Aggregate Salaries and Benefits Adjustment Amount for such Salary Cap Year, the NBA shall not be entitled to reduce player Compensation in such Salary Cap Year or any subsequent Salary Cap Year so as to recover any amounts in excess of the Aggregate Compensation Adjustment Amount.
  \end{enumerate}
\end{enumerate}

\hypertarget{rookie-scale}{%
\chapter{ROOKIE SCALE}\label{rookie-scale}}

\hypertarget{rookie-scale-contracts-for-first-round-picks.}{%
\section{Rookie Scale Contracts for First Round Picks.}\label{rookie-scale-contracts-for-first-round-picks.}}

\begin{enumerate}
\def\labelenumi{(\alph{enumi})}
\item
  Each Rookie Scale Contract between a Team and a First Round Pick shall cover a period of two (2) Seasons, but shall have an Option in favor of the Team for the player's third Season and a second Option in favor of the Team for the player's fourth Season. The Option for the player's third Season shall be exercisable during the period from the day following the last day of the first Season through the immediately following October 31. The Option for the player's fourth Season shall be exercisable during the period from the day following the last day of the second Season through the immediately following October 31. Such Options shall be exercisable by notice to the player that is either personally delivered to the player or his representative or sent by pre-paid certified, registered, or overnight mail to the last known address of the player or his representative, signed by the Team, informing the player that the Team has exercised such Option.
\item
  \begin{enumerate}
  \def\labelenumii{(\roman{enumii})}
  \tightlist
  \item
    The Rookie Salary Scale applicable to a First Round Pick is determined by the first Season to be covered by the player's Rookie Scale Contract. Accordingly, for example, if a player's Rookie Scale Contract commences with the 2005-06 Season, the 2005-06 Rookie Salary Scale shall apply. Within a particular Rookie Salary Scale, a First Round Pick's applicable Rookie Scale Amounts are determined by the player's selection number in the NBA Draft. Accordingly, for example, the Rookie Scale Amounts applicable to the eighth player selected in the first round of the NBA Draft shall be those specified in the applicable Rookie Salary Scale for the eighth pick. Notwithstanding anything to the contrary in this Section 1(b)(i) or in Section 1(b)(ii) below, beginning on January 10 of each Season, an unsigned First Round Pick's applicable Rookie Scale Amount for such Season shall be reduced daily through the end of the Regular Season by the product of the applicable Rookie Scale Amount (as set forth in Exhibit B annexed hereto) multiplied by a fraction, the numerator of which is one (1) and the denominator of which is the total number of days in such Regular Season.
  \item
    Notwithstanding Section 1(b)(i) above, if, pursuant to any provision of this Agreement or the NBA Constitution and By-Laws, one (1) or more Teams is required to forfeit one (1) or more draft picks in the first round of a particular NBA Draft, then:
    (A) the Rookie Salary Scale for the Salary Cap Year immediately following such Draft shall be adjusted by removing one (1) or more Rookie Scale Amounts from the middle of the Rookie Salary Scale, as follows: if one (1) first round pick is forfeited, then the Rookie Scale Amounts that would have been applicable to the 15th player selected in the first round (absent any forfeiture of picks) (hereinafter, the ``15th Pick'') shall be removed from the Rookie Salary Scale; if two (2) first round picks are forfeited, then the Rookie Scale Amounts applicable to the 15th Pick and the pick immediately following the 15th Pick shall be removed from the Rookie Salary Scale; if three (3) first round picks are forfeited, then the Rookie Scale Amounts applicable to the 15th Pick and the picks immediately preceding and immediately following the 15th Pick shall be removed from the Rookie Salary Scale; and if more than three picks are forfeited, additional Rookie Scale Amounts shall be removed from the Rookie Salary Scale in accordance with the foregoing procedure; and
    (B) the Rookie Scale Amounts applicable to players selected in such Draft shall be determined by their selection number under the Rookie Salary Scale as adjusted by Section 1(b)(ii)(A) above. Accordingly, for example, if one First Round Pick were forfeited in the first round of the 2006 Draft, the applicable Rookie Scale Amounts would remain unchanged for the first 14 picks, and the Rookie Scale Amounts applicable to the remaining 15 picks in the first round would be the Rookie Scale Amounts that (absent any forfeiture of picks) would have been applicable to picks 16 through 30.
  \end{enumerate}
\item
  \begin{enumerate}
  \def\labelenumii{(\roman{enumii})}
  \tightlist
  \item
    A Rookie Scale Contract shall provide in each of the two (2) Seasons covered by the Contract and the first Option Year at least 80\% of the applicable Rookie Scale Amount in Current Base Compensation. Components of Salary in excess of 80\%, if any, are subject to individual negotiation, except that (i) in no event may Salary plus Unlikely Bonuses for any Salary Cap Year exceed 120\% of the applicable Rookie Scale Amount, and (ii) a Rookie Scale Contract may not provide for a signing bonus (except for an ``international player'' payment in excess of \$500,000 made in accordance with Article VII, Section 3(e)) or a loan. A Rookie Scale Contract may provide for a payment schedule in any Season that is more favorable to the player than that called for under paragraph 3 of the Uniform Player Contract, subject to the other provisions of this Agreement.
  \item
    A Rookie Scale Contract must provide for Compensation protection for lack of skill and non-insured injury or illness in each of the two (2) Seasons covered by the Contract and the first Option Year to the extent of not less than 80\% of the applicable Rookie Scale Amount. Consistent with the provisions of Article II, Section 4, a Team and a First Round Pick may negotiate additional conditions or limitations applicable to the player's Base Compensation protection, except that lack of skill and non-insured injury or illness protection to the extent of at least 80\% of the applicable Rookie Scale Amount in each of the first two (2) Seasons and the first Option Year shall contain no such individually-negotiated additional conditions or limitations.
  \item
    The terms and conditions that apply to the second Option Year shall be unchanged from all terms and conditions that applied to the player's first Option Year (including but not limited to the percentage of Base Compensation that is protected), except that the Salary, (excluding Incentive Compensation), Likely Bonuses and Unlikely Bonuses for the second Option Year shall be increased over the Salary (excluding Incentive Compensation), Likely Bonuses and Unlikely Bonuses, respectively, for the first Option Year by the applicable percentage specified in Exhibit B hereto.
  \end{enumerate}
\item
  Notwithstanding any other provision of this Agreement, if a trade of a Rookie Scale Contract would, by reason of a trade bonus contained in such Contract, cause the player's Salary plus Unlikely Bonuses for the Salary Cap Year in which such trade occurs to exceed 120\% of the player's applicable Rookie Scale Amount for such Salary Cap Year, such player's trade bonus shall be deemed amended to the extent necessary to reduce the player's Salary plus Unlikely Bonuses for such Salary Cap Year to 120\% of the applicable Rookie Scale Amount.
\end{enumerate}

\hypertarget{rookie-contracts-for-later-signed-first-round-picks.}{%
\section{Rookie Contracts for Later-Signed First Round Picks.}\label{rookie-contracts-for-later-signed-first-round-picks.}}

Except as provided in Section 3 below, a First Round Pick who does not sign with the Team that holds his draft rights for any portion of the three (3) Seasons following the NBA Draft in which he was selected (and who did not play intercollegiate basketball during such period) may enter into either (a) a Rookie Scale Contract in accordance with Section 1 above, or (b) if the Team has Room in excess of the applicable first-year Rookie Scale Amount, a Contract covering no fewer than three (3) Seasons that provides for Salary plus Unlikely Bonuses in the first Salary Cap Year up to the amount of the Team's Room and increases or decreases in Salary and Unlikely Bonuses in subsequent Salary Cap Years in accordance with Article VII, Section 5(c)(1).

\hypertarget{loss-of-draft-rights.}{%
\section{Loss of Draft Rights.}\label{loss-of-draft-rights.}}

If for any reason a Team fails to make a Required Tender to a First Round Pick in accordance with Article X, withdraws a Required Tender to a First Round Pick in accordance with Article X, or renounces a First Round Pick in accordance with Article X, or if a First Round Pick selected in a Subsequent Draft does not sign a Contract for a period of one (1) year following such Subsequent Draft in accordance with Article X, then the rules set forth in Sections 1 and 2 above shall not apply, and such First Round Pick shall become a Rookie Free Agent. In addition, any Team that fails to make a Required Tender to a First Round Pick, withdraws a Required Tender to a First Round Pick, renounces a First Round Pick, or fails to sign within one (1) year a First Round Pick selected in a Subsequent Draft shall be prohibited from signing such player until after he has signed a Player Contract with another NBA Team, and either (a) the player completes the playing services called for under the Contract, or (b) the Contract is terminated in accordance with the NBA waiver procedure.

\hypertarget{players-drafted-prior-to-2005.}{%
\section{Players Drafted Prior To 2005.}\label{players-drafted-prior-to-2005.}}

In lieu of entering into a Rookie Scale Contract in accordance with Section 1 above, a First Round Pick who was drafted prior to the 2005 NBA Draft and the Team that possesses the exclusive rights to negotiate with such player may agree to enter into a Rookie Scale Contract in accordance with the rules set forth in Section 1 above but modified as follows: (a) the Contract shall cover a period of three (3) Seasons, but shall have an Option in favor of the Team for the player's fourth Season; (b) with respect to the third Season covered by such Contract, the rules set forth in Section 1(c) above with respect to the first Option Year shall apply; and (c) the player's applicable Rookie Scale Amounts for all Seasons covered by such Contract (including the Option Year) shall be those set forth in the Rookie Salary Scale contained in the 1999 NBA/NBPA Collective Bargaining Agreement for the Season following the date on which the player was drafted.

\hypertarget{length-of-player-contracts}{%
\chapter{LENGTH OF PLAYER CONTRACTS}\label{length-of-player-contracts}}

\hypertarget{maximum-term.}{%
\section{Maximum Term.}\label{maximum-term.}}

Except where a shorter term is expressly provided for elsewhere in this Agreement, a Player Contract entered into after the date of this Agreement may cover, in the aggregate, up to but no more than five (5) Seasons (including any Season covered by an Option) from the date such Contract is signed; provided, however, that (a) a Player Contract between a Qualifying Veteran Free Agent and his Prior Team may cover, in the aggregate, up to but no more than six (6) Seasons (including any Season covered by an Option) from the date such Contract is signed, and (b) an Extension of a Rookie Scale Contract may cover, in the aggregate, up to but no more than six (6) Seasons (including any Season covered by an Option) from the date such extension is signed.

\hypertarget{computation-of-time.}{%
\section{Computation of Time.}\label{computation-of-time.}}

For purposes of Section 1 above, if a Player Contract or Extension is signed after the beginning of a Season, the Season in which the Contract or Extension is signed shall be counted as one (1) full Season covered by the Contract or Extension.

\hypertarget{player-eligibility-and-nba-draft}{%
\chapter{PLAYER ELIGIBILITY AND NBA DRAFT}\label{player-eligibility-and-nba-draft}}

\hypertarget{player-eligibility.}{%
\section{Player Eligibility.}\label{player-eligibility.}}

\begin{enumerate}
\def\labelenumi{(\alph{enumi})}
\tightlist
\item
  No player may sign a Contract or play in the NBA unless he has been eligible for selection in at least one (1) NBA Draft. No player shall be eligible for selection in more than two (2) NBA Drafts.
\item
  A player shall be eligible for selection in the first NBA Draft with respect to which he has satisfied all applicable requirements of Section 1(b)(i) below and one of the requirements of Section 1(b)(ii) below:

  \begin{enumerate}
  \def\labelenumii{(\roman{enumii})}
  \item
    The player (A) is or will be at least 19 years of age during the calendar year in which the Draft is held, and (B) with respect to a player who is not an international player (defined below), at least one (1) NBA Season has elapsed since the player's graduation from high school (or, if the player did not graduate from high school, since the graduation of the class with which the player would have graduated had he graduated from high school); and
  \item
    \begin{enumerate}
    \def\labelenumiii{(\Alph{enumiii})}
    \tightlist
    \item
      The player has graduated from a four-year college or university in the United States (or is to graduate in the calendar year in which the Draft is held) and has no remaining intercollegiate basketball eligibility; or

      \begin{enumerate}
      \def\labelenumiv{(\Alph{enumiv})}
      \setcounter{enumiv}{1}
      \tightlist
      \item
        The player is attending or previously attended a four-year college or university in the United States, his original class in such college or university has graduated (or is to graduate in the calendar year in which the Draft is held), and he has no remaining intercollegiate basketball eligibility; or
      \item
        The player has graduated from high school in the United States, did not enroll in a four-year college or university in the United States, and four calendar years have elapsed since such player's high school graduation; or
      \item
        The player did not graduate from high school in the United States, and four calendar years have elapsed since the graduation of the class with which the player would have graduated had he graduated from high school; or
      \item
        The player has signed a player contract with a ``professional basketball team not in the NBA'' (defined below) that is located anywhere in the world, and has rendered services under such contract prior to the Draft; or
      \item
        The player has expressed his desire to be selected in the Draft in a writing received by the NBA at least sixty (60) days prior to such Draft (an ``Early Entry'' player); or
      \item
        If the player is an ``international player'' (defined below), and notwithstanding anything contained in subsections (A) through (F) above:
      \end{enumerate}
    \end{enumerate}

    \begin{enumerate}
    \def\labelenumiii{(\arabic{enumiii})}
    \tightlist
    \item
      The player is or will be twenty-two (22) years of age during the calendar year of the Draft; or
    \item
      The player has signed a player contract with a ``professional basketball team not in the NBA'' (defined below) that is located in the United States, and has rendered services under such contract prior to the Draft; or
    \item
      The player has expressed his desire to be selected in the Draft in a writing received by the NBA at least sixty (60) days prior to such Draft (an ``Early Entry'' player).
    \end{enumerate}
  \end{enumerate}
\item
  For purposes of this Article X, an ``international player'' is a player: (i) who has maintained a permanent residence outside of the United States for at least the three (3) years prior to the Draft, while participating in the game of basketball as an amateur or as a professional outside of the United States; (ii) who has never previously enrolled in a college or university in the United States; and (iii) who did not complete high school in the United States.(d) For purposes of this Article X, a ``professional basketball team not in the NBA'' means any team that pays money or compensation of any kind -- in excess of a stipend for living expenses -- to a basketball player for rendering services to such team.
\end{enumerate}

\hypertarget{term-and-timing-of-draft-provisions.}{%
\section{Term and Timing of Draft Provisions.}\label{term-and-timing-of-draft-provisions.}}

An NBA Draft will be held prior to the commencement of each NBA Season covered by the term of this Agreement and, despite the expiration of the other terms of this Agreement pursuant to Article XXXIX, prior to the commencement of the 2011-12 NBA Season (or, if the NBA exercises its option to extend the Agreement pursuant to Article XXXIX, prior to the commencement of the 2012-13 NBA Season). Each such Draft will be held prior to the July 10 preceding the commencement of the NBA Season on a date to be designated by the Commissioner.

\hypertarget{number-of-choices.}{%
\section{Number of Choices.}\label{number-of-choices.}}

\begin{enumerate}
\def\labelenumi{(\alph{enumi})}
\tightlist
\item
  The NBA Draft shall consist of two (2) rounds, with each round consisting of the same number of selections as there will be Teams in the NBA the following Season. Each Team shall be required to exercise any and all draft selections in its possession during each round of the Draft.
\item
  If, pursuant to any provision of this Agreement or the NBA Constitution and By-Laws, any Team is required to forfeit one or more draft pick(s) in a particular NBA Draft, the number of players selected in the applicable round of the Draft will be reduced by the number of such forfeitures. (Thus, for example, if Team A is required to forfeit the ninth pick in the first round of the Draft (at a time when there are thirty (30) NBA Teams), there will only be twenty-nine (29) players selected in the first round of such Draft.) In the event the forfeiture relates to one or more first round picks, the Rookie Salary Scale will be adjusted as set forth in Article VIII, Section 1(b)(2). Other than as specifically agreed to herein, nothing contained in this Agreement shall be deemed to be an agreement of the Players Association to any provision of the NBA Constitution and By-Laws.
\end{enumerate}

\hypertarget{negotiating-rights-to-draft-rookies.}{%
\section{Negotiating Rights to Draft Rookies.}\label{negotiating-rights-to-draft-rookies.}}

\begin{enumerate}
\def\labelenumi{(\alph{enumi})}
\tightlist
\item
  A Team that drafts a player shall, during the period from the date of such NBA Draft (hereinafter, the ``Initial Draft'') to the date of the next Draft (hereinafter, the ``Subsequent Draft''), be the only Team with which such player may negotiate or sign a Player Contract, provided that, on or before the July 15 immediately following the Initial Draft (for a First Round Pick), or in the two (2) weeks before the September 5 immediately following the Initial Draft (for a Second Round Pick), such Team has made a Required Tender to such player. If a Team has made a Required Tender to such a player and the player has not signed a Player Contract within the period between the Initial Draft and the Subsequent Draft, the Team that drafted the player shall lose its exclusive right to negotiate with the player and the player will then be eligible for selection in the Subsequent Draft.
\item
  A Team that, in the Subsequent Draft, drafts a player who (i) was drafted in the Initial Draft, (ii) received a Required Tender from the Team that drafted him in the Initial Draft, and (iii) did not sign a Player Contract with such first Team prior to the Subsequent Draft, shall, during the period from the date of the Subsequent Draft to the date of the next NBA Draft, be the only Team with which such player may negotiate or sign a Player Contract, provided such Team has made a Required Tender to such player by the dates specified in Section 4(a) above. If such player has not signed a Player Contract within the period between the Subsequent Draft and the next NBA Draft with the Team that drafted him in the Subsequent Draft, that Team shall lose its exclusive right, which it obtained in the Subsequent Draft, to negotiate with the player, and the player will become a Rookie Free Agent as of the date of the next NBA Draft.
\item
  If a player is drafted in an Initial Draft and (i) receives a Required Tender, (ii) does not sign a Player Contract with a Team prior to the Subsequent Draft, and (iii) is not drafted by any Team in such Subsequent Draft, the player will become a Rookie Free Agent immediately upon the conclusion of the Subsequent Draft.
\item
  If a player is drafted by a Team in either an Initial or Subsequent Draft and that Team does not make a Required Tender to such player, the player will become a Rookie Free Agent on the July 16 following such Draft (for a First Round Pick) or on the September 6 following such Draft (for a Second Round Pick).
\item
  A Team may at any time withdraw a Required Tender it has made to a player, provided that the player agrees in writing to the withdrawal. In the event that a Required Tender is withdrawn, the player shall thereupon become a Rookie Free Agent.
\item
  A Team that holds the exclusive rights to negotiate with and sign a drafted player may at any time renounce such exclusive rights, except that, if the Team has made a Required Tender to the player, a renunciation shall not be permitted during the time the player has to accept the Required Tender under Article VIII. In order to renounce its exclusive rights with respect to a drafted player, a Team shall provide the NBA with an express, written statement renouncing such exclusive rights. The NBA shall provide a copy of such statement to the Players Association within three (3) business days following its receipt thereof.
\end{enumerate}

\hypertarget{effect-of-contracts-with-other-professional-teams.}{%
\section{Effect of Contracts with Other Professional Teams.}\label{effect-of-contracts-with-other-professional-teams.}}

If a player is drafted by a Team in either an Initial or Subsequent Draft and, during a period in which he may negotiate and sign a Player Contract with only the Team that drafted him, and either (x) is a party to a previously existing player contract with a professional basketball team not in the NBA that covers all or any part of the NBA Season immediately following said Initial or Subsequent Draft, or (y) signs such a player contract, then the following rules will apply:

\begin{enumerate}
\def\labelenumi{(\alph{enumi})}
\tightlist
\item
  Subject to Section 5(b) below, the Team that drafts the player shall retain the exclusive NBA rights to negotiate with and sign him for the period ending one (1) year from the earlier of the following two dates: (i) the date the player notifies such Team that he is available to sign a Player Contract with such Team immediately, provided that such notice will not be effective until the player is under no contractual or other legal impediment to sign and play with such Team for the then-current Season (if applicable) and any future Season; or (ii) the date of the NBA Draft occurring in the twelve-month period from September 1 to August 30 in which the player notifies such Team of his availability and intention to play in the NBA during the Season immediately following said twelve-month period, provided that such notice will not be effective until the player is under no contractual or other legal impediment to sign and play with such Team for the then-current Season (if applicable) and any future Season.
\item
  If, by July 1 of any year, the player notifies the Team that has drafted him that by September 1 of such year he will, immediately thereafter and for any future Season, be under no contractual or other legal impediment to sign and play with such Team, and provided that on such September 1 the player is in fact under no such contractual or other legal impediment, then,in order to retain the exclusive NBA rights to negotiate with and sign the player as provided in Section 5(a), such Team must make a Required Tender to the player by September 10 of such year.
\item
  If the player gives the required notice by July 1 of any year, and the Team that drafted him fails to make a Required Tender by September 10 of such year, the player shall thereupon become a Rookie Free Agent.
\item
  If, during the one-year period of exclusive NBA negotiating rights set forth in Section 5(a) above, the player signs a player contract with a professional basketball team not in the NBA and the player has not made a bona fide effort to negotiate a Player Contract with the Team possessing his exclusive NBA rights or such bona fide effort is made and such Team makes a Required Tender to such player in accordance with Section 5(b) above, then such Team shall retain the exclusive NBA rights to negotiate with and sign the player for additional one-year periods as measured in and in accordance with the provisions of Section 5(a) above.
\item
  If, during the one-year period of exclusive NBA negotiating rights set forth in subsection (a) above, (i) the player signs a player contract with a professional basketball team not in the NBA, (ii) the player has made a bona fide effort to negotiate a Player Contract with the Team possessing his exclusive NBA rights, and (iii) such Team fails to make a Required Tender to such player in accordance with Section 5(b) above, then the player shall thereupon become a Rookie Free Agent.
\item
  If, during the one-year period of exclusive NBA negotiating rights set forth in Section 5(a) above, the Team makes or has made a Required Tender to the player and the player does not sign a player contract with any professional basketball team, then (i) in the case of a player who was previously drafted in an Initial Draft, the next NBA Draft following such one-year period shall be deemed the Subsequent Draft as to such player, and the rules applicable to a player who is subject to a Subsequent Draft will apply, or (ii) in the case of a player who was previously drafted in a Subsequent Draft, such player shall become a Rookie Free Agent at the end of such one-year period.
\item
  Notice under this Section 5 shall be provided in writing by personal delivery or pre-paid certified, registered, or overnight mail sent to the Team's principal address or principal office (as then listed in the NBA's records), to the attention of the Team's general manager.
\end{enumerate}

\hypertarget{application-to-early-entry-players.}{%
\section{Application to ``Early Entry'' Players.}\label{application-to-early-entry-players.}}

If a player who is eligible for the Draft pursuant to Section 1(b)(ii)(F) or (b)(ii)(G)(3) above (an ``Early Entry'' player) is selected in such Draft by a Team, the following rules apply:

\begin{enumerate}
\def\labelenumi{(\alph{enumi})}
\tightlist
\item
  Subject to Section 6(b) below, if the player does not thereafter play intercollegiate basketball, then the Team that drafted him shall, during the period from the date of such Draft to the date of the Draft in which the player would, absent his becoming an Early Entry player, first have been eligible to be selected, be the only Team with which the player may negotiate or sign a Player Contract, provided that such Team makes a Required Tender to the player each year by the date specified in Section 4(a) above. For purposes hereof, the Draft in which such player would, absent his becoming an Early Entry player, first have been eligible to be selected, will be deemed the ``Subsequent Draft'' as to that player, and the rules applicable to a player who has been drafted in a Subsequent Draft will apply. If the player, having been selected in a Draft for which he was eligible as an Early Entry player, has not signed a Player Contract with the Team \textbf{\emph{{[}sic{]}}}
\item
  If the player does thereafter play intercollegiate basketball, then the Team that drafted him shall retain the exclusive NBA rights to negotiate with and sign the player for the period ending one (1) year from the date of the Draft in which the player would, absent his becoming an Early Entry player, first have been eligible to be selected, provided that such Team makes a Required Tender to the player each year by the date specified in Section 4(a) above. For purposes hereof, the Draft in which such player would, absent his becoming an Early Entry player, first have been eligible to be selected, will be deemed the ``Initial Draft'' as to that player. The next NBA Draft shall be deemed the ``Subsequent Draft'' as to that player, and the rules applicable to a player who has been drafted in a Subsequent Draft will apply.
\end{enumerate}

\hypertarget{assignment-of-draft-rights.}{%
\section{Assignment of Draft Rights.}\label{assignment-of-draft-rights.}}

In the event that the exclusive right to negotiate with a player obtained in any NBA Draft is assigned by a Team to another Team, in accordance with NBA procedures, the Team to which such right has been assigned shall have the same, but no greater, right to negotiate with and sign such player as is possessed by the Team assigning such right, and such player shall have the same, but no greater, obligation to the Team to which such right has been assigned as he had to the Team assigning such right.

\hypertarget{general.-2}{%
\section{General.}\label{general.-2}}

\begin{enumerate}
\def\labelenumi{(\alph{enumi})}
\tightlist
\item
  The placement of a Rookie on the Armed Services List, or on any of the other lists described in the NBA By-Laws, or on any other list created by the NBA, shall not extend the period of exclusive negotiating rights which a Team has to any Draft Rookie beyond the
  period specified in this Agreement.
\item
  Nothing contained herein shall prevent the NBA, in accordance with the applicable provisions of the NBA Constitution and By-Laws, from prohibiting or otherwise responding to violations by Teams of the exclusive NBA rights obtained in any NBA Draft, as set forth or referred to in this Article. Other than as specifically agreed to herein, nothing contained in this Agreement shall be deemed to be an agreement by the Players Association to any provision of the NBA Constitution and By-Laws.
\item
  An Early Entry player who is eligible to be selected in the next NBA Draft pursuant to Section 1(b)(ii)(F) or (b)(ii)(G)(3) above shall be entitled to withdraw from such Draft by providing written notice that is received by the NBA ten (10) days prior to such Draft. A player shall not be entitled to withdraw from more than two (2) NBA Drafts.
\item
  Any claim by a player that a Contract offered as a Required Tender pursuant to this Article X fails to meet one or more of the criteria for a Required Tender shall be made by written notice to the Team (with copies sent to the NBA and the Players Association), no later than ten (10) days after the receipt of such Contract by the Players Association. Such notice must set forth the specific changes that the player asserts must be made to the offered Contract in order for it to constitute a Required Tender. Upon receipt of such notice, if the requested changes are necessary to satisfy the requirements of a Required Tender, the Team may within five (5) business days offer the player an amended Contract incorporating the requested changes. If the Team offers such an amended Contract, the player shall be precluded from asserting that such Contract does not constitute a timely and valid Required Tender.
\item
  For purposes of this Article X, any rights afforded to ``a Team that drafts a player'' shall also be afforded to any Team to which such rights are subsequently assigned.
\end{enumerate}

\hypertarget{free-agency}{%
\chapter{FREE AGENCY}\label{free-agency}}

\hypertarget{general-rules.}{%
\section{General Rules.}\label{general-rules.}}

\begin{enumerate}
\def\labelenumi{(\alph{enumi})}
\tightlist
\item
  Subject to the provisions of Article VII, including, but not limited to, Article VII, Section 6(b), and subject further to Article II, Section 14:

  \begin{enumerate}
  \def\labelenumii{(\roman{enumii})}
  \tightlist
  \item
    an Unrestricted Free Agent is free at any time beginning on the first day of the Moratorium Period to negotiate, and free at any time after the last day of the Moratorium Period to enter into, a Player Contract with any Team; and
  \item
    a Restricted Free Agent is free at any time beginning on the first day of the Moratorium Period to negotiate a Player Contract with his Prior Team and to negotiate an Offer Sheet (as defined in Section 5(b) below) with any Team other than his Prior Team, and is free at any time after the last day of the Moratorium Period to enter into a Player Contract with his Prior Team or an Offer Sheet with any Team other than his Prior Team.
  \end{enumerate}
\item
  No compensation obligation of any kind to another Team shall be applicable to any Free Agent. No right of first refusal of any kind shall be applicable to any Free Agent other than a Restricted Free Agent.
\end{enumerate}

\hypertarget{no-individually-negotiated-right-of-first-refusal.}{%
\section{No Individually-Negotiated Right of First Refusal.}\label{no-individually-negotiated-right-of-first-refusal.}}

\begin{enumerate}
\def\labelenumi{(\alph{enumi})}
\tightlist
\item
  No Player Contract, or any Renegotiation, Extension, or other amendment of a Player Contract, executed after the date of this Agreement, may include any individually- \textbf{\emph{{[}sic{]}}}
\item
  No right of first refusal rule, practice, policy, regulation or agreement providing for a right of first refusal shall be applied to any player as a result of that player's entry into a player contract with or the playing with any team in any professional basketball league other than the NBA.
\end{enumerate}

\hypertarget{withholding-services.}{%
\section{Withholding Services.}\label{withholding-services.}}

A player who withholds playing services called for by a Player Contract for more than thirty (30) days after the start of the last Season covered by his Player Contract shall be deemed not to have ``complet{[}ed{]} his Player Contract by rendering the playing services called for thereunder.'' Accordingly, such a player shall not be a Veteran Free Agent and shall not be entitled to negotiate or sign a Player Contract with any other professional basketball team unless and until the Team for which the player last played expressly agrees otherwise.

\hypertarget{qualifying-offers-to-make-certain-players-restricted-free-agents.}{%
\section{Qualifying Offers to Make Certain Players Restricted Free Agents.}\label{qualifying-offers-to-make-certain-players-restricted-free-agents.}}

\begin{enumerate}
\def\labelenumi{(\alph{enumi})}
\item
  \begin{enumerate}
  \def\labelenumii{(\roman{enumii})}
  \tightlist
  \item
    From the day following the Season covered by the second Option Year of a First Round Pick's Rookie Scale Contract through the immediately following June 30, the player's Team may make a Qualifying Offer to the player. If such a Qualifying Offer is made, then, on the July 1 following such Season, the player shall become a Restricted Free Agent, subject to a Right of First Refusal in favor of the Team (``ROFR Team''), as set forth in Section 5 below. If such a Qualifying Offer is not made, then the player shall become an Unrestricted Free Agent on such July 1. If a Team does not timely exercise its Option with respect to the first Option Year or second Option Year of a player's Rookie Scale Contract in accordance with Article VIII, the player shall, following his second or third Season (as the case may be) become an Unrestricted Free Agent.
  \item
    A Team that makes a Qualifying Offer to a player following the second Option Year of his Rookie Scale Contract may elect simultaneously to offer the player an alternative Contract covering six (6) Seasons that provides Salary for the first Salary Cap Year equal to the Maximum Annual Salary under Article II, Section 7(a), with annual increases in Salary equal to 10.5\% of the Salary for the first Salary Cap Year (a ``Maximum Qualifying Offer''). Providing a player with a Maximum Qualifying Offer shall have the consequence described in Section 5(b) below. A Maximum Qualifying Offer shall be subject to the following:
    (A) A Maximum Qualifying Offer shall contain only Base Compensation and no bonuses of any kind.
    (B) A Maximum Qualifying Offer shall state that the player's Base Compensation for the first Season shall equal ``the Maximum Annual Salary applicable to the player in the first Season of the Contract,'' and that the Base Compensation in each of the five (5) subsequent Seasons shall ``be increased by 10.5\% of the Base Compensation for the first Season.'' Such a Contract, if timely accepted by the player in accordance with subsection (ii)(D)below, shall be deemed amended to provide for specific Base Compensation for each Season covered by the Contract, based on the Maximum Annual Salary applicable to the player in the first Season.
    (C) A Maximum Qualifying Offer cannot contain an Option or ETO, and must provide full Base Compensation protection in each Season for lack of skill and non-insured injury or illness (with no individually-negotiated conditions or limitations on such protection).
    (D) The Team's offer of a Maximum Qualifying Offer must remain open for the same period that the player's Qualifying Offer remains open and cannot be withdrawn, except that if the Team withdraws its Qualifying Offer, the Maximum Qualifying Offer shall be deemed to be withdrawn simultaneously.
    (E) A player may accept either his Qualifying Offer or his Maximum Qualifying Offer, but not both.
  \end{enumerate}
\item
  Any Veteran Free Agent (other than a First Round Pick whose first Option Year or second Option Year was not exercised) who will have three (3) or fewer Years of Service as of the June 30 following the end of the last Season covered by his Player Contract will be a Restricted Free Agent if his Prior Team makes a Qualifying Offer to the player at any time from the day following such Season through the immediately following June 30. If such a Qualifying Offer is made, then, on the July 1 following the last Season covered by the player's Player Contract, the player shall become a Restricted Free Agent, subject to a Right of First Refusal in favor of the Team (``ROFR Team''), as set forth in Section 5 below. If such a Qualifying Offer is not made, then the player shall become an Unrestricted Free Agent on such July 1.
\item
  \begin{enumerate}
  \def\labelenumii{(\roman{enumii})}
  \tightlist
  \item
    A player who receives a Qualifying Offer must be given at least until the October 1 following its issuance to accept it, but in no event may a Qualifying Offer be accepted after the March 1 following its issuance. Notwithstanding the preceding sentence, a Qualifying Offer may be withdrawn by the Team at any time through the July 23 following its issuance. If the Qualifying Offer is not withdrawn on or before July 23, it may be withdrawn thereafter but only if the player agrees in writing to the withdrawal. If a Qualifying Offer is withdrawn, the player shall immediately become an Unrestricted Free Agent. If a Qualifying Offer is withdrawn on or after July 24, the Team also shall be deemed to have renounced the player in accordance with Article VII, Section 4(g).
  \item
    If a Qualifying Offer is neither withdrawn nor accepted and the deadline for accepting it passes, the Team's Right of First Refusal shall continue, subject to Section 5(a) below.
  \item
    A player who knows that he has a physical disability that would render him physically unable to perform the playing services required under a Player Contract the following Season may not validly accept a Qualifying Offer received under this Section 4 or Section 5 below, unless the ROFR Team consents after disclosure of such physical disability.Notwithstanding the immediately preceding sentence, a player who knows that he has a physical disability that would render him physically unable to perform the playing services required under a Player Contract the following Season remains subject to the ROFR Team's Right of First Refusal.
  \end{enumerate}
\item
  Any claim that a Contract offered as a Qualifying Offer or a Maximum Qualifying Offer fails to meet one or more of the criteria for a Qualifying Offer or a Maximum Qualifying Offer shall be made by notice to the Team, in writing, no later than ten (10) days after a copy of the Qualifying Offer or Maximum Qualifying Offer was given by the Team or the NBA to the Players Association. Such notice must set forth the specific changes that allegedly must be made to the offered Contract in order for it to constitute a Qualifying Offer or a Maximum Qualifying Offer. Upon receipt of such notice, if the requested changes are necessary to satisfy the requirements of a Qualifying Offer or a Maximum Qualifying Offer, the Team may, within five (5) business days, offer the player an amended Contract incorporating the requested changes. If the Team offers such an amended Contract, the player and the Players Association shall be precluded from asserting that such Contract does not constitute a timely and valid Qualifying Offer or Maximum Qualifying Offer.
\end{enumerate}

\hypertarget{restricted-free-agency.}{%
\section{Restricted Free Agency.}\label{restricted-free-agency.}}

\begin{enumerate}
\def\labelenumi{(\alph{enumi})}
\item
  If a Restricted Free Agent does not sign an Offer Sheet with any Team by March 1 of the Season for which the Qualifying Offer is made, and does not sign a Player Contract with the ROFR Team before that Season ends, then his ROFR Team may reassert its Right of First Refusal for the following Season by extending another Qualifying Offer (on the same terms as the prior Qualifying Offer) on or before the next June 30. A ROFR Team may continue to reassert its Right of First Refusal by following the foregoing procedure in each subsequent year in which that Restricted Free Agent does not sign an Offer Sheet with any Team by March 1 of the Season for which the Qualifying Offer is made, and does not sign a Player Contract with the ROFR Team before that Season ends. In each Season in which a Team reasserts its Right of First Refusal by extending another Qualifying Offer in accordance with this Section 5(a), the Team may also elect to simultaneously provide the player with a Maximum Qualifying Offer (on the same terms as the prior Maximum Qualifying Offer). Any such Qualifying Offer and Maximum Qualifying Offer shall be governed by the provisions of Section 4 above.
\item
  When a Restricted Free Agent receives an offer to sign a Player Contract from a Team other than the ROFR Team (the ``New Team''), which he desires to accept, he shall give to the ROFR Team a completed certificate substantially in the form of Exhibit G annexed hereto (the ``Offer Sheet''), signed by the Restricted Free Agent and the New Team, which shall have attached to it a Uniform Player Contract separately specifying: (i) the ``Principal Terms'' (as defined in Section 5(c) below) of the New Team's offer; and (ii) any non-Principal Terms of the New Team's offer that the ROFR Team is not required to match (as specified in Section 5(c) below) but which would be included in the player's Player Contract with the New Team if the ROFR Team does not exercise its Right of First Refusal. The Offer Sheet must be for a Player Contract with a term of more than one (1) season (not including any Option Year), unless the ROFR team has tendered the player both a Qualifying Offer and a Maximum Qualifying Offer, in which case the Offer Sheet must be for a Player Contract with a term of more than two (2) Seasons (not including any Option Year). In order to extend an Offer Sheet, the New Team must \textbf{\emph{{[}sic{]}}}
\item
  The following rules shall govern the signing of an Offer Sheet by a Restricted Free Agent who has one (1) or two (2) Years of Service:

  \begin{enumerate}
  \def\labelenumii{(\roman{enumii})}
  \tightlist
  \item
    Notwithstanding any other provision of this Agreement, no such Offer Sheet may provide for Salary plus Unlikely Bonuses in the first Salary Cap Year totaling more than 108\% of the Average Player Salary for the prior Salary Cap Year (or if the prior Salary Cap Year's Average Player Salary has not been determined, 108\% of the Estimated Average Player Salary for the prior Salary Cap Year). Annual increases or decreases in Salary and Unlikely Bonuses shall be governed by Article VII, Section 5(c)(1).
  \item
    If an Offer Sheet provides for the maximum allowable amount of Salary for the first two (2) Salary Cap Years pursuant to Section 5(c)(i) above, then, subject to Section 5(c)(iii) below, the Offer Sheet may provide for Salary for the third Salary Cap Year of up to the maximum amount that the player would have been eligible to receive for the third Salary Cap Year absent the restriction in the first sentence of Section 5(c)(i) above and had the player's Salary for the first two (2) Salary Cap Years been the maximum amount permitted under Article II, Section 7(a) and Article VII, Section 5(c)(1). If the Offer Sheet provides for Salary for the third Salary Cap Year in accordance with the foregoing sentence, then, subject to Section 5(c)(iii) below, (A) the player's Salary for each Salary Cap Year after the third Salary Cap Year may increase or decrease in relation to the previous Salary Cap Year's Salary by no more than 6.9\% of the Salary for the third Salary Cap Year, (B) the Offer Sheet cannot contain bonuses of any kind, and (C) the Offer Sheet must provide for 100\% of the Base Compensation in each Season to be protected for lack of skill and non-insured injury or illness.
  \item
    If a Team extends an Offer Sheet in accordance with Section 5(c)(ii) above, then, for purposes of determining whether the Team has Room for the Offer Sheet, the Salary for the first Salary Cap Year covered by the Offer Sheet shall be deemed to equal the average of the aggregate Salaries for such Salary Cap Year and each subsequent Salary Cap Year covered by the Offer Sheet. If the ROFR Team does not exercise its Right of First Refusal, the player's Salary for each Salary Cap Year covered by the Contract with the Team that extended the Offer Sheet shall be deemed to equal the average of the aggregate Salaries for each such Salary Cap Year. If the ROFR Team exercises its Right of First Refusal, the player's Salary for each Salary Cap Year covered by the Contract with the ROFR Team shall be the Salary for such Salary Cap Year as set forth in the Contract.
  \end{enumerate}
\item
  The Principal Terms of an Offer Sheet are only:

  \begin{enumerate}
  \def\labelenumii{(\roman{enumii})}
  \tightlist
  \item
    the term of the Contract;
  \item
    the fixed and specified Compensation that the New Team will pay or lend to the Restricted Free Agent and/or his designees as a signing bonus, Current Base Compensation, and/or Deferred Base Compensation in specified installments on specified dates;
  \item
    Incentive Compensation; provided, however, that the only elements of such Incentive Compensation that shall be included in the Principal Terms are the following: (A) bonuses that qualify as Likely Bonuses based upon the performance of the Team extending the Offer Sheet and the ROFR Team; and (B) Generally Recognized League Honors; and
  \item
    Any allowable amendments to the terms contained in the Uniform Player Contract (e.g., Base Compensation protection, an Early Termination Option, a trade bonus, etc.).
  \end{enumerate}
\item
  If, within seven (7) days from the date it receives an Offer Sheet, the ROFR Team gives to the Restricted Free Agent a ``First Refusal Exercise Notice'' substantially in the form of Exhibit H annexed hereto, then, subject to Section 5(h) below, such Restricted Free Agent and the ROFR Team shall be deemed to have entered into a Player Contract containing all the Principal Terms (but not any terms other than the Principal Terms) included in the Uniform Player Contract attached to the Offer Sheet (except that if the Contract contains an Exhibit 6, such Exhibit 6 shall be deemed deleted). Such Contract may not thereafter be amended in any manner for a period of one (1) year.
\item
  If the ROFR Team does not give the First Refusal Exercise Notice within the aforementioned seven (7) day period, then the player and the New Team shall be deemed to have entered into a Player Contract containing all of the terms and conditions included in the Uniform Player Contract attached to the Offer Sheet (including, if the Contract contains an Exhibit 6, that the player pass a physical examination to be conducted by the Team as a condition precedent to the validity of the Contract). Such Contract may not thereafter be amended in any manner for a period of one (1) year.
\item
  After exercising its Right of First Refusal as described in this Section 5, the ROFR Team may not trade the Restricted Free Agent for one (1) year, without the player's consent. Even with the player's consent, for one (1) year, neither the ROFR Team exercising its Right of First Refusal nor any other Team may trade the player to the Team whose Offer Sheet was matched.
\item
  Any Team may condition its First Refusal Exercise Notice on the player reporting for and passing, in the sole discretion of the Team, a physical examination to be conducted by a physician designated by the Team within two (2) days from its exercise of the Right of First Refusal. In connection with the physical examination, the player must supply all information reasonably requested of him, provide complete and truthful answers to all questions posed to him, and submit to all examinations and tests requested of him. In the event the player does not pass the physical examination: (i) the ROFR Team may withdraw its First Refusal Exercise Notice within two (2) days of such examination; and (ii) if the First Refusal Exercise Notice is withdrawn, the player and the New Team shall be deemed to have entered into a Player Contract in accordance with the provisions of Section 5(f) above. In the event the player does not submit to the requested physical examination within two (2) days of the exercise of the Right of First Refusal then, until such time as the player submits to the requested physical examination and is notified of the results, the ROFR Team's conditional First Refusal Exercise Notice shall remain in effect, except that the ROFR Team may elect at any time to withdraw its First Refusal Exercise Notice, which shall have the effect of invalidating the Offer Sheet and causing the Team that issued the Offer Sheet to be prohibited from signing or acquiring the player for a period of one (1) year from the date the First Refusal Exercise Notice was withdrawn. If the player does not submit to the requested physical examination on or before March 1, the Offer Sheet shall be deemed invalid and the Team that issued the Offer Sheet shall be prohibited from signing or acquiring the player for a period of one (1) year from such March 1.
\item
  A Team shall not be permitted to exercise its Right of First Refusal pursuant to an agreement to trade the Player Contract to another Team pursuant to Article VII, Section 8(e).
\item
  There may be only one (1) Offer Sheet signed by a Restricted Free Agent outstanding at any one time, provided that the Offer Sheet has also been signed by a Team. An Offer Sheet, both before and after it is given to the ROFR Team, may be revoked or withdrawn only upon the written consent of the ROFR Team, the New Team and the Restricted Free Agent. In such event, a Restricted Free Agent shall again be free to negotiate and sign an Offer Sheet with any Team, and any Team shall again be free to negotiate and sign an Offer Sheet with such Restricted Free Agent, subject only to the ROFR Team's renewed Right of First Refusal.
\item
  A Team that holds the Right of First Refusal with respect to a Restricted Free Agent may relinquish such Right of First Refusal at any time except during the period that the player has been given to accept a Qualifying Offer. If a Team relinquishes its Right of First Refusal with respect to a Restricted Free Agent, the player shall immediately become an Unrestricted Free Agent and the Team shall be deemed to have renounced the player in accordance with Article VII, Section 4(g) hereof. In order to relinquish its Right of First Refusal with respect to a Restricted Free Agent, a Team shall provide the NBA with an express, written statement relinquishing such Right of First Refusal. The NBA shall provide a copy of such statement to the Players Association by fax or e-mail within three (3) business days following its receipt thereof.
\item
  An expedited arbitration before the System Arbitrator, whose decision shall be final and binding upon all parties, shall be the exclusive method for resolving any disputes concerning this Section 5. If a dispute arises between the player and either the ROFR Team or the New Team, as the case may be, relating to the contents of an Offer Sheet, and/or whether the binding agreement is between the Restricted Free Agent and the New Team or the Restricted Free Agent and the ROFR Team, such dispute shall immediately be submitted to the System Arbitrator, who shall resolve such dispute within five (5) days.
\item
  A Restricted Free Agent may not give an Offer Sheet to the ROFR Team at any time after the March 1 of the Season for which he has been made a Qualifying Offer.
\item
  On the same day as the giving of an Offer Sheet to the ROFR Team, the ROFR Team shall cause a copy thereof to be given to the NBA, which shall cause a copy thereof to be promptly given to the Players Association. On the same day as the giving of a First Refusal Exercise Notice to the Restricted Free Agent, the ROFR Team shall cause a copy thereof to be given to the New Team, which shall cause a copy thereof to be promptly given to the NBA, which shall cause a copy thereof to be promptly given to the Players Association.
\item
  There may be no consideration of any kind given by one Team to another Team in exchange for a Team's decision to exercise or not to exercise its Right of First Refusal, or in exchange for a Team's decision to submit or not to submit an Offer Sheet to a Restricted Free
  Agent.
\item
  Any Offer Sheet, First Refusal Exercise Notice or other writing required or permitted to be given under this Section 5, shall be either by personal delivery or by pre-paid certified, registered or overnight mail addressed as follows:

  To any NBA Team: addressed to that Team at the principal address of such Team as then listed on the records of the NBA or at the Team's principal office, to the attention of the Team's general manager;

  To the NBA: National Basketball Association, Olympic Tower, 645 Fifth Avenue, New York, NY 10022, Attn: General Counsel;

  To the Players Association: National Basketball Players Association, Two Penn Plaza, Suite 2430, New York, NY 10121, Attn: Counsel.

  To a Restricted Free Agent: to his address listed on the Offer Sheet, and, if the Restricted Free Agent designates a representative on the Offer Sheet and lists such representative's address thereof, a copy shall be sent to such representative at such address.
\item
  An Offer Sheet shall be deemed given only when actually received by the ROFR Team. A First Refusal Exercise Notice shall be deemed given when sent by the ROFR Team. A Qualifying Offer shall be deemed given when sent by the ROFR Team. Other writings required or permitted to be given under this Section 5 shall be deemed given only when actually received by the party to whom addressed.
\end{enumerate}

\hypertarget{option-clauses}{%
\chapter{OPTION CLAUSES}\label{option-clauses}}

\hypertarget{team-options.}{%
\section{Team Options.}\label{team-options.}}

Except as provided by Article VIII, Section 1, a Player Contract shall not contain any option in favor of the Team, except an Option (as defined in Article I, Section 1(ll)) that: (i) is specifically negotiated between a Veteran or a Rookie (other than a First Round Pick) and a Team; (ii) authorizes the extension of such Contract for no more than one (1) year beyond the stated term; (iii) is exercisable only once; and (iv) provides that the Salary (excluding Incentive Compensation), Likely Bonuses and Unlikely Bonuses payable with respect to the Option Year are no less than 100\% of the Salary (excluding Incentive Compensation), Likely Bonuses and Unlikely Bonuses, respectively, payable with respect to the last year of the stated term of such Contract and that all other terms and conditions in the Option Year shall be unchanged from those that applied to the last year of the stated term of such Contract (including, but not limited to, the percentage of Base Compensation that is protected or insured).

\hypertarget{player-options.}{%
\section{Player Options.}\label{player-options.}}

A Player Contract shall not contain any option in favor of the player, except:

\begin{enumerate}
\def\labelenumi{(\alph{enumi})}
\tightlist
\item
  an Option that: (i) is specifically negotiated between a Veteran or a Rookie (other than a First Round Pick) and a Team; (ii) authorizes the extension of such Contract for no more than one (1) year beyond the stated term; (iii) is exercisable only once; and (iv) provides that the Salary (excluding Incentive Compensation), Likely Bonuses and Unlikely Bonuses payable with respect to the Option Year are no less than 100\% of the Salary (excluding Incentive Compensation), Likely Bonuses and Unlikely Bonuses, respectively, payable with respect to the last year of the stated term of such Contract and that all other terms and conditions in the Option Year shall be unchanged from those that applied to the last year of the stated term of such Contract (including, but not limited to, the percentage of Compensation that is protected or insured). If a Player Contract contains an Option in favor of the player and provides, in whole or in part, for Compensation protection or insurance in the Option Year, such Contract must also contain, in Exhibit 2 of the Contract under the heading ``Additional Conditions or Limitations,'' either the language set forth in subsection (A) below or the language set forth in subsection (B) below, but not both, and such language shall define the respective rights and obligations of the player and Team with respect to the subject matter thereof:

  \begin{enumerate}
  \def\labelenumii{(\Alph{enumii})}
  \tightlist
  \item
    ``If this Contract is terminated by Team prior to Player's exercise of the Option described in Exhibit 1 of the Contract, then Player shall be entitled to benefit from the Base Compensation protection or insurance provisions of this Exhibit 2 to the same extent as if the exercise of the Option by Player had occurred prior to Team's termination of the Contract.''
  \item
    ``If this Contract is terminated by Team prior to Player's exercise of the Option described in Exhibit 1 of the Contract, then Team shall be relieved of any obligation to pay Player any Compensation with respect to the Option Year.''
  \end{enumerate}

  No Player Contract that contains the language set forth in subsection (B) above may provide for the Option in favor of the player to be exercisable earlier than the day following the date of the Team's last game of the Season prior to the Option Year; and/or
\item
  an Early Termination Option (or ``ETO'') (as defined in Article I, Section 1(r)), provided that such ETO is exercisable only once and takes effect no earlier than the end of the fourth Season of the Contract. A Contract that does not provide for an ETO when signedmay not be amended to provide for an ETO during the original term of the Contract. If a Team and a player enter into an Extension (other than an Extension of a Rookie Scale Contract), the Contract may simultaneously be amended to provide for an ETO, provided that such ETO is exercisable only once and takes effect no earlier than the end of the fourth Season following (i) the Season during which the Extension is signed, or (ii) if the Extension is signed between Seasons, the date on which the Extension is signed (and, whether or not such an ETO is added, any previously-existing ETO must be eliminated). If a Team and player enter into an Extension of a Rookie Scale Contract, the Contract may simultaneously be amended to provide for an ETO, provided that such ETO is exercisable only once and takes effect no earlier than the end of the fourth Season of the extended term of the Contract.
\end{enumerate}

\hypertarget{no-conditional-options}{%
\section{No Conditional Options}\label{no-conditional-options}}

If a Contract contains any Option or ETO, the right of the Team or player to exercise such Option or ETO must be fixed at the time the Contract (or Extension) is entered into and may not be contingent upon the satisfaction of any individually-negotiated condition. In the case of an ETO, the Effective Season of such ETO also must be fixed at the time the Contract (or Extension) is entered into and may not be contingent upon the satisfaction of any individually-negotiated condition.

\hypertarget{exercise-period.}{%
\section{Exercise Period.}\label{exercise-period.}}

Any ETO must be exercised prior to the July 1 immediately prior to the Effective Season of such ETO. Any Option must be exercised prior to the July 1 immediately prior to the Season covered by the Option, except that an Option in favor of a player who would become a Restricted Free Agent if the Option were not exercised must be exercised prior to the June 25 immediately prior to the Season covered by such Option.

\hypertarget{option-buy-outs.}{%
\section{Option Buy-Outs.}\label{option-buy-outs.}}

Subject to the rules set forth in Article VII, a Player Contract that contains an Option or an ETO may provide for an Option Buy-Out Amount; provided, however, that in no event may an Option Buy-Out Amount exceed, in the case of an Option, 50\% of the Salary called for in the Option Year or, in the case of an ETO, 50\% of the Salary in the first Effective Season of the ETO.

\hypertarget{inapplicability-to-prior-player-contracts.}{%
\section{Inapplicability to Prior Player Contracts.}\label{inapplicability-to-prior-player-contracts.}}

The provisions of this Article XII that differ from the provisions of Article XII of the parties' prior collective bargaining agreement shall not apply to any Player Contract entered into prior to the date of this Agreement, except to the extent that any such Contract is extended following the date of this Agreement.

\hypertarget{option-exercise-notices.}{%
\section{Option Exercise Notices.}\label{option-exercise-notices.}}

The NBA shall provide the Players Association with copies of any Option or ETO exercise or non-exercise notice received by the NBA within two (2) business days of the NBA's receipt of such notice from the Team.

\hypertarget{circumvention}{%
\chapter{CIRCUMVENTION}\label{circumvention}}

\hypertarget{general-prohibitions.}{%
\section{General Prohibitions.}\label{general-prohibitions.}}

\begin{enumerate}
\def\labelenumi{(\alph{enumi})}
\tightlist
\item
  It is the intention of the parties that the provisions agreed to herein, including, without limitation, those relating to the Salary Cap, the Exceptions to the Salary Cap, the scope of Basketball Related Income, the Escrow System, the Rookie Scale, the Right of First Refusal, the Maximum Player Salary, and free agency, be interpreted so as to preserve the essential benefits achieved by both parties to this Agreement. Neither the Players Association, the NBA, nor any Team (or Team Affiliate) or player (or person or entity acting with authority on behalf of such player), shall enter into any agreement, including, without limitation, any Player Contract (including any Renegotiation, Extension, or amendment of a Player Contract), or undertake any action or transaction, including, without limitation, the assignment or termination of a Player Contract, which is, or which includes any term that is, designed to serve the purpose of defeating or circumventing the intention of the parties as reflected by all of the provisions of this Agreement.
\item
  It shall constitute a violation of Section 1(a) above for a Team (or Team Affiliate) to enter into an agreement or understanding with any sponsor or business partner or third-party under which such sponsor, business partner or third-party pays or agrees to pay compensation for basketball services (even if such compensation is ostensibly designated as being for non-basketball services) to a player under Contract to the Team. Such an agreement with a sponsor or business partner or third-party may be inferred where: (i) such compensation from the sponsor or business partner or third-party is substantially in excess of the fair market value of any services to be rendered by the player for such sponsor or business partner or third-party; and (ii) the Compensation in the Player Contract between the player and the Team is substantially below the fair market value of such Contract.
\item
  It shall constitute a violation of Section 1(a) above for a Team (or Team Affiliate) to have a financial arrangement with or offer a financial inducement to any player (not including retired players) not signed to a current Player Contract, except as permitted by this Agreement.
\item
  Nothing contained in Section 1(c) above shall interfere with a Team's obligation to pay a player Deferred Compensation earned under a prior Player Contract.
\end{enumerate}

\hypertarget{no-unauthorized-agreements.}{%
\section{No Unauthorized Agreements.}\label{no-unauthorized-agreements.}}

\begin{enumerate}
\def\labelenumi{(\alph{enumi})}
\tightlist
\item
  At no time shall there be any agreements or transactions of any kind (whether disclosed or undisclosed to the NBA), express or implied, oral or written, or promises, undertakings, representations, commitments, inducements, assurances of intent, or understandings of any kind (whether disclosed or undisclosed to the NBA), between a player (or any person or entity controlled by, related to, or acting with authority on behalf of, such player) and any Team (or Team Affiliate):

  \begin{enumerate}
  \def\labelenumii{(\roman{enumii})}
  \tightlist
  \item
    concerning any future Renegotiation, Extension, or amendment of an existing Player Contract, or entry into a new Player Contract;
  \item
    except as permitted by this Agreement or as set forth in a Uniform Player Contract (provided that the Team has not intentionally delayed submitting such Uniform Player Contract for approval by the NBA), involving compensation or consideration of any kind to be paid, \textbf{\emph{{[}sic{]}}}
  \item
    involving an investment or business opportunity to be furnished or made available to the player, or any person or entity controlled by, related to, or acting with authority on behalf of the player.
  \end{enumerate}
\item
  In addition to the foregoing, it shall be a violation of this Section 2 for any Team (or Team Affiliate) or any player (or any person or entity controlled by, related to, or acting with authority on behalf of, such player) to attempt to enter into or to intentionally solicit any agreement, transaction, promise, undertaking, representation, commitment, inducement, assurance of intent or understanding that would be prohibited by Section 2(a) above.
\item
  A violation of Section 2(a) above may be proven by direct or circumstantial evidence, including, but not limited to, evidence that a Player Contract or any term or provision thereof cannot rationally be explained in the absence of conduct violative of Section 2(a).
\item
  In any proceeding brought before the System Arbitrator pursuant to this Section 2, no adverse inference shall be drawn against the party initiating such proceeding because that party, when it first suspected or believed that a violation of Section 2 may have occurred, deferred the initiation of such proceeding until it had further reason to believe that such a violation had occurred.
\item
  A player will not be found to have committed a violation of Section 2(a)(ii) above if the violation is the Team's intentional delay in submitting a Uniform Player Contract to the NBA and this was done without the player's knowledge.
\end{enumerate}

\hypertarget{penalties.}{%
\section{Penalties.}\label{penalties.}}

\begin{enumerate}
\def\labelenumi{(\alph{enumi})}
\tightlist
\item
  Upon a finding of a violation of Section 1 above by the System Arbitrator, but only following the conclusion of any appeal to the Appeals Panel, the Commissioner shall be authorized to:

  \begin{enumerate}
  \def\labelenumii{(\roman{enumii})}
  \tightlist
  \item
    impose a fine of up to \$2,500,000 (50\% of which shall be payable to the NBA, and 50\% of which shall be payable to the NBPA-Selected Charitable Organization (as defined in Article VI, Section 6(a))) on any Team found to have committed such violation for the first time;
  \item
    impose a fine of up to \$3,000,000 (50\% of which shall be payable to the NBA, and 50\% of which shall be payable to the NBPA-Selected Charitable Organization on any Team found to have committed such violation for at least the second time;
  \item
    direct the forfeiture of one first round draft pick;
  \item
    void any Player Contract, or any Renegotiation, Extension, or amendment of a Player Contract, between any player and any Team when both the player (or any person or entity acting with authority on behalf of such player) and the Team (or Team Affiliate) are found to have committed such violation; and/or
  \item
    void any other transaction or agreement found to have violated Section 1 above.
  \end{enumerate}
\item
  Upon a finding of a violation of Section 2 above by the System Arbitrator, but only following the conclusion of any appeal to the Appeals Panel, the Commissioner shall be authorized to:

  \begin{enumerate}
  \def\labelenumii{(\roman{enumii})}
  \tightlist
  \item
    impose a fine of up to \$5,000,000 on any Team found to have committed such violation (50\% of which shall be payable to the NBA, and 50\% of which shall be payable to the NBPA-Selected Charitable Organization);
  \item
    direct the forfeiture of draft picks;
  \item
    when both the player (or any person or entity acting with authority on behalf of such player) and the Team (or Team Affiliate) are found to have committed such violation, (A) void any Player Contract, or any Renegotiation, Extension, or amendment of a Player Contract, between such player and such Team, (B) impose a fine of up to \$100,000 on any player (50\% of which shall be payable to the NBA, and 50\% of which shall be payable to the NBPA-Selected Charitable Organization), and/or (C) prohibit any future Player Contract, or any Renegotiation, Extension, or amendment of a Player Contract, between such player and such Team;
  \item
    suspend for up to one (1) year any Team personnel found to have willfully engaged in such violation; and/or
  \item
    void any other transaction or agreement found to have violated Section 2 above.
  \end{enumerate}
\item
  In any proceeding before the System Arbitrator in which it is alleged that a player agent or other person or entity acting with authority on behalf of a player has violated Section 2 above, the System Arbitrator shall make a specific determination with respect to such allegation. If the System Arbitrator finds such violation and such finding, if appealed, is affirmed by the Appeals Panel, the System Arbitrator shall refer such finding to the Players Association, which shall accept as binding and conclusive the finding(s) of the System Arbitrator (or, in the case of an appeal, the Appeals Panel) that a violation of Section 2(a) or 2(b) has occurred and shall consider such finding(s) as establishing a violation of the Players Association's regulations applicable to such person or entity. The Players Association represents that it will impose such discipline as is appropriate under the circumstances on the person or entity found to have violated Section 2 above, and that it will promptly notify the NBA of the discipline imposed; provided, however, that in no event shall the penalty imposed upon a player agent found to have violated Section 2 above be less than a one-year suspension of that player agent's certification by the Players Association.
\end{enumerate}

\hypertarget{production-of-tax-materials.}{%
\section{Production of Tax Materials.}\label{production-of-tax-materials.}}

In any proceeding to enforce Section 1 or 2 above, the System Arbitrator shall have the authority, upon good cause shown, to direct any Team, Team Affiliate, or player to produce any tax returns or other relevant tax materials disclosing income figures for the player (non-income figures may be redacted), or disclosing expense figures by the Team or Team Affiliate (non-expense figures may be redacted), which materials shall not be released to the general public or the media and shall be treated as strictly confidential by all parties.

\hypertarget{transactions-with-retired-players.}{%
\section{Transactions With Retired Players.}\label{transactions-with-retired-players.}}

\begin{enumerate}
\def\labelenumi{(\alph{enumi})}
\item
  If (i) a Team or Team Affiliate enters into a transaction after the date of this Agreement with a retired player who played for the Team within the five-year (5) period preceding such transaction and the terms of such transaction provide for the retired player to be paid compensation or consideration in excess of \$10,000 or to be provided with an investment or business opportunity, and, if (ii) the compensation the retired player received from the Team when he was a player was substantially below the then fair market value of such player's basketball services under the Salary Cap system, then the NBA may challenge the transaction, pursuant to the procedures set forth in Section 5(b) below, on the ground that: (A) the compensation or consideration to the retired player substantially exceeds the then fair market value of the services or other consideration provided by the retired player in the transaction; or that (B) the amount of the retired player's investment or the benefit conferred upon the retired player as a result of the investment or business opportunity is not commercially reasonable, given the relative risks and rewards of such investment.
\item
  \begin{enumerate}
  \def\labelenumii{(\roman{enumii})}
  \tightlist
  \item
    Any challenge under this Section 5 shall be filed in writing with a business valuation expert jointly selected by the NBA and the Players Association. In the event the parties cannot agree on the identity of a business valuation expert, a business valuation expert shall be selected in the same manner set forth in Article XXXI, Section 6 for the selection of a Grievance Arbitrator in the absence of an agreement between the parties. The business valuation expert shall conduct a hearing in which the player or retired player, the Team and/or Team Affiliate, the Players Association, and the NBA are afforded the opportunity to appear and participate. The NBA shall have the burden of proof in the proceeding. The business valuation expert may permit discovery of relevant documents necessary to undertake the valuation, and shall render a decision within fifteen (15) days following the conclusion of the hearing. Within ten (10) days of any decision by the business valuation expert, any of the parties may file an appeal with the System Arbitrator, who shall conduct a hearing and render a decision within twenty (20) days of the filing of the appeal. In any such proceeding, the System Arbitrator shall apply an ``arbitrary and capricious'' standard of review. There shall be no right of further appeal to the Appeals Panel.
  \item
    If the NBA prevails in its challenge under this Section 5, the difference between (A) the compensation or consideration received by the retired player, or the value of the investment or business opportunity received by the retired player (net of any contribution by the retired player), and (B) a reasonable estimate of the fair market value of the services or other consideration provided by the retired player, or a reasonable estimate of the fair market value of the investment or business opportunity, in each case as determined by the business valuation expert or the System Arbitrator, as the case may be, shall be included in the Team's Team Salary, subject to the Team's Room and other Salary Cap rules, and further subject to any allocation over time that the business valuation expert or System Arbitrator determines is appropriate. In the event that any amount required to be included in the Team Salary pursuant to this subsection exceeds the Team's Room, the challenged transaction or arrangement shall be rescinded and of no further force and effect.
  \item
    If the NBA prevails in its challenge under this Section 5, and the retired player and the Team and/or Team Affiliate renegotiate or terminate the transaction, any revised terms of the transaction shall be promptly disclosed to the NBA and the Players Association, and may, at the request of the NBA, be re-subjected to the procedures of this Section 5(b).
  \end{enumerate}
\item
  Any information disclosed to the League Office and the Players Association pursuant to the procedures of this Section 5 shall be treated strictly confidential, and shall not be released to the general public or the media.
\end{enumerate}

\hypertarget{anti-collusion-provisions}{%
\chapter{ANTI-COLLUSION PROVISIONS}\label{anti-collusion-provisions}}

\hypertarget{no-collusion.}{%
\section{No Collusion.}\label{no-collusion.}}

Subject to Section 2 below, no NBA Team, its employees or agents, will enter into any contracts, combinations or conspiracies, express or implied, with the NBA or any other NBA Team, their employees or agents: (a) to negotiate or not to negotiate with any Veteran or Rookie; (b) to submit or not to submit an Offer Sheet to any Restricted Free Agent; (c) to offer or not to offer a Player Contract to any Free Agent; (d) to exercise or not to exercise a Right of First Refusal; or (e) concerning the terms or conditions of employment offered to any Veteran or Rookie.

\hypertarget{non-collusive-conduct.}{%
\section{Non-Collusive Conduct.}\label{non-collusive-conduct.}}

The following is a non-exhaustive list of conduct that shall not be deemed a violation of Section 1 above:

\begin{enumerate}
\def\labelenumi{(\alph{enumi})}
\tightlist
\item
  the formulation and negotiation of collective bargaining proposals;
\item
  agreements between NBA Teams necessary to the assignment of a Player Contract of a Veteran or the assignment of the exclusive negotiating rights to a Draft Rookie, where such assignment is contingent upon (i) the signing by the Veteran of an amendment to an existing Player Contract (including, for example, an Extension), or (ii) the signing by the Draft Rookie of a new Player Contract; provided, however, that if such contingency is fulfilled by the Veteran entering into an amended Player Contract (including, for example, an Extension) or the Draft Rookie entering into a new Player Contract, this subsection shall only apply if the assignment is actually consummated;
\item
  an agreement between NBA Teams concerning the signing of a new Player Contract by a Veteran Free Agent with his Prior Team, where such agreement is necessary for the subsequent assignment of the new Player Contract between the agreeing Teams; provided, however, that this Section 2(c) shall apply only if the subsequent assignment is consummated, and only if the agreement and the new Player Contract comply with the provisions of Article VII, Section 8(e);
\item
  conduct authorized by the terms and conditions of the NBA Draft (as set forth in Article X above);
\item
  conduct authorized by any provision of this Agreement or conduct by the NBA League Office, undertaken in good faith, that reflects a reasonable interpretation of this Agreement or a Player Contract; and
\item
  any action taken by the NBA League Office to exclude from the NBA, suspend or discipline any player for any reason authorized or permitted by any provision of this Agreement.
\end{enumerate}

(This subsection, however, shall not affect any other rights of any player or the Players Association to contest such action.)

\hypertarget{individual-negotiations.}{%
\section{Individual Negotiations.}\label{individual-negotiations.}}

No NBA Team shall fail or refuse to negotiate with, or enter into a Player Contract with, any player who is free to negotiate and sign a Player Contract with any NBA Team, on any of the following grounds:

\begin{enumerate}
\def\labelenumi{(\alph{enumi})}
\tightlist
\item
  that the player has previously been subject to the exclusive negotiating rights obtained by another NBA Team in an NBA Draft; or
\item
  that the player has previously refused or failed to enter into a Player Contract containing an Option; or
\item
  that the player has become a Restricted Free Agent or an Unrestricted Free Agent; or
\item
  that the player is or has been subject to a Right of First Refusal.
\end{enumerate}

The fact that a Team has not negotiated with, made any offers to, or entered into any Player Contracts with players who are free to negotiate and sign Player Contracts with any Team, shall not, by itself, be deemed proof that such Team failed or refused to negotiate with, make any offers to, or enter into any Player Contracts with any players on any of the prohibited grounds referred to in this Section 3.

\hypertarget{league-disclosures.}{%
\section{League Disclosures.}\label{league-disclosures.}}

The NBA League Office shall not knowingly communicate or disclose, directly or indirectly, to any NBA Team that another NBA Team has negotiated with or is negotiating with any Restricted Free Agent, unless and until an Offer Sheet (as defined in Article XI, Section 5(b)) shall have been given to the ROFR Team (as defined in Article XI, Section 4(a)), or any Free Agent prior to the execution of a Player Contract with that player.

\hypertarget{enforcement-of-anti-collusion-provisions.}{%
\section{Enforcement of Anti-Collusion Provisions.}\label{enforcement-of-anti-collusion-provisions.}}

\begin{enumerate}
\def\labelenumi{(\alph{enumi})}
\tightlist
\item
  Any player, or the Players Association acting on behalf of a player or players, may bring an action before the System Arbitrator alleging a violation of Section 1 above. Issues of relief and liability shall be determined in the same proceeding (including the amount of damages, pursuant to Section 9 below, if any). The complaining party will bear the burden of demonstrating by a clear preponderance of the evidence that the challenged conduct was in violation of Section 1 above and caused economic injury to such player(s); provided, however, that the Players Association may, in the absence of economic injury to any player, bring an action before the System Arbitrator claiming a violation of Section 1 above (which must be proved by a clear preponderance of the evidence) and seeking only declaratory relief or a direction to cease and desist from the challenged conduct.
\item
  The provisions of this Agreement are not intended to create any substantive rights in any party, other than as provided for herein. This Agreement may be enforced, and any alleged violations may be remedied, only as provided for herein.
\end{enumerate}

\hypertarget{satisfaction-of-burden-of-proof.}{%
\section{Satisfaction of Burden of Proof.}\label{satisfaction-of-burden-of-proof.}}

The failure by a Team or Teams to submit Offer Sheets to Restricted Free Agents, or to make offers or sign Contracts for the playing services of Free Agents shall not, by itself or in combination only with evidence about the playing skills of the player(s) not receiving such offers or contracts, satisfy the burden of proof set forth in Section 5 above. However, such evidence may support a finding of a violation of Section 1 above, but only in combination with other evidence that either by itself or in combination with the evidence referred to in the immediately preceding sentence indicates that the challenged conduct was in violation of Section 1 above and, except in cases where the Players Association seeks only declaratory relief or a direction to cease and desist from the challenged conduct, caused economic injury to such player(s).

\hypertarget{summary-judgment.}{%
\section{Summary Judgment.}\label{summary-judgment.}}

The System Arbitrator may, at any time following the conclusion of any permitted discovery, determine whether or not the complainant's evidence is sufficient to raise a genuineissue of material fact capable of satisfying the standards imposed by Sections 5 and 6 above. If the System Arbitrator determines that complainant's evidence is not so sufficient, he shall dismiss the action.

\hypertarget{remedies-for-economic-injury.}{%
\section{Remedies For Economic Injury.}\label{remedies-for-economic-injury.}}

In the event that an individual player or players, or the Players Association acting on his or their behalf, successfully proves a violation of Section 1 above that has caused economic injury, the player or players determined by the System Arbitrator to have suffered economic injury as a result of the violation will have the right:

\begin{enumerate}
\def\labelenumi{(\alph{enumi})}
\tightlist
\item
  to terminate his (or their) existing Player Contract(s) at his (or their) option (however, such termination shall not take effect until the conclusion of a then-ongoing NBA Season, if any). Such right of termination shall not arise until the recommendation of the System Arbitrator finding a violation is no longer subject to further appeal and must be exercised by the player within thirty (30) days therefrom. If, at the time the Player Contract is terminated, such player would have been an Unrestricted Free Agent pursuant to the provisions of this Agreement, he shall immediately become an Unrestricted Free Agent. If, at the time the Player Contract is terminated, such player would have been a Restricted Free Agent pursuant to the provisions of this Agreement, such player shall immediately become a Restricted Free Agent upon such termination; however, any such player may choose to reinstate his Player Contract at any time up until September 15 of that year; and
\item
  to recover damages as described in Section 9 below. However, if the player terminates his Player Contract under Section 9(a) above and does not reinstate it pursuant thereto, he may not recover damages for the period after such termination takes effect. A player who does not terminate his contract, or who reinstates it pursuant to Section 9(a) above, may recover damages for the entire period of his injury.
\end{enumerate}

\hypertarget{calculation-of-damages.}{%
\section{Calculation of Damages.}\label{calculation-of-damages.}}

Upon any finding of a violation of Section 1 above that has caused economic injury, compensatory damages (i.e., the amount by which any player has been injured as a result of such violation) and non-compensatory damages (i.e., the amount exceeding compensatory damages) shall be awarded as follows:

\begin{enumerate}
\def\labelenumi{(\alph{enumi})}
\tightlist
\item
  Two (2) times the amount of compensatory damages, in the event that all of the Teams found to have violated Section 1 above have committed such a violation for the first time. Any Team found to have committed such a violation for the first time shall be jointly and severally liable for two (2) times the amount of compensatory damages.
\item
  Three (3) times the amount of compensatory damages, in the event that any of the Teams found to have violated Section 1 above have committed such a violation for the second time during the term of this Agreement. In the event that damages are awarded pursuant to this Section 9(b): (i) any Team found to have committed such a violation for the first time shall be jointly and severally liable for two (2) times the amount of compensatory damages; and (ii) any Team found to have committed such a violation for the second time during the term of this Agreement shall be jointly and severally liable for three (3) times the amount of compensatory damages.
\item
  Three (3) times the amount of compensatory damages, plus, for each Team found to have violated Section 1 above for at least the third time during the term of this Agreement, three million dollars (\$3,000,000), in the event that any of the Teams found to have violatedSection 1 above have committed such violation for at least the third time during the term of this Agreement. In the event that damages are awarded pursuant to this Section 9(c): (i) any Team found to have committed such a violation for the first time shall be jointly and severally liable for two (2) times the amount of compensatory damages; (ii) any Team found to have committed such a violation for at least the second time during the term of this Agreement shall be jointly and severally liable for three (3) times the amount of compensatory damages; and (iii) any Team found to have committed such a violation for at least the third time during the term of this Agreement shall, in addition, pay a fine of three million dollars (\$3,000,000).
\end{enumerate}

\hypertarget{payment-of-damages.}{%
\section{Payment of Damages.}\label{payment-of-damages.}}

In the event damages are awarded pursuant to Section 9 above, the amount of compensatory damages shall be paid to the injured player or players. The amount of non-compensatory damages, including any fines, shall be paid to the Players Association, which may use it for any purpose other than to pay it to any player who has received compensatory damages, except that any such player may receive some portion of a non-compensatory damage award as part of a proportional distribution to Players Association members.

\hypertarget{effect-of-damages-on-salary-cap.}{%
\section{Effect of Damages on Salary Cap.}\label{effect-of-damages-on-salary-cap.}}

In the event damages are awarded pursuant to Section 9 above, the amount of non-compensatory damages, including any fines, will not be included in any of the computations described in Article VII above. The amount of compensatory damages awarded will be included in such computations.

\hypertarget{contribution.}{%
\section{Contribution.}\label{contribution.}}

Any Team found liable under Section 1 above shall have the right to seek contribution from any other Team found liable for the same violation in a proceeding before the Commissioner who shall determine what contribution, if any, is fair and equitable. The Commissioner's determination with regard to contribution shall be final and binding upon and unappealable by any Team. A contribution determination by the Commissioner may be appealed by the Players Association to the System Arbitrator, except that if such a determination involves fewer than four (4) Teams found to have committed a violation of Section 1 above and allocates damages equally among the Teams found liable, there shall be no appeal to the System Arbitrator. In the event of a contribution determination by the Commissioner, the NBA shall provide the Players Association with the data and information that the Commissioner used or relied upon in making his determination. Any contribution determination appealed by the Players Association to the System Arbitrator shall be upheld unless it is clearly erroneous.

\hypertarget{no-reimbursement.}{%
\section{No Reimbursement.}\label{no-reimbursement.}}

Any damages awarded pursuant to Section 9 above must be paid by the individual Teams found liable and those Teams may not be reimbursed or indemnified by any other Team or the NBA, except to the extent of any award of contribution made pursuant to Section 12 above.

\hypertarget{costs.}{%
\section{Costs.}\label{costs.}}

In any action brought for an alleged violation of Section 1 above, the System Arbitrator shall order the payment of reasonable attorneys' fees by any party found to have brought such an action or to have asserted a defense to such an action without any reasonable basis for asserting such a claim or defense.

\hypertarget{termination-of-agreement.}{%
\section{Termination of Agreement.}\label{termination-of-agreement.}}

The Players Association shall have the right to terminate this Agreement (pursuant to the procedure set forth in Article XXXIX, Section 3 of this Agreement), under the following circumstances:

\begin{enumerate}
\def\labelenumi{(\alph{enumi})}
\tightlist
\item
  Where there has been a finding or findings of one (1) or more instances of a violation of Section 1 above with respect to any one NBA Season during the term of this Agreement which, either individually or in total, involved five (5) or more Teams and caused injury to five (5) or more players; or
\item
  Where there has been a finding or findings of one (1) or more instances of a violation of Section 1 above with respect to any two (2) consecutive NBA Seasons during the term of this Agreement which, either individually or in total, involved seven (7) or more Teams and caused economic injury to seven (7) or more players. For purposes of this Section 15(b), a player found to have been injured by a violation of Section 1 above in each of two (2) consecutive Seasons shall be counted as an additional player injured by such a violation for each such NBA Season; or
\item
  Where, in a proceeding brought by the Players Association, it is shown by clear and convincing evidence that during the term of this Agreement ten (10) or more Teams have engaged in a violation or violations of Section 1 above, causing economic injury to one or more NBA players. In order to terminate this Agreement pursuant to this subsection (c) and Article XXXIX, Section 3 of this Agreement:

  \begin{enumerate}
  \def\labelenumii{(\roman{enumii})}
  \tightlist
  \item
    the proceeding must be brought by the Players Association; and
  \item
    the NBA and the System Arbitrator must be informed at the outset of any such proceeding that the Players Association is proceeding under this subsection (c) for the purpose of establishing its entitlement to terminate this Agreement.
  \end{enumerate}
\end{enumerate}

\hypertarget{discovery.}{%
\section{Discovery.}\label{discovery.}}

\begin{enumerate}
\def\labelenumi{(\alph{enumi})}
\tightlist
\item
  In any of the actions described in this Article XIV, the System Arbitrator shall grant reasonable and expedited discovery upon the application of any party where, and to the extent, he or she determines it is reasonable to do so. Such discovery may include the production of documents and the taking of depositions.
\item
  Notwithstanding Section 16(a) above, the Players Association and the NBA shall each have the right to obtain discovery upon request in any three (3) proceedings brought under this Article XIV during the term of this Agreement. The scope and extent of such discovery shall be determined by the System Arbitrator.
\end{enumerate}

\hypertarget{time-limits.}{%
\section{Time Limits.}\label{time-limits.}}

Any action under Section 1 above must be brought within 90 days of the time when the player knows or reasonably should have known that he had a claim, or within 90 days of the start of the NBA Season in which a violation of Section 1 above is claimed, whichever is later. In the absence of a System Arbitrator, the complaining party shall file such claim for breach of this Agreement pursuant to Section 301 of the Labor Management Relations Act in either the U.S. District Court for the Southern District of New York or the U.S. District Court for the District of New Jersey. Any party alleged to have violated Section 1 shall have the right, prior to any proceedings on the merits, to make an initial motion to dismiss any complaint that does not comply with the timeliness requirement of this Section 17.

\hypertarget{certifications}{%
\chapter{CERTIFICATIONS}\label{certifications}}

\hypertarget{contract-certification.}{%
\section{Contract Certification.}\label{contract-certification.}}

\begin{enumerate}
\def\labelenumi{(\alph{enumi})}
\tightlist
\item
  Every Player Contract (other than a 10-Day Contract), or any Renegotiation, Extension or other amendment of a Player Contract, entered into during the term of this Agreement shall be accompanied by a certification, sworn to separately by (i) the person who executed the Player Contract on behalf of the Team, (ii) the player, and (iii) any player agent who negotiated the Contract on behalf of the player, under penalties of perjury, that the Player Contract, Renegotiation, Extension, or amendment sets forth all components of a player's Compensation from the Team or any Team Affiliate, and that there are no agreements or transactions of any kind (whether disclosed or undisclosed to the NBA), express or implied, oral or written, or promises, undertakings, representations, commitments, inducements, assurances of intent, or understandings of any kind (whether disclosed or undisclosed to the NBA):

  \begin{enumerate}
  \def\labelenumii{(\roman{enumii})}
  \tightlist
  \item
    concerning any future Renegotiation, Extension, or amendment of an existing Player Contract, or entry into a new Player Contract; or
  \item
    except as permitted by this Agreement or contained in such Uniform Player Contract, involving compensation or consideration of any kind to be paid, furnished or made available to the player, or any person or entity controlled by, related to, or acting with authority on behalf of the player; or
  \item
    involving an investment or business opportunity to be furnished or made available to the player, or any person or entity controlled by, related to, or acting with authority on behalf of the player.
  \end{enumerate}
\item
  Prior to the assignment of any Player Contract of a player who is in the last Salary Cap Year of the Contract (or the last Salary Cap Year before the player has the right to terminate the Contract), the player, the player's agent, and the Team to which such Contract is to be assigned shall each submit to the NBA a certification, sworn to under penalties of perjury, that other than the Player Contract that has been assigned, or as permitted by this Agreement, there are no agreements or transactions of any kind (whether disclosed or undisclosed to the NBA), express or implied, oral or written, or promises, undertakings, representations, commitments, inducements, assurances of intent, or understandings of any kind (whether disclosed or undisclosed to the NBA), between the player (or the player's agent or any person or entity controlled by or related to the player) and the Team to which the Player Contract is to be assigned or a Team Affiliate of the Team to which the Player Contract is to be assigned (i) concerning any future Renegotiation, Extension, or amendment of the Player Contract that has been assigned, or (ii) involving compensation or consideration of any kind (including an investment or business opportunity) to be paid, furnished or made available to the player, or any person or entity controlled by, related to, or acting with authority on behalf of the player.
\item
  If a player, within two (2) years after the assignment of such player's Player Contract, enters into a new Player Contract, or any Renegotiation, Extension, or other amendment of the Player Contract that had been assigned, the Team, the player, and the player's agent shall each submit to the NBA a certification, sworn to under penalties of perjury, that, at the time of the assignment, other than the Player Contract that has been assigned, or as permitted by this Agreement, there were no agreements or transactions of any kind (whether disclosed or undisclosed to the NBA), express or implied, oral or written, or promises, undertakings, representations, commitments, inducements, assurances of intent, or understandings of any kind, (whether disclosed or undisclosed to the NBA), between the player (or the player's agent or any person or entity controlled by or related to the player) and the Team to which the Player Contract has been assigned or a Team Affiliate of the Team to which the Player Contract has been assigned (i) concerning any future Renegotiation, Extension, or amendment of the Player Contract that has been assigned, or (ii) involving compensation or consideration of any kind (including an investment or business opportunity) to be paid, furnished or made available to the player, or any person or entity controlled by, related to, or acting with authority on behalf of the player.
\item
  If an agent, player, or Team fails or refuses to provide a certification called for under this Article XV, the NBA shall have the option, in its sole discretion, to approve or disapprove the transaction in question. In the case of a failure or refusal by an agent, and whether the transaction in question is approved or disapproved, the Players Association shall take appropriate disciplinary action against the agent.
\end{enumerate}

\hypertarget{end-of-season-certification.}{%
\section{End of Season Certification.}\label{end-of-season-certification.}}

\begin{enumerate}
\def\labelenumi{(\alph{enumi})}
\tightlist
\item
  At the conclusion of each NBA Season, a Governor (or Alternate Governor) and the executive primarily responsible for basketball operations on behalf of the Team shall each submit to the NBA a certification, sworn to under penalties of perjury, that the Team has not, to the extent of their knowledge after reasonable inquiry, violated the terms of Article XIV, Section 1, nor received from the NBA League Office any communication disclosing that an NBA Team has negotiated with any Free Agent prior to the execution of a Player Contract with that player. Upon receipt of each such certification, the NBA shall forward a copy of the certification to the Players Association.
\item
  A violation of this Section 2 may be deemed evidence of a violation of Article XIV, Section 1.
\end{enumerate}

\hypertarget{false-certification.}{%
\section{False Certification.}\label{false-certification.}}

Any criminal complaint of perjury filed by the NBA or any Team based upon a certification required pursuant to Section 1 above shall be against the player, the player's agent, and the Team official making such certification.

\hypertarget{mutual-reservation-of-rights}{%
\chapter{MUTUAL RESERVATION OF RIGHTS}\label{mutual-reservation-of-rights}}

Upon the expiration or termination of this Agreement, no person shall be deemed to have waived, by reason of the entry into or effectuation of this Agreement, any other collective bargaining agreement, or any Player Contract, or any of the terms of any of them, or by reason of any practice or course of dealing, their respective rights under law with respect to any issue or their ability to advance any legal argument.

\hypertarget{procedure-with-respect-to-playing-conditions-at-various-facilities}{%
\chapter{PROCEDURE WITH RESPECT TO PLAYING CONDITIONS AT VARIOUS FACILITIES}\label{procedure-with-respect-to-playing-conditions-at-various-facilities}}

\chaptermark{PROCEDURE WITH RESPECT TO \ldots}

When a new franchise is granted, or when an existing franchise moves to another city or a new or different arena, the Players Association shall, upon request and within a reasonable period of time, have the right to inspect the facility to be used by such franchise. Similarly, the Players Association shall, upon reasonable notice to the Team(s) involved and the NBA, have the right to inspect the training camp and practice facilities used by such Team(s). If, following such inspection, the Players Association is of the opinion that the playing conditions at such facility will endanger the health and safety of NBA players, it shall promptly notify the Commissioner and the Team involved in writing. Promptly following the receipt of such notice, representatives of the Players Association and of the Team(s) involved, and the Commissioner or his designee shall meet in an effort to resolve the matter. It is agreed that the failure of the parties to resolve the matter shall not impair the legally binding effect of this Agreement or create any right, during the term of this Agreement, to (a) unilaterally implement any provision concerning such unresolved matter, (b) lockout, or (c) strike. If no resolution satisfactory to the Players Association, the Team(s) involved and the Commissioner is reached, the issue of whether the playing conditions at the facility in question will endanger the health and safety of NBA players will, without interruption of the schedule or training camp or practice activities, immediately be submitted to and determined by the Grievance Arbitrator in accordance with the provisions of Article XXXI; provided, however, that the Grievance Arbitrator need not render an award within 24 hours of the conclusion of the hearing, but shall issue his award as expeditiously as possible under the circumstances.

\hypertarget{travel-accommodations-locker-room-facilities-and-parking}{%
\chapter{TRAVEL ACCOMMODATIONS, LOCKER ROOM FACILITIES AND PARKING}\label{travel-accommodations-locker-room-facilities-and-parking}}

\chaptermark{TRAVEL ACCOMMODATIONS, LOCKER ROOM \ldots}

\hypertarget{hotel-arrangements.}{%
\section{Hotel Arrangements.}\label{hotel-arrangements.}}

\begin{enumerate}
\def\labelenumi{(\alph{enumi})}
\tightlist
\item
  Each Team agrees to use its best efforts to make the following arrangements for its players while they are ``on the road'':

  \begin{enumerate}
  \def\labelenumii{(\roman{enumii})}
  \tightlist
  \item
    to have their baggage picked up by porters;
  \item
    to have them stay in first class hotels; and
  \item
    to have extra-long beds available to them in each hotel.
  \end{enumerate}

  If there is a finding that a Team has committed a willful violation of this Section 1(a), the NBA shall impose a \$5,000 fine on such Team.
\item
  When its players are ``on the road,'' each Team shall provide an individual hotel room for each player.
\end{enumerate}

\hypertarget{first-class-travel.}{%
\section{First Class Travel.}\label{first-class-travel.}}

\begin{enumerate}
\def\labelenumi{(\alph{enumi})}
\tightlist
\item
  Each Team shall provide first class travel accommodations on all trips in excess of one hour, except when such accommodations are not available; provided, however, that a Team's head coach may fly first class in place of a player when eight or more first class seats are provided to players. In the event a Team's head coach flies first class in place of a player, one (1) player, designated by the Players Association, shall be paid the difference between the amount paid by such Team for a first class seat on the flight involved and the cost of the seat purchased for such designated player on that flight.
\item
  If there is a finding that a Team has committed a willful violation of this Section 2, the NBA shall impose a \$5,000 fine on such Team.
\end{enumerate}

\hypertarget{locker-room-facilities.}{%
\section{Locker Room Facilities.}\label{locker-room-facilities.}}

Each Team agrees to provide suitable locker room facilities and to use its best efforts to stabilize the temperature in locker rooms to make it consistent with the temperature on playing courts.

\hypertarget{parking-facilities.}{%
\section{Parking Facilities.}\label{parking-facilities.}}

Each Team agrees to make parking facilities available to its players without charge in connection with games and practices conducted at the facility regularly used by such Team for home games and/or practices.

\hypertarget{union-security-dues-and-check-off}{%
\chapter{UNION SECURITY, DUES AND CHECK-OFF}\label{union-security-dues-and-check-off}}

\hypertarget{membership.}{%
\section{Membership.}\label{membership.}}

As a condition of employment commencing with the execution of this Agreement, for the duration of this Agreement only, and wherever legal: (a) any active player who is or later becomes a member in good standing of the Players Association must maintain his membership in good standing in the Players Association; and (b) any active player (including a player in the future) who is not a member in good standing of the Players Association must, on the 30th day following the beginning of his employment or the 30th day following the execution of this Agreement, whichever is later, pay, pursuant to Section 2 below or otherwise, financial core obligations to the Players Association related to collective bargaining and the administration of collective bargaining agreements (hereinafter referred to as ``financial core fees'').

\hypertarget{check-off.}{%
\section{Check-off.}\label{check-off.}}

Commencing with the execution of this Agreement and for the duration of this Agreement only, each Team, following its receipt of the requisite authorization form, will check-off the initiation fee and annual dues, assessments and financial core fees, as the case may be, in equal installments from the first four payments made thereafter to the player pursuant to paragraph 3(a) of the Uniform Player Contract or from such lesser number of payments made thereafter as provided for by Exhibit 1 to such Contract, for each player for whom a current check-off authorization has been provided to the Team. The Team will forward the check-off monies to the Players Association within fourteen (14) days of each check-off. If the Team fails to do so, interest at 7\% per annum, payable to the Players Association, shall begin to accrue on such check-off monies upon the conclusion of such 14-day period.

\hypertarget{enforcement.}{%
\section{Enforcement.}\label{enforcement.}}

\begin{enumerate}
\def\labelenumi{(\alph{enumi})}
\item
  Upon written notification to the NBA by the Players Association that a player has not paid any initiation fee, dues or financial core fees in violation of Section 1 above, the NBA will raise the matter for discussion with the player and his Team. If there is no resolution of the matter within seven (7) days, then the Team will, upon the written request of the Players Association, suspend the player without pay, wherever legal. Such suspension will continue until the Players Association has notified the Team in writing that the suspended player has satisfied his obligation as contained in Section 1 above. The parties hereby agree that suspension without pay is adopted as a substitute for and in lieu of discharge as the penalty for a violation of the union security clause of this Agreement and that no player will be discharged for a violation of that clause.

  A copy of all notices required by this Section 3(a) will be simultaneously mailed to the player involved and the NBA.
\item
  The term ``member in good standing'' as used in this Article XIX applies only to the payment of dues or any initiation fee and not to any other factors involved in union discipline.
\item
  Other than pursuant to Section 2 above, no Team shall pay any initiation fees, dues, or financial core fees on behalf of any player.
\end{enumerate}

\hypertarget{no-liability.}{%
\section{No Liability.}\label{no-liability.}}

Neither the NBA nor any Team shall be liable for any salary, bonus, or other monetary or non-monetary claims that result from a player being suspended pursuant to the terms of Section 3 above.

\hypertarget{scheduling}{%
\chapter{SCHEDULING}\label{scheduling}}

\hypertarget{training-camp.}{%
\section{Training Camp.}\label{training-camp.}}

\begin{enumerate}
\def\labelenumi{(\alph{enumi})}
\item
  Veteran Players will not be required to attend training camp earlier than 11 a.m. (local time) on the twenty-ninth day prior to the first game of any Regular Season. On such twenty-ninth day, Veterans may only be required to attend a Team dinner and Team meetings, participate in photograph and media sessions, and submit to a physical examination.
\item
  Notwithstanding Section 1(a) above, if a Veteran Player is under contract to a Team that is scheduled during a particular NBA Season to participate outside North America in an Exhibition game or a Regular Season game during the first ten (10) days of the Regular Season, such Veteran Player may be required to attend the training camp conducted in advance of that Regular Season by 11 a.m. (local time) on the thirty-second day prior to the first game of the Regular Season.
\item
  Rookies may be required to attend training camp on a date earlier than the date(s) specified in Sections 1(a) and 1(b) above, but no earlier than ten (10) days prior to the date that Veterans on such Team are required to attend.
\item
  \begin{enumerate}
  \def\labelenumii{(\roman{enumii})}
  \tightlist
  \item
    Team training camps may be held at any location, within or outside the United States and Canada. The NBA shall oversee the arrangements made with respect to any training camp held outside the United States and Canada and the accommodations provided to participating players.
  \item
    The NBA shall be required to notify the Players Association of its intention to conduct a team training camp outside the United States and Canada. Within three (3) business days of its receipt of such notification, the Players Association shall have the right to disapprove such plans, provided that such disapproval may be based solely on a reasonable and well-founded concern that the location of such training camp would be unsafe for players.
  \item
    No Team shall hold its training camp outside the United States and Canada in any two successive Seasons.
  \item
    Players on a Team that holds its training camp outside of the United States and Canada shall have at least one day off following the travel day during which they travel back to the United States from such training camp.
  \end{enumerate}
\item
  \begin{enumerate}
  \def\labelenumii{(\roman{enumii})}
  \tightlist
  \item
    During the first six days of training camp: (A) a Team shall be permitted to conduct no more than two (2) regular practice sessions per day; (B) such session(s) may last an aggregate of no longer than 3.5 hours (excluding time -- not to exceed 30 minutes -- spent stretching and participating in aerobic warm-ups and cool-downs); and (C) if a Team elects to conduct two (2) regular practice sessions during a day, one of the two sessions must be limited to non-contact activities. For the remainder of training camp, a Team shall be permitted to conduct no more than one (1) regular practice session per day and such session may last no longer than 3.5 hours (excluding time -- not to exceed 30 minutes -- spent stretching and participating in aerobic warm-ups and cool downs).
  \item
    If a Team conducts one or two regular practice sessions during a day in accordance with Section 1(e)(i) above, then except as provided in clause (A) of Section 1(e)(iii) below, the Team shall not, at a separate time during the day, conduct, organize or supervise any additional basketball activity on the basketball court.
  \item
    Nothing in Section 1(e)(i) and (ii) above shall be construed to prohibit a Team, on any day of training camp, from conducting one or two regular practice sessions in accordance with Section 1(e)(i) above, plus:
    (A) on-court skills development sessions (e.g., pick-and-roll situations, shooting, passing, etc.) not involving the playing of live defense (i.e., only ``dummy'' defense may be played) and not involving the practicing of four-man or five-man offenses or defenses; and
    (B) team-related or training-related activities (including, but not limited to, weight training, other conditioning sessions (excluding high-impact conditioning drills that are normally conducted during regular practice sessions), video sessions, meetings, and promotional appearances), so long as such additional activities do not include any basketball activity on the basketball court that is organized, supervised, or conducted by the Team.
  \end{enumerate}
\end{enumerate}

\hypertarget{exhibition-games.}{%
\section{Exhibition Games.}\label{exhibition-games.}}

\begin{enumerate}
\def\labelenumi{(\alph{enumi})}
\tightlist
\item
  Exhibition games prior to any Regular Season shall not exceed eight (8) (including intra-squad games for which admission is charged), and Exhibition games during any Regular Season shall not exceed three (3).
\item
  Exhibition games shall not be played on the three (3) days prior to the opening of the Team's Regular Season schedule, on the day prior to a Regular Season game, or on the day prior to and the day following the All-Star Game.
\end{enumerate}

\hypertarget{regular-season-games.}{%
\section{Regular Season Games.}\label{regular-season-games.}}

Each Team agrees that in no event will it play more than 82 Regular Season games.

\hypertarget{location-of-games.}{%
\section{Location of Games.}\label{location-of-games.}}

Exhibition and Regular Season games may be conducted at any location, within or outside the United States and Canada. The NBA shall supervise the arrangements made with respect to games conducted outside the United States and Canada and the accommodations provided to participating players.

\hypertarget{holidays.}{%
\section{Holidays.}\label{holidays.}}

\begin{enumerate}
\def\labelenumi{(\alph{enumi})}
\tightlist
\item
  No Team will be required to play a game on December 25, unless such game is to be telecast or cablecast nationally.
\item
  Games scheduled to be played on January 1 and Good Friday shall not commence prior to 6 p.m. (local time), unless the Players Association consents thereto, which consent shall not be unreasonably withheld. The Players Association will, upon request, consent to the earlier commencement of two (2) games on each of such dates if such games are to be broadcast or cablecast nationally, and provided that the Teams involved are in the same time zone or otherwise in close geographic proximity.
\item
  Teams at home on December 25 and January 1 (each, a ``Holiday'') may, but shall not be required to, conduct a practice on either (or both) of such Holidays, provided: (i) the Team's players have requested that they practice on the Holiday, as communicated to the Team by the Team's player representative; and (ii) within seven (7) days before or after the Holiday, the Team's players are provided with a ``day off'' --- i.e., the Team will not conduct any practice, including any optional practice, on such date, and the Team will not have a scheduled game on such date.
\item
  Teams shall not depart for an away game or series of away games prior to 3 p.m. (local time) on December 25 or January 1, unless reasonable transportation arrangements for such game or games cannot be made at or after 3 p.m. (local time).
\end{enumerate}

\hypertarget{all-star.}{%
\section{All-Star.}\label{all-star.}}

No Team that plays a game on the Thursday prior to the All-Star Game shall play a game on the Tuesday following the All-Star Game or conduct a practice session prior to such Tuesday at 2 p.m. (local time).

\hypertarget{travel.}{%
\section{Travel.}\label{travel.}}

\begin{enumerate}
\def\labelenumi{(\alph{enumi})}
\tightlist
\item
  The NBA and its Teams shall use their best efforts to devise reasonable travel schedules when Team training camps, Exhibition games, and Regular Season games are conducted or played outside the United States and Canada.
\item
  No Team shall be required to play a scheduled game on the same day that such Team has traveled across two (2) time zones, except in unusual circumstances and unless the Players Association consents thereto, which consent shall not be unreasonably withheld.
\end{enumerate}

\hypertarget{nba-all-star-game}{%
\chapter{NBA ALL-STAR GAME}\label{nba-all-star-game}}

\hypertarget{participation}{%
\section{Participation}\label{participation}}

\begin{enumerate}
\def\labelenumi{(\alph{enumi})}
\tightlist
\item
  Any player selected (by any method designated by the NBA) to play in an All-Star Game shall be required to:

  \begin{enumerate}
  \def\labelenumii{(\roman{enumii})}
  \tightlist
  \item
    attend and participate in such Game;
  \item
    attend and participate in one (1) All-Star Skills Competition (but not including the Slam Dunk Competition) that is conducted during the All-Star Weekend on which such Game is held; and
  \item
    attend and participate in every other event conducted in association with such All-Star Weekend, including, but not limited to, a reasonable number of media sessions, television appearances, and promotional appearances.
  \end{enumerate}
\item
  Any player selected (by any method designated by the NBA) to play in a Rookie-Sophomore Game shall be required to:

  \begin{enumerate}
  \def\labelenumii{(\roman{enumii})}
  \tightlist
  \item
    attend and participate in such Game;
  \item
    attend and participate in any All-Star Skills Competition designated by the NBA that is conducted during the All-Star Weekend on which such Game is held; and
  \item
    attend and participate in every other event conducted in association with such All-Star Weekend, including, but not limited to, a reasonable number of media sessions, television appearances, and promotional appearances.
  \end{enumerate}
\item
  Any player who, at the request of the NBA, voluntarily agrees to participate in an All-Star Skills Competition shall be required to attend and participate in such Skills Competition.
\item
  Nothing in this Article XXI shall preclude a player who is an officer or a representative of the Players Association from attending the Players Association's annual meeting during All-Star Weekend or preclude any player from attending the Players Association's All-Star party.
\item
  Notwithstanding anything to the contrary in Section 1(a), (b) or (c) above, a player will not be required to participate in a particular All-Star Game, Rookie-Sophomore Game, or All-Star Skills Competition if he has been excused from participation in the particular event by the Commissioner because (i) he has an injury or illness that renders him physically unable to participate in such Game or Skills Competition, or (ii) for such other reason as the Commissioner may determine in his sole discretion. If the player asserts that he should be excused from participation in a particular All-Star game or event under Section 1(e)(i) above, the Commissioner shall be authorized to require the player to submit to a medical examination to be performed by a physician designated by the NBA, and the determination of whether the player has satisfied Section 1(e)(i) shall be made by such physician in his sole discretion. In the event that a player is excused from participation in an All-Star game or event under Section 1(e)(i)above, he shall thereafter remain on his Team's Inactive List until he is cleared to return to the Active List by the NBA.
\end{enumerate}

\hypertarget{awards.}{%
\section{Awards.}\label{awards.}}

\begin{enumerate}
\def\labelenumi{(\alph{enumi})}
\tightlist
\item
  For their participation in an All-Star Game, players on the winning team shall each receive \$35,000 and players on the losing team shall each receive \$15,000.
\item
  For their participation in a Rookie-Sophomore Game, players on the winning team shall each receive \$15,000 and players on the losing team shall each receive \$5,000.
\item
  For their participation in an All-Star Skills Competition, players shall receive the following amounts:
\end{enumerate}

\begin{longtable}[]{@{}lclc@{}}
\toprule()
Slam Dunk & & Three-Point Shootout & \\
\midrule()
\endhead
1st Place: & \$35,000 & 1st Place: & \$35,000 \\
2nd Place: & \$22,500 & 2nd Place: & \$22,500 \\
3rd Place: & \$16,125 & 3rd Place: & \$15,000 \\
4th Place: & \$16,125 & 4th Place: & \$4,500 \\
& & 5th Place: & \$4,500 \\
& & 6th Place: & \$4,500 \\
\bottomrule()
\end{longtable}

\begin{longtable}[]{@{}lclc@{}}
\toprule()
Skills & & Shooting Stars & \\
\midrule()
\endhead
1st Place: & \$35,000 & Winning Team: & \$45,000 \\
2nd Place: & \$22,500 & 2nd Place Team: & \$33,750 \\
3rd Place: & \$9,000 & 3rd Place Team: & \$16,875 \\
4th Place: & \$9,000 & 4th Place Team: & \$16,875 \\
\bottomrule()
\end{longtable}

\hypertarget{player-guests.}{%
\section{Player Guests.}\label{player-guests.}}

Each player who participates in the All-Star Game, Rookie-Sophomore Game, or any All-Star Skills Competition may invite two (2) guests, who shall be reimbursed for the cost of round-trip first-class air transportation between the home city of the Team by which such player is employed and the site of the All-Star Game, Rookie-Sophomore Game or All-Star Skills Competition.

\hypertarget{players-not-participating-in-all-star-activities.}{%
\section{Players Not Participating in All-Star Activities.}\label{players-not-participating-in-all-star-activities.}}

Players who do not attend or participate in the All-Star Game, Rookie-Sophomore Game, or an All-Star Skills Competition shall have three (3) days off during the All-Star Weekend break.

\hypertarget{all-star-skills-competitions.}{%
\section{All-Star Skills Competitions.}\label{all-star-skills-competitions.}}

The All-Star Skills Competitions that take place during any All-Star Weekend shall be selected by the NBA; provided, however, that before adding any new event to the All-Star Skills Competitions that take place during any All-Star Weekend (i.e., an event different from any conducted by the NBA during any All-Star Weekend held prior to the 2005-06 Season), the NBA shall obtain the consent of the Players Association, which consent shall not be unreasonably withheld.

\hypertarget{medical-treatment-of-players-and-release-of-medical-information}{%
\chapter{MEDICAL TREATMENT OF PLAYERS AND RELEASE OF MEDICAL INFORMATION}\label{medical-treatment-of-players-and-release-of-medical-information}}

\chaptermark{MEDICAL TREATMENT OF PLAYERS \ldots}

\hypertarget{one-surgeon.}{%
\section{One Surgeon.}\label{one-surgeon.}}

Each Team agrees that a player requiring the care and treatment of an orthopedic surgeon will, so far as practicable, be referred to and treated by one (1) orthopedic surgeon (rather than several).

\hypertarget{committee-of-team-physicians.}{%
\section{Committee of Team Physicians.}\label{committee-of-team-physicians.}}

Representatives designated by the Players Association shall participate in meetings of the committee of Team physicians appointed by the NBA for the purpose of discussing matters related to the medical care and treatment of players.

\hypertarget{disclosure-of-medical-or-health-information.}{%
\section{Disclosure of Medical or Health Information.}\label{disclosure-of-medical-or-health-information.}}

\begin{enumerate}
\def\labelenumi{(\alph{enumi})}
\tightlist
\item
  A Team physician may disclose all relevant medical information concerning a player to (i) the General Manager, coaches, and trainers of the Team by which such player is employed, (ii) any entity from which any such Team seeks to procure, or has procured, an insurance policy covering such player's life or any disability, injury or illness such player may suffer or sustain, and (iii) subject to the terms of Section 3(d) below, the media or public on behalf of the Team.
\item
  Should it be requested in connection with the contemplated assignment of a player's Uniform Player Contract to one or more NBA Teams, a Team's physician may furnish all relevant medical information relating to the player to (i) the physicians and General Manager, coaches, and trainers of such other Team or Teams, and (ii) any entity from which any such other Team seeks to procure, or has procured, an insurance policy covering such player's life or any disability, injury or illness such player may suffer or sustain.
\item
  Should a Team assign a player to the NBADL, such Team's physician may furnish all relevant medical information relating to the player to (i) the NBADL, (ii) the physicians and General Manager, head coaches, and trainers of the player's NBADL team, and (ii) any entity from which the Team, the NBADL, or the player's NBADL team seeks to procure, or has procured, an insurance policy covering such player's life or any disability, injury or illness such player may suffer or sustain. In addition, an NBADL team physician may furnish all relevant medical information relating to the player to the physicians and General Manager, coaches, and trainers of the player's Team.
\item
  Subject to Section 3(e) above, each Team may make public medical information relating to the players in its employ, provided that such information relates solely to the reasons why any such player has not been or is not rendering services as a player.
\item
  A player or his immediate family (where appropriate) shall have the right to approve the terms and timing of any public release of medical information relating to any injuries or illnesses suffered by that player that are potentially life- or career-threatening, or that do not arise from the player's participation in NBA games or practices.
\end{enumerate}

\hypertarget{draftees.}{%
\section{Draftees.}\label{draftees.}}

Prior to any NBA Draft, the NBA and/or its Teams, acting jointly, may request that persons eligible for such Draft voluntarily submit to the administration of standardized medical or laboratory tests (other than tests for controlled substances), and intelligence and/or personality tests, the results of which shall be made available to any Team upon request, but which shall be kept confidential from the public and the media. Any person who submits to the administration of such tests may, prior to such Draft, be requested to submit voluntarily to an examination by the physician(s) for an NBA Team(s), but shall not be requested to undergo any further medical or laboratory test administered at the request of the NBA and/or its Teams acting jointly.

\hypertarget{selection-of-team-physician-and-other-health-care-providers.}{%
\section{Selection of Team Physician and Other Health Care Providers.}\label{selection-of-team-physician-and-other-health-care-providers.}}

Each Team has the sole and exclusive discretion to select any doctors, hospitals, clinics, health consultants or other health care providers (``Health Care Providers'') to examine and/or treat players pursuant to the terms of this Agreement and the Uniform Player Contract; provided, however, no Team will engage any such Health Care Provider based primarily on a sponsorship relationship (or lack thereof) with the Team, and without considering the Health Care Provider's qualifications (including, e.g., medical experience and credentials) and the goal of providing high quality care to all of its players.

\hypertarget{exhibition-games-and-off-season-games-and-events}{%
\chapter{EXHIBITION GAMES AND OFF-SEASON GAMES AND EVENTS}\label{exhibition-games-and-off-season-games-and-events}}

\chaptermark{EXHIBITION GAMES AND \ldots}

\hypertarget{exhibition-games.-1}{%
\section{Exhibition Games.}\label{exhibition-games.-1}}

Subject to the provisions of paragraph 2 of the Uniform Player Contract, players shall be required to participate in Exhibition games between an NBA Team and a non-member of the NBA at any location, within or outside the United States, subject to the following conditions:

\begin{enumerate}
\def\labelenumi{(\alph{enumi})}
\tightlist
\item
  The NBA shall supervise the arrangements made with respect to tournaments or series conducted outside the United States and the accommodations provided to NBA players participating in such foreign tournaments or series.
\item
  The NBA shall use its best efforts to establish an Exhibition game schedule pursuant to which excessive travel will be avoided and reasonable periods of time between games will be allotted.
\item
  In any year in which it is played, the annual Basketball Hall of Fame Exhibition game shall be considered as one of the eight (8) Exhibition games prior to the Regular Season referred to in paragraph 2 of the Uniform Player Contract.
\end{enumerate}

\hypertarget{inter-squad-scrimmage.}{%
\section{Inter-squad Scrimmage.}\label{inter-squad-scrimmage.}}

In addition to the Exhibition games provided for by paragraph 2 of the Uniform Player Contract, and during each of the playoff series conducted during the term of this Agreement, any Team that qualifies for the playoffs but is not required to participate in the first round thereof may arrange and require its players to participate in one inter-squad game or scrimmage with another similarly-situated Team, provided that such game or scrimmage is not open to members of the general public.

\hypertarget{off-season-basketball-events.}{%
\section{Off-Season Basketball Events.}\label{off-season-basketball-events.}}

\begin{enumerate}
\def\labelenumi{(\alph{enumi})}
\tightlist
\item
  No player may play in any public off-season basketball game, summer league (e.g., Southern California Pro League or the Rocky Mountain Revue Summer League), or public exhibition or competition of basketball skills (e.g., a slam dunk contest or a ``tour'' organized by an NBA business partner) (each a ``Basketball Event'') unless such Basketball Event is approved in writing by the NBA and complies with the terms and conditions of this Section 3. The NBA will consider an off-season Basketball Event for approval only if a request for such approval is submitted in writing to the NBA, and only if the arrangements made with respect to any such off-season Basketball Event are confirmed in writing to the NBA and satisfy the following requirements, in addition to such other reasonable requirements as the NBA may impose:

  \begin{enumerate}
  \def\labelenumii{(\arabic{enumii})}
  \tightlist
  \item
    General Requirements.

    \begin{enumerate}
    \def\labelenumiii{(\roman{enumiii})}
    \tightlist
    \item
      The Basketball Event takes place on or after July 1, but in no event later than September 15 (or, in the case of a summer league, September 1);
    \item
      Prior to the Basketball Event, each participating player receives the express written consent of his Team to participate in the Basketball Event;
    \item
      The person(s) organizing the Basketball Event obtains disability insurance for the benefit of each participating player's Team, in an amount acceptable to the NBA (provided, however, that this requirement shall not apply to summer leagues); and
    \item
      The names and logos of the NBA and/or any NBA Team are not used or referred to in connection with the Basketball Event, unless the NBA provides express written authorization for such use.
    \end{enumerate}
  \item
    Additional Charitable Game Requirements. The NBA will consider an off-season charitable game for approval only if, in addition to the general requirements set forth in Section 3(a)(1) above and such other reasonable requirements as the NBA may impose, the arrangements made with respect to such charitable game also satisfy the following:

    \begin{enumerate}
    \def\labelenumiii{(\roman{enumiii})}
    \tightlist
    \item
      The Players Association approves the game (which approval shall not be unreasonably withheld);
    \item
      All proceeds from the sale of tickets to the game are used for charitable purposes;
    \item
      The game is officiated by NBA referees assigned by the NBA to officiate the game. The person or entity organizing the game will be responsible for paying the officiating fees and the actual expenses incurred for the referees' lodging and transportation to and from the referees' homes to the site of the game;
    \item
      There is at least one (1) NBA Team trainer and at least one (1) physician present at the game;
    \item
      The name or likeness of an NBA player is not used, or referred to, in advertisements or promotions for or related to the game, except that if the organizer of the game is an NBA player, such organizer-player's name or likeness may be used, or referred to, in such advertisements or promotions;
    \item
      Only current or former professional basketball players participate in the game;
    \item
      The game is not accompanied by an exhibition or competition of basketball skills (such as a slam dunk contest), unless such exhibition or competition has been separately approved in writing by the NBA and the Players Association;
    \item
      Participating players are not paid or compensated (in excess of per diem and actual reasonable expenses incurred in traveling to and participating in the game);
    \item
      The organizer guarantees that the game will produce at least \$100,000 for charity, and, if directed by the NBA and the Players Association, the organizer (or a third party if the organizer itself is a charity) posts security for such amount in a form satisfactory to the NBA and the Players Association which grants the NBA and/or the Players Association the right to sue to recover such amount for the benefit of the charity;
    \item
      The game is played in the United States or Canada; and
    \item
      The organizer agrees to provide the NBA and the Players Association with an audited statement of revenues and expenses, in a form acceptable to the NBA and the Players Association, within sixty (60) days following the game.
    \end{enumerate}
  \item
    Additional Summer League Requirements. The NBA will consider an off-season summer league for approval only if, in addition to the general requirements set forth in Section 3(a)(1) above and such other reasonable requirements as the NBA may impose, the arrangements made with respect to each summer league game in which an NBA player participates also satisfy the following:

    \begin{enumerate}
    \def\labelenumiii{(\roman{enumiii})}
    \tightlist
    \item
      Participating players are not paid or compensated (except as provided under Section 4(c) below);
    \item
      NBA players do not participate in an exhibition or competition of basketball skills (such as a slam dunk contest), unless such exhibition or competition has been separately approved in writing by the NBA;
    \item
      There is at least one (1) trainer or at least one (1) physician or other emergency medical personnel present at the game; and
    \item
      The game is played in the United States or Canada.
    \end{enumerate}
  \end{enumerate}
\item
  Notwithstanding any other terms of this Section 3, and without limiting the right of the NBA to approve all arrangements of a proposed Basketball Event, the NBA may, in its sole discretion, require, as a condition of its approval of a Basketball Event (other than a charitable game or summer league), that the Basketball Event organizer pay an appropriate fee to the NBA prior to the commencement of the Basketball Event.
\item
  For purposes of this Section 3, off-season games in which an NBA player participates on behalf of his national basketball federation as part of an international FIBA competition (e.g., the Olympics and World Championships), and the preparatory Exhibition games in connection therewith, are excluded from the definition of ``Basketball Event;'' provided, however, that such exclusion shall not apply to any preparatory Exhibition game (other than games involving the U.S. national team) played and/or telecast in the United States.
\item
  Notwithstanding anything to the contrary in this Agreement, a Veteran Free Agent remains subject to the provisions of this Section 3 until the September 1 following the last Season of his Player Contract; provided, however, that any such Veteran Free Agent shall be permitted to sign a contract with and play in basketball games for a team in a professional basketball league other than the NBA beginning on the July 1 immediately following such Season (or prior to July 1 if approved in writing by the NBA).
\item
  The NBA shall have the exclusive right to (and to authorize third parties to) telecast or broadcast by radio any Basketball Event (in whole or in part) that is approved for NBA player participation in accordance with this Section 3.
\end{enumerate}

\hypertarget{summer-leagues.}{%
\section{Summer Leagues.}\label{summer-leagues.}}

\begin{enumerate}
\def\labelenumi{(\alph{enumi})}
\tightlist
\item
  No NBA Team may simultaneously enroll more than four (4) Veterans in any summer basketball league during an off-season. For purposes of this Section 4(a), the following players are not considered Veterans:

  \begin{enumerate}
  \def\labelenumii{(\arabic{enumii})}
  \tightlist
  \item
    a player who has never signed a Player Contract or whose first Player Contract begins with the Season immediately following the off-season in which such summer league is to be conducted;
  \item
    a player not under contract to an NBA Team at the time he enrolls in such summer league;
  \item
    a player under contract to an NBA Team but who missed twenty-five (25) or more of the Team's games during the Regular Season immediately preceding such off-season due to injury or illness; and
  \item
    a player who played for a team in the NBA Development League or any other U.S.-based professional league during all, or any portion, of the Regular Season immediately preceding such off-season.
  \end{enumerate}
\item
  Prior to playing in a summer basketball league, each player who is under contract with a Team for the following Season shall be provided by his Team, and requested to sign a ``Form Regarding Summer League Participation'' as attached hereto as Exhibit E.
\item
  The only compensation that may be paid by a Team or any person or entity affiliated with a Team to a player participating in a summer basketball league is a reasonable expense allowance for: (1) meals, but no greater than that set forth in Article III, Section 2; (2) lodging; and (3) transportation to and from the player's home to the site of the summer league,and to and from the site of the player's lodging during the summer league to the site of summer-league-related activities. In addition, the Team may purchase a disability insurance policy for the player covering the term of the applicable summer league.
\item
  No Team shall schedule, and no player shall participate in, a summer basketball league that is scheduled to extend, or does in fact extend, past September 1 of any calendar year.
\end{enumerate}

\hypertarget{prohibition-of-no-trade-contracts}{%
\chapter{PROHIBITION OF NO-TRADE CONTRACTS}\label{prohibition-of-no-trade-contracts}}

\hypertarget{general-limitation.}{%
\section{General Limitation.}\label{general-limitation.}}

No Player Contract may contain any prohibition or limitation of an NBA Team's right to assign such Contract to another NBA Team.

\hypertarget{exceptions-to-general-limitation.}{%
\section{Exceptions to General Limitation.}\label{exceptions-to-general-limitation.}}

Notwithstanding the provisions of Section 1 of this Article XXIV:

\begin{enumerate}
\def\labelenumi{(\alph{enumi})}
\tightlist
\item
  A Player Contract may contain (in Exhibit 4 to such Player Contract) a provision entitling a Player to earn Compensation if the player's Uniform Player Contract is traded (``trade bonus'') subject to the following:

  \begin{enumerate}
  \def\labelenumii{(\roman{enumii})}
  \tightlist
  \item
    A trade bonus shall be payable only the first time that the Contract is traded (and not as a result of any subsequent trade); provided, however, that if a Contract is signed in connection with an agreement to trade the Contract in accordance with Article VII, Section 8(e) and the Contract contains a trade bonus, the bonus shall not apply to such initial trade but shall instead be payable only if the Contract is traded a second time (and not as a result of any subsequent trade).
  \item
    A trade bonus shall not exceed 15\% of the Base Compensation remaining to be earned by the player pursuant to the Contract at the time of the trade (excluding an Option Year if not yet exercised).
  \item
    The only allowable amendment to Exhibit 4 to a Uniform Player Contract shall be the specification of the amount of the trade bonus to be paid to the player, expressed as either (A) a specified percentage of the Base Compensation remaining to be earned under the Contract at the time of the trade (excluding an Option Year if not yet exercised), or (B) a specified dollar amount not to exceed a specified percentage of Base Compensation remaining to be earned under the Contract at the time of the trade (excluding an Option Year if not yet exercised).
  \item
    A Contract that does not contain a trade bonus when signed cannot be amended to add one, except that: (A) if the Contract is extended (other than pursuant to an agreement to trade the extended Contract in accordance with Article VII, Section 8(e)), the Contract may be amended simultaneously to provide for a trade bonus that will be payable only the first time that the Contract is traded following the signing of the Extension (and not as a result of any subsequent trade) and (B) if the Contract is extended pursuant to an agreement to trade the extended Contract in accordance with Article VII, Section 8(e), the Contract may be amended simultaneously to provide for a trade bonus that shall not apply to such initial trade but shall instead be payable only if the extended Contract is traded a second time (and not as a result of a subsequent trade).
  \end{enumerate}
\item
  A Player Contract entered into by a player who has eight (8) or more Years of Service in the NBA and who has rendered four (4) or more Years of Service for the Team entering into such Contract may contain a prohibition or limitation of such Team's right to trade such Contract to another NBA Team.
\end{enumerate}

\hypertarget{limitation-on-deferred-compensation}{%
\chapter{LIMITATION ON DEFERRED COMPENSATION}\label{limitation-on-deferred-compensation}}

\hypertarget{general-limitation.-1}{%
\section{General Limitation.}\label{general-limitation.-1}}

No NBA Team may sign a Player Contract with any player under which more than 30\% of Compensation is Deferred Compensation. For purposes of this provision only, Deferred Compensation shall mean Deferred Compensation during the period commencing more than two (2) years after the playing term covered by a Player Contract.

\hypertarget{attribution.}{%
\section{Attribution.}\label{attribution.}}

All Player Contracts shall specify the Season(s) to which any Deferred Compensation is attributable.

\hypertarget{rabbi-trusts.}{%
\section{Rabbi Trusts.}\label{rabbi-trusts.}}

\begin{enumerate}
\def\labelenumi{(\alph{enumi})}
\tightlist
\item
  Notwithstanding Section 1, a Player Contract may provide for an annuity to be purchased by the Team that will pay the Player (or his designees) an amount of Deferred Compensation in excess of 30\% of Compensation, provided that:

  \begin{enumerate}
  \def\labelenumii{(\roman{enumii})}
  \tightlist
  \item
    The Team and the Player agree with respect to the form and terms of the annuity instrument and the institution from which it is purchased;
  \item
    Ownership of the annuity and all related aspects are structured in a manner that qualifies the arrangement as a tax deferred (``rabbi'') trust, in the opinion of the NBA's tax advisor; and
  \item
    The total cost of the annuity and the schedule of payment of such costs are specified in the Player Contract.
  \end{enumerate}
\item
  Notwithstanding anything to the contrary contained in Section 3(a) above:

  \begin{enumerate}
  \def\labelenumii{(\roman{enumii})}
  \tightlist
  \item
    If the institution obligated to make payment under the annuity fails to do so for any reason (other than non-compliance by the Team with the provisions of the annuity contract), the Team shall thereupon become obligated to pay to the Player as Deferred Compensation an amount, if any, equal to the unpaid portion of the purchase price of the annuity for which the Team remains obligated; and
  \item
    If the creditors of the Team and not the Player receive payments under the annuity, the Team shall thereupon become obligated to pay to the Player as Deferred Compensation an amount equal to the full purchase price of the annuity.
  \end{enumerate}
\end{enumerate}

\hypertarget{team-rules}{%
\chapter{TEAM RULES}\label{team-rules}}

Each Team may maintain or establish rules with which its players shall comply at all times, whether on or off the playing floor; provided, however, that such rules are in writing, are reasonable, and do not violate the provisions of this Agreement or the Uniform Player Contract.

\hypertarget{right-of-set-off}{%
\chapter{RIGHT OF SET-OFF}\label{right-of-set-off}}

\hypertarget{set-off-calculation.}{%
\section{Set-off Calculation.}\label{set-off-calculation.}}

\begin{enumerate}
\def\labelenumi{(\alph{enumi})}
\item
  When a Team (``First Team'') terminates a Player Contract (``First Contract'') in circumstances where the First Team, following the termination, continues to be liable for Compensation called for by the First Contract (including any Deferred Compensation), the First Team's liability for such Compensation shall be reduced pro rata by a portion of the compensation earned by the player (for services as a player) from any professional basketball team or teams (the ``Subsequent Team(s)'') during each Salary Cap Year covered by the First Contract (including, but not limited to, compensation earned but not paid during such period). The reduction in the First Team's liability shall be calculated for each Salary Cap Year (or partial Salary Cap Year) as follows:

  Step 1: Calculate the total compensation earned by the player (for services as a player) from the Subsequent Team(s) during the Salary Cap Year (or partial Salary Cap Year).

  Step 2: Subtract from the result in Step 1 (i) if the player had zero (0) Years of Service at the time the First Contract was terminated, the Minimum Annual Salary applicable to such player for the Salary Cap Year in which the First Contract was terminated, or (ii) if the player had one (1) or more Years of Service at the time the First Contract was terminated, the Minimum Annual Salary applicable to a player with one (1) Year of Service for the Salary Cap Year in which the First Contract was terminated.

  Step 3: If the result in Step 2 is a negative amount, there is no reduction in the First Team's liability for the relevant Salary Cap Year (or partial Salary Cap Year). If the result in Step 2 is a positive amount, the reduction in the First Team's liability for the relevant Salary Cap Year (or partial Salary Cap Year) shall equal 50\% (fifty percent) of such amount.
\item
  For the purposes of this Article, a ``professional basketball team'' shall mean any team in any country that pays money or compensation of any kind to a basketball player for rendering services to such team (other than a reasonable stipend limited to basic living expenses). For purposes of this Article, ``compensation'' earned by a player shall include all forms of compensation (including, without limitation, any non-cash compensation) other than benefits comparable to the type of benefits (e.g., medical and dental insurance) provided to an NBA player in accordance with Article IV above, travel and moving expenses, and any car and housing provided temporarily by a professional basketball team to the player during the period of time for which the player renders services to such team.
\end{enumerate}

\hypertarget{successive-terminations.}{%
\section{Successive Terminations.}\label{successive-terminations.}}

In the event of successive terminations by NBA Teams of Player Contracts involving the same player, the Team first to terminate shall be entitled to the right of set-off provided for by this Article XXVII until its compensation liability has been eliminated in its entirety, and the right of set-off shall then pass in order to the Team(s) terminating any subsequent Contract(s).

\hypertarget{deferred-compensation.}{%
\section{Deferred Compensation.}\label{deferred-compensation.}}

In calculating the amount of set-off to which a Team may be entitled pursuant to this Article, Deferred Compensation payable to a player for or with respect to a period covered by the terminated Contract shall be discounted on an annual basis by a percentage equal to the prime rate as set by Citibank, N.A. and in effect at the time the agreement providing for such Deferred Compensation was made.

\hypertarget{waiver-of-set-off-right.}{%
\section{Waiver of Set-off Right.}\label{waiver-of-set-off-right.}}

A Team and a player may agree in an amendment to an already-existing Player Contract to modify or eliminate the set-off right provided in this Article, but only pursuant to and to the extent allowed by Article II, Section 3(l ).

\hypertarget{broadcast-or-telecast-rights}{%
\chapter{BROADCAST OR TELECAST RIGHTS}\label{broadcast-or-telecast-rights}}

\hypertarget{league-rights.}{%
\section{League Rights.}\label{league-rights.}}

During the term of this Agreement, the Players Association agrees that the NBA, Properties, Media Ventures, and NBA Teams have the right to use, distribute, or license any performance by the players, under this Agreement or the Uniform Player Contract, for any form of broadcast or telecast, including over-the-air television, cable television, pay television, direct broadcast satellite television, and any form of cassette, cartridge, or disk system, or other means of distribution known or unknown.

\hypertarget{no-suit.}{%
\section{No Suit.}\label{no-suit.}}

The Players Association, for itself and present and future NBA players, covenants not to sue (or finance any suit against) the NBA, Properties, Media Ventures, and any NBA Team, or, any of their respective past, present and future owners (direct and indirect) acting in their capacity as owners of any of the foregoing entities, officers, directors, trustees, employees, agents, attorneys, licensees, successors, heirs, administrators, executors and assigns, with respect to the use, distribution, or license, for any form of broadcast or telecast, including over-the-air television, cable television, pay television, or direct broadcast satellite television, and any form of cassette, cartridge, or disk system, or other means of distribution known or unknown, of any performances by any player rendered under this Agreement or prior collective bargaining agreements, or under Player Contracts made pursuant thereto, during any period up to and including the day following the last Playoff game of the 2010-11 NBA Season (or, if the NBA exercises its option to extend this Agreement, up to and including the day following the last Playoff game of the 2011-12 NBA Season).

\hypertarget{reservation-of-rights.}{%
\section{Reservation of Rights.}\label{reservation-of-rights.}}

The Players Association expressly reserves its rights to bargain collectively on the subject described in Section 1 above at the expiration of this Agreement. Such reservation shall not, however, preclude the NBA from contending that the subject described in Section 1 above is not a mandatory subject of collective bargaining.

\hypertarget{miscellaneous}{%
\chapter{MISCELLANEOUS}\label{miscellaneous}}

\hypertarget{active-roster-size.}{%
\section{Active Roster Size.}\label{active-roster-size.}}

Each Team agrees to have twelve (12) players on its Active List and to have a minimum of eight (8) players on the bench for all Regular Season games. Notwithstanding the foregoing, any Team may from time to time as appropriate, but for no more than two (2) consecutive weeks at a time during the Regular Season, have eleven (11) players on its Active List.

\hypertarget{inactive-roster-size.}{%
\section{Inactive Roster Size.}\label{inactive-roster-size.}}

Each Team agrees to have one (1) player on its Inactive List for all Regular Season games. Notwithstanding the foregoing, any Team may from time to time as appropriate, but for no more than two (2) consecutive weeks at any time during the Regular Season, have no players on its Inactive List.

\hypertarget{minimum-league-wide-roster.}{%
\section{Minimum League-Wide Roster.}\label{minimum-league-wide-roster.}}

\begin{enumerate}
\def\labelenumi{(\alph{enumi})}
\item
  For each Regular Season covered by this Agreement, NBA Teams shall, in the aggregate, employ an average of no less than fourteen (14) players per Team.
\item
  The NBA's satisfaction (or not) of Section 3(a) above shall be measured following each Regular Season as follows:

  STEP 1: For each player signed to a Player Contract (including a Rest-of-Season or 10-Day Contract) during a Regular Season, determine the number of days during such Regular Season that such player was carried on his Team's Active List or Inactive List (hereinafter ``Duty Days'').

  STEP 2: Determine the total Duty Days for all players for such Regular Season by adding together the results for each player from Step 1.

  STEP 3: Multiply (x) the number of NBA Teams that played games during the applicable Regular Season, by (y) 2,300 days.

  STEP 4: If, for a Regular Season, (x) the result in Step 2 above is equal to or greater than the result in Step 3 above, then the NBA has satisfied its obligation under subsection (a) above for such Regular Season; or (y) the result in Step 2 above is less than the result in Step 3 above (the difference hereinafter described as ``the Shortfall''), then the NBA has failed to satisfy its obligation under Section 3(a) above for such Regular Season.
\item
  If the NBA fails to satisfy its obligation under Section 3(a) above for a Regular Season, then it shall be required to make a payment to the Players Association equal to (i) the Shortfall, times (ii) the Minimum Player Salary for a player with one (1) Year of Service divided by the number of calendar days during such Regular Season. Any such payment shall be made to the Players Association by August 1 following the applicable Season, and shall be distributed by the Players Association to all NBA players who were on the Active List or Inactive List of an NBA team during the applicable Regular Season, on such proportional basis as may reasonably be determined by the Players Association.
\end{enumerate}

\hypertarget{playing-rules-and-officiating.}{%
\section{Playing Rules and Officiating.}\label{playing-rules-and-officiating.}}

\begin{enumerate}
\def\labelenumi{(\alph{enumi})}
\tightlist
\item
  One representative of the Players Association shall be permitted to attend the meetings of and have a vote on the NBA Competition Committee with respect to issues relating to the NBA playing rules and officiating.
\item
  The Players Association may on behalf of the players annually submit to the Commissioner one (1) written critique of referees, without reference to any individual referee.
\end{enumerate}

\hypertarget{playoffs.}{%
\section{Playoffs.}\label{playoffs.}}

\begin{enumerate}
\def\labelenumi{(\alph{enumi})}
\tightlist
\item
  The number of Teams participating in the playoffs shall equal sixteen (16). Notwithstanding the foregoing, the NBA shall have the right to increase the number of Teams participating in the playoffs.
\item
  Each round of the playoffs shall be played in a best-of-seven-games format.
\end{enumerate}

\hypertarget{game-tickets.}{%
\section{Game Tickets.}\label{game-tickets.}}

\begin{enumerate}
\def\labelenumi{(\alph{enumi})}
\tightlist
\item
  In the event that a Team provides home-game tickets to its players, seat locations must be allocated to players based on seniority, with the most senior players (based on years of NBA service) receiving the most favorable seat locations.
\item
  NBA Teams shall provide four (4) tickets to authorized representatives of the Players Association to any home game at box office prices, provided notice of such request is given at least forty-eight (48) hours before the game.
\item
  Each Team agrees to provide retired players with three (3) or more years of NBA service with the opportunity to purchase two (2) tickets at box office prices to its NBA home games, and to hold such tickets for such players, provided tickets are available and the retired players provide the Team with forty-eight (48) hours advance notice of their desire for such tickets.
\end{enumerate}

\hypertarget{release-for-fighting.}{%
\section{Release for Fighting.}\label{release-for-fighting.}}

Each NBA Team (hereinafter ``such Team'') hereby releases and waives every claim it may have against any player employed by other NBA Teams for injuries sustained by any player in the employ of such Team which arise out of or in connection with any fighting or other formof violent and/or unsportsmanlike conduct during the course of any Exhibition, Regular Season, and/or Playoff game.

\hypertarget{limitation-on-player-ownership.}{%
\section{Limitation on Player Ownership.}\label{limitation-on-player-ownership.}}

During the term of this Agreement, no NBA player may acquire or hold a direct or indirect interest in the ownership of any NBA Team; provided, however, that any player may own shares of any publicly-traded company that directly or indirectly owns an NBA Team.

\hypertarget{nondisclosure.}{%
\section{Nondisclosure.}\label{nondisclosure.}}

The parties agree that (a) the economic terms of any individual Uniform Player Contract entered into by a Team and a player, and (b) any information contained in, or disclosed to the Players Association in connection with an Audit Report, Draft Audit Report, Interim Audit Report, Interim Escrow Audit Report, BRI Report, Escrow Schedule, or Notice to Escrow Agent, shall not be disclosed to the media by (i) the NBA, its Teams, or their respective employees, or (ii) the Players Association, NBA players, or their respective employees, agents, or representatives.

\hypertarget{implementation-of-agreement.}{%
\section{Implementation of Agreement.}\label{implementation-of-agreement.}}

\begin{enumerate}
\def\labelenumi{(\alph{enumi})}
\tightlist
\item
  The NBA and the Players Association will use their respective best efforts to have NBA Teams and NBA players comply with the terms and provisions of this Agreement.
\item
  The NBA and the Players Association shall use their respective best efforts and take all reasonable steps to cooperate to defend the enforceability of this Agreement against any challenge thereto.
\end{enumerate}

\hypertarget{additional-canadian-provisions.}{%
\section{Additional Canadian Provisions.}\label{additional-canadian-provisions.}}

\begin{enumerate}
\def\labelenumi{(\alph{enumi})}
\tightlist
\item
  The bases upon which a player may be disciplined or discharged or a Player Contract terminated, as set forth in this Agreement and/or in the Uniform Player Contract, shall constitute just and reasonable cause within the meaning of any applicable Canadian statute (federal or provincial) and, to the extent this Agreement or the Uniform Player Contract provides specific penalties for such conduct, those penalties shall apply.
\item
  During the term of this Agreement, the NBA and Players Association shall consult regularly about issues relating to the workplace which affect the parties or any player bound by this Agreement.
\item
  If and to the extent Sections 48 and 49 of the Ontario Labour Relations Act are or may be found applicable to this Agreement, the parties agree that the provisions thereof shall apply only to disputes between the Toronto Raptors and players for the Toronto Raptors.
\item
  The parties acknowledge and agree that a player employed by an NBA Team pursuant to the provisions of a Uniform Player Contract, a 10-Day Contract, or a Rest-of-Season Contract is and/or shall be deemed to be an ``employee hired on the basis that his employment is to terminate on the expiry of a definite term or the completion of a specific task'' within the meaning of paragraph 1 of Section 2(1) of Ontario Regulation 288/01 under the Ontario Employment Standards Act, 2000, so as to render inapplicable to NBA players the provisions of Sections 54-62 of such Act.
\item
  The parties acknowledge and agree that the severance benefits provided to players pursuant to this Agreement (including the provisions of Player Contracts that provide, in certain circumstances, for the continued payment of Salary to a player following the termination of a Player Contract) constitute and/or shall be deemed to constitute a settlement binding on the player within the meaning of Section 6 of the Ontario Employment Standards Act, 2000, and/or ``an amount paid to an employee for loss of employment under a provision of an employment contract based upon length of employment, length of service or seniority'' within the meaning of paragraph 2 of Section 65(8) of the Ontario Employment Standards Act, 2000, so as to render inapplicable to NBA players the provisions of Sections 63-66 of such Act.
\item
  Upon the NBA's request, the Players Association shall cooperate with the NBA in a reasonable manner in connection with any effort the NBA may make to seek an exemption from any Canadian (federal or provincial) law or regulation affecting the employment relationship that is inconsistent with the provisions of this Agreement or any other agreement between the Players Association and the NBA (or NBA Properties) or between any player and any NBA Team.
\item
  All players employed by NBA Teams shall be paid in U.S. dollars, regardless of where such Teams are located.
\end{enumerate}

\hypertarget{no-strike-and-no-lockout-provisions-and-other-undertakings}{%
\chapter{NO-STRIKE AND NO-LOCKOUT PROVISIONS AND OTHER UNDERTAKINGS}\label{no-strike-and-no-lockout-provisions-and-other-undertakings}}

\chaptermark{NO-STRIKE AND NO-LOCKOUT \ldots}

\hypertarget{no-strike.}{%
\section{No Strike.}\label{no-strike.}}

During the term of this Agreement, neither the Players Association nor its members shall engage in any strikes, cessations or stoppages of work, or any other similar interference with the operations of the NBA or any of its Teams. Notwithstanding the foregoing, nothing in this Section 1 shall impair the rights accorded the Players Association by Article XXXIX, Section 3 (Termination by Players Association/Anti-Collusion) or Section 6 (Mutual Right of Termination).

\hypertarget{no-lockout.}{%
\section{No Lockout.}\label{no-lockout.}}

During the term of this Agreement, neither the NBA nor its Teams shall engage in any lockouts, cessations or stoppages of work or any other similar interference with the employment of NBA players by NBA Teams. Notwithstanding the foregoing, nothing in this Section 2 shall impair the rights accorded the NBA by Article XXXIX, Section 14 (NBA Option to Reopen/TV Revenues), Section 5 (Termination by NBA/Force Majeure), or Section 6 (Mutual Right of Termination).

\hypertarget{no-breach-of-player-contracts.}{%
\section{No Breach of Player Contracts.}\label{no-breach-of-player-contracts.}}

The Players Association agrees that it will not engage in any concerted activities to breach, induce the breach of, or threaten to breach or induce the breach of, any Player Contract.

\hypertarget{best-efforts-of-players-association.}{%
\section{Best Efforts of Players Association.}\label{best-efforts-of-players-association.}}

The Players Association will use its best efforts: (a) to prevent each player from rendering, or threatening to render, services as a professional basketball player for another professional basketball team during the term of a Player Contract between such player and the Team for which he plays (except as said Player Contract may be assigned, sold, or transferred in accordance with the provisions of such Player Contract or this Agreement); (b) to prevent each player from refusing, or threatening to refuse, to participate in any scheduled Exhibition game, Regular Season game, All-Star Game, Rookie-Sophomore Game, All-Star Skills Competition, or Playoff game; (c) to prevent each player from refusing, or threatening to refuse, to report, within the time required, to a team in the NBA Development League (``NBADL'') when the player has been assigned to an NBADL Team in accordance with the provisions of this Agreement, and to prevent each such player from refusing, or threatening to refuse, to participate in any scheduled
NBADL game; (d) to prevent each player from otherwise breaching, or threatening to breach, his Player Contract; and (e) to prevent each player from making any demand upon the NBA or any of its Teams, including, but not limited to, a demand (accompanied by threats that the player will render services as a professional basketball player for another professional basketball team during the term of his Player Contract) that such Player Contract be renegotiated during the term thereof; provided, however, that this provision is not intended to prevent any player from entering into negotiations with a Team, in accordance with Article VII, with respect to the compensation to be paid to said player for the Season(s) following the last playing Season covered by any Player Contract, or renewal or extension thereof.

\hypertarget{grievance-and-arbitration-procedure-and-special-procedures-with-respect-to-disputes-involving-player-discipline}{%
\chapter{GRIEVANCE AND ARBITRATION PROCEDURE AND SPECIAL PROCEDURES WITH RESPECT TO DISPUTES INVOLVING PLAYER DISCIPLINE}\label{grievance-and-arbitration-procedure-and-special-procedures-with-respect-to-disputes-involving-player-discipline}}

\chaptermark{GRIEVANCE AND ARBITRATION \ldots}

\hypertarget{scope.}{%
\section{Scope.}\label{scope.}}

\begin{enumerate}
\def\labelenumi{(\alph{enumi})}
\item
  \begin{enumerate}
  \def\labelenumii{(\roman{enumii})}
  \tightlist
  \item
    Except as provided otherwise by this Agreement or by paragraph 9 of the Uniform Player Contract, the Grievance Arbitrator shall have exclusive jurisdiction to determine, in accordance with procedures set forth in this Article XXXI, any and all disputes involving the interpretation or application of, or compliance with, the provisions of this Agreement or the provisions of a Player Contract, including a dispute concerning the validity of a Player Contract. Any such dispute subject to the exclusive jurisdiction of the Grievance Arbitrator shall hereinafter be referred to as a ``Grievance.''
  \item
    The Grievance Arbitrator shall also have jurisdiction to resolve disputes arising under the Agreement and Declaration of Trust Establishing the National Basketball Players Association/National Basketball Association Supplemental Benefit Plan and the Agreement and Declaration of Trust Establishing the National Basketball Players Association/National Basketball Association Labor-Management Cooperation and Education Trust in accordance with the provisions of such agreements and declarations of trust. In connection with the resolution of such disputes, to the extent there is any conflict between the provisions of such agreements and declarations of trust and the provisions of this Agreement, the provisions of such agreements and declarations of trust shall control.
  \end{enumerate}
\item
  Notwithstanding the provisions of Section 1(a) above:

  \begin{enumerate}
  \def\labelenumii{(\roman{enumii})}
  \tightlist
  \item
    Disputes arising under Articles VII, VIII, X, XI, XII, XIII, XIV, XV, XVI, XXXVII, XXXIX, and XL shall (except as otherwise specifically provided by Article VII, Section 3(d)(5)) be determined by the System Arbitrator provided for in Article XXXII; and
  \item
    Disputes involving (A) a fine or suspension imposed upon a player by the Commissioner (or his designee) for conduct on the playing court (as defined in Section 8(c) below), or (B) action taken by the Commissioner (or his designee) concerning the preservation of the integrity of, or maintenance of public confidence in, the game of basketball, shall be resolved in accordance with the provisions set forth in Section 8 below.
  \end{enumerate}
\end{enumerate}

\hypertarget{initiation.}{%
\section{Initiation.}\label{initiation.}}

\begin{enumerate}
\def\labelenumi{(\alph{enumi})}
\tightlist
\item
  Grievances may be initiated, as set forth below, by a player, a Team, the NBA, or the Players Association, except that the Players Association may not initiate a Grievance involving player discipline without the approval of the player(s) concerned.
\item
  No party may initiate a Grievance until and unless it has first discussed the matter with the party or parties against whom the Grievance is to be initiated in an attempt to settle it.
\item
  A Grievance must be initiated, in accordance with the provisions of Section 2(d) below, within thirty (30) days from the date of the occurrence upon which the Grievance is based, or within thirty (30) days from the date upon which the facts of the matter became known or reasonably should have become known to the party initiating the Grievance, whichever is later.
\item
  Subject to the provisions of Sections 2(a)-(c) above: (i) a player or the Players Association may initiate a Grievance (A) against the NBA by filing written notice thereof with the NBA, and (B) against a Team, by filing written notice thereof with the Team and the NBA; (ii) a Team may initiate a Grievance by filing written notice thereof with the Players Association and furnishing copies of such notice to the player(s) involved and to the NBA; and (iii) the NBA may initiate a Grievance by filing written notice thereof with the Players Association and furnishing copies of such notice to the player(s) and Team(s) involved. Any such notice shall expressly state that the party is initiating a Grievance pursuant to Article XXXI, Section 2 above.
\end{enumerate}

\hypertarget{hearings.}{%
\section{Hearings.}\label{hearings.}}

\begin{enumerate}
\def\labelenumi{(\alph{enumi})}
\tightlist
\item
  Upon at least thirty (30) days' written notice to the other side, the NBA and the Players Association may arrange to have a hearing scheduled on a date that is mutually convenient to the parties to the dispute, the NBA, the Players Association, and the Grievance Arbitrator; provided, however, that if the NBA and the Players Association cannot agree on a hearing date, the Grievance Arbitrator shall set a reasonable hearing date that follows the expiration of the thirty-day notice period. Only the NBA and the Players Association may schedule or postpone hearings before the Grievance Arbitrator.
\item
  Notwithstanding the provisions of Section 3(a) above, during each Salary Cap Year covered by this Agreement, the Players Association and the NBA shall each have the right, upon a showing of need, to have two (2) Grievances scheduled for hearing on or after the tenth day following service of the notice provided for by Section 3(a) above. The provisions of this subsection (b) shall not limit or otherwise affect the rights of the NBA or the Players Association pursuant to Section 12 below.
\item
  If a Grievance is scheduled for hearing under this Article XXXI, and the hearing date is thereafter postponed at the request of either the NBA or the Players Association, the postponement fee (if any) of the Grievance Arbitrator will be borne by the party requesting the postponement, unless that party objects and the Grievance Arbitrator finds that the request for such postponement was for good cause. Should good cause be found, the parties will share any postponement fee equally.
\item
  In any Grievance matter, neither the NBA nor the Players Association may request or be granted more than one (1) postponement of a hearing previously scheduled under this Article XXXI. If a party which has been granted a postponement of a hearing fails to attend a subsequently scheduled hearing in the same Grievance matter, the Grievance shall be resolved against that party.
\item
  If (i) a hearing of a Grievance is not scheduled to take place within one (1) year from the initiation of the Grievance, or (ii) in the circumstance where the initial date set for the hearing has been postponed, if a second hearing in that Grievance is not scheduled to take place within two (2) years from the initiation of the Grievance, then the Grievance shall be dismissed with prejudice.
\item
  For purposes of computing time under this Section 3, the time shall be tolled during any period when there is no Grievance Arbitrator or when the grieving party has been unable to schedule a hearing (after making efforts to do so) because the Grievance Arbitrator is unavailable.
\item
  Hearings before the Grievance Arbitrator shall be held in New York (alternating between the NBA and Players Association offices). All such hearings shall be conducted in accordance with the Labor Arbitration Rules of the American Arbitration Association; provided, however, that in the event of any conflict between such Rules and the provisions of this Agreement, the provisions of this Agreement shall control.
\end{enumerate}

\hypertarget{procedure.}{%
\section{Procedure.}\label{procedure.}}

\begin{enumerate}
\def\labelenumi{(\alph{enumi})}
\tightlist
\item
  Not later than seven (7) days prior to the hearing, the parties shall submit to the Grievance Arbitrator a joint statement of the issue(s) in dispute. If the parties cannot agree on such a joint statement, each party may submit to the Grievance Arbitrator a separate statement setting forth the disputed issue(s), and such separate statement shall be delivered to the other party or parties at the same time it is submitted to the Grievance Arbitrator.
\item
  During each Salary Cap Year covered by this Agreement, the NBA and the Players Association shall each be entitled, as a matter of right, in connection with one (1) proceeding brought pursuant to this Article XXXI, to the discovery, in advance of a hearing, of non-privileged documents from any adverse party (or parties) in such proceeding. The party (or parties) to whom a request for document discovery is made shall have the obligation to produce only documents that are directly relevant and material to the core issue(s) in dispute, and shall not be obligated to produce documents merely because the production of such documents would be reasonably calculated to lead to the discovery of relevant or admissible evidence.
\item
  Not later than three (3) business days prior to the hearing, the parties shall exchange witness lists, relevant documents and other evidentiary materials, and citations of legal authorities that each side intends to rely on in its affirmative case. Absent a showing of good cause, no party may proffer, refer to, or rely on the testimony of any witness, any document or other evidentiary material in its affirmative case that has not been identified to the other side as required by this subsection.
\item
  The Grievance Arbitrator shall grant the request of any party to file a pre-hearing and/or post-hearing brief, unless an opposing party demonstrates that the filing of briefs is unreasonable in the circumstances. If the Grievance Arbitrator grants a request to file pre-hearing briefs, such briefs shall be served on the adverse party (or parties) and filed with the Grievance Arbitrator not later than three (3) business days prior to the hearing. No pre-hearing brief shall exceed ten (10) pages in length, and the rules applicable in the United States District Court for the Southern District of New York with respect to the calculation of pages, the size of font, margins and the like shall apply.
\end{enumerate}

\hypertarget{arbitrators-decision-and-award.}{%
\section{Arbitrator's Decision and Award.}\label{arbitrators-decision-and-award.}}

\begin{enumerate}
\def\labelenumi{(\alph{enumi})}
\tightlist
\item
  Except as set forth in Section 12 below, the Grievance Arbitrator shall render an Award as soon as practicable. The Award shall be accompanied by a written opinion, or, if both the NBA and the Players Association agree, the written opinion may follow within a reasonable time thereafter. In no event shall the Award and written opinion be issued more than thirty (30) days following the conclusion of a Grievance hearing (or, where applicable, following the date designated by the Grievance Arbitrator for the submission of post-hearing briefs). The Award shall constitute full, final and complete disposition of the Grievance, and shall be binding upon the player(s) and Team(s) involved and the parties to this Agreement.
\item
  In addition to such other limitations as may be imposed on him/her by this Agreement, the Grievance Arbitrator shall have jurisdiction and authority only to: (i) interpret, apply, or determine compliance with the provisions of this Agreement; (ii) interpret, apply or determine compliance with the provisions of Player Contracts; (iii) determine the validity of Player Contracts; (iv) award damages in connection with a proceeding provided for in Section 11 below; (v) award declaratory relief in connection with a proceeding initiated by a Team to determine whether such Team may properly terminate a Player Contract pursuant to paragraph 16(a) of such Contract, and what, if any, liability such Team would incur as a result of such termination; and (vi) resolve disputes arising under Article VII, Section 3(d)(5), Article XXII, Section 5, Article XXVI, and Article XXXIII in the manner set forth therein. Notwithstanding the foregoing or any other provision of this Agreement or the Uniform Player Contract, the Grievance Arbitrator shall not have jurisdiction or authority to add to, detract from, or alter in any way the provisions of this Agreement (including the provisions of this subsection) or any Player Contract. Nor, in the absence of agreement by the NBA and the Players Association, shall the Grievance Arbitrator have jurisdiction or authority to resolve questions of substantive, as opposed to procedural, arbitrability. Questions of substantive arbitrability shall include the question of whether an arbitrator provided for by the terms of this Agreement, as opposed to the Commissioner (or his designee), has jurisdiction to hear or resolve a particular dispute and such questions shall be determined in a judicial proceeding to be venued in the United States District Court for the Southern District of New York.
\end{enumerate}

\hypertarget{appointment-and-replacement-of-grievance-arbitrator.}{%
\section{Appointment and Replacement of Grievance Arbitrator.}\label{appointment-and-replacement-of-grievance-arbitrator.}}

\begin{enumerate}
\def\labelenumi{(\alph{enumi})}
\tightlist
\item
  The parties to this Agreement shall agree upon the appointment of a Grievance Arbitrator, who shall serve for the duration of this Agreement; provided, however, that as of September 1, 2006, and as of each successive September 1, either of the parties to this Agreement may discharge the Grievance Arbitrator by serving written notice upon him and upon the other party to this Agreement during the period July 27 through August 1 immediately preceding each such September 1; and provided, further, that as of the April 30 of the last Season covered by this Agreement (or any extension thereof), either of the parties may discharge the Grievance Arbitrator by serving written notice upon him and upon the other party to this Agreement during the period March 26 through March 31 immediately preceding such April 30. A Grievance Arbitrator as to whom a notice of discharge has been served shall continue to have jurisdiction only with respect to (i) Grievances as to which a hearing has been commenced or scheduled for a date certain and (ii) Grievances filed within the thirty (30) day period preceding the service of a notice of discharge; provided, however, that a hearing with respect to Grievances referred to in this subsection (ii) must commence no later than thirty (30) days following the effective date of the Grievance Arbitrator's discharge.
\item
  If the Grievance Arbitrator is discharged (or resigns), the parties shall agree upon a successor Grievance Arbitrator. In the absence of such agreement, the parties shall jointly request the International Institute for Conflict Prevention and Resolution (the ``CPR Institute'') (or such other organization(s) as the parties may agree upon) to submit to the parties a list of eleven (11) attorneys, none of whom shall have, nor whose firm shall have, represented within the past five (5) years any professional athletes; agents or other representatives of professional athletes; labor organizations representing athletes; sports leagues, governing bodies,or their affiliates; sports teams or their affiliates; or owners in any professional sport. If, within seven (7) days from the receipt of such list, the parties fail to agree upon the selection of a Grievance Arbitrator from among the names on such list, they shall return that list, with up to five (5) names deleted there from by each party, to the CPR Institute (or such other organization as the parties may have agreed upon), and the CPR Institute (or such other organization) shall choose a new Grievance Arbitrator from the names remaining on such list.
\end{enumerate}

\hypertarget{injury-grievances.}{%
\section{Injury Grievances.}\label{injury-grievances.}}

\begin{enumerate}
\def\labelenumi{(\alph{enumi})}
\tightlist
\item
  If a party to a dispute arising under paragraphs 7, 16(a)(iii), 16(b), or 16(c) of a Uniform Player Contract so elects, the NBA and the Players Association shall agree upon a neutral physician or (in the absence of such agreement) jointly request that the President of the American College of Orthopedic Surgeons (or such other similar organization as the NBA and the Players Association agree may be most appropriate to the issues in dispute) designate a physician who has no relationship with any party covered by this Agreement who shall, for purposes of the dispute, serve as an independent medical expert and consultant to the Grievance Arbitrator. Such independent medical expert shall conduct a physical examination of the player; review such medical records and reports relating to the player that bear on the issues in dispute; and prepare a written report of the player's medical condition, which report shall address any specific medical questions submitted to the independent medical expert by joint agreement of the parties or by the Grievance Arbitrator. Any reports, opinions, or conclusions of the independent medical expert shall be provided in writing to the parties in advance of any hearing scheduled pursuant to Section 3 above. The opinions and conclusions of the independent medical expert shall be accorded such weight as the Grievance Arbitrator deems appropriate. The fees and costs of the independent medical expert shall be borne equally by both sides.
\item
  During the course of any arbitration proceeding, the Grievance Arbitrator may, by appropriate process, require any person (including, but not limited to, a Team and a Team physician, and a player and any physician consulted by such player) to provide to the player or that player's Team, as the case may be, all medical information in the possession of any person relating to the subject matter of the arbitration.
\end{enumerate}

\hypertarget{special-procedures-with-respect-to-player-discipline.}{%
\section{Special Procedures with Respect to Player Discipline.}\label{special-procedures-with-respect-to-player-discipline.}}

\begin{enumerate}
\def\labelenumi{(\alph{enumi})}
\tightlist
\item
  A dispute involving (i) a fine of \$50,000 or less or a suspension of twelve (12) games or less (or both such fine and suspension) imposed upon a player by the Commissioner (or his designee) for conduct on the playing court (as defined in Section 8(c) below), or (ii) action taken by the Commissioner (or his designee) (A) concerning the preservation of the integrity of, or the maintenance of public confidence in, the game of basketball and (B) resulting in a financial impact on the player of \$50,000 or less, shall not give rise to a Grievance, shall not be subject to a hearing before, or resolution by, the Grievance Arbitrator, and shall not be determined by arbitration; but instead shall be processed exclusively as an ``Appeal'' before the Commissioner (or his designee) as follows:

  \begin{enumerate}
  \def\labelenumii{(\arabic{enumii})}
  \tightlist
  \item
    Within twenty (20) days following written notification of the action taken by the Commissioner (or his designee), a player affected thereby or the Players Association may appeal in writing to the Commissioner.
  \item
    Upon the written request of the Players Association, the Commissioner shall designate a time and place for hearing as soon as is reasonably practicable following his receipt of the notice of appeal.
  \item
    As soon as reasonably practicable, but not later than twenty (20) days, following the conclusion of such hearing, the Commissioner shall render a written decision, which decision shall constitute full, final and complete disposition of the dispute, and shall be binding upon the player(s) and Team(s) involved and the parties to this Agreement.
  \item
    In the event such appeal involves a fine and/or suspension imposed by the Commissioner's designee, the Commissioner, as a consequence of such appeal and hearing, shall have authority only to affirm or reduce such fine and/or suspension, and shall not have authority to increase such fine and/or suspension.
  \end{enumerate}
\item
  A dispute involving (i) a fine of more than \$50,000 and/or a suspension of more than twelve (12) games that is imposed upon a player by the Commissioner (or his designee) for conduct on the playing court, or (ii) an action taken by the Commissioner (or his designee) that (A) concerns the preservation of the integrity of, or the maintenance of public confidence in, the game of basketball and (B) results in a financial impact on the player of more than \$50,000, shall be processed and determined in the same manner as a Grievance under Sections 2-6 above; provided, however, that the Grievance Arbitrator shall apply an ``arbitrary and capricious'' standard of review.
\item
  As used in this Agreement, ``conduct on the playing court'' shall mean conduct in any area within an arena (including, but not limited to, locker rooms, vomitories, loading docks, and other back-of-house and underground areas, including those used by television production and other vehicles) at, during or in connection with an NBA Exhibition, All-Star, Regular Season or Playoff game. (By way of example and not limitation, conduct ``at'' and/or ``in connection with'' an NBA game shall include conduct engaged in by a player within an arena from the time the player arrives at the arena for an NBA game until the time the player has left the premises of the arena following the conclusion of such game.) Conduct engaged in by a player outside an arena (such as, for example, in a parking lot adjacent to an arena) shall not constitute ``conduct on the playing court.''
\item
  In the event a matter filed as a Grievance in accordance with the provisions of this Article gives rise to issues involving the integrity of, or public confidence in, the game of basketball, and the financial impact on the player of the action being grieved is \$50,000 or less, the Commissioner may, at any stage of its processing, order that the matter be withdrawn from such processing and thereafter be processed in accordance with the appeal procedure provided in Section 8(a) above.
\end{enumerate}

\hypertarget{procedure-with-respect-to-fine-and-suspension-amounts.}{%
\section{Procedure with Respect to Fine and Suspension Amounts.}\label{procedure-with-respect-to-fine-and-suspension-amounts.}}

In the event that a Grievance or an Appeal challenging a Commissioner or Team-imposed fine and/or suspension is filed in accordance with this Article, the amount of any fine or salary lost by virtue of the suspension shall be deposited in a separate interest-bearing account maintained for such fines or suspension-related amounts. The NBA shall provide written notice to the Players Association of the date and amount of each deposit made pursuant to this Section 9 by delivering to the Players Association monthly statements reflecting the investment activity in such account. In the absence of agreement between the NBA and the Players Association, the Grievance Arbitrator (in resolving a Grievance, and in a manner consistent with his determination of such Grievance) or the Commissioner (or his designee) (in resolving an Appeal,and in a manner consistent with his determination of such Appeal) shall determine the amount of the deposited funds to be payable to the player, the Team, or the NBA, and any interest earned on such deposit shall be allocated to the parties in proportion thereto.

\hypertarget{disputes-with-respect-to-the-terms-of-a-player-contract.}{%
\section{Disputes with Respect to the Terms of a Player Contract.}\label{disputes-with-respect-to-the-terms-of-a-player-contract.}}

\begin{enumerate}
\def\labelenumi{(\alph{enumi})}
\tightlist
\item
  If either the NBA or the Players Association asserts that a term or provision of a Player Contract is not permitted by this Agreement, either may have the dispute involving such Contract term or provision resolved by initiating a Grievance. If such a Grievance is initiated by the NBA, the thirty-day time period referred to in Section 2(c) above shall commence with the date upon which the NBA received the Player Contract (or amendment thereto) containing the disputed term or provision. If such a Grievance is initiated by the Players Association, the thirty-day time period referred to in Section 2(c) above shall commence with the date upon which the Player Contract (or amendment thereto) containing the disputed term or provision was first made available for inspection by the Players Association.
\item
  If, as a result of the Grievance and Arbitration procedure, a Player Contract is found to contain a term or provision that is not permitted by this Agreement, then (i) such term or provision shall be deleted from the Player Contract and have no force or effect, and the Player Contract shall in all other respects remain valid and binding upon the parties thereto, and (ii) if the Team and the player agree to reform or revise the Player Contract within thirty days of the Grievance Arbitrator's decision, such reformation or revision shall be exempted from the rules governing Renegotiations contained in Article VII, Section 7(c).
\item
  Nothing set forth above shall affect in any manner the Commissioner's authority with respect to the approval or disapproval of Player Contracts pursuant to paragraph11 of the Uniform Player Contract; and the fact that the Commissioner has approved or not disapproved a Player Contract containing a term or provision not permitted by this Agreement shall not be referred to in the course of the Grievance and Arbitration procedure and shall not be considered in any manner or for any purpose by the Grievance Arbitrator in connection with a dispute concerning that Player Contract.
\end{enumerate}

\hypertarget{disputes-with-respect-to-players-under-contract-who-withhold-playing-services.}{%
\section{Disputes with Respect to Players Under Contract Who Withhold Playing Services.}\label{disputes-with-respect-to-players-under-contract-who-withhold-playing-services.}}

In addition to any other rights a Team may have under contract or law, including those under paragraph 9 of a Uniform Player Contract, a Team may recover damages in a proceeding before the Grievance Arbitrator when a player who is party to a currently effective Player Contract fails or refuses to render the services called for under the Player Contract. In any such proceeding, where the Grievance Arbitrator determines that damages are continuing to accrue at the time of the hearing, the Arbitrator shall award such damages (if any) as the Team has by then sustained, and the hearing shall remain open to enable the submission of proof on the issue of continuing damages.

\hypertarget{expedited-procedure.}{%
\section{Expedited Procedure.}\label{expedited-procedure.}}

\begin{enumerate}
\def\labelenumi{(\alph{enumi})}
\tightlist
\item
  Notwithstanding the foregoing, in the event of a dispute arising under Article XVII, Article XXX, or Article XXXI, Section 11 of this Agreement, or under paragraph 15 of a Uniform Player Contract (but only insofar as such paragraph provides), or in the event of an alleged breach by a player of paragraph 9 of a Uniform Player Contract, the NBA or the Players Association may request that such dispute or alleged breach be referred immediately to the Grievance Arbitrator. In any such case, the dispute or alleged breach shall be asserted by notice in writing or by facsimile given to the other party or parties, the NBA, the Players Association, and the Grievance Arbitrator.
\item
  The Grievance Arbitrator shall convene a hearing with respect to such dispute or alleged breach at the earliest possible time, but in no event later than 24 hours following his receipt of such notice. If the Grievance Arbitrator is not immediately available and the parties are unable to agree upon another arbitrator to hear and resolve such dispute, the parties shall select an arbitrator in accordance with the procedures set forth in Section 6(b) above.
\item
  The award, which shall be issued not later than twenty-four (24) hours after the conclusion of the hearing, shall be in writing and may be issued with or without opinion. If any party desires an opinion, one shall be issued but its issuance shall not delay compliance with or enforcement of the award. The award shall constitute full, final and complete disposition of the dispute or alleged breach, and shall be binding upon the player(s) and Team(s) involved and the parties to this Agreement.
\item
  The failure of any party to attend the hearing as scheduled shall not delay the hearing, and the Grievance Arbitrator (or an arbitrator selected in accordance with the procedures set forth in Section 6(b) above, as the case may be) shall be authorized to proceed to take evidence and issue an award as though such party were present.
\end{enumerate}

\hypertarget{threshold-amount-for-certain-grievances.}{%
\section{Threshold Amount for Certain Grievances.}\label{threshold-amount-for-certain-grievances.}}

A dispute concerning a fine or suspension (or a combination thereof) imposed by a Team may be heard and resolved by the Grievance Arbitrator only if it results in a financial impact on the player of more than \$5,000. A dispute concerning a fine or suspension (or a combination thereof) imposed by the Commissioner (or his designee) other than for conduct on the playing court (as defined in Section 8(c) above) may be heard and resolved by the Grievance Arbitrator only if it results in a financial impact on the player of more than \$50,000.

\hypertarget{miscellaneous.-1}{%
\section{Miscellaneous.}\label{miscellaneous.-1}}

\begin{enumerate}
\def\labelenumi{(\alph{enumi})}
\tightlist
\item
  Each of the time limits set forth herein may be extended by mutual agreement of the parties involved.
\item
  In any meeting or hearing provided for by this Article XXXI, a player may be accompanied by a representative of the Players Association who may participate in such meeting or hearing and represent the player. In any such meeting or hearing, the NBA and any Team involved may attend and be accompanied by a representative who may participate in such meeting or hearing and represent the NBA and any such Team.
\item
  The parties recognize that a player may be subjected to disciplinary action for just cause by his Team or by the Commissioner (or his designee). Therefore, in Grievances regarding discipline, the issue to be resolved shall be whether there has been just cause for the penalty imposed.
\item
  Nothing contained herein shall excuse a player from prompt compliance with any discipline imposed upon him. If discipline imposed upon a player is determined to be improper by a final disposition under this Article XXXI, the player shall promptly be made whole.
\item
  Nothing contained in this Article XXXI shall be deemed to limit or impair the right of the NBA or any Team to impose discipline upon a player(s) or to take any other action not inconsistent with the provisions of a Player Contract or this Agreement.
\item
  Subject to Section 3(c) above, all costs of arbitration, including the fees and expenses of the Grievance Arbitrator, shall be borne equally by the parties thereto; but each party shall bear the cost of its own witnesses, counsel, and the like.
\item
  A Team shall not be required to terminate a Player Contract under the NBA waiver procedure as a condition precedent to the filing of a Grievance with respect to such Player Contract. To the extent that the decision of the Impartial Arbitrator in In re Otis Birdsong, Dec.~No.~87-2, May 14, 1987, is inconsistent with the foregoing, it is hereby overruled.
\item
  In a proceeding involving the interpretation of a Player Contract, no Uniform Player Contract (whether signed during the term of this Agreement or during the term of any prior collective bargaining agreement between the parties), or amendment thereto, other than the Player Contract or amendment that is the subject of dispute shall be admissible as evidence of the meaning of, or of the parties' intentions with respect to, any individually-negotiated terms or provisions in the Player Contract or amendment that is the subject of dispute.
\end{enumerate}

\hypertarget{system-arbitration}{%
\chapter{SYSTEM ARBITRATION}\label{system-arbitration}}

\hypertarget{jurisdiction-and-authority.}{%
\section{Jurisdiction and Authority.}\label{jurisdiction-and-authority.}}

The NBA and the Players Association shall agree upon a System Arbitrator, who shall have exclusive jurisdiction to determine any and all disputes arising under Articles VII (except as otherwise specifically provided by Article VII, Section 3(d)(5)), VIII, X, XI, XII, XIII, XIV, XV, XVI, XXXVII, XXXIX, and XL of this Agreement, and those made subject to his jurisdiction by Sections 9 and 10 of this Article. In addition, in the event of a disagreement between the NBA and the Players Association, the System Arbitrator shall have exclusive jurisdiction to determine whether the System Arbitrator, the Grievance Arbitrator or some other arbitrator provided for by the provisions of this Agreement has jurisdiction to hear or resolve a particular dispute.

\hypertarget{initiation.-1}{%
\section{Initiation.}\label{initiation.-1}}

\begin{enumerate}
\def\labelenumi{(\alph{enumi})}
\tightlist
\item
  Subject to Article XIV, Section 5, System Arbitrations may be initiated, as set forth below, only by the NBA or the Players Association.
\item
  No party may initiate a System Arbitration until and unless it has first discussed the matter with the other party in an attempt to settle it.
\item
  A System Arbitration must be initiated within three (3) years from the date of the act or omission upon which the System Arbitration is based, or within three (3) years from the date upon which such act or omission became known or reasonably should have become known to the party initiating the System Arbitration, whichever is later.
\item
  Either the NBA or the Players Association may initiate a System Arbitration by filing written notice thereof with the System Arbitrator and serving a copy of such notice on the other party.
\end{enumerate}

\hypertarget{hearings.-1}{%
\section{Hearings.}\label{hearings.-1}}

\begin{enumerate}
\def\labelenumi{(\alph{enumi})}
\tightlist
\item
  The System Arbitrator shall hold hearings on alleged violations of the Articles set forth in Section 1 above. Except as otherwise provided in Article XI, Section 5(l) and Sections 9 and 10 below, awards issued by the System Arbitrator shall be subject to review by the Appeals Panel, in the manner and in accordance with the procedures set forth in Sections 3 and 8 of this Article XXXII.
\item
  The System Arbitrator shall make findings of fact and award appropriate relief including, without limitation, damages and specific performance. The System Arbitrator shall render an award as soon as practicable, and the award shall be accompanied by a written opinion. Notwithstanding the foregoing, if the System Arbitrator determines that expedition so requires, he shall accompany the award with a written summary of the grounds upon which the award is based, and a full written opinion may follow within a reasonable time thereafter. In no event shall the award and written opinion be issued more than thirty (30) days following the date upon which the record of a System Arbitration proceeding is closed (or, where applicable, the date designated by the System Arbitrator for the submission of post-hearing briefs).
\item
  The System Arbitrator shall have authority to order the production of documents, the conduct of pre-hearing depositions, and the attendance of witnesses at the hearing with respect to the NBA and the Players Association, and/or any player or Team. The System Arbitrator shall have the authority to compel the attendance of witnesses and the production of documents at any hearing within the jurisdiction of the System Arbitrator in accordance with the New York C.P.L.R.
\item
  An award of the System Arbitrator shall upon its issuance constitute the full, final and complete disposition of the dispute, shall be binding upon the parties to this Agreement and upon any player(s) or Team(s) involved, and shall be followed by them unless (in cases where this Agreement provides for an appeal to the Appeals Panel) a notice of appeal is served by the appealing party upon the responding party and filed with the System Arbitrator within ten (10) days of the date of the award of the System Arbitrator appealed from. If and when an award of the System Arbitrator is reversed or modified by the Appeals Panel, the effect of such reversal or modification shall be deemed by the parties to be retroactive to the time of issuance of the award of the System Arbitrator. The parties may seek appropriate relief to effectuate and enforce this provision.
\item
  The System Arbitrator shall not have jurisdiction or authority to add to, detract from, or alter in any way the provisions of this Agreement or any Player Contract. Nor, except for the authority conferred upon him by the second sentence of Section 1 above (or unless the NBA and the Players Association otherwise agree), shall the System Arbitrator have jurisdiction or authority to resolve questions of substantive, as opposed to procedural, arbitrability (which shall include the question of whether an arbitrator provided for by the terms of this Agreement, as opposed to the Commissioner (or his designee), has jurisdiction to hear or resolve a particular dispute), which shall be determined in a judicial proceeding to be venued in the United States District Court for the Southern District of New York.
\end{enumerate}

\hypertarget{costs-relating-to-system-arbitration.}{%
\section{Costs Relating to System Arbitration.}\label{costs-relating-to-system-arbitration.}}

\begin{enumerate}
\def\labelenumi{(\alph{enumi})}
\tightlist
\item
  The compensation of the System Arbitrator and the costs and expenses incurred in connection with any proceeding brought before the System Arbitrator shall be borne equally by the parties to this Agreement; provided, however, that each participant in such proceeding shall bear its own attorneys' fees and litigation costs.
\item
  Notwithstanding the provisions of Section 4(a) above, if a matter is scheduled for hearing under this Article XXXII, and the hearing date is thereafter postponed at the request of either the NBA or the Players Association, the postponement fee (if any) of the System Arbitrator will be borne by the party requesting the postponement unless that party objects and the System Arbitrator finds that the request for such postponement was for good cause. Should good cause be found, the parties will share any postponement fee equally.
\end{enumerate}

\hypertarget{procedure-for-system-arbitration.}{%
\section{Procedure for System Arbitration.}\label{procedure-for-system-arbitration.}}

All matters before the System Arbitrator shall be heard and determined in an expedited manner, provided that such expedition is reasonable under the circumstances. A proceeding may be commenced upon seventy-two (72) hours' written notice (or upon shorter notice if ordered by the System Arbitrator) served upon the party against whom the proceeding is brought and filed with the System Arbitrator. All such notices and all orders and notices issued and directed by the System Arbitrator shall be served on the NBA, counsel for the NBA, the Players Association, counsel for the Players Association, and any counsel appearing for individual NBA players or individual NBA Teams. In any proceeding commenced pursuant to Article XIV, Section 5, the Players Association (on its own behalf and/or on behalf of a player) and the NBA (on its own behalf and/or on behalf of a Team) shall have the right to participate.

\hypertarget{selection-of-system-arbitrator.}{%
\section{Selection of System Arbitrator.}\label{selection-of-system-arbitrator.}}

\begin{enumerate}
\def\labelenumi{(\alph{enumi})}
\tightlist
\item
  In the event that the Players Association and the NBA cannot agree on the identity of a System Arbitrator, the parties shall jointly request the International Institute for Conflict Prevention and Resolution (the ``CPR Institute'') (or such other organization(s) as the parties may have agreed upon) to submit to the parties a list of eleven (11) attorneys, none of whom shall have, nor whose firm shall have, represented within the past five (5) years any professional athletes; agents or other representatives of professional athletes; labor organizations representing athletes; sports leagues, governing bodies, or their affiliates; sports teams or their affiliates; or owners in any professional sport. If the parties cannot within seven (7) days from the receipt of such list agree to the identity of the System Arbitrator from among the names on such list, they shall return said list, with up to five (5) names deleted therefrom by each party, to the CPR Institute (or such other organization as the parties may have agreed upon), which shall choose from the remaining names on the list the identity of the System Arbitrator.
\item
  The first System Arbitrator selected under the provisions of this Agreement shall serve until June 30, 2006. Thereafter, the System Arbitrator shall serve for continually renewing two-year terms unless notice of termination is given either by the NBA or by the Players Association. Notice of termination of the System Arbitrator shall be given to the other party, and to the System Arbitrator, during the period May 10 through May 15 immediately preceding the end of any term. Following the giving of such notice, a new System Arbitrator shall be selected in accordance with the procedures set forth in Section 6(a) above. A System Arbitrator as to whom a notice of termination has been given shall continue to have jurisdiction only with respect to (i) System Arbitrations in which a hearing has been commenced or scheduled for a date certain, and (ii) System Arbitrations initiated (in accordance with theprovisions of Section 2 above) within the thirty (30) day period preceding the service of the notice of termination; provided, however, that a hearing with respect to System Arbitrations referred to in this subsection (ii) must commence no later than thirty (30) days following the end of a System Arbitrator's term.
\end{enumerate}

\hypertarget{selection-of-appeals-panel.}{%
\section{Selection of Appeals Panel.}\label{selection-of-appeals-panel.}}

There shall be a three-member Appeals Panel for each appeal noticed from an award of the System Arbitrator. In the event the Players Association and the NBA cannot agree upon the members of such a panel, the parties will jointly request the CPR Institute (or such other organization(s) as the parties may agree) to submit to the parties a list of fifteen (15) attorneys (none of whom shall have, nor whose firm shall have, represented within the past five (5) years any professional athletes; agents or other representatives of professional athletes; labor organizations representing athletes; sports leagues, governing bodies, or their affiliates; sports teams or their affiliates; or owners in any professional sport). If the parties cannot within seven (7) days from the receipt of such list agree to the identity of the Appeals Panel from among the names on such list, they shall meet and alternate striking one (1) name at a time from the list until three (3) names on the list remain. The three (3) remaining names on the list shall comprise the Appeals Panel for that particular appeal. The compensation of the members of the Appeals Panel and the costs of proceedings before the Appeals Panel shall be borne equally by the parties to this Agreement; provided, however, that each participant in an Appeals Panel proceeding shall bear its own attorneys' fees and litigation costs.

\hypertarget{procedure-relating-to-appeals-of-determination-by-the-system-arbitrator.}{%
\section{Procedure Relating to Appeals of Determination by the System Arbitrator.}\label{procedure-relating-to-appeals-of-determination-by-the-system-arbitrator.}}

\begin{enumerate}
\def\labelenumi{(\alph{enumi})}
\tightlist
\item
  Any party seeking to appeal (in whole or in part) an award of the System Arbitrator must serve on the other party and file with the System Arbitrator a notice of appeal, within ten (10) days of the date of the award appealed from. The timely service and filing of a notice of appeal shall automatically stay the award of the System Arbitrator pending resolution by the Appeals Panel.
\item
  Following the timely service and filing of a notice of appeal, the NBA and the Players Association shall attempt to agree upon a briefing schedule. In the absence of such agreement, the briefing schedule shall be set by the Appeals Panel; provided, however, that any party seeking to appeal (in whole or in part) from an award of the System Arbitrator shall be afforded no less than fifteen (15) and no more than twenty-five (25) days from the date of the issuance of such award, or the date of the issuance of the System Arbitrator's written opinion, or the date upon which the members of the Appeals Panel have been selected in accordance with the provisions of Section 7 above, whichever is latest, to serve on the opposing party and file with the Appeals Panel its brief in support thereof; and provided further that the responding party or parties shall be afforded the same aggregate amount of time to serve and file its or their responding brief(s). The Appeals Panel shall schedule oral argument on the appeal(s) no less than five (5) and no more than ten (10) days following the service and filing of the responding brief(s), and shall issue a written decision within thirty (30) days from the date of argument.
\item
  The Appeals Panel shall review the findings of fact and conclusions of law made by the System Arbitrator using the standards of review employed by the U.S. Court of Appeals for the Second Circuit. The decision of the Appeals Panel shall constitute full, final, \textbf{\emph{{[}sic{]}}}
\end{enumerate}

\hypertarget{special-procedure-for-disputes-with-respect-to-interim-audit-reports.}{%
\section{Special Procedure for Disputes with Respect to Interim Audit Reports.}\label{special-procedure-for-disputes-with-respect-to-interim-audit-reports.}}

\begin{enumerate}
\def\labelenumi{(\alph{enumi})}
\tightlist
\item
  Notwithstanding any of the other provisions of this Agreement, at the request of either the NBA or the Players Association, and irrespective of which party may commence the proceeding, the procedures set forth in this Section 9 shall apply to the resolution of any disputes with respect to an Interim Audit Report, including but not limited to disputes concerning any Escrow Information set forth in an Interim Audit Report. If in connection with such disputes, there is any conflict between the procedures set forth in this Section 9 and those set forth elsewhere in this Agreement, the procedures set forth in this Section shall control.
\item
  A proceeding before the System Arbitrator shall be commenced, in the manner provided for by Sections 2(d) and 5 above, no more than thirty (30) days following the delivery by the Accountants of the Interim Audit Report with respect to any dispute or claim concerning (i) the amount(s) of BRI or Total Salaries (or portions thereof) as to which the Accountants have completed their review and which is the subject of a good faith dispute between the parties, (ii) the amount(s) of BRI or Total Salaries (or portions thereof) as to which the Accountants have not completed their review and with respect to which the parties have a good faith disagreement, (iii) such Escrow Information as is included in the Interim Audit Report as to which the parties have a good faith disagreement, and/or (iv) all other disputes (including but not limited to disputes over the amounts and includability of any revenues or expenses included or excluded from the Interim Audit Report) of which the parties were aware or reasonably should have been aware, at the time the proceeding was commenced, based upon the \textbf{\emph{{[}sic{]}}}
\item
  A party's failure to commence a proceeding before the System Arbitrator within the thirty-day (30) period provided for by Section 9(b) above with respect to the disputes or claims enumerated therein shall forever bar that party from asserting or seeking relief of any kind for any such dispute or claim; provided, however, that the provisions of Section 9(b) above and this Section 9(c) shall not bar a party from commencing a proceeding before the System Arbitrator and seeking appropriate relief, subject to the limitations imposed by Section 2 above:

  \begin{enumerate}
  \def\labelenumii{(\roman{enumii})}
  \tightlist
  \item
    With respect to a dispute or claim concerning an Interim Audit Report as to which such party was not aware or reasonably should not have been aware, based upon the materials referred to in Section 9(b) above, during the thirty-day (30) period following the delivery of such Interim Audit Report; or
  \item
    With respect to any dispute or claim relating to a subsequent Salary Cap Year, including, but not limited to, any dispute concerning the includability or non-includability in BRI of a category or type of revenue or the allowance or disallowance of a category or type of expense, without regard to whether, based upon the materials referred to in Section 9(b) above (other than a BRI Report, Draft Audit Report or Interim Audit Report), the party was or reasonably should have been aware of such dispute or claim during the thirty-day (30) period following the delivery of such Interim Audit Report.(iii) Subject to Section 9(c)(ii) above, no determination made by the System Arbitrator or the Appeals Panel (as the case may be) in a proceeding commenced pursuant to Section 9(c)(i) or (ii) above shall affect any calculations made pursuant to Article VII, Section 12.
  \end{enumerate}
\item
  Where a hearing before the System Arbitrator is provided for by this Section 9, such hearing shall be conducted within fifteen (15) days from the commencement of the proceeding, and the System Arbitrator shall render an award and issue a written decision as soon as possible, but in no event later than fifteen (15) days following the close of the hearing. Where a right to appeal from the System Arbitrator's award is provided for by this Section 9, any party seeking to appeal (in whole or in part) from such an award shall serve and file a notice of appeal therefrom within five (5) days from the date of such award and shall serve and file its brief in support of such appeal within fifteen (15) days from the date of the System Arbitrator's award or within five (5) days from the date upon which the members of the Appeals Panel have been selected, whichever is later. The party opposing such appeal shall serve and file its brief in opposition within ten (10) days following its receipt of the brief in support of the appeal. The Appeals Panel shall schedule oral argument at its discretion, but shall issue a written decision within twenty (20) days following its receipt of the brief from the party opposing the appeal.
\item
  Any dispute concerning the amounts (as opposed to the includability) of any revenues or expenses to be included in an Interim Audit Report (hereinafter referred to as ``Disputed Adjustments'') shall, whenever such Disputed Adjustments for all Teams are adverse to the party asserting the dispute in an aggregate amount of less than \$10 million for any Season covered by this Agreement, be resolved by the Accountants; and the determination of the Accountants shall constitute full, final and complete disposition of the dispute and shall bebinding upon the parties to this Agreement. Notwithstanding the foregoing, any Disputed Adjustments that involve the interpretation, validity or application of this Agreement shall be resolved by the System Arbitrator and shall be appealable to the Appeals Panel in accordance with the provisions of Section 9(d) above.
\item
  If the Disputed Adjustments for all Teams are adverse to the party asserting the dispute in an aggregate amount of \$10 million or more but less than \$15 million for any Season covered by this Agreement, the determination of the System Arbitrator shall constitute full, final and complete disposition of the dispute and shall be binding upon the parties to this Agreement, and there shall be no appeal to the Appeals Panel. Notwithstanding the foregoing, any Disputed Adjustments that involve the interpretation, validity or application of this Agreement shall be resolved by the System Arbitrator and shall be appealable to the Appeals Panel in accordance with the provisions of Section 9(d) above.
\item
  If the Disputed Adjustments for all Teams are adverse to the party asserting the dispute in an aggregate amount of \$10 million or more but less than \$15 million for any Season covered by this Agreement, and if the party asserting such dispute does not prevail before the System Arbitrator, then that party shall pay all of the fees and expenses of the System Arbitrator and the reasonable costs and expenses, including attorneys' fees, of the other party for its defense of the proceeding; provided, however, that if each party has asserted a dispute upon which it has not prevailed, all such fees, expenses and costs shall be borne in the manner provided for by Section 4 above.
\item
  All other disputes involving an Interim Audit Report (including but not limited to disputes over the amounts and includability of any revenues or expenses to be included in such Reports) and the Escrow Information shall be resolved by the System Arbitrator and shall be appealable to the Appeals Panel in accordance with the provisions of Section 9(d) above.
\end{enumerate}

\hypertarget{special-procedure-for-disputes-with-respect-to-the-escrow-schedules.}{%
\section{Special Procedure for Disputes with Respect to the Escrow Schedules.}\label{special-procedure-for-disputes-with-respect-to-the-escrow-schedules.}}

\begin{enumerate}
\def\labelenumi{(\alph{enumi})}
\tightlist
\item
  Notwithstanding any of the other provisions of this Agreement, the procedures set forth in this Section 10 shall apply to the resolution of any disputes with respect to the Escrow Schedules described in Article VII, Section 12. If in connection with such disputes, there is any conflict between the procedures set forth in this Section 10 and those set forth elsewhere in this Agreement, the procedures set forth in this Section shall control.
\item
  In the event of any dispute with respect to the Escrow Schedules, the proceeding before the System Arbitrator shall be commenced, in the manner provided for by Sections 2(d) and 5 above no more than seven (7) days following the transmittal to the Players Association of any of such schedules.
\item
  The hearing before the System Arbitrator with respect to a dispute concerning the Escrow Schedules shall be conducted within ten (10) days following the commencement of the proceeding and the briefs of the parties, if any, shall be filed before the opening of the hearing on a date or dates set by the System Arbitrator. The hearing shall be conducted on an expedited basis and, unless the parties otherwise agree or a party demonstrates that such limitation will result in undue prejudice, will not last longer than two (2) full days.
\item
  If in connection with the Escrow Schedules, there is a dispute between the NBA and the Players Association and the amount in controversy is \$5 million or less, the determination of the System Arbitrator shall constitute full, final and complete disposition of the dispute and shall be binding upon the parties to this Agreement, and there shall be no appeal to the Appeals Panel. If with respect to such dispute the amount in controversy is more than \$5 million, either party may appeal a determination of the System Arbitrator to the Appeals Panel.
\item
  In connection with any dispute concerning the Escrow Schedules, the System Arbitrator shall render an award and issue a written decision as soon as possible, but in no event later than ten (10) days following the close of the hearing. When the award is issued, the System Arbitrator shall set forth the basis therefore either in a written opinion or orally at a conference with the parties (which conference may be conducted by telephone) of which a stenographic record shall be made. Any party seeking to appeal (in whole or in part) from an award of the System Arbitrator rendered pursuant to Section 10(d) above shall serve and file a notice of appeal therefrom within two (2) business days from the date of such award. The party seeking to appeal shall serve and file its brief in support of such appeal within ten (10) days from the date of the System Arbitrator's award or within three (3) days from the date upon which the members of the Appeals Panel have been selected, whichever is later. The party opposing such appeal shall serve and file its brief in opposition within ten (10) days following its receipt of the brief in support of the appeal. The Appeals Panel shall schedule oral argument at its discretion, but shall issue a written decision within twenty (20) days following its receipt of the brief from the party opposing the appeal.
\end{enumerate}

\hypertarget{anti-drug-program}{%
\chapter{ANTI-DRUG PROGRAM}\label{anti-drug-program}}

\hypertarget{definitions.-2}{%
\section{Definitions.}\label{definitions.-2}}

As used in this Article XXXIII, the following terms shall have the following meanings:

\begin{enumerate}
\def\labelenumi{(\alph{enumi})}
\tightlist
\item
  ``Authorization for Testing'' shall mean a notice issued by the Independent Expert pursuant to the provisions of Section 5 below, in the form annexed hereto as Exhibit I-1 to this Agreement.
\item
  ``Come Forward Voluntarily'' shall mean that a player has directly communicated to the NBA, the Players Association, or the Medical Director his desire to enter the Program and seek treatment for a problem involving the use of a Prohibited Substance.
\item
  ``Counselors'' or ``Anti-Drug Counselors'' shall mean the persons selected by the Medical Director to provide counseling and other treatment to players in the Program.
\item
  ``Diuretics'' shall mean any of the substances listed as diuretics on Exhibit I-2 to this Agreement.
\item
  ``Drugs of Abuse'' shall mean any of the substances listed as drugs of abuse on Exhibit I-2 to this Agreement.
\item
  ``Drugs of Abuse Program'' shall mean (i) the testing program for Drugs of Abuse set forth in this Article XXXIII, and (ii) the education, treatment, and counseling program for Drugs of Abuse established by the Medical Director (after consultation with the NBA and the Players Association), which may contain such elements---including, but not limited to, urine,blood, breath, or other testing for Prohibited Substances other than SPEDs---as may be determined by the Medical Director in his or her professional judgment.
\item
  ``First-Year Player'' shall mean a player under Contract to an NBA Team who, prior to the then-current Season, has not been on the roster of an NBA Team following the first game of a Regular Season.
\item
  ``In-Patient Facility'' shall mean such treatment center or other facility as may be selected by the Medical Director and agreed upon by the NBA and the Players Association.
\item
  ``Independent Expert'' or ``Expert'' shall mean the person selected by the NBA and the Players Association in accordance with Section 2(b) below.
\item
  ``Marijuana Program'' shall mean (i) the testing program for marijuana set forth in this Article XXXIII, and (ii) the education, treatment, and counseling program for marijuana established by the Medical Director (after consultation with the NBA and the Players Association), which may contain such elements---including, but not limited to, urine, blood, breath, or other testing for Prohibited Substances other than SPEDs---as may be determined by the Medical Director in his or her professional judgment.
\item
  ``Medical Director'' shall mean the person selected by the NBA and the Players Association in accordance with Section 2(a) below.
\item
  ``Prohibited Substance'' shall mean any of the substances listed on Exhibit I-2 to this Agreement and any other substance added to such Exhibit under the provisions of Section 15 below.
\item
  ``Program'' shall mean this Anti-Drug Program, and shall include the Drugs of Abuse Program, the Marijuana Program, and the SPED Program.
\item
  ``Prohibited Substances Committee'' shall mean the committee selected by the NBA and the Players Association in accordance with Section 2(d) below.
\item
  ``SPED'' shall mean any of the steroids, performance-enhancing drugs and masking agents listed on Exhibit I-2 to this Agreement.
\item
  ``SPED Program'' shall mean the (i) testing program for SPEDs set forth in this Article XXXIII, and (ii) the education, treatment, and counseling program for SPEDs established by the Medical Director (after consultation with the NBA and the Players Association), which may contain such elements---including, but not limited to, urine, blood, breath or other testing for SPEDs (but not for any other Prohibited Substance)---as may be determined by the Medical Director in his or her professional judgment.
\item
  ``Tender'' shall mean an offer of a Uniform Player Contract, signed by the Team, that is either personally delivered to the player or his representative or sent by prepaid certified, registered, or overnight mail to the last known address of the player or his representative.
\item
  ``Veteran Player'' shall mean any player who is not a First-Year Player.
\end{enumerate}

\hypertarget{administration.}{%
\section{Administration.}\label{administration.}}

\begin{enumerate}
\def\labelenumi{(\alph{enumi})}
\item
  The NBA and the Players Association shall jointly select a Medical Director who shall be a person experienced in the field of testing and treatment for substance abuse. The Medical Director shall have the responsibility, among other duties, for selecting and supervising the Counselors and other personnel necessary for the effective implementation of the Program, for evaluating and treating players subject to the Program, and for otherwise managing and overseeing the Program, subject to the control of the NBA and the Players Association. To the extent practicable, the Medical Director shall select qualified retired NBA players to serve as Counselors.
\item
  The NBA and the Players Association shall jointly select an Independent Expert who shall be a person experienced in the field of substance abuse detection and enforcement and shall have the responsibility for issuing Authorizations for Testing in accordance with Section 5 below.
\item
  The Medical Director and the Independent Expert shall each serve for the duration of this Agreement, unless either the NBA or the Players Association has, by September 1 of any year covered by this Agreement, served written notice of discharge upon the other party and, as appropriate, the Medical Director and/or the Independent Expert. Such notice of discharge shall be effective as of the immediately following September 30; provided, however, that if the parties do not reach agreement by such September 30 as to who shall serve thereafter as the Medical Director and/or the Independent Expert, as the case may be, each party shall, by the immediately following October 15, appoint a person who shall have no relationship to or affiliation with that party. Such persons shall then have until the immediately following December 1 to agree on the appointment of a new Medical Director and/or Independent Expert. Until a new Medical Director and/or Independent Expert has been appointed, the previous Medical Director and/or Independent Expert shall continue to serve.
\item
  \begin{enumerate}
  \def\labelenumii{(\roman{enumii})}
  \tightlist
  \item
    The NBA and the Players Association shall form a Prohibited Substances Committee, which shall be comprised of one (1) representative from the NBA, one (1) representative from the Players Association, and three (3) individuals jointly selected by the NBA and the Players Association who shall be experts in the field of testing and treatment for drugs of abuse and performance-enhancing substances. The members of this Committee shall serve for the duration of the Agreement.
  \item
    The members of the Prohibited Substances Committee shall meet (either in person or by conference call) at least once each Season and once each Off-Season (the ``Annual Meetings''). The Annual Meetings shall be scheduled by the NBA after consultation with the NBPA. At the Annual Meetings, the Committee shall review the Program's list of Prohibited Substances, discuss general anti-doping issues (including, but not limited to, advances in drug testing science and technology, and modifications to relevant anti-doping policies of other sports organizations). The Committee shall also make recommendations to the NBA and NBPA for changes to the list of Prohibited Substances (including the determination of laboratory analysis cutoff levels).
  \item
    As of September 1, 2005, and as of each successive September 1, either of the parties to this Agreement may discharge any jointly-selected member of the Prohibited Substances Committee by serving thirty (30) days' prior notice upon him and upon the other party to this Agreement. In the case \textbf{\emph{{[}sic{]}}}
  \end{enumerate}
\item
  Unless specifically stated otherwise in this Article XXXIII, all costs of the Program in excess of those covered by the NBA Players Group Health Plan, including the fees and expenses of the Medical Director, the Independent Expert, and the Prohibited Substances Committee shall be shared equally by the NBA and Players Association. The Players Association's share shall be paid by the NBA and included in Player Benefits under Article IV, Section 5(j) of this Agreement. The fees and expenses incurred by the NBA in conducting testing pursuant to Sections 5 and 6 below shall be borne by the NBA.
\item
  Any and all disputes arising under this Article XXXIII shall be resolved in accordance with Article XXXI, Sections 2-6 and 14 of this Agreement; provided, however, that in any challenge to a decision, recommendation, or other conduct of the Medical Director, Independent Expert, or Prohibited Substances Committee, or in any challenge to an action or process over which the Medical Director has supervision, the Grievance Arbitrator shall apply an ``arbitrary and capricious'' standard of review; and provided further that nothing in this Section 2(f) shall limit or otherwise affect paragraph 19 of the Uniform Player Contract.
\end{enumerate}

\hypertarget{confidentiality.}{%
\section{Confidentiality.}\label{confidentiality.}}

\begin{enumerate}
\def\labelenumi{(\alph{enumi})}
\tightlist
\item
  Other than as reasonably required in connection with the suspension or disqualification of a player, the NBA, the Teams, and the Players Association, and all of their members, affiliates, agents, consultants, and employees, are prohibited from publicly disclosing information about the diagnosis, treatment, prognosis, test results, compliance, or the fact of participation of a player in the Program (``Program Information''). If a player is suspended or disqualified for conduct involving a Drug of Abuse or marijuana, the NBA shall not publicly disclose the particular Prohibited Substance involved, absent the agreement of the Players Association or the prior disclosure of such information by the player (or by a person authorized by the player to disclose such information). If a player is suspended or disqualified for conduct involving a SPED, the particular SPED shall be publicly disclosed along with the announcement of the applicable penalty.
\item
  The Medical Director and the Counselors, and all of their affiliates, agents, consultants, and employees, are prohibited from publicly disclosing Program Information; provided, however, that the Medical Director shall not be prohibited from disclosing such information to the NBA and the Players Association.
\item
  The Independent Expert is prohibited from publicly disclosing any information supplied to him by the NBA or the Players Association pursuant to Section 5 below.
\item
  Members of the Prohibited Substances Committee are prohibited from publicly disclosing any information obtained by them in connection with their duties as Committee members. If a jointly-selected member of the Committee violates this Section 3(d), he shall be immediately discharged from the Committee.
\item
  Any Program Information that is publicly disclosed (i) under Section 3(a) above, (ii) by the player, (iii) with the player's authorization, or (iv) through release by sources other than the NBA, NBA Teams, the Players Association, the Medical Director, the Counselors, the Independent Expert, or the Prohibited Substances Committee, or any of their members, affiliates, agents, consultants, and employees, will, after such disclosure, no longer be subject to the confidentiality provisions of this Section 3.
\item
  Other than as reasonably required by the Reasonable Cause Testing procedure set forth in Section 5 below, neither the NBA nor the Players Association shall divulge to any other person or entity (including their respective members, affiliates, agents, consultants, employees, and the player and Team involved):

  \begin{enumerate}
  \def\labelenumii{(\roman{enumii})}
  \tightlist
  \item
    that it has received information regarding the use, possession, or distribution of a Prohibited Substance by a player;
  \item
    that it is considering requesting, has requested, or has had a conference with the Independent Expert concerning the suspected use, possession, or distribution of a Prohibited Substance by a player;
  \item
    any information disclosed to the Independent Expert; or
  \item
    the results of any conference with the Independent Expert.
  \end{enumerate}
\item
  Notwithstanding anything to the contrary contained in Section 3(a)-(f) above, the NBA and the Players Association shall promptly advise and make available to each other all information either of them may have in their possession, custody, or control that provides cause to believe that a player is engaged in the use, possession, or distribution of a Prohibited Substance.
\item
  Nothing contained in this Section 3 shall prohibit a Team from providing to the NBA information concerning whether a player is engaged in the use, possession, or distribution of a Prohibited Substance.
\end{enumerate}

\hypertarget{testing.}{%
\section{Testing.}\label{testing.}}

\begin{enumerate}
\def\labelenumi{(\alph{enumi})}
\tightlist
\item
  Testing conducted pursuant to this Article XXXIII, whether by the NBA or the Medical Director, shall be conducted in compliance with scientifically accepted analytical techniques. Such testing shall also comply with the collection procedures described in Exhibit I-3 to this Agreement and such additional procedures and protocols as may be established by the NBA (after consultation with the Players Association) or the Medical Director (after consultation with the NBA and the Players Association). The NBA and the Medical Director (after consultation with the NBA and the Players Association) are authorized to retain such consultants and support services as are necessary and appropriate to administer and conduct such testing.
\item
  All tests conducted pursuant to this Article XXXIII shall be analyzed by laboratories selected by the NBA and the Players Association, approved by the Medical Director, and certified by the World Anti-Doping Agency, the Substance Abuse and Mental Health Services Administration (SAMHSA), or the International Olympic Committee.
\item
  Any test conducted pursuant to this Article XXXIII will be considered ``positive'' for a Prohibited Substance under the following circumstances:

  \begin{enumerate}
  \def\labelenumii{(\roman{enumii})}
  \tightlist
  \item
    If the test is for a Prohibited Substance other than a SPED or Diuretic and it is confirmed by laboratory analysis at the levels established at the time of the test by SAMHSA; provided, however, if there is no confirmatory level established by SAMHSA for one or more of such Prohibited Substances at the time of the test, then the level for such Prohibited Substance shall be: amphetamines and their analogs---500 ng/ml; cocaine metabolites---150 ng/ml; LSD---200 pg/ml; marijuana metabolites---15 ng/ml; MDMA---500 ng/ml; opiate metabolites---2000 ng/ml; heroin metabolite (6-acety1morphine)---10 ng/ml (only if the opiate metabolites are in excess of 2000 ng/ml); phencyclidine---25 ng/ml.
  \item
    If the test is for a SPED, and it is confirmed by laboratory analysis at the levels set forth in Exhibit I-4.
  \item
    If a player refuses to submit to a test or cooperate fully with the testing process, without a reasonable explanation satisfactory to the Medical Director; provided, however, that the NBA will use its best efforts (A) to have the drug testing collectors immediately notify the NBA when any player refuses to submit to a test or cooperate fully with the testing process, and (B) to provide such information to the Players Association as soon as possible thereafter; and provided, further, that (C) following any player's refusal to submit to a test or failure to cooperate fully with the testing process, the drug testing collector shall wait ninety (90) minutes at the collection site, and (D) if the player submits to the test and cooperates fully with the testing process within such additional time, then his earlier refusal or failure to cooperate shall be excused and the test shall not be deemed positive under this Section 4(c).
  \item
    If the player fails to submit to a scheduled test, without a reasonable explanation satisfactory to the Medical Director.
  \item
    If the player attempts to substitute, dilute, or adulterate a specimen sample or in any other manner alter a test result (other than by testing positive for a Diuretic).
  \item
    If the test is positive for a Diuretic, and it is confirmed by laboratory analysis at any detectable level.
  \end{enumerate}
\item
  The NBA shall promptly notify the Players Association of any positive test conducted by the NBA, and shall thereafter notify the player. The Medical Director shall promptly notify the player of any positive test conducted by the Medical Director; provided, however, that if the positive test will result in a penalty to be imposed on the player, the Medical Director shall notify the NBA and the Players Association of the positive test result and the NBA shall thereafter notify the player of such result and such penalty.
\item
  Any player who is notified of a positive test pursuant to Section 4(d) above may, within five (5) business days of such notification, inform the NBA and the Players Association that he requests testing of the split or ``B'' sample of his specimen. Any such test shall be subject to the provisions of this Section 4 and shall be performed within ten (10) business days of the player's request. The test of the ``B'' sample will be performed at a laboratory other than the laboratory that performed the test on the original or ``A'' sample.
\item
  Any positive test pursuant to Sections 4(c)(i), (ii), or (vi) above shall be reviewed by the Medical Director. If the Medical Director determines, in his professional judgment, that there is a valid alternative medical explanation for such positive test result, then the test shall be deemed negative.
\item
  If the test result for any player is reported by the laboratory as ``invalid'' ``endogenous steroids abnormally low or absent,'' or a similar designation, the NBA shall promptly notify the Players Association, and shall thereafter notify the player. In the event of such a test result, the player shall be required to submit to another test on a date determined by the NBA that is not more than thirty (30) days after the date of the original test (the ``Re-Test''). If the Re-Test results in (i) a positive test for a Drug of Abuse or a positive test under Section 4(c)(iii), (iv) or (v) above, the player shall be immediately be dismissed and disqualified from any association with the NBA or its Teams in accordance with the provisions of Section 11(a) below; (ii) a positive test for marijuana, the player shall suffer the applicable consequences set forth in Section 8 below; or (iii) a positive test for a SPED or Diuretic, the player shall suffer the applicable consequences set forth in Section 9 below. The original test will not be counted towards the number of tests to be administered to that player for that Season under Section 6 (Random Testing) below.
\end{enumerate}

\hypertarget{reasonable-cause-testing-or-hearing.}{%
\section{Reasonable Cause Testing or Hearing.}\label{reasonable-cause-testing-or-hearing.}}

\begin{enumerate}
\def\labelenumi{(\alph{enumi})}
\tightlist
\item
  In the event that either the NBA or the Players Association has information that gives it reasonable cause to believe that a player is engaged in the use, possession, or distribution of a Prohibited Substance, including information that a First-Year Player may have been engaged in such conduct during the period beginning three (3) months prior to his entry into the NBA, such party shall request a conference with the other party and the Independent Expert, which shall be held within twenty-four (24) hours or as soon thereafter as the Expert is available. Upon hearing the information presented, the Independent Expert shall immediately decide whether there is reasonable cause to believe that the player in question has been engaged in the use, possession, or distribution of a Prohibited Substance. If the Independent Expert decides that such reasonable cause exists, the Expert shall thereupon issue an Authorization for Testing with respect to such player.
\item
  In evaluating the information presented to him, the Independent Expert shall use his independent judgment based upon his experience in substance abuse detection and enforcement. The parties acknowledge that the type of information to be presented to the Independent Expert is likely to consist of reports of conversations with third parties of the type generally considered by law enforcement authorities to be reliable sources, and that such sources might not otherwise come forward if their identities were to become known. Accordingly, neither the NBA nor the Players Association shall be required to divulge to each other or to the Independent Expert the names (or other identifying characteristics) of their sources of information regarding the use, possession, or distribution of a Prohibited Substance, and the absence of such identification of sources, standing alone, shall not constitute a basis for the Expert to refuse to issue an Authorization for Testing with respect to a player. In conferences with the Independent Expert, the player involved shall not be identified by name until such time as the Expert has determined to issue an Authorization for Testing with respect to such player.
\item
  Immediately upon the Independent Expert's issuance of an Authorization for Testing with respect to a particular player, the NBA shall arrange for such player to undergo testing for Drugs of Abuse (if the Authorization for Testing was based on information regarding the use, possession, or distribution of a Drug of Abuse), for marijuana (if the authorization for Testing was based on information regarding the player's use, possession, or distribution of marijuana), or for SPEDs (if the Authorization for Testing was based on information regarding the player's use, possession, or distribution of a SPED) no more than four (4) times during the six-week period commencing with the issuance of the Authorization for Testing. Such testing may be administered at any time, in the discretion of the NBA, without prior notice to the player.
\item
  In the event that the player tests positive for a Drug of Abuse pursuant to this Section 5, or tests positive pursuant to Section 4(c)(iii), (iv) or (v) above in connection with testing conducted pursuant to this Section 5, he shall immediately be dismissed and disqualified from any association with the NBA or any of its Teams in accordance with the provisions of Section 11(a) below. If the player tests positive for marijuana or a SPED pursuant to this Section 5, he shall enter the Program and suffer the applicable consequences set forth in Sections 8 or 9 below, as the case may be. If the player tests positive for a Diuretic, he shall suffer the applicable consequences of a positive test for the Prohibited Substance for which the Authorization for Testing was issued.
\item
  In the event that either the NBA or the Players Association determines that there is sufficient evidence to demonstrate that, within the previous year, a player has engaged in the use, possession, or distribution of a Prohibited Substance, or has received treatment for use of a Prohibited Substance other than in accordance with the terms of this Article XXXIII, it may, in lieu of requesting the testing procedure set forth in Section 5(a)-(d) above, request a hearing on the matter before the Grievance Arbitrator. If the Grievance Arbitrator concludes that, within the previous year, the Player has used, possessed, or distributed a Prohibited Substance, or has received treatment other than in accordance with the terms of this Article XXXIII, the player shall immediately be dismissed and disqualified from any association with the NBA or any of its Teams in accordance with the provisions of Section 11(a) below, notwithstanding the fact that the player has not undergone the testing procedure set forth in this Section 5; provided, however, that if the Grievance Arbitrator concludes that the player has used or possessed marijuana or a SPED, he shall enter the Program and suffer the applicable consequences set forth in Sections 8 or 9 below, as the case may be.
\end{enumerate}

\hypertarget{random-testing.}{%
\section{Random Testing.}\label{random-testing.}}

\begin{enumerate}
\def\labelenumi{(\alph{enumi})}
\item
  In addition to the testing procedures set forth in Section 5 above, a player shall be required to undergo testing for Prohibited Substances at any time, without prior notice to the player, no more than four (4) times each Season. The scheduling of testing and collection of urine samples will be conducted according to a random player selection procedure by a third-party organization, and neither the NBA, the Players Association, any Team or any player will have any involvement in selecting the players to be tested or will receive prior notice of the testing schedule; provided, however, that it shall not be a violation of the foregoing for the third-party organization (or a specimen collector for the same) to provide advance notice of a scheduled collection to an NBA Team Security Representative, so long as such notice does not identify the player(s) who will be tested and seeks merely to facilitate access of the collector to the testing location.
\item
  \begin{enumerate}
  \def\labelenumii{(\roman{enumii})}
  \tightlist
  \item
    In the event that a First-Year Player who tests positive for a Drug of Abuse pursuant to this Section 6, he shall immediately be dismissed and disqualified from any association with the NBA or its Teams for a period of one (1) year, his Player Contract shall be rendered null and void and of no further force or effect (subject to the provisions of Paragraph 8 of the Uniform Player Contract), and he shall enter Stage 1 of the Drugs of Abuse Program. Such dismissal and disqualification shall be mandatory and may not be rescinded or reduced by the player's Team or the NBA.
  \item
    During any period while a First-Year Player is dismissed and disqualified from the NBA under Section 6(b)(i) above, and so long as such player is in compliance with his in-patient or aftercare obligations under the Program (as determined by the Medical Director), he shall receive from his Team a reasonable and necessary living expense stipend to be agreed upon by the NBA and the Players Association which (A) shall not exceed twenty-five percent (25\%) of the Salary that the player would otherwise have been entitled to earn for the period of his dismissal and disqualification and (B) shall not be payable for more than one (1) year from the date of such dismissal and disqualification.
  \item
    Any First-Year Player who tests positive for marijuana or a SPED pursuant to this Section 6, shall suffer the applicable consequences set forth in Sections 8 or 9 below, as the case may be. Any First-Year Player who tests positive for a Diuretic pursuant to this Section 6, shall suffer the applicable consequences set forth in Section 9 below.
  \end{enumerate}
\item
  In the event that a Veteran Player tests positive for a Drug of Abuse pursuant to this Section 6, he shall immediately be dismissed and disqualified from any association with the NBA or any of its Teams in accordance with the provisions of Section 11(a) below. If the player tests positive for marijuana or a SPED pursuant to this Section 6, he shall enter the Program and suffer the applicable consequences set forth in Sections 8 or 9 below, as the case may be. If the player tests positive for a Diuretic pursuant to this Section 6, he shall enter the SPED Program and suffer the applicable consequences set forth in Section 9 below.
\item
  In the event that any player tests ``positive'' pursuant to Section 4(c)(iii), (iv) or (v) above in connection with testing conducted pursuant to this Section 6, that positive test result shall be considered a positive test result for a Drug of Abuse, and the player shall immediately be dismissed and disqualified from any association with the NBA or any of its Teams in accordance with the provisions of Section 11(a) below.
\end{enumerate}

\hypertarget{drugs-of-abuse-program.}{%
\section{Drugs of Abuse Program.}\label{drugs-of-abuse-program.}}

\begin{enumerate}
\def\labelenumi{(\alph{enumi})}
\item
  \textbf{Voluntary Entry.}

  \begin{enumerate}
  \def\labelenumii{(\roman{enumii})}
  \tightlist
  \item
    A player may enter the Drugs of Abuse Program voluntarily at any time by Coming Forward Voluntarily for a problem involving the use of a Drug of Abuse; provided, however, that a player may not Come Forward Voluntarily (A) until he has been selected in an NBA Draft or invited to an NBA training camp; (B) during any period in which an Authorization for Testing as to that player remains in effect pursuant to Section 5 above; (C) during any period in which he remains subject to in-patient or aftercare treatment in Stage 1 of the Drugs of Abuse Program; or (D) after he has reached Stage 2 of the Drugs of Abuse Program.
  \item
    If a player who has not previously entered the Drugs of Abuse Program Comes Forward Voluntarily for a problem involving the use of a Drug of Abuse, he shall enter Stage 1 of the Drugs of Abuse Program.
  \item
    If a player who has not previously entered Stage 2 of the Drugs of Abuse Program, but who has been notified by the Medical Director that he has successfully completed Stage 1 of that Program, Comes Forward Voluntarily for a problem involving the use of a Drug of Abuse, he shall enter Stage 2 of the Drugs of Abuse Program.
  \item
    No penalty of any kind will be imposed on a player as a result of having Come Forward Voluntarily for a problem involving the use of a Drug of Abuse. The foregoing sentence shall not preclude the imposition of a penalty under Section 7(c)(iv) below as a result of the player's entering Stage 2 of the Drugs of Abuse Program, or any penalty called for by this Article XXXIII as a result of conduct by the player that occurs after he has Come Forward Voluntarily.
  \end{enumerate}
\item
  \textbf{Stage 1.}

  \begin{enumerate}
  \def\labelenumii{(\roman{enumii})}
  \tightlist
  \item
    Any player who has entered Stage 1 of the Drugs of Abuse Program shall be required to submit to an evaluation by the Medical Director, provide (or cause to be provided) to the Medical Director such relevant medical and treatment records as the Medical Director may request, and commence the treatment and testing program prescribed by the Medical Director.
  \item
    If a player, within ten (10) days of the date on which he was notified that he had entered Stage 1 of the Drugs of Abuse Program and without a reasonable excuse, fails to comply (in the professional judgment of the Medical Director) with any of the obligations set forth in Section 7(b)(i) above, he shall be suspended until such time as the Medical Director determines that he has fully complied with Section 7(b)(i) above. If such noncompliance continues without a reasonable excuse (in the professional judgment of the Medical Director) for thirty (30) days from the date on which the Player was notified that he had entered Stage 1 of the Drugs of Abuse Program, the player shall, following notice of the player's non-compliance by the Medical Director to the NBA and then by the NBA to the player's Team (notwithstanding the provisions of Section 3 above), (A) advance to Stage 2 of the Drugs of Abuse Program, or (B) the player's Team may, notwithstanding any term or provision in or amendment to the player's Uniform Player Contract, elect to terminate such Contract without any further obligation to pay Compensation, except to pay the Compensation (either Current or Deferred) that may have been earned by the player to the date of termination.
  \item
    Except as provided in this Article XXXIII, no penalty of any kind will be imposed on a player while he is in Stage 1 of the Drugs of Abuse Program and, provided he complies with the terms of his prescribed treatment, he will continue to receive his Compensation during the term of his treatment for a period of up to three (3) months of care in an In-Patient Facility and such aftercare as may be required by the Medical Director.
  \end{enumerate}
\item
  \textbf{Stage 2.}

  \begin{enumerate}
  \def\labelenumii{(\roman{enumii})}
  \tightlist
  \item
    Any player who has entered Stage 2 of the Drugs of Abuse Program shall be required to submit to an evaluation by the Medical Director, provide (or cause to be provided) to the Medical Director such relevant medical and treatment records as the Medical Director may request, and commence the treatment and testing program prescribed by the Medical Director.
  \item
    If a player, within thirty (30) days of the date on which he was notified that he had entered Stage 2 of the Drugs of Abuse Program and without a reasonable excuse, fails to comply (in the professional judgment of the Medical Director) with any of the obligations set forth in Section 7(c)(i) above, he shall immediately be dismissed and disqualified from any association with the NBA or any of its Teams in accordance with the provisions of Section 11(a) below.
  \item
    A player in Stage 2 of the Drugs of Abuse Program shall be suspended during the period of his in-patient treatment and for at least the first six (6) months of his aftercare treatment. The player shall remain suspended during any subsequent period in which he is undergoing treatment that, in the professional judgment of the Medical Director, prevents him from rendering the playing services called for by his Uniform Player Contract.
  \item
    Any subsequent use, possession, or distribution of a Drug of Abuse by a player in Stage 2, even if voluntarily disclosed, or any conduct by a player in Stage 2 that results in his advancing one Stage in the Drugs of Abuse Program, shall result in the player being immediately dismissed and disqualified from any association with the NBA or any of its Teams in accordance with the provisions of Section 11(a) below.
  \end{enumerate}
\item
  \textbf{Treatment and Testing Program.}

  A player who enters the Drugs of Abuse Program shall be required to comply with such in-patient and aftercare program as may be prescribed and supplemented from time to time by the Medical Director. Such program may include random testing for Prohibited Substances other than SPEDs, and for alcohol, and such non-testing elements as may be determined in the professional judgment of the Medical Director.
\end{enumerate}

\hypertarget{marijuana-program.}{%
\section{Marijuana Program.}\label{marijuana-program.}}

\begin{enumerate}
\def\labelenumi{(\alph{enumi})}
\item
  \textbf{Voluntary Entry.}

  \begin{enumerate}
  \def\labelenumii{(\roman{enumii})}
  \tightlist
  \item
    A player may enter the Marijuana Program voluntarily at any time by Coming Forward Voluntarily; provided, however, that a player may not Come Forward Voluntarily for a problem involving the use of marijuana (A) until he has been selected in an NBA Draft or invited to an NBA training camp; (B) during any period in which an Authorization for Testing as to that player remains in effect pursuant to Section 5 above; or (C) during any period in which he remains subject to in-patient or aftercare treatment in the Marijuana Program.
  \item
    If a player who has not previously entered the Marijuana Program, or a player who has been notified by the Medical Director that he has successfully completed that Program, Comes Forward Voluntarily for a problem involving the use of marijuana, he shall enter the Marijuana Program.
  \item
    No penalty of any kind will be imposed on a player as a result of having Come Forward Voluntarily for a problem involving the use of marijuana. The foregoing sentence shall not preclude the imposition of any penalty called for by this Article XXXIII as a result of conduct by the player that occurs after he has Come Forward Voluntarily.
  \end{enumerate}
\item
  \textbf{Treatment.}

  \begin{enumerate}
  \def\labelenumii{(\roman{enumii})}
  \tightlist
  \item
    A player who enters the Marijuana Program shall be required to submit to an evaluation by the Medical Director, provide (or cause to be provided) to the Medical Director such relevant medical and treatment records as the Medical Director may request, and commence the treatment and testing program prescribed by the Medical Director. Such program may include random testing for Prohibited Substances other than SPEDs, and for alcohol, and such non-testing elements as may be determined in the professional judgment of the Medical Director.
  \item
    If a player, within five (5) days of the date on which he was notified that he had entered the Marijuana Program and without a reasonable excuse, fails to comply (in the professional judgment of the Medical Director) with any of the obligations set forth in the first sentence of Section 8(b)(i) above, he shall be fined \$10,000; if the player thereafter fails to comply, without a reasonable excuse, with such obligations (in the professional judgment of the Medical Director) within eight (8) days of such notification, he shall be fined an additional \$10,000; and for each additional day beyond the 8th day that the player, without a reasonable excuse, fails to comply with such obligations (in the professional judgment of the Medical Director), he shall be fined an additional \$10,000. The total amount of such fines may not exceed the player's total Compensation.
  \end{enumerate}
\item
  \textbf{Penalties.}

  Any player who (i) tests positive for marijuana pursuant to Section 5 (Reasonable Cause Testing), Section 6 (Random Testing), or Section 14 (Additional Bases for Testing), (ii) is adjudged by the Grievance Arbitrator pursuant to Section 5(e) above to have used or possessed marijuana, or (iii) has been convicted of (including a plea of guilty, no contest or nolo contendere to) the use or possession of marijuana in violation of the law, shall suffer the following penalties:

  \begin{enumerate}
  \def\labelenumii{(\Alph{enumii})}
  \tightlist
  \item
    For the first such violation, the player shall be required to enter the Marijuana Program;
  \item
    For the second such violation, the player shall be fined \$25,000 and, if the player is not then subject to in-patient or aftercare treatment in the Marijuana Program, be required to enter the Marijuana Program;
  \item
    For the third such violation, the player shall be suspended for five (5) games and, if the player is not then subject to in-patient oraftercare treatment in the Marijuana Program, be required to enter the Marijuana Program; and
  \item
    For any subsequent violation, the player shall be suspended for five (5) games longer than his immediately preceding suspension for violating the Marijuana Program and, if the player is not then subject to in-patient or aftercare treatment in the Marijuana Program, be required to enter the Marijuana Program.
  \end{enumerate}
\end{enumerate}

\hypertarget{steroids-performance-enhancing-drugs-and-masking-agents-program.}{%
\section{Steroids, Performance-Enhancing Drugs and Masking Agents Program.}\label{steroids-performance-enhancing-drugs-and-masking-agents-program.}}

\begin{enumerate}
\def\labelenumi{(\alph{enumi})}
\item
  \textbf{Voluntary Entry.}

  \begin{enumerate}
  \def\labelenumii{(\roman{enumii})}
  \tightlist
  \item
    A player may enter the SPED Program voluntarily at any time by Coming Forward Voluntarily; provided, however, that a player may not Come Forward Voluntarily for a problem involving the use of a SPED (A) until he has been selected in an NBA Draft or invited to an NBA training camp; (B) during any period in which an Authorization for testing as to that player remains in effect pursuant to Section 5 above; or (C) during any period in which he remains subject to in-patient or aftercare treatment in the SPED Program.
  \item
    If a player who has not previously entered the SPED Program, Comes Forward Voluntarily for a problem involving the use of a SPED, he shall enter the SPED Program.(iii) No penalty of any kind will be imposed on a player as a result of having Come Forward Voluntarily for a problem involving the use of a SPED. The foregoing sentence shall not preclude the imposition of any penalty called for by this Article XXXIII as a result of conduct by the player that occurs after he has Come Forward Voluntarily.
  \end{enumerate}
\item
  \textbf{Treatment.}

  \begin{enumerate}
  \def\labelenumii{(\roman{enumii})}
  \tightlist
  \item
    A player who enters the SPED Program shall be required to submit to an evaluation by the Medical Director, provide (or cause to be provided) to the Medical Director such relevant medical and treatment records as the Medical Director may request, and commence the treatment and testing program prescribed by the Medical Director. Such program may include random testing for SPEDs and such non-testing elements as may be determined in the professional judgment of the Medical Director.
  \item
    If a player, within five (5) days of the date on which he was notified that he had entered the SPED Program and without a reasonable excuse, fails to comply (in the professional judgment of the Medical Director) with any of the obligations set forth in the first sentence of Section 9(b)(i) above, he shall be fined \$10,000; if the player, without a reasonable excuse, thereafter fails to comply with such obligations (in the professional judgment of the Medical Director) within eight (8) days of such notification, he shall be fined an additional \$10,000; and for each additional day beyond the 8th day that the player, without a reasonable excuse, fails to comply with such obligations (in the professional judgment of the Medical Director), he shall be fined an additional \$10,000. The total amount of such fines shall not exceed the player's total Compensation.
  \end{enumerate}
\item
  \textbf{Penalties.}

  Any player who (i) tests positive for a SPED pursuant to Section 5 (Reasonable Cause Testing), Section 6 (Random Testing), or Section 14 (Additional Bases for Testing), or (ii) is adjudged by the Grievance Arbitrator pursuant to Section 5(e) above to have used or possessed a SPED, shall suffer the following penalties:

  \begin{enumerate}
  \def\labelenumii{(\Alph{enumii})}
  \tightlist
  \item
    For the first such violation, the player shall be suspended for ten (10) games and required to enter the SPED Program;
  \item
    For the second such violation, the player shall be suspended for twenty-five (25) games and, if the player is not then subject to in-patient or aftercare treatment in the SPED Program, be required to enter the SPED Program;
  \item
    for the third such violation, the player shall be suspended for one (1) year from the date of such violation and, if the player is not then subject to in-patient or aftercare treatment in the SPED Program, be required to enter the SPED Program; and
  \item
    for the fourth such violation, the player shall be immediately dismissed and disqualified from any association with the NBA or any of its Teams in accordance with the provisions of Section 11(a) below.
  \end{enumerate}
\end{enumerate}

\hypertarget{noncompliance-with-treatment.}{%
\section{Noncompliance with Treatment.}\label{noncompliance-with-treatment.}}

\begin{enumerate}
\def\labelenumi{(\alph{enumi})}
\item
  \textbf{Drugs of Abuse.}

  \begin{enumerate}
  \def\labelenumii{(\roman{enumii})}
  \tightlist
  \item
    Any player who, after entering Stage 1 or Stage 2 of the Drugs of Abuse Program, fails to comply with his treatment or his aftercare program as prescribed and determined by the Medical Director, shall be suspended. Such suspension shall continue until the player has, in the professional judgment of the Medical Director, resumed full compliance with his treatment program.
  \item
    Notwithstanding Section 10(a) above, any player who in the professional judgment of the Medical Director, after entering Stage 1 or Stage 2 of the Drugs of Abuse Program, fails to comply with his treatment program through (A) a pattern of behavior that demonstrates a mindful disregard for his treatment responsibilities, or (B) a positive test for a Prohibited Substance other than a SPED that is not clinically expected by the Medical Director, shall suffer the following penalties:
    (1) if the player is in Stage 1 of the Drugs of Abuse Program, he shall advance to Stage 2 and be suspended until, in the professional judgment of the Medical Director, he has resumed full compliance with his treatment program; or
    (2) if the player already is in Stage 2 of the Drugs of Abuse Program, he shall immediately be dismissed and disqualified from any association with the NBA or any of its Teams in accordance with the provisions of Section 11(a) below.
  \end{enumerate}
\item
  \textbf{Marijuana.}

  \begin{enumerate}
  \def\labelenumii{(\roman{enumii})}
  \tightlist
  \item
    Any player who, after entering the Marijuana Program, fails to comply (without a reasonable excuse) with his treatment program as prescribed and determined by the Medical Director, shall be fined \$5,000 for each day that he fails to comply. Such fines shall continue until the player has, in the professional judgment of the Medical Director, resumed full compliance with his treatment program. The total amount of such fines shall not exceed the player's total Compensation.
  \item
    Notwithstanding Section 10(b)(i) above, any player who, after entering the Marijuana Program, fails to comply with his treatment program as prescribed and determined by the Medical Director through (A) a pattern of behavior that demonstrates a mindful disregard for his treatment responsibilities, or (B) a positive test for marijuana that is not clinically expected by the Medical Director, shall suffer the following penalties:
    (1) if the player has not previously been fined \$25,000 under Section 8(c) above or this Section 10(b)(ii), a fine of \$25,000;
    (2) if the player has previously been fined \$25,000 under Section 8(c) above or this Section 10(b)(ii), a suspension of five (5) games; or
    (3) if the player has previously been suspended for five (5) or more games under Section 8(c) above or this Section 10(b)(ii), a suspension that is at least five (5) games longer than his immediately-preceding suspension and that shall continue until, in the professional judgment of the Medical Director, the player resumes full compliance with his treatment program.
  \item
    In addition to any consequence to the player under Section 10(b)(ii) above, any player who has entered the Marijuana Program but not the Drugs of Abuse Program, and tests positive for a Drug of Abuse in any test conducted by the Medical Director, shall enter Stage 1 of the Drugs of Abuse Program.
  \end{enumerate}
\item
  \textbf{SPEDs.}

  \begin{enumerate}
  \def\labelenumii{(\roman{enumii})}
  \tightlist
  \item
    Any player who, after entering the SPED Program, fails to comply (without a reasonable excuse) with his treatment program as prescribed and determined by the Medical Director, shall be fined \$5,000 per day for each day that he fails to comply. Such fines shall continue until the player has, in the professional judgment of the Medical Director, resumed full compliance with his treatment program. The total amount of such fines shall not exceed the player's total Compensation.
  \item
    Notwithstanding Section 10(c)(i) above, any player who, after entering the SPED Program, fails to comply with his treatment program as prescribed and determined by the Medical Director through (A) a pattern of behavior that demonstrates a mindful disregard for his treatment responsibilities, or (B) a positive test for a SPED that is not clinically expected by the Medical Director, shall suffer the following penalties:
    (1) if the player has not previously been suspended for ten (10) games under Section 9(c) above or this Section 10(c)(ii), a suspension of ten (10) games;
    (2) if the player has previously been suspended for ten (10) games under Section 9(c) above or this Section 10(c)(ii), a suspension of twenty-five (25) games;
    (3) if the player has previously been suspended for twenty-five (25) games under Section 9(c) above or this Section 10(c)(ii), a suspension of one (1) year from the date of such violation; or
    (4) if the player has been previously suspended for one (1) year under Section 9(c) above or this Section 10(c)(ii), the player shall be immediately dismissed and disqualified from any association with the NBA or any of its Teams in accordance with the provisions of Section 11(a) below.
  \end{enumerate}
\item
  \textbf{Directed Testing.}

  Any player who, after entering the Program, and without a reasonable explanation satisfactory to the Medical Director, (i) fails to appear for any of his Team's scheduled games, or (ii) misses, during any consecutive seven-day (7) period, any two (2) airplane flights on which his team is scheduled to travel, any two (2) Team practices, or a combination of any one (1) practice and any one (1) Team flight, shall immediately submit to a urine test to be conducted by the NBA. If any test conducted pursuant to this Section 10(d) is positive: (x) for a Drug of Abuse or pursuant to Section 4(c)(iii), (iv) or (v) above (for a player in the Drugs of Abuse Program), then the player shall suffer the applicable consequence set forth in Section 10(a)(ii) above; (y) for marijuana or pursuant to Section 4(c)(iii), (iv) or (v) above (for a player in the Marijuana Program), then the player shall suffer the applicable consequence set forth in Section 10(b)(ii) above; or (z) for a SPED or pursuant to Section 4(c)(iii), (iv) or (v) above (for a player in the SPED Program), then the player will suffer the applicable consequence set forth in Section 10(c)(ii) above. If any test conducted pursuant to this Section 10(d) is positive for a Diuretic, then the player shall suffer the applicable consequences of a positive test for the Prohibited Substance for which he entered the Program.
\end{enumerate}

\hypertarget{dismissal-and-disqualification.}{%
\section{Dismissal and Disqualification.}\label{dismissal-and-disqualification.}}

\begin{enumerate}
\def\labelenumi{(\alph{enumi})}
\tightlist
\item
  A player who, under the terms of this Agreement, is ``dismissed and disqualified from any association with the NBA or any of its Teams in accordance with the provisions of Section 11(a)'' shall, without exception, immediately be so dismissed and disqualified for a period of not less than two (2) years, and such player's Player Contract shall be rendered null and void and of no further force or effect (subject to the provisions of paragraph 8 of the Uniform Player Contract). Such dismissal and disqualification shall be mandatory and may not be rescinded or reduced by the player's Team or the NBA.
\item
  In addition to any other provision of this Agreement requiring that a player be dismissed and disqualified from any association with the NBA or any of its Teams in accordance with the provisions of Section 11(a) above, a player will also be dismissed and disqualified under Section 11(a) above if he is convicted of (including a plea of guilty, no contest, or nolo contendere to) a crime involving the use or possession of a Prohibited Substance other than marijuana.
\end{enumerate}

\hypertarget{reinstatement.}{%
\section{Reinstatement.}\label{reinstatement.}}

\begin{enumerate}
\def\labelenumi{(\alph{enumi})}
\item
  After a period of at least two (2) years from the time of a player's dismissal and disqualification under Section 11(a) above, and after a period of at least one (1) year from the date of a First-Year Player's dismissal and disqualification under Section 6(b) above, such player may apply for reinstatement as a player in the NBA. However, such player shall have no right to reinstatement under any circumstance and the reinstatement shall be granted only with the prior approval of both the NBA and the Players Association, which shall not be unreasonably withheld. The approval of the NBA and the Players Association shall rest in their absolute and sole discretion, and their decision shall be final, binding, and unappealable. Among the factors that may be considered by the NBA and the Players Association in determining whether to grant reinstatement are (without limitation): the circumstances surrounding the player's dismissal and disqualification; whether the player has satisfactorily completed a treatment and rehabilitation program; the player's conduct since his dismissal, including the extent to which the player has since comported himself as a suitable role model for youth; and whether the player is judged to possess the requisite qualities of good character and morality.
\item
  For a First-Year Player, the NBA and the Players Association will consider an application for reinstatement only if the player has, in the opinion of the Medical Director, successfully completed any in-patient treatment and/or aftercare prescribed by the Medical Director. For a Veteran Player, the NBA and the Players Association will consider any application for reinstatement only if the player can demonstrate, by proof of random urine testing acceptable to the Medical Director (conducted on at least a weekly basis), that he has not tested positive (i) for a Prohibited Substance within the twelve (12) months prior to the submission of his application for reinstatement and during any period while his application is being reviewed, and (ii) if the Medical Director deems it necessary in his or her professional judgment, for alcohol for the six (6) months prior to the submission of his application for reinstatement and during any period while his application is being reviewed.
\item
  The granting of an application for reinstatement may be conditioned upon random testing of the player or such other terms as may be agreed upon by the NBA and the Players Association, whether or not such terms are contemplated by the terms of this Article XXXIII.
\item
  In the event that the application for reinstatement of a First-Year Player dismissed and disqualified pursuant to Section 6(b) above is approved, such player, by reason of his Player Contract having been rendered null and void pursuant to Section 6(b) above, shall be deemed not to have completed his Player Contract by rendering the playing services called for thereunder. Accordingly, such player shall not be a Free Agent and shall not be entitled to negotiate or sign a Player Contract with any NBA Team, except as specifically provided in this Section 12.
\item
  \begin{enumerate}
  \def\labelenumii{(\roman{enumii})}
  \tightlist
  \item
    A First-Year Player who has been reinstated pursuant to this Section 12 shall, immediately upon such reinstatement, notify the Team to which he was under contract at the time of his dismissal and disqualification (the ``previous Team''). Upon receipt of such notification, and subject to Section 12(e)(ii) below, the previous Team shall then have thirty (30) days in which to make a Tender to the player with a stated term of at least one (1) full NBA Season (or, in the event that the Tender is made during a Season, of at least the remainder of that Season) and calling for at least the Minimum Player Salary then applicable to that player but not more than the Salary provided for in Section 12(e)(ii) below. If the previous Team makes such a Tender, it shall, for a period of one (1) year from the date of the Tender, be the only NBA Team with which the player may negotiate and sign a Player Contract. If the player does not sign a Player Contract with the previous Team within the year following such Tender, the player shall thereupon be deemed a Restricted Free Agent, subject to a Right of First Refusal. If the previous Team fails to make a Tender, the player shall become an Unrestricted Free Agent.
  \item
    Notwithstanding anything to the contrary in Section 12(e)(i) above, the 30-day period for the previous Team to make a Tender shall be tolled if (A) on the date the player serves the notice required by Section 12(e)(i), he is under contract to a professional basketball team not in the NBA, or (B) the player signs a contract with a professional basketball team not in the NBA at any point after the date on which the player serves the notice required by Section 12(e)(i) and before the \textbf{\emph{{[}sic{]}}}
  \item
    A First-Year Player who is reinstated pursuant to this Section 12 may enter into a Player Contract with his previous Team that provides for a Salary and Unlikely Bonuses for the first Season of up to the Player's Salary and Unlikely Bonuses, respectively, for the Salary Cap Year in which he was dismissed and disqualified (reduced on a pro rata basis if the first Season of the new Contract is a partial Season), even if the Team has a Team Salary at or above the Salary Cap or such Player Contract causes the Team to have a Team Salary above the Salary Cap. If the player and the previous Team enter into such Player Contract and such Contract covers more than one Season, increases and decreases in Salary for Seasons following the first Season shall be governed by Article VII, Section 5(c)(1); provided, however, that if the player who is reinstated was dismissed and disqualified during the term of his Rookie Scale Contract, then (A) the number of Seasons in the player's new Contract may not exceed two (2) Seasons plus two(2) Option Years in favor of the Team, and the Salary and Unlikely Bonuses called for in any Season of the player's new Contract, including any Option Year, may not exceed the Salary and Unlikely Bonuses called for during the corresponding Season of his Rookie Scale Contract, and (B) if the new Contract contains terms identical to those contained in the remaining Seasons of the Player's Rookie Scale Contract at the time he was dismissed and disqualified, and the Team exercises all Option Year(s) available under the new Contract, then the player's Team shall retain the same rights with respect to such new Contract as it would have retained under Article XI following the completion of the player's Rookie Scale Contract.
  \end{enumerate}
\item
  \begin{enumerate}
  \def\labelenumii{(\roman{enumii})}
  \tightlist
  \item
    A Veteran Player who has been reinstated pursuant to this Section 12 shall, immediately upon such reinstatement, notify the Team to which he was under contract at the time of his dismissal and disqualification (the ``previous Team''). Upon receipt of such notification, and subject to Section 12(f)(ii) below, the previous Team shall then have thirty (30) days in which to make a Tender to the player with a stated term of at least one (1) full NBA Season (or, in the event the Tender is made during a Season, of at least the rest of that Season) and calling for a Salary in the first Season covered by the Tender at least equal to the lesser of (A) the player's Salary for the Salary Cap Year in which he was dismissed and disqualified, or (B) the Estimated Average Player Salary during the then-current Season, in either case reduced on a pro rata basis if the first Season covered by the Tender is a partial Season, but not greater than the Salary provided in Section 12(f)(iii) below. If the previous Team makes such a Tender, it shall, for a period of one (1) year from the date of the Tender, be the only NBA Team with which the player may negotiate and sign a Player Contract. If the player does not sign a Player Contract with the previous Team within the year following such Tender, then the player shall thereupon be deemed a Restricted or an Unrestricted Free Agent, in accordance with the provisions of Article XI. If the previous Team fails to make a Required Tender, the player shall become an Unrestricted Free Agent.
  \item
    Notwithstanding anything to the contrary in Section 12(f)(i) above, the 30-day period for the previous Team to make a Tender shall be tolled if (A) on the date the player serves the notice required by Section 12(f)(i), he is under contract to a professional basketball team not in the NBA, or (B) the player signs a contract with a professional basketball team not in the NBA at any point after the date on which he serves the notice required by Section 12(f)(i) and before the date on which the previous Team makes a Tender. If the 30-day period for making a Tender is tolled pursuant to the preceding sentence, the period shall remain tolled until the date on which the player notifies the Team that he is available to sign a Player Contract with and begin rendering playing services for such Team immediately, provided that such notice will not be effective until the player is under no contractual or other legal impediment to sign with and begin rendering playing services for such Team.
  \item
    A Veteran Player who is reinstated pursuant to this Section 12 may enter into a Player Contract with his previous Team that provides for a Salary and Unlikely Bonuses for the first Season of up to the player's Salary and Unlikely Bonuses, respectively, for the Salary Cap Year in which he was dismissed and disqualified (reduced on a pro rata basis if the first Season of the new Contract is a partial Season), even if the Team has a Team Salary at or above the Salary Cap or such Player Contract causes the Team to have a Team Salary above the Salary Cap. If the player and the previous Team enter into such Player Contract and such Contract covers more than one Season, increases and decreases in Salary for Seasons following the first Season shall be governed by Article VII, Section 5(c)(i); provided, however, that if the player who is reinstated was dismissed and disqualified during the term of his Rookie Scale Contract, then (A) the number of Seasons in the Player's new Contract may not exceed the number of Seasons (including the Option Year in favor of the Team) that remained under the player's Rookie Scale Contract at the time he was dismissed and disqualified, and the Salary called for in any Season of the Player's new Contract (including any Option Year), may not exceed the Salary called for during the corresponding Season of his Rookie Scale Contract, and (B) if the new Contract contains terms identical to those contained in the remaining Seasons of the player's Rookie Scale Contract at the time he was dismissed and disqualified, and the player's Team ultimately exercises the Option Year available under the new Contract, then such Team shall retain the same rights with respect to such new Contract as it would have retained under Article XI following the completion of the player's Rookie Scale Contract.
  \end{enumerate}
\end{enumerate}

\hypertarget{exclusivity-of-the-program.}{%
\section{Exclusivity of the Program.}\label{exclusivity-of-the-program.}}

\begin{enumerate}
\def\labelenumi{(\alph{enumi})}
\tightlist
\item
  Except as expressly provided in this Article XXXIII, there shall be no other screening or testing for Prohibited Substances conducted by the NBA or any Team, and no player may undergo such screening or testing; provided, however, that, in a medical emergency, team physicians may test players solely for diagnostic purposes in order to provide satisfactory medical care. The results of any diagnostic drug testing conducted pursuant to the preceding sentence shall not be used for any other purpose by the player's Team or the NBA. If any Team is found to have tested a player in violation of this Section 13, the NBA will impose a substantial fine not to exceed \$750,000 upon such Team pursuant to the NBA's Constitution and By-Laws.
\item
  The penalties set forth in this Article XXXIII shall be the exclusive penalties to be imposed upon a player for the use, possession or distribution of a Prohibited Substance.
\item
  No Uniform Player Contract entered into after the date hereof shall include any term or provision that modifies, contradicts, changes, or is inconsistent with paragraph 8 of such Contract (including any condition or limitation on salary protection, as precluded by Article II, Section 4(n)) or provides for the testing of a player for illegal substances. Any term or provision of a currently effective Uniform Player Contract that is inconsistent with paragraph 8 of such Contract shall be deemed null and void only to the extent of the inconsistency.
\end{enumerate}

\hypertarget{additional-bases-for-testing.}{%
\section{Additional Bases for Testing.}\label{additional-bases-for-testing.}}

\begin{enumerate}
\def\labelenumi{(\alph{enumi})}
\tightlist
\item
  Any player who seeks treatment outside the Program for a problem involving a Prohibited Substance shall, as directed by the NBA (after notice to the Players Association), submit himself to an evaluation by the Medical Director and provide (or cause to be provided) to the Medical Director such medical and treatment records as the Medical Director may request. The Medical Director may, in his or her professional judgment, also require such a player, without prior notice, to submit to testing for Prohibited Substances, provided that the frequency of such testing shall not exceed three (3) times per week and the duration of such testing shall not exceed one (1) year from the date of the player's initial evaluation by the Medical Director.
\item
  Any player who is a subject to in-patient or aftercare treatment in the Program and is formally charged with ``driving while intoxicated,'' ``driving under the influence of alcohol,'' or any other crime or offense involving suspected alcohol or illegal substance use shall, provided that the NBA has advised the Players Association, be required to submit to a urine test, to be conducted by the NBA, within seven (7) days of being so charged.
\item
  If, pursuant to Section 14(a) above, a player (i) tests positive for a Drug of Abuse; (ii) tests positive pursuant to Section 4(c)(iii), (iv) or (v) above; or (iii) refuses or fails to submit to an evaluation or provide (or cause to be provided) the information requested by the Medical Director, but does not Come Forward Voluntarily within 60 days of being requested to do so by the NBA (with notice to the Players Association), or if, pursuant to Section 14(b) above, a player tests positive for a Drug of Abuse, then, in either case, the player shall advance two stages in the Drugs of Abuse Program---i.e., the player shall enter Stage 2 of the Drugs of Abuse Program (if the player had not previously entered Stage 1 of such Program), and the player shall be dismissed and disqualified from any association with the NBA or any of its Teams in accordance with the provisions of Section 11(a) above (if the player had previously entered Stage 1 or Stage 2 of such Program).
\item
  If, pursuant to Section 14(a) or (b) above, a player tests positive for marijuana or a SPED, he shall suffer the applicable consequences set forth in Sections 8 or 9 above, as the case may be. If, pursuant to Section 14 (a) or (b) above, a player tests positive for a Diuretic, he shall suffer the applicable consequences set forth in Section 9 above.
\item
  If a player is or, within the previous six (6) months, has been in possession of any device or product used or designed for substituting, diluting, or adulterating a specimen sample, that player shall be required to undergo testing for Prohibited Substances no more than four (4) times during the six-week period following his notification by the NBA of the commencement of such testing. If the player tests positive for a Drug of Abuse, he shall be dismissed and disqualified from any association with the NBA or any of its Teams in accordance with the provisions of Section 11(a) above. If the player tests positive for marijuana or a SPED, he shall suffer the applicable consequences set forth in Sections 8 or 9 above, as the case may be. If the player tests positive for a Diuretic, he shall suffer the applicable consequences set forth in Section 9, above.
\item
  Nothing in this Section 14 shall limit or otherwise affect any of the provisions of Section 5 (Reasonable Cause Testing).
\end{enumerate}

\hypertarget{additional-prohibited-substances.}{%
\section{Additional Prohibited Substances.}\label{additional-prohibited-substances.}}

\begin{enumerate}
\def\labelenumi{(\alph{enumi})}
\tightlist
\item
  Any steroid or performance-enhancing drug that is declared illegal during the term of this Agreement will automatically be added to the list of Prohibited Substances as a SPED.
\item
  At any time during the term of this Agreement, either the NBA or the Players Association may convene a meeting of the Prohibited Substances Committee to request that a substance or substances be added to the list of Prohibited Substances set forth on Exhibit I-2 to this Agreement. Any such addition of a Prohibited Substance may only include a substance that is or is reasonably likely to be physically harmful to Players and is or is reasonably likely to be improperly performance-enhancing. The determination of the Committee to add to the list of Prohibited Substances shall be made by a majority vote of all five Committee members, and shall be final, binding, and unappealable.
\item
  Players will receive notice of any addition to the list of Prohibited Substances six (6) months prior to the date on which such addition becomes effective under this Article XXXIII.
\end{enumerate}

\hypertarget{recognition-clause}{%
\chapter{RECOGNITION CLAUSE}\label{recognition-clause}}

The NBA recognizes the Players Association as the exclusive collective bargaining representative of persons who are employed by NBA Teams as professional basketball players (and/or who may become so employed during the term of this Agreement or any extension thereof); and the Players Association warrants that it is duly empowered to enter into this Agreement for and on behalf of such persons. The NBA and the Players Association agree that, notwithstanding the foregoing, such persons and NBA Teams may, on an individual basis, bargain with respect to and agree upon the provisions of Player Contracts, but only as and to the extent permitted by this Agreement.

\hypertarget{savings-clause}{%
\chapter{SAVINGS CLAUSE}\label{savings-clause}}

In the event that any provision hereof is found to be inconsistent with the Internal Revenue Code of 1986, as amended (or the rules and regulations issued thereunder (the ``Code'')), the National Labor Relations Act, any other federal, state, provincial, or local statute or ordinance, or the rules and regulations of any other government agency, or is determined to have an adverse effect upon the right of the NBA (or any successor entity) to a tax exemption under Section 501(c)(6) of the Code (or any successor section of like import), then the parties hereto agree to make such changes as are necessary to avoid such inconsistency or to obtain or maintain such exemption retaining, to the extent possible, the intention of such provision.

\hypertarget{player-agents}{%
\chapter{PLAYER AGENTS}\label{player-agents}}

\hypertarget{approval-of-player-contracts.}{%
\section{Approval of Player Contracts.}\label{approval-of-player-contracts.}}

The NBA shall not approve any Player Contract between a player and a Team unless such player: (a) is represented in the negotiations with respect to such Player Contract by an agent or representative duly certified by the Players Association in accordance with the Players Association's Agent Regulation Program and authorized to represent him; or (b) acts on his own behalf in negotiating such Player Contract.

\hypertarget{fines.}{%
\section{Fines.}\label{fines.}}

The NBA shall impose a fine of \$20,000 upon any Team that negotiates a Player Contract with an agent or representative not certified by the Players Association in accordance with the Players Association's Agent Regulation Program if, at the time of such negotiations, such Team either (a) knows that such agent or representative has not been so certified or (b) fails to make reasonable inquiry of the NBA as to whether such agent or representative has been so certified. Notwithstanding the preceding sentence, in no event shall any Team be subject to a fine if the Team negotiates a Player Contract with an agent or representative designated as the player's authorized agent on the then-current agent list provided by the Players Association to the NBA in accordance with Section 4 below.

\hypertarget{indemnity.}{%
\section{Indemnity.}\label{indemnity.}}

The Players Association agrees to indemnify and hold harmless the NBA, its Teams and each of its and their respective past, present and future owners (direct and indirect) acting in their capacity as Team owners, officers, directors, trustees, employees, successors, agents, attorneys, heirs, administrators, executors and assigns, from any and all claims of any kind arising from or relating to (a) the Players Association's Agent Regulation Program, and (b) the provisions of this Article, including, without limitation, any judgments, costs and settlements, provided that the Players Association is immediately notified of such claim in writing (and, in no event later than five (5) days from the receipt thereof), is given the opportunity to assume the defense thereof, and the NBA and/or its Teams (whichever is sued) use their best efforts to defend such claim, and do not admit liability with respect to and do not settle such claim without the prior written consent of the Players Association.

\hypertarget{agent-lists.}{%
\section{Agent Lists.}\label{agent-lists.}}

The Players Association agrees to provide the NBA League Office with a list of (a) all agents certified under the Players Association's Agent Regulation Program, and (b) the players represented by each such agent. Such list shall be updated once every two (2) weeks from the day after the NBA Finals to the first day of the next succeeding Regular Season and shall be updated once every month at all other times.

\hypertarget{confirmation-by-the-players-association.}{%
\section{Confirmation by the Players Association.}\label{confirmation-by-the-players-association.}}

If the NBA has reason to believe that the agent representing a player in Contract negotiations is not a certified agent or is not the agent authorized to represent the player, the NBA may, at its election, request in writing from the Players Association confirmation as to whether the agent who represented the player in the Contract negotiations is in fact the player's certified representative. If within three (3) business days of the date the Players Association receives such written request, the NBA does not receive a written response from the Players Association stating that the agent who represented the player is not the player's certified representative, then the NBA shall be free to act as if the agent is the player's confirmed certified representative.

\hypertarget{group-licensing-rights}{%
\chapter{GROUP LICENSING RIGHTS}\label{group-licensing-rights}}

\hypertarget{rights-granted.}{%
\section{Rights Granted.}\label{rights-granted.}}

The Players Association, on behalf of present and future NBA players, agrees that NBA Properties, Inc., during the term of the Agreement between NBA Properties, Inc., and the National Basketball Players Association, dated as of September 18, 1995, as amended January 20, 1999 and July 29, 2005 (the ``Group License Agreement''), has the exclusive right to use the ``Player's Attributes'' of each NBA player as such term is defined in, for such group licensing purposes as are set forth in, and in accordance with the terms of the Group License Agreement.

\hypertarget{player-appearances.}{%
\section{Player Appearances.}\label{player-appearances.}}

A player may, during each Contract Year covered by a Player Contract to which he is a party, be required (a) to make up to four (4) appearances at the request of and in connection with licensing arrangements made by NBA Properties, Inc., in accordance with the terms of the Group License Agreement, and (b) to make up to two (2) additional appearances at the request of NBA Properties in accordance with paragraph 13(d) of a Uniform Player Contract and Article II, Section 8. Any appearance that a player is required to make shall comply with the terms of Article II, Section 8, and when a player makes an appearance in accordance with this Section, he shall be paid at least \$2,500. When a player fails, without reasonable excuse, to appear or reasonably to cooperate during an appearance at any of the licensing appearances referred to in this Section, he may be fined for each failure in an amount up to \$20,000.

\hypertarget{uniform.}{%
\section{Uniform.}\label{uniform.}}

\begin{enumerate}
\def\labelenumi{(\alph{enumi})}
\tightlist
\item
  During any NBA game or practice, including warm-up periods and going to and from the locker room to the playing floor, a player shall wear only the Uniform as supplied by his Team. For purposes of the preceding sentence only, ``Uniform'' means all clothing and other items (such as kneepads, wristbands and headbands, but not including Sneakers) worn by a player during an NBA game or practice. ``Sneakers'' means athletic shoes of the type worn by players while playing an NBA game.
\item
  Other than as may be incorporated into his Uniform and the manufacturer's identification incorporated into his Sneakers, a player may not, during any NBA game, display any commercial, promotional, or charitable name, mark, logo or other identification, including but not limited to on his body, on his hair, or otherwise.
\end{enumerate}

\hypertarget{integration-entire-agreement-interpretation-choice-of-law}{%
\chapter{INTEGRATION, ENTIRE AGREEMENT, INTERPRETATION, CHOICE OF LAW}\label{integration-entire-agreement-interpretation-choice-of-law}}

\chaptermark{INTEGRATION, ENTIRE AGREEMENT \ldots}

\hypertarget{integration-entire-agreement.}{%
\section{Integration, Entire Agreement.}\label{integration-entire-agreement.}}

This Agreement, together with the exhibits hereto, constitutes the entire understanding between the parties and all understandings, conversations and communications, proposals, and counterproposals, oral and written (including any draft of this Agreement) between the Members of the NBA and the Players Association, or on behalf of them, are merged into and superseded by this Agreement and shall be of no force or effect, except as expressly provided herein. No such understandings, conversations, communications, proposals, counterproposals or drafts shall be referred to in any proceeding by the parties. Further, no understanding contained in this Agreement shall be modified, altered or amended, except by a writing signed by the party against whom enforcement is sought.

\hypertarget{interpretation.}{%
\section{Interpretation.}\label{interpretation.}}

\begin{enumerate}
\def\labelenumi{(\alph{enumi})}
\tightlist
\item
  The NBA and Players Association recognize and acknowledge that there are and may continue to be (i) a collective bargaining relationship between WNBA, LLC (``WNBA'') and the Women's National Basketball Players Association (``WNBPA'') and (ii) a business arrangement between the NBA Development League (``NBADL'') and the Players Association, each of which is separate and distinct from the collective bargaining relationship between the NBA and the Players Association.
\item
  The NBA and the Players Association agree that this Agreement shall be interpreted without reference: (i) to any past, present or future WNBA/WNBPA collective bargaining agreement (or to any other past, present or future agreement between the WNBA or WNBA Enterprises, LLC, on the one hand, and the WNBPA on the other) or to any past, present, or future Standard Player Contract, Team Marketing and Promotional Agreement, or WNBA Marketing and Promotional Agreement (collectively, ``WNBA Agreements''); (ii) to any past, present or future agreement between the NBADL and the Players Association; (iii) to any of the provisions of such agreements or contracts; (iv) to the fact that a subject was not or is not covered by or included in any such agreements or contracts; and/or (v) to any judicial, arbitral, or administrative decision interpreting any of such agreements or contracts.
\item
  The parties agree that they will make no reference to any of the WNBA Agreements, NBADL/Players Association agreements, contracts or decisions referred to in Section 2(b) above, or to the fact that a particular provision was not or is not included in any such agreement or contract, or to any practice or policy of the WNBA (or WNBA Enterprises, LLC), the NBADL, or the WNBPA, in any arbitral, judicial, administrative, or other proceeding concerning the interpretation or enforcement of this Agreement, including, without limitation, a proceeding brought under Articles XXXI or XXXII of this Agreement. The parties further agree that no such agreement, contract, provision (or absence of provisions), decision, practice, or policy may be relied upon by any decision maker in such proceedings.
\end{enumerate}

\hypertarget{choice-of-law.}{%
\section{Choice of Law.}\label{choice-of-law.}}

This Agreement is made under and shall be governed by the internal law of the State of New York, except where federal law may govern.

\hypertarget{term-of-agreement}{%
\chapter{TERM OF AGREEMENT}\label{term-of-agreement}}

\hypertarget{expiration-date.}{%
\section{Expiration Date.}\label{expiration-date.}}

This Agreement shall be effective from July 29, 2005 and, unless extended or terminated pursuant to the provisions of this Article XXXIX, shall continue in full force and effect through June 30, 2011.

\hypertarget{nba-option-to-extend.}{%
\section{NBA Option to Extend.}\label{nba-option-to-extend.}}

The NBA shall have the option to extend this Agreement for one (1) year (i.e., through June 30, 2012) by serving written notice of its exercise of such option on the Players Association on or before December 15, 2010.

\hypertarget{termination-by-players-associationanti-collusion.}{%
\section{Termination by Players Association/Anti-Collusion.}\label{termination-by-players-associationanti-collusion.}}

\begin{enumerate}
\def\labelenumi{(\alph{enumi})}
\tightlist
\item
  In the event the conditions of Article XIV, Section 15 are satisfied, the Players Association shall have the right to terminate this Agreement by serving written notice of its exercise of such right within thirty (30) days after the System Arbitrator's report finding the requisite conditions (pursuant to Article XIV, Section 15) becomes final and any appeals therefrom have been exhausted or, in the absence of a System Arbitrator, by serving such written notice upon the NBA within thirty (30) days after any decision by a court finding the requisite conditions (pursuant to Article XIV, Section 15). In the latter situation, if the finding of the court is reversed on appeal, the Agreement shall be immediately reinstated and both parties reserve their rights with respect to any conduct by the other party during the period from the date of service of the termination notice to the date upon which the Agreement was reinstated.
\item
  If the Players Association exercises the right accorded it by Section 3(a) above, this Agreement shall terminate as of the June 30 immediately following the service of the termination notice.
\end{enumerate}

\hypertarget{termination-by-nbanational-tv-revenues.}{%
\section{Termination by NBA/National TV Revenues.}\label{termination-by-nbanational-tv-revenues.}}

\begin{enumerate}
\def\labelenumi{(\alph{enumi})}
\tightlist
\item
  For the purposes of this provision: (i) ``National TV Revenues'' shall mean the rights fees or other non-contingent payments stated in the NBA's third-party national broadcast network (e.g., ABC) and cable network (e.g., TNT or ESPN) television agreements (each, a ``National TV Agreement''); and (ii) ``Other Media Income'' shall mean the aggregate net income earned by any League-related entity (as defined in Article VII, Section 1(a)(1)) (but excluding net income attributable to ownership interests in any such League-related entity that is not owned by the NBA, NBA Properties, NBA Media Ventures, LLC and/or a group of NBA Teams) or by the NBA on behalf of the Teams from agreements that provide for the transmission of live (or delayed) NBA games, on a domestic or international basis, by means of television, radio, internet and any other mode of delivery referenced in Article VII, Section 1(a)(1)(ii), net of reasonable and customary expenses related thereto.
\item
  If, during the term of this Agreement, (i) the sum of the average annual National TV Revenues provided for under the Successor Agreements (as defined in Article VII, Section 1(c)(2)), plus 104.5\% of Other Media Income for the most recent Salary Cap Year, will be at least 35\% less than (ii) the sum of the average annual National TV Revenues provided for under the NBA/ABC and NBA/TBS Agreements (which, for purposes of this provision only, the parties agree is \$713 million), plus Other Media Income for the 2004-05 Salary Cap Year, the NBA shall have the right to terminate this Agreement effective as of the June 30 immediately preceding the first Season covered by the Successor Agreements, by providing written notice of such termination to the Players Association at least sixty (60) days prior to such June 30. During the period following delivery of such written notice of termination, the NBA and the Players Association shall engage in good faith negotiations for the purpose of entering into a successor agreement and the provisions of Article XXX shall remain in full force and effect.
\end{enumerate}

\hypertarget{termination-by-nbaforce-majeure.}{%
\section{Termination by NBA/Force Majeure.}\label{termination-by-nbaforce-majeure.}}

\begin{enumerate}
\def\labelenumi{(\alph{enumi})}
\tightlist
\item
  ``Force Majeure Event'' shall mean the occurrence of any of the following events or conditions, provided that such event or condition either (i) makes it impossible for the NBA to perform its obligations under this Agreement, or (ii) frustrates the underlying purpose of this Agreement, or (iii) makes it economically impracticable for the NBA to perform its obligations under this Agreement: wars or war-like action (whether actual or threatened and whether conventional or other, including, but not limited to, chemical or biological wars or war-like action); sabotage, terrorism or threats of sabotage or terrorism; explosions; epidemics; and any governmental order or action (civil or military); provided, however, that none of the foregoing enumerated events or conditions is within the reasonable control of the NBA or an NBA Team.
\item
  In addition to any other rights a Team or the NBA may have by contract or by law, if a Force Majeure Event occurs and the NBA, as a result, ceases or suspends its operations for a definite or indefinite period and directs that no games (whether Exhibition, Regular Season, or Playoff games) be played by NBA Teams during such period (the ``Force Majeure Period''), then, for each missed Exhibition, Regular Season, or Playoff game during the Force Majeure Period, the Compensation payable to each player who was on a Team roster at the time of the Force Majeure Event shall be reduced by 1/94.6th of the player's Compensation for the \textbf{\emph{{[}sic{]}}}
\item
  In the event that Section 5(b) above applies, the applicable Compensation reduction from each player shall be withheld by the player's Team from the first Compensation payment (or payments, if the first such payment is insufficient to satisfy the reduction) that is (or are) due or to become due to such player following the commencement of the Force Majeure Period (whether under the Player Contract that was in existence at the commencement of the Force Majeure Period or any subsequent Player Contract between the player and the Team). If such Compensation payment (or payments) is (or are) insufficient to cover the Compensation reduction required by subsection (b) above, then either (i) the player shall promptly pay the difference directly to the Team (``old Team''), or (ii) if he subsequently enters into a Player Contract with another NBA Team (``new Team''), such difference shall be withheld from the first available Compensation payment (or payments, if the first such payment is insufficient to satisfy the remaining reduction) that is (or are) due to the player from the new Team and shall be remitted by the new Team to the old Team.
\item
  Upon the occurrence of a Force Majeure Event satisfying the terms of Section 5(a) above, the NBA shall have the right to terminate this Agreement as of the sixtieth day following delivery to the Players Association of a written notice of termination, which must be delivered to the Players Association within sixty days of the Force Majeure Event. During the sixty-day period following delivery of such written notice of termination, the NBA and the Players Association shall engage in good faith negotiations for the purpose of entering into a successor agreement, and during such period the provisions of Article XXX shall remain in full force and effect.
\end{enumerate}

\hypertarget{mutual-right-of-termination.}{%
\section{Mutual Right of Termination.}\label{mutual-right-of-termination.}}

If at any time during the term of this Agreement any provision contained in Article VII, X, XI and XIV of this Agreement is enjoined, vacated, declared null and void or is rendered unenforceable by any court of competent jurisdiction, then either the NBA or the Players Association shall have the right to terminate this Agreement by serving upon the other party written notice of termination at least sixty (60) days prior to the effective date of such termination.

\hypertarget{no-obligation-to-terminate-no-waiver.}{%
\section{No Obligation to Terminate; No Waiver.}\label{no-obligation-to-terminate-no-waiver.}}

The grant to either party of a right or option to terminate pursuant to the provisions of this Article XXXIX shall not carry with it the obligation to exercise that right or option; and the failure of the NBA or the Players Association to exercise any right or option to terminate this Agreement with respect to any playing Season in accordance with this Article XXXIX shall not be deemed a waiver of or in any way impair or prejudice the NBA or the Players Association's right or option, if any, to terminate this Agreement in accordance with this Article with respect to any succeeding Season.

\hypertarget{expansion}{%
\chapter{EXPANSION}\label{expansion}}

The NBA may determine during the term of this Agreement to expand the number of Teams and to have existing Teams make available for assignment to any such Expansion Teams the Player Contracts of a certain number of Veterans under substantially the same terms and in substantially the same manner that Player Contracts were made available to the Charlotte expansion Team pursuant to the 1999 NBA/NBPA Collective Bargaining Agreement; provided, however, that any change shall be subject to the approval of the Players Association, which shall not be unreasonably withheld.

\hypertarget{player-team-cooperation}{%
\chapter{PLAYER-TEAM COOPERATION}\label{player-team-cooperation}}

\hypertarget{player-team-growth-and-development-committee.}{%
\section{Player-Team Growth and Development Committee.}\label{player-team-growth-and-development-committee.}}

\begin{enumerate}
\def\labelenumi{(\alph{enumi})}
\tightlist
\item
  A Player-Team Growth and Development Committee (hereinafter ``Development Committee'') shall be established for the purpose of examining and discussing issues relating to, among other things: (i) the implementation of this Agreement; (ii) the continued enhancement of relations between players and Teams; (iii) the promotion of the NBA, NBA players and Teams, and the game of basketball; and (iv) the future growth and success of the NBA.
\item
  The Development Committee shall consist of eight (8) members: the Commissioner of the NBA and three (3) NBA Team owners; and the Executive Director of the Players Association and three (3) NBA players. The respective owners and players who shall be members of the Development Committee will be selected, and the lengths of their terms fixed, under such rules as the NBA and the Players Association separately establish. The original members of the Committee will be selected within thirty (30) days following the execution of this Agreement. The Development Committee will hold regular face-to-face meetings at least twice each year on dates and at sites mutually agreeable to the Committee members. The meetings may be attended by staff members and advisors of the NBA and the Players Association.
\end{enumerate}

\hypertarget{nba-development-league}{%
\chapter{NBA DEVELOPMENT LEAGUE}\label{nba-development-league}}

\hypertarget{nbadl-work-assignments.}{%
\section{NBADL Work Assignments.}\label{nbadl-work-assignments.}}

\begin{enumerate}
\def\labelenumi{(\alph{enumi})}
\tightlist
\item
  An NBA Team may at any time assign a player on its Active List or Inactive List to a NBA Development League (``NBADL'') team, provided that the player has either zero (0) or one (1) Years of Service at the time of the assignment. Upon such assignment (``NBADL Work Assignment''), the player will be placed on the NBA Team's Inactive List, and shall (i) report to the NBADL team (and render for the NBADL team such services as the player is required to render for the NBA Team under his Uniform Player Contract and this Agreement), and (ii), at the direction of the NBA Team, subsequently return and report to, and resume the performance of services for, the NBA Team. An NBADL Work Assignment commences when the player reports in-person to the NBADL team, and ends either when the player, upon being recalled, reports back to his NBA Team or when the NBADL season concludes.
\item
  No player shall be given more than three (3) NBADL Work Assignments during any NBA Season.
\item
  No NBA Team shall issue an NBADL Work Assignment for the purpose of disciplining a player for misconduct or for retaliating against a player for exercising any right that he has under this Agreement or the Uniform Player Contract.
\item
  The NBA may establish reasonable rules regarding the assignment and recall of players to the NBADL provided that such rules do not violate the provisions of this Article XLII.
\end{enumerate}

\hypertarget{reporting-requirements-for-nbadl-work-assignments.}{%
\section{Reporting Requirements for NBADL Work Assignments.}\label{reporting-requirements-for-nbadl-work-assignments.}}

\begin{enumerate}
\def\labelenumi{(\alph{enumi})}
\tightlist
\item
  In order to initiate an NBADL Work Assignment or terminate such assignment and recall the player, the NBA Team shall provide the player, the NBA, and the Players Association with written notice. The player shall report to the NBADL team or NBA Team, whichever is applicable, within forty-eight (48) hours after such notice is received by the player.
\item
  If the player, without a reasonable excuse, does not report to the NBADL team or NBA Team, whichever is applicable, within the time provided in Section 2(a) above, the player may be fined and/or suspended without pay by the NBA Team until such time as he reports. In addition, such failure to report, without a reasonable excuse, shall constitute conduct prejudicial to the NBA under Article 35(e) of the NBA Constitution.
\end{enumerate}

\hypertarget{travel-and-relocation-expenses.}{%
\section{Travel and Relocation Expenses.}\label{travel-and-relocation-expenses.}}

A player's NBA Team shall be obligated to reimburse the player for his ordinary and reasonable expenses incurred in (a) traveling to and, when recalled, from the NBADL team to begin and/or end any NBADL Work Assignment, and (b) relocating to and, if recalled, from the NBADL team's home location to begin and/or end any NBADL Work Assignment that extends beyond a period of thirty (30) days. During any NBADL Work Assignment, the player will be provided with housing or a housing subsidy in accordance with the NBADL housing policy.

\hypertarget{terms-of-nbadl-work-assignment.}{%
\section{Terms of NBADL Work Assignment.}\label{terms-of-nbadl-work-assignment.}}

\begin{enumerate}
\def\labelenumi{(\alph{enumi})}
\item
  \textbf{General Terms.}

  During or in connection with any NBADL Work Assignment, and except as expressly set forth in, or limited or modified by, this Article, a player shall (i) accept and be subject to the work requirements and conditions applicable to NBADL players (as such requirements and conditions may change from time to time), and (ii) continue to be subject to the terms and obligations and entitled to the benefits and rights (including, without limitation, Years of Service and free agency rights) of his Uniform Player Contract and this Agreement.
\item
  \textbf{Compensation and Benefits.}

  \begin{enumerate}
  \def\labelenumii{(\roman{enumii})}
  \tightlist
  \item
    During or in connection with any NBADL Work Assignment, the player (A) shall continue to receive the Compensation called for by his Uniform Player Contract, and (B) shall not receive (or accept) any compensation of any kind from the NBADL or any NBADL team other than as expressly set forth in this Article. The player's performance in the NBADL shall not be considered for purposes of any Incentive Compensation contained in his Uniform Player Contract.
  \item
    Any Compensation protection or insurance provided to a player in his Uniform Player Contract shall remain in effect during an NBADL Work Assignment. For purposes of Article II, Section 4, an injury sustained while participating in a basketball practice or game for an NBADL team shall be deemed an injury sustained while participating in a basketball practice or game for the NBA Team.
  \item
    During or in connection with any NBADL Work Assignment, a player (A) shall continue to be eligible to receive the benefits set forth in Article IV of this Agreement to the extent that such player would have been eligible to receive such benefits under this Agreement absent the NBADL Work Assignment, and (B) shall not be eligible to receive (and shall not accept) any benefits from the NBADL or any NBADL team, unless expressly set forth in this Article.
  \item
    To the extent necessary, any plans and/or policies described in Article IV of this Agreement shall be amended to implement the provisions of Section 4(b)(iii) of this Article.
  \end{enumerate}
\item
  \textbf{Meal Expense.}
  While on the road with his NBADL team during an NBADL Work Assignment, the player (i) shall receive the meal expense allowance applicable to NBA players, in accordance with the terms of Article III, Section 2 of this Agreement, and (ii) shall not receive (or accept) any meal expense or per diem from the NBADL or any NBADL team.
\item
  \textbf{Travel Accommodations.}
  During an NBADL Work Assignment, the player shall be provided with the same travel accommodations (including, but not limited to, transportation and hotel arrangements for ``road'' games) that are provided to NBADL players pursuant to applicable NBADL policies, except that: (i) the player shall not be required to share a hotel room; and (ii) the player shall be permitted to fly first class when traveling by air with his NBADL team to road games to the extent first class seats are available on his NBADL team's flight.
\item
  \textbf{Conduct and Discipline.}

  \begin{enumerate}
  \def\labelenumii{(\roman{enumii})}
  \tightlist
  \item
    During any NBADL Work Assignment, the player will: (A) observe and comply with all rules and policies of the NBADL or his NBADL team at all times, whether on or off the playing floor; (B) give his best services, as well as his loyalty, to the NBADL team; (C) be neatly and fully attired in public; (D) conduct himself on and off the court according to the highest standards of honesty, citizenship, and sportsmanship; and (E) not do anything that, in theopinion of the Commissioner of the NBA, is materially detrimental or materially prejudicial to the best interests of the NBA Team, the NBA, the NBADL or the NBADL team.
  \item
    During or in connection with any NBADL Work Assignment, the NBADL, the player's NBADL team, the NBA and the player's NBA Team may impose a fine and/or suspension on the player for the violation of NBADL or NBADL team rules or policies or for any conduct impairing the faithful and thorough discharge of the duties incumbent upon the player. Any disciplinary action taken by the NBA or an NBA Team in response to any act or conduct of a player during an NBADL Work Assignment will supersede disciplinary action taken by the NBADL or any NBADL team in response to such act or conduct. The amount of any such fine and/or suspension that may be imposed by the NBA or an NBA Team shall be governed by the terms of this Agreement and the Uniform Player Contract and shall not be limited by any NBADL rules, policies, practices, procedures, or fine schedules.
  \item
    A fine or suspension imposed by the NBADL or NBADL team in connection with a player's NBADL Work Assignment may be heard and resolved by the Grievance Arbitrator pursuant to Article XXXI of this Agreement only if it results in a financial impact to the player of more than \$5,000.
  \item
    For purposes of paragraph 16(a)(ii) of a player's Uniform Player Contract, during or in connection with any NBADL Work Assignment, (A) the terms ``any official or employee of the Team or the NBA (other than another player)'' will be construed to include, without limitation, any official or employee of the NBADL or the player's NBADL team (other than another player), and (B) the terms ``any NBA game or event'' will be construed to include, without limitation, any NBADL game or event.
  \end{enumerate}
\item
  \textbf{Medical Treatment and Physical Condition.}

  \begin{enumerate}
  \def\labelenumii{(\roman{enumii})}
  \tightlist
  \item
    The NBADL and/or NBADL team may make public medical information about a player on an NBADL Work Assignment, provided that such information relates solely to the reasons why any such player has not been or is not performing services with the NBADL team. With respect to the foregoing, a player or his immediate family (where appropriate) shall have the right to approve the terms and timing of any public release of medical information relating to any injuries or illnesses suffered by the player that are potentially life- or career threatening, or that do not arise from the player's participation in NBADL games or practices.
  \item
    For purposes of paragraphs 7, 16(a)(iii), 16(b), and 16(c) of the player's Uniform Player Contract, the terms ``basketball practice or game played for the Team'' or ``playing for the Team'' will be construed to include, without limitation, any practice or game played in the NBADL during an NBADL Work Assignment.
  \end{enumerate}
\item
  \textbf{Prohibited Substances.}

  During any NBADL Work Assignment, the player (i) shall be subject to Article XXXIII (Anti-Drug Program) of this Agreement and paragraph 8 of the Uniform Player Contract, and (ii) shall not be subject to any anti-drug program maintained by the NBADL.
\item
  \textbf{Player Attributes and Performances.}

  Notwithstanding anything to the contrary in this Agreement, the Uniform Player Contract, or the Group License Agreement, with respect to any player who serves or has served on an NBADL Work Assignment:
  (i) the NBA and its related entities (including, without limitation, NBA Teams), and the NBADL and its related entities (including, without limitation, NBADL teams),shall have the right to use, and to license others to use, such player's Player Attributes (as defined in the Group License Agreement) in connection with:
  (A) any advertising, marketing, or collateral materials or marketing programs conducted by the NBADL or any NBADL team that is intended to promote (1) any game in which an NBADL team participates or any NBADL game telecast or broadcast (including NBADL pre-season, exhibition, regular season, or playoff games), (2) the NBADL, its teams or its players, or (3) the sport of basketball; and
  (B) the manufacturing, distribution, sale, advertising (in any media whatsoever), marketing, or promotion of any product or service, subject to the applicable terms of the Group License Agreement.
  (ii) the NBA and its related entities (including, without limitation, NBA Teams), and the NBADL and its related entities (including, without limitation, NBADL teams), shall have the right to use, and to license others to use, any performance of such player in connection with any form of broadcast or telecast, including over-the-air television, cable television, pay television, direct broadcast satellite television, and any form of cassette, cartridge, disk system, or other means of distribution known or unknown.
\item
  \textbf{Promotional Activities.}

  In connection with a player's NBADL Work Assignment, the rights accorded to the NBA and his NBA Team under paragraph 13(a) of the Uniform Player Contract shall extend, without t limitation, to the NBADL and his NBADL team, and any promotional appearances such player is required to make while on his NBADL Work Assignment shall count against the appearances the player is obligated to provide to the NBA and his NBA Team under Article II, Section 8; provided, however, that such player will be required to provide two (2) additional promotional appearances while on assignment each Season to the NBADL or his NBADL team.
\end{enumerate}

\hypertarget{miscellaneous.-2}{%
\section{Miscellaneous.}\label{miscellaneous.-2}}

\begin{enumerate}
\def\labelenumi{(\alph{enumi})}
\tightlist
\item
  With respect to the duties and obligations of players under paragraph 5 of the Uniform Player Contract (relating to Article 35 of the NBA Constitution) during or in connection with any NBADL Work Assignment:

  \begin{enumerate}
  \def\labelenumii{(\roman{enumii})}
  \tightlist
  \item
    the terms ``game'' or ``games'' in Article 35(b) and (c) of the Constitution will be construed to include, without limitation, any game played by an NBADL team;
  \item
    the term ``basketball'' or ``game of basketball'' in Article 35(c) and (d) of the Constitution will be construed to include, without limitation, the NBADL or any of its teams;
  \item
    the prohibition concerning wagering in Article 35(f) of the Constitution will extend, without limitation, to any game played by an NBADL team; and
  \item
    the Commissioner's authority to act pursuant to paragraph 5(e) of the Uniform Player Contract will extend, without limitation, to any game played by an NBADL team.
  \end{enumerate}
\item
  A player shall not directly or indirectly own or hold any interest in the NBADL or any NBADL team unless authorized by the NBA.(c) At the conclusion of each Season covered by this Agreement, the NBA and the Players Association shall meet to discuss issues concerning the operation of this Article.
\end{enumerate}

\hypertarget{other}{%
\chapter{OTHER}\label{other}}

\hypertarget{headings-and-organization.}{%
\section{Headings and Organization.}\label{headings-and-organization.}}

The headings and organization of this Agreement are solely for the convenience of the parties, and shall not be deemed part of, or considered in construing or interpreting, this Agreement.

\hypertarget{time-periods.}{%
\section{Time Periods.}\label{time-periods.}}

Unless specifically stated otherwise, the specification of any time period in this Agreement shall include any non-business days within such period, except that any deadline falling on a Saturday, Sunday, or Federal Holiday shall be deemed to fall on the following business day.

\hypertarget{exhibits.}{%
\section{Exhibits.}\label{exhibits.}}

All of the Exhibits hereto are an integral part of this Agreement and of the agreement of the parties thereto.

NATIONAL BASKETBALL ASSOCIATION\\
By: /s/ David J. Stern\\
David J. Stern\\
Commissioner

NATIONAL BASKETBALL PLAYERS ASSOCIATION\\
By: /s/ ANTONIO DAVIS\\
Antonio Davis\\
President

\hypertarget{appendix-appendix}{%
\appendix}


\hypertarget{national-basketball-association-uniform-player-contract}{%
\chapter{NATIONAL BASKETBALL ASSOCIATION UNIFORM PLAYER CONTRACT}\label{national-basketball-association-uniform-player-contract}}

THIS AGREEMENT made this \_\_\_\_\_ day of\_\_\_\_\_\_\_\_\_\_\_\_\_\_\_\_\_\_\_, is by and between \_\_\_\_\_\_\_\_\_\_\_\_\_\_\_\_\_\_\_\_\_\_\_\_ (hereinafter called the ``Team''), a member of the National Basketball Association (hereinafter called the ``NBA'' or ``League'') and \_\_\_\_\_\_\_\_\_\_\_\_\_\_\_\_\_\_\_ , an individual whose address is shown below (hereinafter called the ``Player''). In consideration of the mutual promises hereinafter contained, the parties hereto promise and agree as follows:

\begin{enumerate}
\def\labelenumi{\arabic{enumi}.}
\item
  \textbf{TERM.}

  The Team hereby employs the Player as a skilled basketball player for a term of \_\_\_\_ year(s) from the 1st day of September \_\_\_\_.
\item
  \textbf{SERVICES.}

  The services to be rendered by the Player pursuant to this Contract shall include: (a) training camp, (b) practices, meetings, workouts, and skill or conditioning sessions conducted by the Team during the Season, (c) games scheduled for the Team during any Regular Season, (d) Exhibition games scheduled by the Team or the League during and prior to any Regular Season, (e) if the Player is invited to participate, the NBA's All-Star Game (including the Rookie-Sophomore Game) and every event conducted in association with such All-Star Game, but only in accordance with Article XXI of the Collective Bargaining Agreement currently in effect between the NBA and the National Basketball Players Association (hereinafter the ``CBA''), (f) Playoff games scheduled by the League subsequent to any Regular Season, (g) promotional and commercial activities of the Team and the League as set forth in this Contract and the CBA, and (h) any NBADL Work Assignment in accordance with Article XLII of the CBA.
\item
  \textbf{COMPENSATION.}
\end{enumerate}

\begin{enumerate}
\def\labelenumi{(\alph{enumi})}
\tightlist
\item
  Subject to paragraph 3(b) below, the Team agrees to pay the Player for rendering the services and performing the obligations described herein the Compensation described in Exhibit 1 or Exhibit 1A hereto (less all amounts required to be withheld by any governmental authority, and exclusive of any amount(s) which the Player shall be entitled to receive from the Player Playoff Pool). Unless otherwise provided in Exhibit 1, such Compensation shall be paid in twelve (12) equal semi-monthly payments beginning with the first of said payments on November 15th of each year covered by the Contract and continuing with such payments on the first and fifteenth of each month until said Compensation is paid in full.
\item
  The Team agrees to pay the Player \$1,500 per week, pro rata, less all amounts required to be withheld by any governmental authority, for each week (up to a maximum of four (4) weeks for Veterans and up to a maximum of five (5) weeks for Rookies) prior to the Team's first Regular Season game that the Player is in attendance at training camp or Exhibition games; provided, however, that no such payments shall be made if, prior to the date on which he is required to attend training camp, the Player has been paid \$10,000 or more in Compensation with respect to the NBA Season scheduled to commence immediately following such training camp. Any Compensation paid by the Team pursuant to this subparagraph shall be considered an advance against any Compensation owed to the Player pursuant to paragraph 3(a) above, and the first scheduled payment of such Compensation (or such subsequent payments, if the first scheduled payment is not sufficient) shall be reduced by the amount of such advance.
\item
  The Team will not pay and the Player will not accept any bonus or anything of value on account of the Team's winning any particular NBA game or series of games or attaining a certain position in the standings of the League as of a certain date, other than the final standing of the Team.
\end{enumerate}

\begin{enumerate}
\def\labelenumi{\arabic{enumi}.}
\setcounter{enumi}{3}
\item
  \textbf{EXPENSES.}

  The Team agrees to pay all proper and necessary expenses of the Player, including the reasonable lodging expenses of the Player while playing for the Team ``on the road'' and during the training camp period (defined for this paragraph only to mean the period from the first day of training camp through the day of the Team's first Exhibition game) for as long as the Player is not then living at home. The Player, while ``on the road'' (and during the training camp period, only if the Player is not then living at home and the Team does not pay for meals directly), shall be paid a meal expense allowance as set forth in the CBA. No deductions from such meal expense allowance shall be made for meals served on an airplane. During the training camp period (and only if the Player is not then living at home and the Team does not pay for meals directly), the meal expense allowance shall be paid in weekly installments commencing with the first week of training camp. For the purposes of this paragraph, the Player shall be considered to be ``on the road'' from the time the Team leaves its home city until the time the Team arrives back at its home city.
\item
  \textbf{CONDUCT.}
\end{enumerate}

\begin{enumerate}
\def\labelenumi{(\alph{enumi})}
\tightlist
\item
  The Player agrees to observe and comply with all Team rules, as maintained or promulgated in accordance with the CBA, at all times whether on or off the playing floor. Subject to the provisions of the CBA, such rules shall be part of this Contract as fully as if herein written and shall be binding upon the Player.
\item
  The Player agrees: (i) to give his best services, as well as his loyalty, to the Team, and to play basketball only for the Team and its assignees; (ii) to be neatly and fully attired in public; (iii) to conduct himself on and off the court according to the highest standards of honesty, citizenship, and sportsmanship; and (iv) not to do anything that is materially detrimental or materially prejudicial to the best interests of the Team or the League.
\item
  For any violation of Team rules, any breach of any provision of this Contract, or for any conduct impairing the faithful and thorough discharge of the duties incumbent upon the Player, the Team may reasonably impose fines and/or suspensions on the Player in accordance with the terms of the CBA.
\item
  The Player agrees to be bound by Article 35 of the NBA Constitution, a copy of which, as in effect on the date of this Contract, is attached hereto. The Player acknowledges that the Commissioner is empowered to impose fines upon and/or suspend the Player for causes and in the manner provided in such Article, provided that such fines and/or suspensions are consistent with the terms of the CBA.
\item
  The Player agrees that if the Commissioner, in his sole judgment, shall find that the Player has bet, or has offered or attempted to bet, money or anything of value on the outcome of any game participated in by any team which is a member of the NBA, the Commissioner shall have the power in his sole discretion to suspend the Player indefinitely or to expel him as a player for any member of the NBA, and the Commissioner's finding and decision shall be final, binding, conclusive, and unappealable.
\item
  The Player agrees that he will not, during the term of this Contract, directly or indirectly, entice, induce, or persuade, or attempt to entice, induce, or persuade, any player or coach who is under contract to any NBA team to enter into negotiations for or relating to his services as a basketball player or coach, nor shall he negotiate for or contract for such services, except with the prior written consent of such team. Breach of this subparagraph, in addition to the remedies available to the Team, shall be punishable by fine and/or suspension to be imposed by the Commissioner.
\item
  When the Player is fined and/or suspended by the Team or the NBA, he shall be given notice in writing (with a copy to the Players Association), stating the amount of the fine or the duration of the suspension and the reasons therefor.
\end{enumerate}

\begin{enumerate}
\def\labelenumi{\arabic{enumi}.}
\setcounter{enumi}{5}
\tightlist
\item
  \textbf{WITHHOLDING.}
\end{enumerate}

\begin{enumerate}
\def\labelenumi{(\alph{enumi})}
\tightlist
\item
  In the event the Player is fined and/or suspended by the Team or the NBA, the Team shall withhold the amount of the fine or, in the case of a suspension, the amount provided in Article VI of the CBA from any Current Base Compensation due or to become due to the Player with respect to the contract year in which the conduct resulting in the fine and/or the suspension occurred (or a subsequent contract year if the Player has received all Current Base Compensation due to him for the then current contract year). If, at the time the Player is fined and/or suspended, the Current Base Compensation remaining to be paid to the Player under this Contract is not sufficient to cover such fine and/or suspension, then the Player agrees promptly to pay the amount directly to the Team. In no case shall the Player permit any such fine and/or suspension to be paid on his behalf by anyone other than himself.
\item
  Any Current Base Compensation withheld from or paid by the Player pursuant to this paragraph 6 shall be retained by the Team or the League, as the case may be, unless the Player contests the fine and/or suspension by initiating a timely Grievance in accordance with the provisions of the CBA. If such Grievance is initiated and it satisfies Article XXXI, Section 13 of the CBA, the amount withheld from the Player shall be placed in an interest-bearing account, pursuant to Article XXXI, Section 9 of such Agreement, pending the resolution of the Grievance.
\end{enumerate}

\begin{enumerate}
\def\labelenumi{\arabic{enumi}.}
\setcounter{enumi}{6}
\tightlist
\item
  \textbf{PHYSICAL CONDITION.}
\end{enumerate}

\begin{enumerate}
\def\labelenumi{(\alph{enumi})}
\tightlist
\item
  The Player agrees to report at the time and place fixed by the Team in good physical condition and to keep himself throughout each NBA Season in good physical condition.
\item
  If the Player, in the judgment of the Team's physician, is not in good physical condition at the date of his first scheduled game for the Team, or if, at the beginning of or during any Season, he fails to remain in good physical condition (unless such condition results directly from an injury sustained by the Player as a direct result of participating in any basketball practice or game played for the Team during such Season), so as to render the Player, in the judgment of the Team's physician, unfit to play skilled basketball, the Team shall have the right to suspend such Player until such time as, in the judgment of the Team's physician, the Player is in sufficiently good physical condition to play skilled basketball. In the event of such suspension, the Base Compensation payable to the Player for any Season during such suspension shall be reduced in the same proportion as the length of the period during which, in the judgment of the Team's physician, the Player is unfit to play skilled basketball, bears to the length of such Season. Nothing in this subparagraph shall authorize the Team to suspend the Player solely because the Player is injured or ill.
\item
  If, during the term of this Contract, the Player is injured as a direct result of participating in any basketball practice or game played for the Team, the Team will pay the Player's reasonable hospitalization and medical expenses (including doctor's bills), provided that the hospital and doctor are selected by the Team, and provided further that the Team shall be obligated to pay only those expenses incurred as a direct result of medical treatment caused solely by and relating directly to the injury sustained by the Player. Subject to the provisions set forth in Exhibit 3, if in the judgment of the Team's physician, the Player's injuries resulted directly from playing for the Team and render him unfit to play skilled basketball, then, so long as such unfitness continues, but in no event after the Player has received his full Base Compensation for the Season in which the injury was sustained, the Team shall pay to the Player the Base Compensation prescribed in Exhibit 1 to this Contract for such Season. The Team's obligations hereunder shall be reduced by (i) any workers' compensation benefits, which, to the extent permitted by law, the Player hereby assigns to the Team, and (ii) any insurance provided for by the Team whether paid or payable to the Player.
\item
  The Player agrees to provide to the Team's coach, trainer, or physician prompt notice of any injury, illness, or medical condition suffered by him that is likely to affect adversely the Player's ability to render the services required under this Contract, including the time, place, cause, and nature of such injury, illness, or condition.
\item
  Should the Player suffer an injury, illness, or medical condition, he will submit himself to a medical examination, appropriate medical treatment by a physician designated by the Team, and such rehabilitation activities as such physician may specify. Such examination when made at the request of the Team shall be at its expense, unless made necessary by some act or conduct of the Player contrary to the terms of this Contract.
\item
  The Player agrees (i) to submit to a physical examination at the commencement and conclusion of each Contract year hereunder, and at such other times as reasonably determined by the Team to be medically necessary, and (ii) at the commencement of this Contract, and upon the request of the Team, to provide a complete prior medical history.
\item
  The Player agrees to supply complete and truthful information in connection with any medical examinations or requests for medical information authorized by this Contract.
\item
  A Player who consults a physician other than a physician designated by the Team shall give notice of such consultation to the Team and shall authorize and direct such other physician to provide the Team with all information it may request concerning any condition that in the judgment of the Team's physician may affect the Player's ability to play skilled basketball.
\item
  If and to the extent necessary to enable or facilitate the disclosure of medical information as provided for by this Contract or Article XXII of the CBA, the Player shall execute such individual authorization(s) as may be requested by the Team or as may be required by health care providers who examine or treat the Player.
\end{enumerate}

\begin{enumerate}
\def\labelenumi{\arabic{enumi}.}
\setcounter{enumi}{7}
\item
  \textbf{PROHIBITED SUBSTANCES.}

  The Player acknowledges that this Contract may be terminated in accordance with the express provisions of Article XXXIII (Anti-Drug Program) of the CBA, and that any such termination will result in the Player's immediate dismissal and disqualification from any employment by the NBA and any of its teams. Notwithstanding any terms or provisions of this Contract (including any amendments hereto), in the event of such termination, all obligations of the Team, including obligations to pay Compensation, shall cease, except the obligation of the Team to pay the Player's earned Compensation (whether Current or Deferred) to the date of termination.
\item
  \textbf{UNIQUE SKILLS.}

  The Player represents and agrees that he has extraordinary and unique skill and ability as a basketball player, that the services to be rendered by him here under cannot be replaced or the loss thereof adequately compensated for in money damages, and that any breach by the Player of this Contract will cause irreparable injury to the Team, and to its assignees. Therefore, it is agreed that in the event it is alleged by the Team that the Player is playing, attempting or threatening to play, or negotiating for the purpose of playing, during the term of this Contract, for any other person, firm, entity, or organization, the Team and its assignees (in addition to any other remedies that may be available to them judicially or by way of arbitration) shall have the right to obtain from any court or arbitrator having jurisdiction such equitable relief as may be appropriate, including a decree enjoining the Player from any further such breach of this Contract, and enjoining the Player from playing basketball for any other person, firm, entity, or organization during the term of this Contract. The Player agrees that this right may be enforced by the Team or the NBA. In any suit, action, or arbitration proceeding brought to obtain such equitable relief, the Player does hereby waive his right, if any, to trial by jury, and does hereby waive his right, if any, to interpose any counterclaim or set-off for any cause whatever.
\item
  \textbf{ASSIGNMENT.}
\end{enumerate}

\begin{enumerate}
\def\labelenumi{(\alph{enumi})}
\tightlist
\item
  The Team shall have the right to assign this Contract to any other NBA team and the Player agrees to accept such assignment and to faithfully perform and carry out this Contract with the same force and effect as if it had been entered into by the Player with the assignee team instead of with the Team.
\item
  In the event that this Contract is assigned to any other NBA team, all reasonable expenses incurred by the Player in moving himself and his family to the home territory of the team to which such assignment is made, as a result thereof, shall be paid by the assignee team.
\item
  In the event that this Contract is assigned to another NBA team, the Player shall forthwith be provided notice orally or in writing, delivered to the Player personally or delivered or mailed to his last known address, and the Player shall report to the assignee team within forty-eight (48) hours after said notice has been received (if the assignment is made during a Season), within one (1) week after said notice has been received (if the assignment is made between Seasons), or within such longer time for reporting as may be specified in said notice. The NBA shall also promptly notify the Players Association of any such assignment. The Player further agrees that, immediately upon reporting to the assignee team, he will submit upon request to a physical examination conducted by a physician designated by the assignee team.
\item
  If the Player, without a reasonable excuse, does not report to the team to which this Contract has been assigned within the time provided in subsection (c) above, then (i) upon consummation of the assignment, the Player may be disciplined by the assignee team or, if the assignment is not consummated or is voided as a result of the Player's failure to so report, by the assignor Team, and (ii) such conduct shall constitute conduct prejudicial to the NBA under Article 35(d) of the NBA Constitution, and shall therefore subject the Player to discipline from the NBA in accordance with such Article.
\end{enumerate}

\begin{enumerate}
\def\labelenumi{\arabic{enumi}.}
\setcounter{enumi}{10}
\tightlist
\item
  \textbf{VALIDITY AND FILING.}
\end{enumerate}

\begin{enumerate}
\def\labelenumi{(\alph{enumi})}
\tightlist
\item
  This Contract shall be valid and binding upon the Team and the Player immediately upon its execution.
\item
  The Team agrees to file a copy of this Contract, and/or any amendment(s) thereto, with the Commissioner of the NBA as soon as practicable by facsimile and overnight mail, but in no event may such filing be made more than forty-eight (48) hours after the execution of this Contract and/or amendment(s).
\item
  If pursuant to the NBA Constitution and By-Laws or the CBA, the Commissioner disapproves this Contract (or amendment) within ten (10) days after the receipt thereof in his office by overnight mail, this Contract (or amendment) shall thereupon terminate and be of no further force or effect and the Team and the Player shall thereupon be relieved of their respective rights and liabilities thereunder. If the Commissioner's disapproval is subsequently overturned in any proceeding brought under the arbitration provisions of the CBA (including any appeals), the Contract shall again be valid and binding upon the Team and the Player, and the Commissioner shall be afforded another ten-day period to disapprove the Contract (based on the Team's Room at the time the Commissioner's disapproval is overturned) as set forth in the foregoing sentence. The NBA will promptly inform the Players Association if the Commissioner disapproves this Contract.
\end{enumerate}

\begin{enumerate}
\def\labelenumi{\arabic{enumi}.}
\setcounter{enumi}{11}
\item
  \textbf{PROHIBITED ACTIVITIES.}

  The Player and the Team acknowledge and agree that the Player's participation in certain other activities may impair or destroy his ability and skill as a basketball player, and the Player's participation in any game or exhibition of basketball other than at the request of the Team may result in injury to him. Accordingly, the Player agrees that he will not, without the written consent of the Team, engage in any activity that a reasonable person would recognize as involving or exposing the participant to a substantial risk of bodily injury including, but not limited to: (i) sky-diving, hang gliding, snow skiing, rock or mountain climbing (as distinguished from hiking), rappelling, and bungee jumping; (ii) any fighting, boxing, or wrestling; (iii) driving or riding on a motorcycle or moped; (iv) riding in or on any motorized vehicle in any kind of race or racing contest; (v) operating an aircraft of any kind; (vi) engaging in any other activity excluded or prohibited by or under any insurance policy which the Team procures against the injury, illness or disability to or of the Player, or death of the Player, for which the Player has received written notice from the Team prior to the execution of this Contract; or (vii) participating in any game or exhibition of basketball, football, baseball, hockey, lacrosse, or other team sport or competition. If the Player violates this Paragraph 12, he shall be subject to discipline imposed by the Team and/or the Commissioner of the NBA. Nothing contained herein shall be intended to require the Player to obtain the written consent of the Team in order to enable the Player to participate in, as an amateur, the sports of golf, tennis, handball, swimming, hiking, softball, volleyball, and other similar sports that a reasonable person would not recognize as involving or exposing the participant to a substantial risk of bodily injury.
\item
  \textbf{PROMOTIONAL ACTIVITIES.}
\end{enumerate}

\begin{enumerate}
\def\labelenumi{(\alph{enumi})}
\tightlist
\item
  The Player agrees to allow the Team, the NBA, or a League-related entity to take pictures of the Player, alone or together with others, for still photographs, motion pictures, or television, at such reasonable times as the Team, the NBA or the League-related entity may designate. No matter by whom taken, such pictures may be used in any manner desired by either the Team, the NBA, or the League-related entity for publicity or promotional purposes. The rights in any such pictures taken by the Team, the NBA, or the League-related entity shall belong to the Team, the NBA, or the League-related entity, as their interests may appear.
\item
  The Player agrees that, during any year of this Contract, he will not make public appearances, participate in radio or television programs, permit his picture to be taken, write or sponsor newspaper or magazine articles, or sponsor commercial products without the written consent of the Team, which shall not be withheld except in the reasonable interests of the Team or the NBA. The foregoing shall be interpreted in accordance with the decision in Portland Trail Blazers v. Darnell Valentine and Jim Paxson, Decision 86-2 (August 13, 1986).
\item
  Upon request, the Player shall consent to and make himself available for interviews by representatives of the media conducted at reasonable times.
\item
  In addition to the foregoing, and subject to the conditions and limitations set forth in Article II, Section 8 of the CBA, the Player agrees to participate, upon request, in all other reasonable promotional activities of the Team, the NBA, and any League-related entity. For each such promotional appearance made on behalf of a commercial sponsor of the Team, the Team agrees to pay the Player \$2,500 or, if the Team agrees, such higher amount that is consistent with the Team's past practice and not otherwise unreasonable.
\end{enumerate}

\begin{enumerate}
\def\labelenumi{\arabic{enumi}.}
\setcounter{enumi}{13}
\tightlist
\item
  \textbf{GROUP LICENSE.}
\end{enumerate}

\begin{enumerate}
\def\labelenumi{(\alph{enumi})}
\tightlist
\item
  The Player hereby grants to NBA Properties, Inc.~(and its related entities) the exclusive rights to use the Player's Player Attributes as such term is defined and for such group licensing purposes as are set forth in the Agreement between NBA Properties, Inc.~and the National Basketball Players Association, made as of September 18, 1995 and amended January 20, 1999 and July 29, 2005 (the ``Group License''), a copy of which will, upon his request, be furnished to the Player; and the Player agrees to make the appearances called for by such Agreement.
\item
  Notwithstanding anything to the contrary contained in the Group License or this Contract, NBA Properties (and its related entities) may use, in connection with League Promotions, the Player's (i) name or nickname and/or (ii) the Player's Player Attributes (as defined in the Group License) as such Player Attributes may be captured in game action footage or photographs. NBA Properties (and its related entities) shall be entitled to use the Player's Player Attributes individually pursuant to the preceding sentence and shall not be required to use the Player's Player Attributes in a group or as one of multiple players. As used herein, League Promotion shall mean any advertising, marketing, or collateral materials or marketing programs conducted by the NBA, NBA Properties (and its related entities) or any NBA team that is intended to promote (A) any game in which an NBA team participates or game telecast, cablecast or broadcast (including Pre-Season, Exhibition, Regular Season, and Playoff games), (B) the NBA, its teams, or its players, or (C) the sport of basketball.
\end{enumerate}

\begin{enumerate}
\def\labelenumi{\arabic{enumi}.}
\setcounter{enumi}{14}
\item
  \textbf{TEAM DEFAULT.}

  In the event of an alleged default by the Team in the payments to the Player provided for by this Contract, or in the event of an alleged failure by the Team to perform any other material obligation that it has agreed to perform hereunder, the Player shall notify both the Team and the League in writing of the facts constituting such alleged default or alleged failure. If neither the Team nor the League shall cause such alleged default or alleged failure to be remedied within five (5) days after receipt of such written notice, the National Basketball Players Association shall, on behalf of the Player, have the right to request that the dispute concerning such alleged default or alleged failure be referred immediately to the Grievance Arbitrator in accordance with the provisions of the CBA. If, as a result of such arbitration, an award issues in favor of the Player, and if neither the Team nor the League complies with such award within ten (10) days after the service thereof, the Player shall have the right, by a further written notice to the Team and the League, to terminate this Contract.
\item
  \textbf{TERMINATION.}
\end{enumerate}

\begin{enumerate}
\def\labelenumi{(\alph{enumi})}
\tightlist
\item
  The Team may terminate this Contract upon written notice to the Player if the Player shall:

  \begin{enumerate}
  \def\labelenumii{(\roman{enumii})}
  \tightlist
  \item
    at any time, fail, refuse, or neglect to conform his personal conduct to standards of good citizenship, good moral character (defined here to mean not engaging in acts of moral turpitude, whether or not such acts would constitute a crime), and good sportsmanship, to keep himself in first class physical condition, or to obey the Team's training rules;
  \item
    at any time commit a significant and inexcusable physical attack against any official or employee of the Team or the NBA (other than another player), or any person in attendance at any NBA game or event, considering the totality of the circumstances, including (but not limited to) the degree of provocation (if any) that may have led to the attack, the nature and scope of the attack, the Player's state of mind at the time of the attack, and the extent of any injury resulting from the attack;
  \item
    at any time, fail, in the sole opinion of the Team's management, to exhibit sufficient skill or competitive ability to qualify to continue as a member of the Team; provided, however, (A) that if this Contract is terminated by the Team, in accordance with the provisions of this subparagraph, prior to January 10 of any Season, and the Player, at the time of such termination, is unfit to play skilled basketball as the result of an injury resulting directly from his playing for the Team, the Player shall (subject to the provisions set forth in Exhibit 3) continue to receive his full Base Compensation, less all workers' compensation benefits (which, to the extent permitted by law, and if not deducted from the Player's Compensation by the Team, the Player hereby assigns to the Team) and any insurance provided for by the Team paid or payable to the Player by reason of said injury, until such time as the Player is fit to play skilled basketball, but not beyond the Season during which such termination occurred; and provided, further, (B) that if this Contract is terminated by the Team, in accordance with the provisions of this subparagraph, during the period from the January 10 of any Season through the end of such Season, the Player shall be entitled to receive his full Base Compensation for said Season; or
  \item
    at any time, fail, refuse, or neglect to render his services hereunder or in any other manner materially breach this Contract.
  \end{enumerate}
\item
  If this Contract is terminated by the Team by reason of the Player's failure to render his services hereunder due to disability caused by an injury to the Player resulting directly from his playing for the Team and rendering him unfit to play skilled basketball, and notice of such injury is given by the Player as provided herein, the Player shall (subject to the provisions set forth in Exhibit 3) be entitled to receive his full Base Compensation for the Season in which the injury was sustained, less all workers' compensation benefits (which, to the extent permitted by law, and if not deducted from the Player's Compensation by the Team, the Player hereby assigns to the Team) and any insurance provided for by the Team paid or payable to the Player by reason of said injury.
\item
  Notwithstanding the provisions of paragraph 16(b) above, if this Contract is terminated by the Team prior to the first game of a Regular Season by reason of the Player's failure to render his services hereunder due to an injury or condition sustained or suffered during a preceding Season, or after such Season but prior to the Player's participation in any basketball practice or game played for the Team, payment by the Team of any Compensation earned through the date of termination under paragraph 3(b) above, payment of the Player's board, lodging, and expense allowance during the training camp period, payment of the reasonable traveling expenses of the Player to his home city, and the expert training and coaching provided by the Team to the Player during the training season shall be full payment to the Player.
\item
  If this Contract is terminated by the Team during the period designated by the Team for attendance at training camp, payment by the Team of any Compensation earned through the date of termination under paragraph 3(b) above, payment of the Player's board, lodging, and expense allowance during such period to the date of termination, payment of the reasonable traveling expenses of the Player to his home city, and the expert training and coaching provided by the Team to the Player during the training season shall be full payment to the Player.
\item
  If this Contract is terminated by the Team after the first game of a Regular Season, except in the case provided for in subparagraphs (a)(iii) and (b) of this paragraph 16, the Player shall be entitled to receive as full payment hereunder a sum of money which, when added to the salary which he has already received during such Season, will represent the same proportionate amount of the annual sum set forth in Exhibit 1 hereto as the number of days of such Regular Season then past bears to the total number of days of such Regular Season, plus the reasonable traveling expenses of the Player to his home.
\item
  If the Team proposes to terminate this Contract in accordance with subparagraph (a) of this paragraph 16, it must first comply with the following waiver procedure:

  \begin{enumerate}
  \def\labelenumii{(\roman{enumii})}
  \tightlist
  \item
    The Team shall request the NBA Commissioner to request waivers from all other clubs. Such waiver request may not be withdrawn.
  \item
    Upon receipt of the waiver request, any other team may claim assignment of this Contract at such waiver price as may be fixed by the League, the priority of claims to be determined in accordance with the NBA Constitution and By-Laws.
  \item
    If this Contract is so claimed, the Team agrees that it shall, upon the assignment of this Contract to the claiming team, notify the Player of such assignment as provided in paragraph 10(c) hereof, and the Player agrees he shall report to the assignee team as provided in said paragraph 10(c).
  \item
    If the Contract is not claimed prior to the expiration of the waiver period, it shall terminate and the Team shall promptly deliver written notice of termination to the Player.
  \item
    The NBA shall promptly notify the Players Association of the disposition of any waiver request.
  \item
    To the extent not inconsistent with the foregoing provisions of this subparagraph (f), the waiver procedures set forth in the NBA Constitution and By-Laws, a copy of which, as in effect on the date of this Contract, is attached hereto, shall govern.
  \end{enumerate}
\item
  Upon any termination of this Contract by the Player, all obligations of the Team to pay Compensation shall cease on the date of termination, except the obligation of the Team to pay the Player's Compensation to said date.
\end{enumerate}

\begin{enumerate}
\def\labelenumi{\arabic{enumi}.}
\setcounter{enumi}{16}
\item
  \textbf{DISPUTES.}

  In the event of any dispute arising between the Player and the Team relating to any matter arising under this Contract, or concerning the performance or interpretation thereof (except for a dispute arising under paragraph 9 hereof), such dispute shall be resolved in accordance with the Grievance and Arbitration Procedure set forth in Article XXXI of the CBA.
\item
  \textbf{PLAYER NOT A MEMBER.}

  Nothing contained in this Contract or in any provision of the NBA Constitution and By-Laws shall be construed to constitute the Player a member of the NBA or to confer upon him any of the rights or privileges of a member thereof.
\item
  \textbf{RELEASE.}

  The Player hereby releases and waives any and all claims he may have, or that may arise during the term of this Contract, against (a) the NBA and its related entities, the NBADL and its related entities, and every member of the NBA or the NBADL, and every director, officer, owner, stockholder, trustee, partner, and employee of the NBA, NBADL and their respective related entities and/or any member of the NBA or NBADL and their related entities (excluding persons employed as players by any such member), and (b) any person retained by the NBA and/or the Players Association in connection with the NBA/NBPA Anti-Drug Program, the Grievance Arbitrator, the System Arbitrator, and any other arbitrator or expert retained by the NBA and/or the Players Association under the terms of the CBA, in both cases (a) and (b) above, arising out of, or in connection with, and whether or not by negligence, (i) any injury that is subject to the provisions of paragraph 7 hereof, (ii) any fighting or other form of violent and/or unsportsmanlike conduct occurring during the course of any practice, any NBADL game, and/or any NBA Exhibition, Regular Season, and/or Playoff game (in all cases on or adjacent to the playing floor or in or adjacent to any facility used for such practices or games), (iii) the testing procedures or the imposition of any penalties set forth in paragraph 8 hereof and in the NBA/NBPA Anti-Drug Program, or (iv) any injury suffered in the course of his employment as to which he has or would have a claim for workers' compensation benefits. The foregoing shall not apply to any claim of medical malpractice against a Team-affiliated physician or other medical personnel.
\item
  \textbf{ENTIRE AGREEMENT.}

  This Contract (including any Exhibits hereto) contains the entire agreement between the parties and, except as provided in the CBA, sets forth all components of the Player's Compensation from the Team or any Team Affiliate, and there are no other agreements or transactions of any kind (whether disclosed or undisclosed to the NBA), express or implied, oral or written, or promises, undertakings, representations, commitments, inducements, assurances of intent, or understandings of any kind (whether disclosed or undisclosed to the NBA) (a) concerning any future Renegotiation, Extension, or other amendment of this Contract or the entry into any new Player Contract, or (b) involving compensation or consideration of any kind (including, without limitation, an investment or business opportunity) to be paid, furnished, or made available to the Player, or any person or entity controlled by, related to, or acting with authority on behalf of the Player, by the Team or any Team Affiliate.
\end{enumerate}

\newpage

\textbf{\emph{EXAMINE THIS CONTRACT CAREFULLY BEFORE SIGNING IT.}}

THIS CONTRACT INCLUDES EXHIBITS \_\_\_\_\_\_\_\_, WHICH ARE ATTACHED HERETO AND MADE A PART HEREOF.

IN WITNESS WHEREOF the Player has hereunto signed his name and the Team has caused this Contract to be executed by its duly authorized officer.

\begin{longtable}[]{@{}ll@{}}
\toprule()
\endhead
Dated: \_\_\_\_\_\_\_\_\_\_\_\_\_\_\_\_\_\_\_\_\_ & By: \_\_\_\_\_\_\_\_\_\_\_\_\_\_\_\_\_\_\_\_\_\_\_\_\_\_\_\_ \\
& Title: \_\_\_\_\_\_\_\_\_\_\_\_\_\_\_\_\_\_\_\_\_\_\_\_\_\_\_\_ \\
& Team: \_\_\_\_\_\_\_\_\_\_\_\_\_\_\_\_\_\_\_\_\_\_\_\_\_\_\_\_ \\
& \\
Dated: \_\_\_\_\_\_\_\_\_\_\_\_\_\_\_\_\_\_\_\_\_ & By: \_\_\_\_\_\_\_\_\_\_\_\_\_\_\_\_\_\_\_\_\_\_\_\_\_\_\_\_ \\
& Player: \_\_\_\_\_\_\_\_\_\_\_\_\_\_\_\_\_\_\_\_\_\_\_\_\_\_\_\_ \\
& Player's Address: \\
& \_\_\_\_\_\_\_\_\_\_\_\_\_\_\_\_\_\_\_\_\_\_\_\_\_\_\_\_\_\_\_\_\_\_\_\_ \\
& \_\_\_\_\_\_\_\_\_\_\_\_\_\_\_\_\_\_\_\_\_\_\_\_\_\_\_\_\_\_\_\_\_\_\_\_ \\
\bottomrule()
\end{longtable}

\newpage

\hypertarget{excerpt-from-nba-constitution}{%
\subsection{EXCERPT FROM NBA CONSTITUTION}\label{excerpt-from-nba-constitution}}

\hypertarget{misconduct}{%
\subsubsection{MISCONDUCT}\label{misconduct}}

\begin{enumerate}
\def\labelenumi{\arabic{enumi}.}
\setcounter{enumi}{34}
\tightlist
\item
  The provisions of this Article 35 shall govern all Players in the Association, hereinafter referred to as ``Players.''

  \begin{enumerate}
  \def\labelenumii{(\alph{enumii})}
  \tightlist
  \item
    Each Member shall provide and require in every contract with any of its Players that they shall be bound and governed by the provisions of this Article. Each Member, at the direction of the Board of Governors or the Commissioner, as the case may be, shall take such action as the Board or the Commissioner may direct in order to effectuate the purposes of this Article.
  \item
    The Commissioner shall direct the dismissal and perpetual disqualification from any further association with the Association or any of its Members, of any Player found by the Commissioner after a hearing to have been guilty of offering, agreeing, conspiring, aiding or attempting to cause any game of basketball to result otherwise than on its merits.
  \item
    If in the opinion of the Commissioner any act or conduct of a Player at or during an Exhibition, Regular Season, or Playoff game has been prejudicial to or against the best interests of the Association or the game of basketball, the Commissioner shall impose upon such Player a fine not exceeding \$50,000, or may order for a time the suspension of any such Player from any connection or duties with Exhibition, Regular Season, or Playoff games, or he may order both such fine and suspension.
  \item
    The Commissioner shall have the power to suspend for a definite or indefinite period, or to impose a fine not exceeding \$50,000, or inflict both such suspension and fine upon any Player who, in his opinion, (i) shall have made or caused to be made any statement having, or that was designed to have, an effect prejudicial or detrimental to the best interests of basketball or of the Association or of a Member, or (ii) shall have been guilty of conduct that does not conform to standards of morality or fair play, that does not comply at all times with all federal, state, and local laws, or that is prejudicial or detrimental to the Association.
  \item
    Any Player who, directly or indirectly, entices, induces, persuades or attempts to entice, induce, or persuade any Player, Coach, Trainer, General Manager or any other person who is under contract to any other Member of the Association to enter into negotiations for or relating to his services or negotiates or contracts for such services shall, on being charged with such tampering, be given an opportunity to answer such charges after due notice and the Commissioner shall have the power to decide whether or not the charges have been sustained; in the event his decision is that the charges have been sustained, then the Commissioner shall have the power to suspend such Player for a definite or indefinite period, or to impose a fine not exceeding \$50,000, or inflict both such suspension and fine upon any such Player.
  \item
    Any Player who, directly or indirectly, wagers money or anything of value on the outcome of any game played by a Team in the league operated by the Association shall, on being charged with such wagering, be given an opportunity to answer such charges after due notice, and the decision of the Commissioner shall be final, binding and conclusive and unappealable. The penalty for such offense shall be within the absolute and sole discretion of the Commissioner and may include a fine, suspension, expulsion and/or perpetual disqualification from further association with the Association or any of its Members.
  \item
    Except for a penalty imposed under Paragraph (f) of this Article 35: (i) any challenge by a Team to the decisions and acts of the Commissioner pursuant to Article 35 shall be appealable to the Board of Governors, who shall determine such appeals in accordance with such rules and regulations as may be adopted by the Board in its absolute and sole discretion, and (ii) any challenge by a Player to the decisions or acts of the Commissioner pursuant to Article 35 shall be governed by the provisions of Article XXXI of the NBA/NBPA Collective Bargaining Agreement then in effect.
  \end{enumerate}
\end{enumerate}

\newpage

\hypertarget{excerpt-from-nba-by-laws}{%
\subsection{EXCERPT FROM NBA BY-LAWS}\label{excerpt-from-nba-by-laws}}

5.01. \emph{Waiver Right.} Except for sales and trading between Members in accordance with these By-Laws, no Member shall sell, option, or otherwise assign the contract with, right to the services of, or right to negotiate with, a Player without complying with the waiver procedure prescribed by this Constitution and By-Laws.

5.02. \emph{Waiver Price.} The waiver price shall be \$1,000 per Player.

5.03. \emph{Waiver Procedure.} A Member desiring to secure waivers on a Player shall notify the Commissioner or the Commissioner's designee, who shall, on behalf of such Member, immediately notify all other Members of the waiver request. Such Player shall be assumed to have been waived unless a Member shall notify the Commissioner or the Commissioner's designee in accordance with Section 5.04 of a claim to the rights to such Player. Once a Member has notified the Commissioner or the Commissioner's designee of its desire to secure waivers on a Player, such notice may not be withdrawn. A Player remains the financial responsibility of the Member placing him on waivers until the waiver period set by the Commissioner or the Commissioner's designee has expired.

5.04. \emph{Waiver Period.} If the Commissioner or the Commissioner's designee distributes notice of request for waiver at any time between August 15 and the end of the next Season, any Members wishing to claim rights to the Player shall do so by giving notice by telephone and in a Writing of such claim to the Commissioner or the Commissioner's designee within forty-eight (48) hours after the time of such notice. If the Commissioner or the Commissioner's designee distributes notice of request for waiver at any other time, any Member wishing to claim rights to the Player shall do so by providing Written Notice of such claim to the Commissioner or the Commissioner's designee within seven (7) days after the date of such notice. A Team may not withdraw a claim to the rights to a Player on waivers.

5.05. \emph{Waiver Preferences.}
(a) In the event that more than one (1) Member shall have claimed the rights to a Player placed on waivers, the claiming Member with the lowest team standing at the time the waiver was requested shall be entitled to acquire the rights to such Player. If the request for waiver shall occur after the last day of the Season and before 11:59 p.m. eastern time on the following November 30, the standings at the close of the previous Season shall govern.
(b) If the winning percentage of two (2) claiming Teams are the same, then the tie shall be determined, if possible, on the basis of the Regular Season Games between the two (2) Teams during the Season or during the preceding Season, as the case may be. If still tied, a toss of a coin shall determine priority. For the purpose of determining standings, both Conferences of the Association shall be deemed merged and a consolidated standing shall control.

5.06. \emph{Players Acquired Through Waivers.} A Member who has acquired the rights and title to the contract of a Player through the waiver procedure may not sell or trade such rights for a period of thirty (30) days after the acquisition thereof; provided, however, that if the rights to such Player were acquired between Seasons, the 30-day period described herein shall begin on the first day of the next succeeding Season.

5.07. \emph{Additional Waiver Rules.} The Commissioner or the Board of Governors may from time to time adopt additional rules (supplementary to those set forth in this Section 5) with respect to the operation of the waiver procedure. Such rules shall not be inconsistent with the provisions of this Section 5 and shall apply to but shall not be limited to the mechanics of notice, inadvertent omission of notification to a Member, and rules of construction as to time.

\newpage

\hypertarget{agent-certification}{%
\subsection{AGENT CERTIFICATION}\label{agent-certification}}

(To be completed only if Player was represented by an agent who negotiated the terms of this Contract.)

I, the undersigned, having negotiated this Contract on behalf of \_\_\_\_\_\_\_\_\_\_\_\_\_\_\_, do hereby swear and certify, under penalties of perjury, that the terms of Paragraph 20 of this Contract (``Entire Agreement'') are true and correct to the best of my knowledge and belief.

\begin{longtable}[]{@{}l@{}}
\toprule()
\endhead
Player Representative: \_\_\_\_\_\_\_\_\_\_\_\_\_\_\_\_\_ \\
State of \_\_\_\_\_\_\_\_\_\_\_\_\_\_\_\_\_\_\_\_\_\_\_\_\_\_\_\_\_\_\_ \\
County of \_\_\_\_\_\_\_\_\_\_\_\_\_\_\_\_\_\_\_\_\_\_\_\_\_\_\_\_\_\_ \\
\bottomrule()
\end{longtable}

On \_\_\_\_\_\_\_\_\_\_\_\_\_\_\_\_\_\_\_\_\_, before me personally came \_\_\_\_\_\_\_\_\_\_\_\_\_\_\_\_\_\_\_\_\_ and acknowledged to me that he/she had executed the foregoing Agent Certification.

\begin{longtable}[]{@{}l@{}}
\toprule()
\endhead
Notary Public: \_\_\_\_\_\_\_\_\_\_\_\_\_\_\_\_\_\_\_\_\_\_\_\_ \\
\bottomrule()
\end{longtable}

\newpage

\hypertarget{uniform-player-contract-1}{%
\section{UNIFORM PLAYER CONTRACT}\label{uniform-player-contract-1}}

\hypertarget{exhibit-1-compensation}{%
\subsection{Exhibit 1 --- Compensation}\label{exhibit-1-compensation}}

Player: \_\_\_\_\_\_\_\_\_\_\_\_\_\_\_\_\_\_\_\_\_\_\_\_\_\_\\
Team: \_\_\_\_\_\_\_\_\_\_\_\_\_\_\_\_\_\_\_\_\_\_\_\_\_\_\_\_\\
Date: \_\_\_\_\_\_\_\_\_\_\_\_\_\_\_\_\_\_\_\_\_\_\_\_\_\_\_\_

\begin{longtable}[]{@{}
  >{\centering\arraybackslash}p{(\columnwidth - 4\tabcolsep) * \real{0.1071}}
  >{\centering\arraybackslash}p{(\columnwidth - 4\tabcolsep) * \real{0.4464}}
  >{\centering\arraybackslash}p{(\columnwidth - 4\tabcolsep) * \real{0.4464}}@{}}
\toprule()
\begin{minipage}[b]{\linewidth}\centering
Season
\end{minipage} & \begin{minipage}[b]{\linewidth}\centering
Current Base Compensation
\end{minipage} & \begin{minipage}[b]{\linewidth}\centering
Deferred Base Compensation
\end{minipage} \\
\midrule()
\endhead
\_\_\_\_\_\_\_\_ & \_\_\_\_\_\_\_\_\_\_\_\_\_\_\_\_\_\_\_\_\_\_\_ & \_\_\_\_\_\_\_\_\_\_\_\_\_\_\_\_\_\_\_\_\_\_\_\_\_ \\
\_\_\_\_\_\_\_\_ & \_\_\_\_\_\_\_\_\_\_\_\_\_\_\_\_\_\_\_\_\_\_\_ & \_\_\_\_\_\_\_\_\_\_\_\_\_\_\_\_\_\_\_\_\_\_\_\_\_ \\
\_\_\_\_\_\_\_\_ & \_\_\_\_\_\_\_\_\_\_\_\_\_\_\_\_\_\_\_\_\_\_\_ & \_\_\_\_\_\_\_\_\_\_\_\_\_\_\_\_\_\_\_\_\_\_\_\_\_ \\
\_\_\_\_\_\_\_\_ & \_\_\_\_\_\_\_\_\_\_\_\_\_\_\_\_\_\_\_\_\_\_\_ & \_\_\_\_\_\_\_\_\_\_\_\_\_\_\_\_\_\_\_\_\_\_\_\_\_ \\
\_\_\_\_\_\_\_\_ & \_\_\_\_\_\_\_\_\_\_\_\_\_\_\_\_\_\_\_\_\_\_\_ & \_\_\_\_\_\_\_\_\_\_\_\_\_\_\_\_\_\_\_\_\_\_\_\_\_ \\
\bottomrule()
\end{longtable}

\textbf{Payment Schedule} (if different from paragraph 3):

Current Base:

Deferred Base:

\textbf{Signing Bonus} (include dates of payment):

\textbf{Incentive Compensation} (include dates of payment):

\textbf{Other Arrangements:}

\begin{longtable}[]{@{}ll@{}}
\toprule()
Initialed: & \\
\midrule()
\endhead
\_\_\_\_\_\_\_\_\_\_\_\_\_\_ & \_\_\_\_\_\_\_\_\_\_\_\_\_\_ \\
Player & Team \\
\bottomrule()
\end{longtable}

\newpage

\hypertarget{exhibit-1a-compensation-minimum-player-salary}{%
\subsection{Exhibit 1A --- Compensation: Minimum Player Salary}\label{exhibit-1a-compensation-minimum-player-salary}}

Player: \_\_\_\_\_\_\_\_\_\_\_\_\_\_\_\_\_\_\_\_\_\_\_\_\_\_\_\_\\
Team: \_\_\_\_\_\_\_\_\_\_\_\_\_\_\_\_\_\_\_\_\_\_\_\_\_\_\_\_\\
Date: \_\_\_\_\_\_\_\_\_\_\_\_\_\_\_\_\_\_\_\_\_\_\_\_\_\_\_\_

\begin{longtable}[]{@{}
  >{\centering\arraybackslash}p{(\columnwidth - 4\tabcolsep) * \real{0.1071}}
  >{\centering\arraybackslash}p{(\columnwidth - 4\tabcolsep) * \real{0.4464}}
  >{\centering\arraybackslash}p{(\columnwidth - 4\tabcolsep) * \real{0.4464}}@{}}
\toprule()
\begin{minipage}[b]{\linewidth}\centering
Season
\end{minipage} & \begin{minipage}[b]{\linewidth}\centering
Current Base Compensation
\end{minipage} & \begin{minipage}[b]{\linewidth}\centering
Deferred Base Compensation
\end{minipage} \\
\midrule()
\endhead
\_\_\_\_\_\_\_\_ & \_\_\_\_\_\_\_\_\_\_\_\_\_\_\_\_\_\_\_\_\_\_\_ & \_\_\_\_\_\_\_\_\_\_\_\_\_\_\_\_\_\_\_\_\_\_\_\_\_ \\
\_\_\_\_\_\_\_\_ & \_\_\_\_\_\_\_\_\_\_\_\_\_\_\_\_\_\_\_\_\_\_\_ & \_\_\_\_\_\_\_\_\_\_\_\_\_\_\_\_\_\_\_\_\_\_\_\_\_ \\
\_\_\_\_\_\_\_\_ & \_\_\_\_\_\_\_\_\_\_\_\_\_\_\_\_\_\_\_\_\_\_\_ & \_\_\_\_\_\_\_\_\_\_\_\_\_\_\_\_\_\_\_\_\_\_\_\_\_ \\
\_\_\_\_\_\_\_\_ & \_\_\_\_\_\_\_\_\_\_\_\_\_\_\_\_\_\_\_\_\_\_\_ & \_\_\_\_\_\_\_\_\_\_\_\_\_\_\_\_\_\_\_\_\_\_\_\_\_ \\
\_\_\_\_\_\_\_\_ & \_\_\_\_\_\_\_\_\_\_\_\_\_\_\_\_\_\_\_\_\_\_\_ & \_\_\_\_\_\_\_\_\_\_\_\_\_\_\_\_\_\_\_\_\_\_\_\_\_ \\
\bottomrule()
\end{longtable}

\textbf{This Contract is intended to provide for a Base Compensation for the \_\_\_\_\_\_\_\_\_\_\_\_\_\_ Season(s) equal to the Minimum Player Salary for such Season(s) (with no bonuses of any kind) and shall be deemed amended to the extent necessary to so provide.}

\textbf{Payment Schedule} (if different from paragraph 3):

\textbf{Other Arrangements:}

\begin{longtable}[]{@{}ll@{}}
\toprule()
Initialed: & \\
\midrule()
\endhead
\_\_\_\_\_\_\_\_\_\_\_\_\_\_ & \_\_\_\_\_\_\_\_\_\_\_\_\_\_ \\
Player & Team \\
\bottomrule()
\end{longtable}

\newpage

\hypertarget{exhibit-2-compensation-protection-or-insurance}{%
\subsection{Exhibit 2 --- Compensation Protection or Insurance}\label{exhibit-2-compensation-protection-or-insurance}}

\begin{longtable}[]{@{}l@{}}
\toprule()
\endhead
Player: \_\_\_\_\_\_\_\_\_\_\_\_\_\_\_\_\_\_\_\_\_\_\_\_\_\_\_\_\_\_\_\_ \\
Team: \_\_\_\_\_\_\_\_\_\_\_\_\_\_\_\_\_\_\_\_\_\_\_\_\_\_\_\_\_\_\_\_\_\_ \\
Date: \_\_\_\_\_\_\_\_\_\_\_\_\_\_\_\_\_\_\_\_\_\_\_\_\_\_\_\_\_\_\_\_\_\_ \\
\bottomrule()
\end{longtable}

\begin{longtable}[]{@{}
  >{\centering\arraybackslash}p{(\columnwidth - 6\tabcolsep) * \real{0.0806}}
  >{\centering\arraybackslash}p{(\columnwidth - 6\tabcolsep) * \real{0.2419}}
  >{\centering\arraybackslash}p{(\columnwidth - 6\tabcolsep) * \real{0.2742}}
  >{\centering\arraybackslash}p{(\columnwidth - 6\tabcolsep) * \real{0.4032}}@{}}
\toprule()
\begin{minipage}[b]{\linewidth}\centering
Season
\end{minipage} & \begin{minipage}[b]{\linewidth}\centering
Type of Protection
\end{minipage} & \begin{minipage}[b]{\linewidth}\centering
Amount of Protection or Insurance
\end{minipage} & \begin{minipage}[b]{\linewidth}\centering
Conditions or Limitations
\end{minipage} \\
\midrule()
\endhead
\_\_\_\_\_ & \_\_\_\_\_\_\_\_\_\_\_ & \_\_\_\_\_\_\_\_\_\_\_\_\_\_\_ & \_\_\_\_\_\_\_\_\_\_\_\_\_\_\_\_\_\_ \\
\_\_\_\_\_ & \_\_\_\_\_\_\_\_\_\_\_ & \_\_\_\_\_\_\_\_\_\_\_\_\_\_\_ & \_\_\_\_\_\_\_\_\_\_\_\_\_\_\_\_\_\_ \\
\_\_\_\_\_ & \_\_\_\_\_\_\_\_\_\_\_ & \_\_\_\_\_\_\_\_\_\_\_\_\_\_\_ & \_\_\_\_\_\_\_\_\_\_\_\_\_\_\_\_\_\_ \\
\_\_\_\_\_ & \_\_\_\_\_\_\_\_\_\_\_ & \_\_\_\_\_\_\_\_\_\_\_\_\_\_\_ & \_\_\_\_\_\_\_\_\_\_\_\_\_\_\_\_\_\_ \\
\_\_\_\_\_ & \_\_\_\_\_\_\_\_\_\_\_ & \_\_\_\_\_\_\_\_\_\_\_\_\_\_\_ & \_\_\_\_\_\_\_\_\_\_\_\_\_\_\_\_\_\_ \\
\bottomrule()
\end{longtable}

\begin{longtable}[]{@{}ll@{}}
\toprule()
Initialed: & \\
\midrule()
\endhead
\_\_\_\_\_\_\_\_\_\_\_\_\_\_ & \_\_\_\_\_\_\_\_\_\_\_\_\_\_ \\
Player & Team \\
\bottomrule()
\end{longtable}

\newpage

\hypertarget{exhibit-3-prior-injury-exclusion}{%
\subsection{Exhibit 3 --- Prior Injury Exclusion}\label{exhibit-3-prior-injury-exclusion}}

\begin{longtable}[]{@{}l@{}}
\toprule()
\endhead
Player: \_\_\_\_\_\_\_\_\_\_\_\_\_\_\_\_\_\_\_\_\_\_\_\_\_\_\_\_\_\_\_\_ \\
Team: \_\_\_\_\_\_\_\_\_\_\_\_\_\_\_\_\_\_\_\_\_\_\_\_\_\_\_\_\_\_\_\_\_\_ \\
Date: \_\_\_\_\_\_\_\_\_\_\_\_\_\_\_\_\_\_\_\_\_\_\_\_\_\_\_\_\_\_\_\_\_\_ \\
\bottomrule()
\end{longtable}

The Player's right to receive his Compensation as set forth in paragraphs 7(c), 16(a)(iii), 16(b) of this Contract, or otherwise is limited or eliminated with respect to the following reinjury of the injury or aggravation of the condition set forth below:

\begin{longtable}[]{@{}l@{}}
\toprule()
Describe injury or condition: \\
\midrule()
\endhead
\_\_\_\_\_\_\_\_\_\_\_\_\_\_\_\_\_\_\_\_\_\_\_\_\_\_\_\_\_\_\_\_\_\_\_\_\_\_\_\_\_\_\_\_\_\_\_\_\_\_\_\_\_\_\_\_\_\_\_\_\_ \\
\_\_\_\_\_\_\_\_\_\_\_\_\_\_\_\_\_\_\_\_\_\_\_\_\_\_\_\_\_\_\_\_\_\_\_\_\_\_\_\_\_\_\_\_\_\_\_\_\_\_\_\_\_\_\_\_\_\_\_\_\_ \\
\_\_\_\_\_\_\_\_\_\_\_\_\_\_\_\_\_\_\_\_\_\_\_\_\_\_\_\_\_\_\_\_\_\_\_\_\_\_\_\_\_\_\_\_\_\_\_\_\_\_\_\_\_\_\_\_\_\_\_\_\_ \\
\_\_\_\_\_\_\_\_\_\_\_\_\_\_\_\_\_\_\_\_\_\_\_\_\_\_\_\_\_\_\_\_\_\_\_\_\_\_\_\_\_\_\_\_\_\_\_\_\_\_\_\_\_\_\_\_\_\_\_\_\_ \\
\_\_\_\_\_\_\_\_\_\_\_\_\_\_\_\_\_\_\_\_\_\_\_\_\_\_\_\_\_\_\_\_\_\_\_\_\_\_\_\_\_\_\_\_\_\_\_\_\_\_\_\_\_\_\_\_\_\_\_\_\_ \\
\_\_\_\_\_\_\_\_\_\_\_\_\_\_\_\_\_\_\_\_\_\_\_\_\_\_\_\_\_\_\_\_\_\_\_\_\_\_\_\_\_\_\_\_\_\_\_\_\_\_\_\_\_\_\_\_\_\_\_\_\_ \\
\_\_\_\_\_\_\_\_\_\_\_\_\_\_\_\_\_\_\_\_\_\_\_\_\_\_\_\_\_\_\_\_\_\_\_\_\_\_\_\_\_\_\_\_\_\_\_\_\_\_\_\_\_\_\_\_\_\_\_\_\_ \\
\_\_\_\_\_\_\_\_\_\_\_\_\_\_\_\_\_\_\_\_\_\_\_\_\_\_\_\_\_\_\_\_\_\_\_\_\_\_\_\_\_\_\_\_\_\_\_\_\_\_\_\_\_\_\_\_\_\_\_\_\_ \\
\_\_\_\_\_\_\_\_\_\_\_\_\_\_\_\_\_\_\_\_\_\_\_\_\_\_\_\_\_\_\_\_\_\_\_\_\_\_\_\_\_\_\_\_\_\_\_\_\_\_\_\_\_\_\_\_\_\_\_\_\_ \\
\_\_\_\_\_\_\_\_\_\_\_\_\_\_\_\_\_\_\_\_\_\_\_\_\_\_\_\_\_\_\_\_\_\_\_\_\_\_\_\_\_\_\_\_\_\_\_\_\_\_\_\_\_\_\_\_\_\_\_\_\_ \\
\_\_\_\_\_\_\_\_\_\_\_\_\_\_\_\_\_\_\_\_\_\_\_\_\_\_\_\_\_\_\_\_\_\_\_\_\_\_\_\_\_\_\_\_\_\_\_\_\_\_\_\_\_\_\_\_\_\_\_\_\_ \\
\bottomrule()
\end{longtable}

\begin{longtable}[]{@{}
  >{\raggedright\arraybackslash}p{(\columnwidth - 0\tabcolsep) * \real{1.0000}}@{}}
\toprule()
\begin{minipage}[b]{\linewidth}\raggedright
Describe the extent to which liability for Compensation is limited or eliminated:
\end{minipage} \\
\midrule()
\endhead
\_\_\_\_\_\_\_\_\_\_\_\_\_\_\_\_\_\_\_\_\_\_\_\_\_\_\_\_\_\_\_\_\_\_\_\_\_\_\_\_\_\_\_\_\_\_\_\_\_\_\_\_\_\_\_\_\_\_\_\_\_ \\
\_\_\_\_\_\_\_\_\_\_\_\_\_\_\_\_\_\_\_\_\_\_\_\_\_\_\_\_\_\_\_\_\_\_\_\_\_\_\_\_\_\_\_\_\_\_\_\_\_\_\_\_\_\_\_\_\_\_\_\_\_ \\
\_\_\_\_\_\_\_\_\_\_\_\_\_\_\_\_\_\_\_\_\_\_\_\_\_\_\_\_\_\_\_\_\_\_\_\_\_\_\_\_\_\_\_\_\_\_\_\_\_\_\_\_\_\_\_\_\_\_\_\_\_ \\
\_\_\_\_\_\_\_\_\_\_\_\_\_\_\_\_\_\_\_\_\_\_\_\_\_\_\_\_\_\_\_\_\_\_\_\_\_\_\_\_\_\_\_\_\_\_\_\_\_\_\_\_\_\_\_\_\_\_\_\_\_ \\
\_\_\_\_\_\_\_\_\_\_\_\_\_\_\_\_\_\_\_\_\_\_\_\_\_\_\_\_\_\_\_\_\_\_\_\_\_\_\_\_\_\_\_\_\_\_\_\_\_\_\_\_\_\_\_\_\_\_\_\_\_ \\
\_\_\_\_\_\_\_\_\_\_\_\_\_\_\_\_\_\_\_\_\_\_\_\_\_\_\_\_\_\_\_\_\_\_\_\_\_\_\_\_\_\_\_\_\_\_\_\_\_\_\_\_\_\_\_\_\_\_\_\_\_ \\
\_\_\_\_\_\_\_\_\_\_\_\_\_\_\_\_\_\_\_\_\_\_\_\_\_\_\_\_\_\_\_\_\_\_\_\_\_\_\_\_\_\_\_\_\_\_\_\_\_\_\_\_\_\_\_\_\_\_\_\_\_ \\
\_\_\_\_\_\_\_\_\_\_\_\_\_\_\_\_\_\_\_\_\_\_\_\_\_\_\_\_\_\_\_\_\_\_\_\_\_\_\_\_\_\_\_\_\_\_\_\_\_\_\_\_\_\_\_\_\_\_\_\_\_ \\
\_\_\_\_\_\_\_\_\_\_\_\_\_\_\_\_\_\_\_\_\_\_\_\_\_\_\_\_\_\_\_\_\_\_\_\_\_\_\_\_\_\_\_\_\_\_\_\_\_\_\_\_\_\_\_\_\_\_\_\_\_ \\
\_\_\_\_\_\_\_\_\_\_\_\_\_\_\_\_\_\_\_\_\_\_\_\_\_\_\_\_\_\_\_\_\_\_\_\_\_\_\_\_\_\_\_\_\_\_\_\_\_\_\_\_\_\_\_\_\_\_\_\_\_ \\
\_\_\_\_\_\_\_\_\_\_\_\_\_\_\_\_\_\_\_\_\_\_\_\_\_\_\_\_\_\_\_\_\_\_\_\_\_\_\_\_\_\_\_\_\_\_\_\_\_\_\_\_\_\_\_\_\_\_\_\_\_ \\
\bottomrule()
\end{longtable}

\begin{longtable}[]{@{}ll@{}}
\toprule()
Initialed: & \\
\midrule()
\endhead
\_\_\_\_\_\_\_\_\_\_\_\_\_\_ & \_\_\_\_\_\_\_\_\_\_\_\_\_\_ \\
Player & Team \\
\bottomrule()
\end{longtable}

\newpage

\hypertarget{exhibit-4-trade-payments}{%
\subsection{Exhibit 4 --- Trade Payments}\label{exhibit-4-trade-payments}}

Player: \_\_\_\_\_\_\_\_\_\_\_\_\_\_\_\_\_\_\_\_\_\_\_\_\_\_\_\_\\
Team: \_\_\_\_\_\_\_\_\_\_\_\_\_\_\_\_\_\_\_\_\_\_\_\_\_\_\_\_\\
Date: \_\_\_\_\_\_\_\_\_\_\_\_\_\_\_\_\_\_\_\_\_\_\_\_\_\_\_\_

In the event this Contract is traded by the Team executing the Contract to another NBA Team, the Player shall be entitled to receive from the assignor Team, within thirty (30) days of the date of such trade, the following payment:

\begin{longtable}[]{@{}l@{}}
\toprule()
\endhead
\_\_\_\_\_\_\_\_\_\_\_\_\_\_\_\_\_\_\_\_\_\_\_\_\_\_\_\_\_\_\_\_\_\_\_\_\_\_\_\_\_\_\_\_\_\_\_\_\_\_\_\_\_\_\_\_\_\_\_\_\_ \\
\_\_\_\_\_\_\_\_\_\_\_\_\_\_\_\_\_\_\_\_\_\_\_\_\_\_\_\_\_\_\_\_\_\_\_\_\_\_\_\_\_\_\_\_\_\_\_\_\_\_\_\_\_\_\_\_\_\_\_\_\_ \\
\_\_\_\_\_\_\_\_\_\_\_\_\_\_\_\_\_\_\_\_\_\_\_\_\_\_\_\_\_\_\_\_\_\_\_\_\_\_\_\_\_\_\_\_\_\_\_\_\_\_\_\_\_\_\_\_\_\_\_\_\_ \\
\_\_\_\_\_\_\_\_\_\_\_\_\_\_\_\_\_\_\_\_\_\_\_\_\_\_\_\_\_\_\_\_\_\_\_\_\_\_\_\_\_\_\_\_\_\_\_\_\_\_\_\_\_\_\_\_\_\_\_\_\_ \\
\_\_\_\_\_\_\_\_\_\_\_\_\_\_\_\_\_\_\_\_\_\_\_\_\_\_\_\_\_\_\_\_\_\_\_\_\_\_\_\_\_\_\_\_\_\_\_\_\_\_\_\_\_\_\_\_\_\_\_\_\_ \\
\_\_\_\_\_\_\_\_\_\_\_\_\_\_\_\_\_\_\_\_\_\_\_\_\_\_\_\_\_\_\_\_\_\_\_\_\_\_\_\_\_\_\_\_\_\_\_\_\_\_\_\_\_\_\_\_\_\_\_\_\_ \\
\_\_\_\_\_\_\_\_\_\_\_\_\_\_\_\_\_\_\_\_\_\_\_\_\_\_\_\_\_\_\_\_\_\_\_\_\_\_\_\_\_\_\_\_\_\_\_\_\_\_\_\_\_\_\_\_\_\_\_\_\_ \\
\_\_\_\_\_\_\_\_\_\_\_\_\_\_\_\_\_\_\_\_\_\_\_\_\_\_\_\_\_\_\_\_\_\_\_\_\_\_\_\_\_\_\_\_\_\_\_\_\_\_\_\_\_\_\_\_\_\_\_\_\_ \\
\_\_\_\_\_\_\_\_\_\_\_\_\_\_\_\_\_\_\_\_\_\_\_\_\_\_\_\_\_\_\_\_\_\_\_\_\_\_\_\_\_\_\_\_\_\_\_\_\_\_\_\_\_\_\_\_\_\_\_\_\_ \\
\_\_\_\_\_\_\_\_\_\_\_\_\_\_\_\_\_\_\_\_\_\_\_\_\_\_\_\_\_\_\_\_\_\_\_\_\_\_\_\_\_\_\_\_\_\_\_\_\_\_\_\_\_\_\_\_\_\_\_\_\_ \\
\_\_\_\_\_\_\_\_\_\_\_\_\_\_\_\_\_\_\_\_\_\_\_\_\_\_\_\_\_\_\_\_\_\_\_\_\_\_\_\_\_\_\_\_\_\_\_\_\_\_\_\_\_\_\_\_\_\_\_\_\_ \\
\_\_\_\_\_\_\_\_\_\_\_\_\_\_\_\_\_\_\_\_\_\_\_\_\_\_\_\_\_\_\_\_\_\_\_\_\_\_\_\_\_\_\_\_\_\_\_\_\_\_\_\_\_\_\_\_\_\_\_\_\_ \\
\_\_\_\_\_\_\_\_\_\_\_\_\_\_\_\_\_\_\_\_\_\_\_\_\_\_\_\_\_\_\_\_\_\_\_\_\_\_\_\_\_\_\_\_\_\_\_\_\_\_\_\_\_\_\_\_\_\_\_\_\_ \\
\_\_\_\_\_\_\_\_\_\_\_\_\_\_\_\_\_\_\_\_\_\_\_\_\_\_\_\_\_\_\_\_\_\_\_\_\_\_\_\_\_\_\_\_\_\_\_\_\_\_\_\_\_\_\_\_\_\_\_\_\_ \\
\_\_\_\_\_\_\_\_\_\_\_\_\_\_\_\_\_\_\_\_\_\_\_\_\_\_\_\_\_\_\_\_\_\_\_\_\_\_\_\_\_\_\_\_\_\_\_\_\_\_\_\_\_\_\_\_\_\_\_\_\_ \\
\bottomrule()
\end{longtable}

\begin{longtable}[]{@{}ll@{}}
\toprule()
Initialed: & \\
\midrule()
\endhead
\_\_\_\_\_\_\_\_\_\_\_\_\_\_ & \_\_\_\_\_\_\_\_\_\_\_\_\_\_ \\
Player & Team \\
\bottomrule()
\end{longtable}

\newpage

\hypertarget{exhibit-5-other-activities}{%
\subsection{Exhibit 5 --- Other Activities}\label{exhibit-5-other-activities}}

\begin{longtable}[]{@{}l@{}}
\toprule()
\endhead
Player: \_\_\_\_\_\_\_\_\_\_\_\_\_\_\_\_\_\_\_\_\_\_\_\_\_\_\_\_\_\_\_\_ \\
Team: \_\_\_\_\_\_\_\_\_\_\_\_\_\_\_\_\_\_\_\_\_\_\_\_\_\_\_\_\_\_\_\_\_\_ \\
Date: \_\_\_\_\_\_\_\_\_\_\_\_\_\_\_\_\_\_\_\_\_\_\_\_\_\_\_\_\_\_\_\_\_\_ \\
\bottomrule()
\end{longtable}

Notwithstanding the provisions of paragraph 12 of this Contract, the Player and the Team agree that the Player need not obtain the consent of the Team in order to engage in the activities set forth below:

\begin{longtable}[]{@{}l@{}}
\toprule()
\endhead
\_\_\_\_\_\_\_\_\_\_\_\_\_\_\_\_\_\_\_\_\_\_\_\_\_\_\_\_\_\_\_\_\_\_\_\_\_\_\_\_\_\_\_\_\_\_\_\_\_\_\_\_\_\_\_\_\_\_\_\_\_ \\
\_\_\_\_\_\_\_\_\_\_\_\_\_\_\_\_\_\_\_\_\_\_\_\_\_\_\_\_\_\_\_\_\_\_\_\_\_\_\_\_\_\_\_\_\_\_\_\_\_\_\_\_\_\_\_\_\_\_\_\_\_ \\
\_\_\_\_\_\_\_\_\_\_\_\_\_\_\_\_\_\_\_\_\_\_\_\_\_\_\_\_\_\_\_\_\_\_\_\_\_\_\_\_\_\_\_\_\_\_\_\_\_\_\_\_\_\_\_\_\_\_\_\_\_ \\
\_\_\_\_\_\_\_\_\_\_\_\_\_\_\_\_\_\_\_\_\_\_\_\_\_\_\_\_\_\_\_\_\_\_\_\_\_\_\_\_\_\_\_\_\_\_\_\_\_\_\_\_\_\_\_\_\_\_\_\_\_ \\
\_\_\_\_\_\_\_\_\_\_\_\_\_\_\_\_\_\_\_\_\_\_\_\_\_\_\_\_\_\_\_\_\_\_\_\_\_\_\_\_\_\_\_\_\_\_\_\_\_\_\_\_\_\_\_\_\_\_\_\_\_ \\
\_\_\_\_\_\_\_\_\_\_\_\_\_\_\_\_\_\_\_\_\_\_\_\_\_\_\_\_\_\_\_\_\_\_\_\_\_\_\_\_\_\_\_\_\_\_\_\_\_\_\_\_\_\_\_\_\_\_\_\_\_ \\
\_\_\_\_\_\_\_\_\_\_\_\_\_\_\_\_\_\_\_\_\_\_\_\_\_\_\_\_\_\_\_\_\_\_\_\_\_\_\_\_\_\_\_\_\_\_\_\_\_\_\_\_\_\_\_\_\_\_\_\_\_ \\
\_\_\_\_\_\_\_\_\_\_\_\_\_\_\_\_\_\_\_\_\_\_\_\_\_\_\_\_\_\_\_\_\_\_\_\_\_\_\_\_\_\_\_\_\_\_\_\_\_\_\_\_\_\_\_\_\_\_\_\_\_ \\
\_\_\_\_\_\_\_\_\_\_\_\_\_\_\_\_\_\_\_\_\_\_\_\_\_\_\_\_\_\_\_\_\_\_\_\_\_\_\_\_\_\_\_\_\_\_\_\_\_\_\_\_\_\_\_\_\_\_\_\_\_ \\
\_\_\_\_\_\_\_\_\_\_\_\_\_\_\_\_\_\_\_\_\_\_\_\_\_\_\_\_\_\_\_\_\_\_\_\_\_\_\_\_\_\_\_\_\_\_\_\_\_\_\_\_\_\_\_\_\_\_\_\_\_ \\
\_\_\_\_\_\_\_\_\_\_\_\_\_\_\_\_\_\_\_\_\_\_\_\_\_\_\_\_\_\_\_\_\_\_\_\_\_\_\_\_\_\_\_\_\_\_\_\_\_\_\_\_\_\_\_\_\_\_\_\_\_ \\
\_\_\_\_\_\_\_\_\_\_\_\_\_\_\_\_\_\_\_\_\_\_\_\_\_\_\_\_\_\_\_\_\_\_\_\_\_\_\_\_\_\_\_\_\_\_\_\_\_\_\_\_\_\_\_\_\_\_\_\_\_ \\
\_\_\_\_\_\_\_\_\_\_\_\_\_\_\_\_\_\_\_\_\_\_\_\_\_\_\_\_\_\_\_\_\_\_\_\_\_\_\_\_\_\_\_\_\_\_\_\_\_\_\_\_\_\_\_\_\_\_\_\_\_ \\
\_\_\_\_\_\_\_\_\_\_\_\_\_\_\_\_\_\_\_\_\_\_\_\_\_\_\_\_\_\_\_\_\_\_\_\_\_\_\_\_\_\_\_\_\_\_\_\_\_\_\_\_\_\_\_\_\_\_\_\_\_ \\
\_\_\_\_\_\_\_\_\_\_\_\_\_\_\_\_\_\_\_\_\_\_\_\_\_\_\_\_\_\_\_\_\_\_\_\_\_\_\_\_\_\_\_\_\_\_\_\_\_\_\_\_\_\_\_\_\_\_\_\_\_ \\
\bottomrule()
\end{longtable}

\begin{longtable}[]{@{}ll@{}}
\toprule()
Initialed: & \\
\midrule()
\endhead
\_\_\_\_\_\_\_\_\_\_\_\_\_\_ & \_\_\_\_\_\_\_\_\_\_\_\_\_\_ \\
Player & Team \\
\bottomrule()
\end{longtable}

\newpage

\hypertarget{exhibit-6-physical-exam}{%
\subsection{Exhibit 6 --- Physical Exam}\label{exhibit-6-physical-exam}}

\begin{longtable}[]{@{}l@{}}
\toprule()
\endhead
Player: \_\_\_\_\_\_\_\_\_\_\_\_\_\_\_\_\_\_\_\_\_\_\_\_\_\_\_\_\_\_\_\_ \\
Team: \_\_\_\_\_\_\_\_\_\_\_\_\_\_\_\_\_\_\_\_\_\_\_\_\_\_\_\_\_\_\_\_\_\_ \\
Date: \_\_\_\_\_\_\_\_\_\_\_\_\_\_\_\_\_\_\_\_\_\_\_\_\_\_\_\_\_\_\_\_\_\_ \\
\bottomrule()
\end{longtable}

The Player and the Team agree that this Contract will be invalid and of no force and effect unless the Player passes, in the sole discretion of a physician designated by the Team, a physical examination in accordance with Article II, Section 12(h) of the CBA that is (i) conducted within three (3) business days of the execution of this Contract, and (ii) the results of which are reported by the Team to the player within six (6) business days of the execution of this Contract. The Player agrees to supply complete and truthful information in connection with any such examinations.

\begin{longtable}[]{@{}ll@{}}
\toprule()
Initialed: & \\
\midrule()
\endhead
\_\_\_\_\_\_\_\_\_\_\_\_\_\_ & \_\_\_\_\_\_\_\_\_\_\_\_\_\_ \\
Player & Team \\
\bottomrule()
\end{longtable}

\newpage

\hypertarget{exhibit-7-substitution-for-upc-paragraph-7b}{%
\subsection{Exhibit 7 --- Substitution for UPC Paragraph 7(b)}\label{exhibit-7-substitution-for-upc-paragraph-7b}}

\begin{longtable}[]{@{}l@{}}
\toprule()
\endhead
Player: \_\_\_\_\_\_\_\_\_\_\_\_\_\_\_\_\_\_\_\_\_\_\_\_\_\_\_\_\_\_\_\_ \\
Team: \_\_\_\_\_\_\_\_\_\_\_\_\_\_\_\_\_\_\_\_\_\_\_\_\_\_\_\_\_\_\_\_\_\_ \\
Date: \_\_\_\_\_\_\_\_\_\_\_\_\_\_\_\_\_\_\_\_\_\_\_\_\_\_\_\_\_\_\_\_\_\_ \\
\bottomrule()
\end{longtable}

Paragraph 7(b) is hereby deleted and the following shall be substituted in place and instead thereof:

\begin{quote}
``7. (b) The Player agrees, notwithstanding any other provision of this Contract, that he will to the best of his ability maintain himself in physical condition sufficient to play skilled basketball at all times. If the Player, in the reasonable judgment of the physician designated for that purpose by the Team, is not in good physical condition at the date of his first scheduled game for the Team, or if, at the beginning of or during any Season, he fails to remain in good physical condition, in either event so as to render the Player unfit in the reasonable judgment of said physician to play skilled basketball, the Team shall have the right to suspend the Player for successive one-week periods until the Player, in the reasonable judgment of the Team's physician, is in good physical condition; provided, however, that at the end of each such one-week period of suspension, if the Team notifies the Player, orally or in writing, that in its reasonable judgment it believes the Player is still not in good physical condition, and if the Player so requests, then the Player shall be examined by a physician or physicians designated for such purpose by the President, or any Vice President if the President is not available, of the American Society of Orthopedic Physicians, or equivalent organization (the''Reviewing Physician''), whose sole judgment concerning the physical condition of the Player to play skilled basketball shall be binding upon the Team and the Player for purposes of this paragraph. The suspension of the Player shall be terminated promptly upon the failure of the Team to give the Player the notice required at the end of the one-week period or upon the finding of said Reviewing Physician that the Player is in physical condition sufficient to play skilled basketball. In the event of a suspension permitted hereunder, the Compensation (excluding any signing bonus or Incentive Compensation) payable to the Player for any Season during such suspension shall be reduced in the same proportion as the length of the period of disability so determined bears to the length of the Season. Nothing in this paragraph 7(b) shall authorize the Team to suspend the Player solely because the Player is injured or ill.''
\end{quote}

\begin{longtable}[]{@{}ll@{}}
\toprule()
Initialed: & \\
\midrule()
\endhead
\_\_\_\_\_\_\_\_\_\_\_\_\_\_ & \_\_\_\_\_\_\_\_\_\_\_\_\_\_ \\
Player & Team \\
\bottomrule()
\end{longtable}

\hypertarget{exhibit-8-sign-and-trade}{%
\subsection{Exhibit 8 --- Sign and Trade}\label{exhibit-8-sign-and-trade}}

\begin{longtable}[]{@{}l@{}}
\toprule()
\endhead
Player: \_\_\_\_\_\_\_\_\_\_\_\_\_\_\_\_\_\_\_\_\_\_\_\_\_\_\_\_\_\_\_\_ \\
Team: \_\_\_\_\_\_\_\_\_\_\_\_\_\_\_\_\_\_\_\_\_\_\_\_\_\_\_\_\_\_\_\_\_\_ \\
Date: \_\_\_\_\_\_\_\_\_\_\_\_\_\_\_\_\_\_\_\_\_\_\_\_\_\_\_\_\_\_\_\_\_\_ \\
\bottomrule()
\end{longtable}

The Player and the Team agree that this {[}Contract{]} {[}amendment{]} will be invalid and of no force and effect unless the Contract is traded to the {[}assignee team{]} within forty-eight (48) hours of its execution, and all conditions to such trade are ultimately satisfied.

\begin{longtable}[]{@{}ll@{}}
\toprule()
Initialed: & \\
\midrule()
\endhead
\_\_\_\_\_\_\_\_\_\_\_\_\_\_ & \_\_\_\_\_\_\_\_\_\_\_\_\_\_ \\
Player & Team \\
\bottomrule()
\end{longtable}

\hypertarget{rookie-scales}{%
\chapter{ROOKIE SCALES}\label{rookie-scales}}

\newpage

\hypertarget{nba-rookie-scale-000s}{%
\section{2005-06 NBA Rookie Scale (\$000's)}\label{nba-rookie-scale-000s}}

\begin{longtable}[]{@{}
  >{\centering\arraybackslash}p{(\columnwidth - 10\tabcolsep) * \real{0.0806}}
  >{\raggedright\arraybackslash}p{(\columnwidth - 10\tabcolsep) * \real{0.1452}}
  >{\raggedright\arraybackslash}p{(\columnwidth - 10\tabcolsep) * \real{0.1613}}
  >{\raggedright\arraybackslash}p{(\columnwidth - 10\tabcolsep) * \real{0.1613}}
  >{\centering\arraybackslash}p{(\columnwidth - 10\tabcolsep) * \real{0.2258}}
  >{\centering\arraybackslash}p{(\columnwidth - 10\tabcolsep) * \real{0.2258}}@{}}
\toprule()
\begin{minipage}[b]{\linewidth}\centering
Pick
\end{minipage} & \begin{minipage}[b]{\linewidth}\raggedright
1st Year Salary
\end{minipage} & \begin{minipage}[b]{\linewidth}\raggedright
2nd Year Salary
\end{minipage} & \begin{minipage}[b]{\linewidth}\raggedright
3rd Year Option Salary
\end{minipage} & \begin{minipage}[b]{\linewidth}\centering
4th Year Option: Percentage Increase Over 3rd Year Salary
\end{minipage} & \begin{minipage}[b]{\linewidth}\centering
Qualifying Offer: Percentage Increase Over 4th Year Salary
\end{minipage} \\
\midrule()
\endhead
1 & 3,617.1 & 3,888.3 & 4,159.6 & 26.1\% & 30.0\% \\
2 & 3,236.3 & 3,479.0 & 3,721.7 & 26.2\% & 30.5\% \\
3 & 2,906.2 & 3,124.2 & 3,342.2 & 26.4\% & 31.2\% \\
4 & 2,620.2 & 2,816.8 & 3,013.3 & 26.5\% & 31.9\% \\
5 & 2,372.8 & 2,550.7 & 2,728.7 & 26.7\% & 32.6\% \\
6 & 2,155.1 & 2,316.8 & 2,478.4 & 26.8\% & 33.4\% \\
7 & 1,967.4 & 2,114.9 & 2,262.5 & 27.0\% & 34.1\% \\
8 & 1,802.4 & 1,937.5 & 2,072.7 & 27.2\% & 34.8\% \\
9 & 1,656.8 & 1,781.0 & 1,905.3 & 27.4\% & 35.5\% \\
10 & 1,573.9 & 1,691.9 & 1,810.0 & 27.5\% & 36.2\% \\
11 & 1,495.2 & 1,607.3 & 1,719.5 & 32.7\% & 36.9\% \\
12 & 1,420.4 & 1,526.9 & 1,633.5 & 37.8\% & 37.6\% \\
13 & 1,349.4 & 1,450.6 & 1,551.8 & 42.9\% & 38.3\% \\
14 & 1,282.0 & 1,378.1 & 1,474.3 & 48.1\% & 39.1\% \\
15 & 1,217.8 & 1,309.1 & 1,400.5 & 53.3\% & 39.8\% \\
16 & 1,157.0 & 1,243.7 & 1,330.5 & 53.4\% & 40.5\% \\
17 & 1,099.1 & 1,181.5 & 1,264.0 & 53.6\% & 41.2\% \\
18 & 1,044.2 & 1,122.5 & 1,200.8 & 53.8\% & 41.9\% \\
19 & 997.1 & 1,071.9 & 1,146.7 & 54.0\% & 42.6\% \\
20 & 957.3 & 1,029.0 & 1,100.8 & 54.2\% & 43.3\% \\
21 & 918.9 & 987.9 & 1,056.8 & 59.3\% & 44.1\% \\
22 & 882.2 & 948.3 & 1,014.5 & 64.5\% & 44.8\% \\
23 & 847.0 & 910.5 & 974.0 & 69.7\% & 45.5\% \\
24 & 813.0 & 874.0 & 935.0 & 74.9\% & 46.2\% \\
25 & 780.5 & 839.0 & 897.6 & 80.1\% & 46.9\% \\
26 & 754.7 & 811.2 & 867.8 & 80.3\% & 47.6\% \\
27 & 732.8 & 787.8 & 842.8 & 80.4\% & 48.3\% \\
28 & 728.4 & 783.0 & 837.6 & 80.5\% & 49.0\% \\
29 & 723.1 & 777.3 & 831.5 & 80.5\% & 50.0\% \\
30 & 717.8 & 771.7 & 825.5 & 80.5\% & 50.0\% \\
\bottomrule()
\end{longtable}

\newpage

\hypertarget{nba-rookie-scale-000s-1}{%
\section{2006-07 NBA Rookie Scale (\$000's)}\label{nba-rookie-scale-000s-1}}

\begin{longtable}[]{@{}
  >{\centering\arraybackslash}p{(\columnwidth - 10\tabcolsep) * \real{0.0806}}
  >{\raggedright\arraybackslash}p{(\columnwidth - 10\tabcolsep) * \real{0.1452}}
  >{\raggedright\arraybackslash}p{(\columnwidth - 10\tabcolsep) * \real{0.1613}}
  >{\raggedright\arraybackslash}p{(\columnwidth - 10\tabcolsep) * \real{0.1613}}
  >{\centering\arraybackslash}p{(\columnwidth - 10\tabcolsep) * \real{0.2258}}
  >{\centering\arraybackslash}p{(\columnwidth - 10\tabcolsep) * \real{0.2258}}@{}}
\toprule()
\begin{minipage}[b]{\linewidth}\centering
Pick
\end{minipage} & \begin{minipage}[b]{\linewidth}\raggedright
1st Year Salary
\end{minipage} & \begin{minipage}[b]{\linewidth}\raggedright
2nd Year Salary
\end{minipage} & \begin{minipage}[b]{\linewidth}\raggedright
3rd Year Option Salary
\end{minipage} & \begin{minipage}[b]{\linewidth}\centering
4th Year Option: Percentage Increase Over 3rd Year Salary
\end{minipage} & \begin{minipage}[b]{\linewidth}\centering
Qualifying Offer: Percentage Increase Over 4th Year Salary
\end{minipage} \\
\midrule()
\endhead
1 & 3,751.0 & 4,032.4 & 4,313.7 & 26.1\% & 30.0\% \\
2 & 3,356.1 & 3,607.8 & 3,859.5 & 26.2\% & 30.5\% \\
3 & 3,013.9 & 3,239.9 & 3,466.0 & 26.4\% & 31.2\% \\
4 & 2,717.3 & 2,921.1 & 3,124.9 & 26.5\% & 31.9\% \\
5 & 2,460.7 & 2,645.2 & 2,829.8 & 26.7\% & 32.6\% \\
6 & 2,234.9 & 2,402.6 & 2,570.2 & 26.8\% & 33.4\% \\
7 & 2,040.2 & 2,193.2 & 2,346.3 & 27.0\% & 34.1\% \\
8 & 1,869.1 & 2,009.3 & 2,149.5 & 27.2\% & 34.8\% \\
9 & 1,718.1 & 1,847.0 & 1,975.8 & 27.4\% & 35.5\% \\
10 & 1,632.2 & 1,754.6 & 1,877.0 & 27.5\% & 36.2\% \\
11 & 1,550.6 & 1,666.8 & 1,783.1 & 32.7\% & 36.9\% \\
12 & 1,473.0 & 1,583.5 & 1,694.0 & 37.8\% & 37.6\% \\
13 & 1,399.4 & 1,504.3 & 1,609.3 & 42.9\% & 38.3\% \\
14 & 1,329.5 & 1,429.2 & 1,528.9 & 48.1\% & 39.1\% \\
15 & 1,262.9 & 1,357.6 & 1,452.3 & 53.3\% & 39.8\% \\
16 & 1,199.8 & 1,289.8 & 1,379.8 & 53.4\% & 40.5\% \\
17 & 1,139.8 & 1,225.3 & 1,310.8 & 53.6\% & 41.2\% \\
18 & 1,082.8 & 1,164.1 & 1,245.3 & 53.8\% & 41.9\% \\
19 & 1,034.1 & 1,111.6 & 1,189.2 & 54.0\% & 42.6\% \\
20 & 992.7 & 1,067.2 & 1,141.6 & 54.2\% & 43.3\% \\
21 & 953.0 & 1,024.4 & 1,095.9 & 59.3\% & 44.1\% \\
22 & 914.9 & 983.5 & 1,052.1 & 64.5\% & 44.8\% \\
23 & 878.3 & 944.2 & 1,010.1 & 69.7\% & 45.5\% \\
24 & 843.1 & 906.4 & 969.6 & 74.9\% & 46.2\% \\
25 & 809.4 & 870.1 & 930.8 & 80.1\% & 46.9\% \\
26 & 782.6 & 841.3 & 900.0 & 80.3\% & 47.6\% \\
27 & 760.0 & 817.0 & 874.0 & 80.4\% & 48.3\% \\
28 & 755.4 & 812.0 & 868.7 & 80.5\% & 49.0\% \\
29 & 749.9 & 806.1 & 862.3 & 80.5\% & 50.0\% \\
30 & 744.4 & 800.2 & 856.1 & 80.5\% & 50.0\% \\
\bottomrule()
\end{longtable}

\newpage

\hypertarget{nba-rookie-scale-000s-2}{%
\section{2007-08 NBA Rookie Scale (\$000's)}\label{nba-rookie-scale-000s-2}}

\begin{longtable}[]{@{}
  >{\centering\arraybackslash}p{(\columnwidth - 10\tabcolsep) * \real{0.0806}}
  >{\raggedright\arraybackslash}p{(\columnwidth - 10\tabcolsep) * \real{0.1452}}
  >{\raggedright\arraybackslash}p{(\columnwidth - 10\tabcolsep) * \real{0.1613}}
  >{\raggedright\arraybackslash}p{(\columnwidth - 10\tabcolsep) * \real{0.1613}}
  >{\centering\arraybackslash}p{(\columnwidth - 10\tabcolsep) * \real{0.2258}}
  >{\centering\arraybackslash}p{(\columnwidth - 10\tabcolsep) * \real{0.2258}}@{}}
\toprule()
\begin{minipage}[b]{\linewidth}\centering
Pick
\end{minipage} & \begin{minipage}[b]{\linewidth}\raggedright
1st Year Salary
\end{minipage} & \begin{minipage}[b]{\linewidth}\raggedright
2nd Year Salary
\end{minipage} & \begin{minipage}[b]{\linewidth}\raggedright
3rd Year Option Salary
\end{minipage} & \begin{minipage}[b]{\linewidth}\centering
4th Year Option: Percentage Increase Over 3rd Year Salary
\end{minipage} & \begin{minipage}[b]{\linewidth}\centering
Qualifying Offer: Percentage Increase Over 4th Year Salary
\end{minipage} \\
\midrule()
\endhead
1 & 3,885.0 & 4,176.4 & 4,467.7 & 26.1\% & 30.0\% \\
2 & 3,476.0 & 3,736.7 & 3,997.4 & 26.2\% & 30.5\% \\
3 & 3,121.5 & 3,355.6 & 3,589.7 & 26.4\% & 31.2\% \\
4 & 2,814.3 & 3,025.4 & 3,236.5 & 26.5\% & 31.9\% \\
5 & 2,548.5 & 2,739.7 & 2,930.8 & 26.7\% & 32.6\% \\
6 & 2,314.8 & 2,488.4 & 2,662.0 & 26.8\% & 33.4\% \\
7 & 2,113.1 & 2,271.6 & 2,430.1 & 27.0\% & 34.1\% \\
8 & 1,935.9 & 2,081.1 & 2,226.2 & 27.2\% & 34.8\% \\
9 & 1,779.5 & 1,912.9 & 2,046.4 & 27.4\% & 35.5\% \\
10 & 1,690.5 & 1,817.3 & 1,944.0 & 27.5\% & 36.2\% \\
11 & 1,605.9 & 1,726.4 & 1,846.8 & 32.7\% & 36.9\% \\
12 & 1,525.6 & 1,640.1 & 1,754.5 & 37.8\% & 37.6\% \\
13 & 1,449.3 & 1,558.0 & 1,666.7 & 42.9\% & 38.3\% \\
14 & 1,376.9 & 1,480.2 & 1,583.5 & 48.1\% & 39.1\% \\
15 & 1,308.0 & 1,406.1 & 1,504.2 & 53.3\% & 39.8\% \\
16 & 1,242.7 & 1,335.8 & 1,429.0 & 53.4\% & 40.5\% \\
17 & 1,180.5 & 1,269.1 & 1,357.6 & 53.6\% & 41.2\% \\
18 & 1,121.5 & 1,205.6 & 1,289.7 & 53.8\% & 41.9\% \\
19 & 1,071.0 & 1,151.3 & 1,231.6 & 54.0\% & 42.6\% \\
20 & 1,028.2 & 1,105.3 & 1,182.4 & 54.2\% & 43.3\% \\
21 & 987.0 & 1,061.0 & 1,135.1 & 59.3\% & 44.1\% \\
22 & 947.5 & 1,018.6 & 1,089.7 & 64.5\% & 44.8\% \\
23 & 909.7 & 977.9 & 1,046.2 & 69.7\% & 45.5\% \\
24 & 873.2 & 938.7 & 1,004.2 & 74.9\% & 46.2\% \\
25 & 838.3 & 901.2 & 964.1 & 80.1\% & 46.9\% \\
26 & 810.6 & 871.3 & 932.1 & 80.3\% & 47.6\% \\
27 & 787.1 & 846.2 & 905.2 & 80.4\% & 48.3\% \\
28 & 782.3 & 841.0 & 899.7 & 80.5\% & 49.0\% \\
29 & 776.6 & 834.9 & 893.1 & 80.5\% & 50.0\% \\
30 & 771.0 & 828.8 & 886.6 & 80.5\% & 50.0\% \\
\bottomrule()
\end{longtable}

\newpage

\hypertarget{nba-rookie-scale-000s-3}{%
\section{2008-09 NBA Rookie Scale (\$000's)}\label{nba-rookie-scale-000s-3}}

\begin{longtable}[]{@{}
  >{\centering\arraybackslash}p{(\columnwidth - 10\tabcolsep) * \real{0.0806}}
  >{\raggedright\arraybackslash}p{(\columnwidth - 10\tabcolsep) * \real{0.1452}}
  >{\raggedright\arraybackslash}p{(\columnwidth - 10\tabcolsep) * \real{0.1613}}
  >{\raggedright\arraybackslash}p{(\columnwidth - 10\tabcolsep) * \real{0.1613}}
  >{\centering\arraybackslash}p{(\columnwidth - 10\tabcolsep) * \real{0.2258}}
  >{\centering\arraybackslash}p{(\columnwidth - 10\tabcolsep) * \real{0.2258}}@{}}
\toprule()
\begin{minipage}[b]{\linewidth}\centering
Pick
\end{minipage} & \begin{minipage}[b]{\linewidth}\raggedright
1st Year Salary
\end{minipage} & \begin{minipage}[b]{\linewidth}\raggedright
2nd Year Salary
\end{minipage} & \begin{minipage}[b]{\linewidth}\raggedright
3rd Year Option Salary
\end{minipage} & \begin{minipage}[b]{\linewidth}\centering
4th Year Option: Percentage Increase Over 3rd Year Salary
\end{minipage} & \begin{minipage}[b]{\linewidth}\centering
Qualifying Offer: Percentage Increase Over 4th Year Salary
\end{minipage} \\
\midrule()
\endhead
1 & 4,019.0 & 4,320.4 & 4,621.8 & 26.1\% & 30.0\% \\
2 & 3,595.8 & 3,865.5 & 4,135.2 & 26.2\% & 30.5\% \\
3 & 3,229.2 & 3,471.3 & 3,713.5 & 26.4\% & 31.2\% \\
4 & 2,911.4 & 3,129.7 & 3,348.1 & 26.5\% & 31.9\% \\
5 & 2,636.4 & 2,834.2 & 3,031.9 & 26.7\% & 32.6\% \\
6 & 2,394.6 & 2,574.2 & 2,753.8 & 26.8\% & 33.4\% \\
7 & 2,186.0 & 2,349.9 & 2,513.9 & 27.0\% & 34.1\% \\
8 & 2,002.6 & 2,152.8 & 2,303.0 & 27.2\% & 34.8\% \\
9 & 1,840.8 & 1,978.9 & 2,117.0 & 27.4\% & 35.5\% \\
10 & 1,748.8 & 1,879.9 & 2,011.1 & 27.5\% & 36.2\% \\
11 & 1,661.3 & 1,785.9 & 1,910.5 & 32.7\% & 36.9\% \\
12 & 1,578.2 & 1,696.6 & 1,815.0 & 37.8\% & 37.6\% \\
13 & 1,499.3 & 1,611.8 & 1,724.2 & 42.9\% & 38.3\% \\
14 & 1,424.4 & 1,531.3 & 1,638.1 & 48.1\% & 39.1\% \\
15 & 1,353.1 & 1,454.6 & 1,556.1 & 53.3\% & 39.8\% \\
16 & 1,285.5 & 1,381.9 & 1,478.3 & 53.4\% & 40.5\% \\
17 & 1,221.2 & 1,312.8 & 1,404.4 & 53.6\% & 41.2\% \\
18 & 1,160.2 & 1,247.2 & 1,334.2 & 53.8\% & 41.9\% \\
19 & 1,107.9 & 1,191.0 & 1,274.1 & 54.0\% & 42.6\% \\
20 & 1,063.6 & 1,143.4 & 1,223.2 & 54.2\% & 43.3\% \\
21 & 1,021.0 & 1,097.6 & 1,174.2 & 59.3\% & 44.1\% \\
22 & 980.2 & 1,053.7 & 1,127.2 & 64.5\% & 44.8\% \\
23 & 941.1 & 1,011.7 & 1,082.2 & 69.7\% & 45.5\% \\
24 & 903.4 & 971.1 & 1,038.9 & 74.9\% & 46.2\% \\
25 & 867.2 & 932.3 & 997.3 & 80.1\% & 46.9\% \\
26 & 838.5 & 901.4 & 964.3 & 80.3\% & 47.6\% \\
27 & 814.3 & 875.4 & 936.4 & 80.4\% & 48.3\% \\
28 & 809.3 & 870.0 & 930.7 & 80.5\% & 49.0\% \\
29 & 803.4 & 863.7 & 923.9 & 80.5\% & 50.0\% \\
30 & 797.6 & 857.4 & 917.2 & 80.5\% & 50.0\% \\
\bottomrule()
\end{longtable}

\newpage

\hypertarget{nba-rookie-scale-000s-4}{%
\section{2009-10 NBA Rookie Scale (\$000's)}\label{nba-rookie-scale-000s-4}}

\begin{longtable}[]{@{}
  >{\centering\arraybackslash}p{(\columnwidth - 10\tabcolsep) * \real{0.0806}}
  >{\raggedright\arraybackslash}p{(\columnwidth - 10\tabcolsep) * \real{0.1452}}
  >{\raggedright\arraybackslash}p{(\columnwidth - 10\tabcolsep) * \real{0.1613}}
  >{\raggedright\arraybackslash}p{(\columnwidth - 10\tabcolsep) * \real{0.1613}}
  >{\centering\arraybackslash}p{(\columnwidth - 10\tabcolsep) * \real{0.2258}}
  >{\centering\arraybackslash}p{(\columnwidth - 10\tabcolsep) * \real{0.2258}}@{}}
\toprule()
\begin{minipage}[b]{\linewidth}\centering
Pick
\end{minipage} & \begin{minipage}[b]{\linewidth}\raggedright
1st Year Salary
\end{minipage} & \begin{minipage}[b]{\linewidth}\raggedright
2nd Year Salary
\end{minipage} & \begin{minipage}[b]{\linewidth}\raggedright
3rd Year Option Salary
\end{minipage} & \begin{minipage}[b]{\linewidth}\centering
4th Year Option: Percentage Increase Over 3rd Year Salary
\end{minipage} & \begin{minipage}[b]{\linewidth}\centering
Qualifying Offer: Percentage Increase Over 4th Year Salary
\end{minipage} \\
\midrule()
\endhead
1 & 4,152.9 & 4,464.4 & 4,775.9 & 26.1\% & 30.0\% \\
2 & 3,715.7 & 3,994.4 & 4,273.1 & 26.2\% & 30.5\% \\
3 & 3,336.8 & 3,587.1 & 3,837.3 & 26.4\% & 31.2\% \\
4 & 3,008.4 & 3,234.1 & 3,459.7 & 26.5\% & 31.9\% \\
5 & 2,724.3 & 2,928.6 & 3,132.9 & 26.7\% & 32.6\% \\
6 & 2,474.4 & 2,660.0 & 2,845.6 & 26.8\% & 33.4\% \\
7 & 2,258.8 & 2,428.2 & 2,597.6 & 27.0\% & 34.1\% \\
8 & 2,069.4 & 2,224.6 & 2,379.8 & 27.2\% & 34.8\% \\
9 & 1,902.2 & 2,044.9 & 2,187.5 & 27.4\% & 35.5\% \\
10 & 1,807.1 & 1,942.6 & 2,078.1 & 27.5\% & 36.2\% \\
11 & 1,716.7 & 1,845.4 & 1,974.2 & 32.7\% & 36.9\% \\
12 & 1,630.9 & 1,753.2 & 1,875.5 & 37.8\% & 37.6\% \\
13 & 1,549.3 & 1,665.5 & 1,781.7 & 42.9\% & 38.3\% \\
14 & 1,471.9 & 1,582.3 & 1,692.7 & 48.1\% & 39.1\% \\
15 & 1,398.2 & 1,503.1 & 1,608.0 & 53.3\% & 39.8\% \\
16 & 1,328.4 & 1,428.0 & 1,527.6 & 53.4\% & 40.5\% \\
17 & 1,261.9 & 1,356.6 & 1,451.2 & 53.6\% & 41.2\% \\
18 & 1,198.9 & 1,288.8 & 1,378.7 & 53.8\% & 41.9\% \\
19 & 1,144.9 & 1,230.7 & 1,316.6 & 54.0\% & 42.6\% \\
20 & 1,099.1 & 1,181.5 & 1,263.9 & 54.2\% & 43.3\% \\
21 & 1,055.1 & 1,134.2 & 1,213.3 & 59.3\% & 44.1\% \\
22 & 1,012.9 & 1,088.8 & 1,164.8 & 64.5\% & 44.8\% \\
23 & 972.5 & 1,045.4 & 1,118.3 & 69.7\% & 45.5\% \\
24 & 933.5 & 1,003.5 & 1,073.5 & 74.9\% & 46.2\% \\
25 & 896.2 & 963.4 & 1,030.6 & 80.1\% & 46.9\% \\
26 & 866.5 & 931.4 & 996.4 & 80.3\% & 47.6\% \\
27 & 841.4 & 904.5 & 967.6 & 80.4\% & 48.3\% \\
28 & 836.3 & 899.0 & 961.7 & 80.5\% & 49.0\% \\
29 & 830.2 & 892.5 & 954.7 & 80.5\% & 50.0\% \\
30 & 824.2 & 886.0 & 947.8 & 80.5\% & 50.0\% \\
\bottomrule()
\end{longtable}

\newpage

\hypertarget{nba-rookie-scale-000s-5}{%
\section{2010-11 NBA Rookie Scale (\$000's)}\label{nba-rookie-scale-000s-5}}

\begin{longtable}[]{@{}
  >{\centering\arraybackslash}p{(\columnwidth - 10\tabcolsep) * \real{0.0806}}
  >{\raggedright\arraybackslash}p{(\columnwidth - 10\tabcolsep) * \real{0.1452}}
  >{\raggedright\arraybackslash}p{(\columnwidth - 10\tabcolsep) * \real{0.1613}}
  >{\raggedright\arraybackslash}p{(\columnwidth - 10\tabcolsep) * \real{0.1613}}
  >{\centering\arraybackslash}p{(\columnwidth - 10\tabcolsep) * \real{0.2258}}
  >{\centering\arraybackslash}p{(\columnwidth - 10\tabcolsep) * \real{0.2258}}@{}}
\toprule()
\begin{minipage}[b]{\linewidth}\centering
Pick
\end{minipage} & \begin{minipage}[b]{\linewidth}\raggedright
1st Year Salary
\end{minipage} & \begin{minipage}[b]{\linewidth}\raggedright
2nd Year Salary
\end{minipage} & \begin{minipage}[b]{\linewidth}\raggedright
3rd Year Option Salary
\end{minipage} & \begin{minipage}[b]{\linewidth}\centering
4th Year Option: Percentage Increase Over 3rd Year Salary
\end{minipage} & \begin{minipage}[b]{\linewidth}\centering
Qualifying Offer: Percentage Increase Over 4th Year Salary
\end{minipage} \\
\midrule()
\endhead
1 & 4,286.9 & 4,608.4 & 4,929.9 & 26.1\% & 30.0\% \\
2 & 3,835.6 & 4,123.2 & 4,410.9 & 26.2\% & 30.5\% \\
3 & 3,444.4 & 3,702.8 & 3,961.1 & 26.4\% & 31.2\% \\
4 & 3,105.5 & 3,338.4 & 3,571.3 & 26.5\% & 31.9\% \\
5 & 2,812.2 & 3,023.1 & 3,234.0 & 26.7\% & 32.6\% \\
6 & 2,554.2 & 2,745.8 & 2,937.4 & 26.8\% & 33.4\% \\
7 & 2,331.7 & 2,506.6 & 2,681.4 & 27.0\% & 34.1\% \\
8 & 2,136.1 & 2,296.3 & 2,456.5 & 27.2\% & 34.8\% \\
9 & 1,963.6 & 2,110.8 & 2,258.1 & 27.4\% & 35.5\% \\
10 & 1,865.3 & 2,005.2 & 2,145.1 & 27.5\% & 36.2\% \\
11 & 1,772.1 & 1,905.0 & 2,037.9 & 32.7\% & 36.9\% \\
12 & 1,683.5 & 1,809.7 & 1,936.0 & 37.8\% & 37.6\% \\
13 & 1,599.3 & 1,719.2 & 1,839.2 & 42.9\% & 38.3\% \\
14 & 1,519.4 & 1,633.3 & 1,747.3 & 48.1\% & 39.1\% \\
15 & 1,443.3 & 1,551.6 & 1,659.8 & 53.3\% & 39.8\% \\
16 & 1,371.2 & 1,474.0 & 1,576.9 & 53.4\% & 40.5\% \\
17 & 1,302.6 & 1,400.3 & 1,498.0 & 53.6\% & 41.2\% \\
18 & 1,237.5 & 1,330.3 & 1,423.1 & 53.8\% & 41.9\% \\
19 & 1,181.8 & 1,270.4 & 1,359.0 & 54.0\% & 42.6\% \\
20 & 1,134.5 & 1,219.6 & 1,304.7 & 54.2\% & 43.3\% \\
21 & 1,089.1 & 1,170.8 & 1,252.5 & 59.3\% & 44.1\% \\
22 & 1,045.6 & 1,124.0 & 1,202.4 & 64.5\% & 44.8\% \\
23 & 1,003.8 & 1,079.1 & 1,154.4 & 69.7\% & 45.5\% \\
24 & 963.6 & 1,035.9 & 1,108.1 & 74.9\% & 46.2\% \\
25 & 925.1 & 994.4 & 1,063.8 & 80.1\% & 46.9\% \\
26 & 894.4 & 961.5 & 1,028.6 & 80.3\% & 47.6\% \\
27 & 868.6 & 933.7 & 998.9 & 80.4\% & 48.3\% \\
28 & 863.3 & 928.0 & 992.7 & 80.5\% & 49.0\% \\
29 & 857.0 & 921.3 & 985.5 & 80.5\% & 50.0\% \\
30 & 850.8 & 914.6 & 978.4 & 80.5\% & 50.0\% \\
\bottomrule()
\end{longtable}

\newpage

\hypertarget{nba-rookie-scale-000s-6}{%
\section{2011-12 NBA Rookie Scale (\$000's)}\label{nba-rookie-scale-000s-6}}

\begin{longtable}[]{@{}
  >{\centering\arraybackslash}p{(\columnwidth - 10\tabcolsep) * \real{0.0806}}
  >{\raggedright\arraybackslash}p{(\columnwidth - 10\tabcolsep) * \real{0.1452}}
  >{\raggedright\arraybackslash}p{(\columnwidth - 10\tabcolsep) * \real{0.1613}}
  >{\raggedright\arraybackslash}p{(\columnwidth - 10\tabcolsep) * \real{0.1613}}
  >{\centering\arraybackslash}p{(\columnwidth - 10\tabcolsep) * \real{0.2258}}
  >{\centering\arraybackslash}p{(\columnwidth - 10\tabcolsep) * \real{0.2258}}@{}}
\toprule()
\begin{minipage}[b]{\linewidth}\centering
Pick
\end{minipage} & \begin{minipage}[b]{\linewidth}\raggedright
1st Year Salary
\end{minipage} & \begin{minipage}[b]{\linewidth}\raggedright
2nd Year Salary
\end{minipage} & \begin{minipage}[b]{\linewidth}\raggedright
3rd Year Option Salary
\end{minipage} & \begin{minipage}[b]{\linewidth}\centering
4th Year Option: Percentage Increase Over 3rd Year Salary
\end{minipage} & \begin{minipage}[b]{\linewidth}\centering
Qualifying Offer: Percentage Increase Over 4th Year Salary
\end{minipage} \\
\midrule()
\endhead
1 & 4,420.9 & 4,752.4 & 5,084.0 & 26.1\% & 30.0\% \\
2 & 3,955.4 & 4,252.1 & 4,548.7 & 26.2\% & 30.5\% \\
3 & 3,552.1 & 3,818.5 & 4,084.9 & 26.4\% & 31.2\% \\
4 & 3,202.5 & 3,442.7 & 3,682.9 & 26.5\% & 31.9\% \\
5 & 2,900.1 & 3,117.6 & 3,335.1 & 26.7\% & 32.6\% \\
6 & 2,634.0 & 2,831.6 & 3,029.1 & 26.8\% & 33.4\% \\
7 & 2,404.6 & 2,584.9 & 2,765.2 & 27.0\% & 34.1\% \\
8 & 2,202.9 & 2,368.1 & 2,533.3 & 27.2\% & 34.8\% \\
9 & 2,024.9 & 2,176.8 & 2,328.7 & 27.4\% & 35.5\% \\
10 & 1,923.6 & 2,067.9 & 2,212.2 & 27.5\% & 36.2\% \\
11 & 1,827.4 & 1,964.5 & 2,101.5 & 32.7\% & 36.9\% \\
12 & 1,736.1 & 1,866.3 & 1,996.5 & 37.8\% & 37.6\% \\
13 & 1,649.3 & 1,773.0 & 1,896.6 & 42.9\% & 38.3\% \\
14 & 1,566.9 & 1,684.4 & 1,801.9 & 48.1\% & 39.1\% \\
15 & 1,488.4 & 1,600.1 & 1,711.7 & 53.3\% & 39.8\% \\
16 & 1,414.1 & 1,520.1 & 1,626.2 & 53.4\% & 40.5\% \\
17 & 1,343.3 & 1,444.1 & 1,544.8 & 53.6\% & 41.2\% \\
18 & 1,276.2 & 1,371.9 & 1,467.6 & 53.8\% & 41.9\% \\
19 & 1,218.7 & 1,310.1 & 1,401.5 & 54.0\% & 42.6\% \\
20 & 1,170.0 & 1,257.7 & 1,345.5 & 54.2\% & 43.3\% \\
21 & 1,123.1 & 1,207.4 & 1,291.6 & 59.3\% & 44.1\% \\
22 & 1,078.2 & 1,159.1 & 1,240.0 & 64.5\% & 44.8\% \\
23 & 1,035.2 & 1,112.8 & 1,190.5 & 69.7\% & 45.5\% \\
24 & 993.7 & 1,068.2 & 1,142.8 & 74.9\% & 46.2\% \\
25 & 954.0 & 1,025.5 & 1,097.1 & 80.1\% & 46.9\% \\
26 & 922.4 & 991.5 & 1,060.7 & 80.3\% & 47.6\% \\
27 & 895.7 & 962.9 & 1,030.1 & 80.4\% & 48.3\% \\
28 & 890.2 & 957.0 & 1,023.8 & 80.5\% & 49.0\% \\
29 & 883.8 & 950.0 & 1,016.3 & 80.5\% & 50.0\% \\
30 & 877.3 & 943.1 & 1,008.9 & 80.5\% & 50.0\% \\
\bottomrule()
\end{longtable}

\newpage

\hypertarget{minimum-annual-salary-scale}{%
\chapter{MINIMUM ANNUAL SALARY SCALE}\label{minimum-annual-salary-scale}}

\begin{longtable}[]{@{}
  >{\raggedright\arraybackslash}p{(\columnwidth - 14\tabcolsep) * \real{0.1047}}
  >{\raggedright\arraybackslash}p{(\columnwidth - 14\tabcolsep) * \real{0.1279}}
  >{\raggedright\arraybackslash}p{(\columnwidth - 14\tabcolsep) * \real{0.1279}}
  >{\raggedright\arraybackslash}p{(\columnwidth - 14\tabcolsep) * \real{0.1279}}
  >{\raggedright\arraybackslash}p{(\columnwidth - 14\tabcolsep) * \real{0.1279}}
  >{\raggedright\arraybackslash}p{(\columnwidth - 14\tabcolsep) * \real{0.1279}}
  >{\raggedright\arraybackslash}p{(\columnwidth - 14\tabcolsep) * \real{0.1279}}
  >{\raggedright\arraybackslash}p{(\columnwidth - 14\tabcolsep) * \real{0.1279}}@{}}
\toprule()
\begin{minipage}[b]{\linewidth}\raggedright
Years of Service
\end{minipage} & \begin{minipage}[b]{\linewidth}\raggedright
2005-06
\end{minipage} & \begin{minipage}[b]{\linewidth}\raggedright
2006-07
\end{minipage} & \begin{minipage}[b]{\linewidth}\raggedright
2007-08
\end{minipage} & \begin{minipage}[b]{\linewidth}\raggedright
2008-09
\end{minipage} & \begin{minipage}[b]{\linewidth}\raggedright
2009-10
\end{minipage} & \begin{minipage}[b]{\linewidth}\raggedright
2010-11
\end{minipage} & \begin{minipage}[b]{\linewidth}\raggedright
2011-12
\end{minipage} \\
\midrule()
\endhead
0 & 398,762 & 412,718 & 427,163 & 442,114 & 457,588 & 473,604 & 490,180 \\
1 & 641,748 & 664,209 & 687,456 & 711,517 & 736,420 & 762,195 & 788,872 \\
2 & 719,373 & 744,551 & 770,610 & 797,581 & 825,497 & 854,389 & 884,293 \\
3 & 745,248 & 771,331 & 798,328 & 826,269 & 855,189 & 885,120 & 916,100 \\
4 & 771,123 & 798,112 & 826,046 & 854,957 & 884,881 & 915,852 & 947,907 \\
5 & 835,810 & 865,063 & 895,341 & 926,678 & 959,111 & 992,680 & 1,027,424 \\
6 & 900,498 & 932,015 & 964,636 & 998,398 & 1,033,342 & 1,069,509 & 1,106,941 \\
7 & 965,185 & 998,967 & 1,033,930 & 1,070,118 & 1,107,572 & 1,146,337 & 1,186,459 \\
8 & 1,029,873 & 1,065,918 & 1,103,225 & 1,141,838 & 1,181,803 & 1,223,166 & 1,265,976 \\
9 & 1,035,000 & 1,071,225 & 1,108,718 & 1,147,523 & 1,187,686 & 1,229,255 & 1,272,279 \\
10+ & 1,138,500 & 1,178,348 & 1,219,590 & 1,262,275 & 1,306,455 & 1,352,181 & 1,399,507 \\
\bottomrule()
\end{longtable}

\hypertarget{bri-expense-ratios}{%
\chapter{BRI EXPENSE RATIOS}\label{bri-expense-ratios}}

Article VII. Section 1(a)(l)(v), (vi)

\begin{longtable}[]{@{}lc@{}}
\toprule()
Category & Ratio of Expenses to Revenues \\
\midrule()
\endhead
Novelties and Concessions & 50\% \\
Game Parking & Accountants to determine \\
Game Programs & 25\% \\
Team Sponsorships and Promotions & 34\% \\
In-arena signage & Accountants to determine \\
In-arena Club & Accountants to determine \\
\bottomrule()
\end{longtable}

Article VII. Section 1(a)(viii)

\begin{longtable}[]{@{}lc@{}}
\toprule()
Category & Ratio of Expenses to Revenues \\
\midrule()
\endhead
Sponsorships & 19\% \\
NBA Entertainment & 35\% \\
International Television & 22\% \\
Special Events & 100\% \\
\bottomrule()
\end{longtable}

\hypertarget{notice-to-veteran-players-concerning-summer-leagues}{%
\chapter{NOTICE TO VETERAN PLAYERS CONCERNING SUMMER LEAGUES}\label{notice-to-veteran-players-concerning-summer-leagues}}

\begin{enumerate}
\def\labelenumi{\arabic{enumi}.}
\item
  Under the Uniform Player Contract and the Collective Bargaining Agreement between the NBA and the Players Association, the Team cannot require players to participate in any summer league.
\item
  The failure of a player to participate in a summer league will not, by itself, prejudice or disadvantage such player in his Team standing or relationship.
\item
  The Team reserves the right to determine how many and which players it may enroll in any summer league.
\end{enumerate}

We would appreciate your signing in the space provided below to acknowledge that you have freely chosen to participate in summer league play on a voluntary basis during the summer of \_\_\_\_.

\begin{longtable}[]{@{}lc@{}}
\toprule()
Agreed to and Accepted: & \\
\midrule()
\endhead
\_\_\_\_\_\_\_\_\_\_\_\_\_\_\_\_\_\_\_\_\_ & \_\_\_\_\_\_ \\
(Name of Player) & \\
\_\_\_\_\_\_\_\_\_\_\_\_\_\_\_\_\_\_\_\_\_ & \_\_\_\_\_\_ \\
(Date) & \\
\bottomrule()
\end{longtable}

\hypertarget{f}{%
\chapter{F}\label{f}}

{[}Intentionally left blank{]}

\hypertarget{offer-sheet}{%
\chapter{OFFER SHEET}\label{offer-sheet}}

\begin{longtable}[]{@{}
  >{\raggedright\arraybackslash}p{(\columnwidth - 2\tabcolsep) * \real{0.5000}}
  >{\raggedright\arraybackslash}p{(\columnwidth - 2\tabcolsep) * \real{0.5000}}@{}}
\toprule()
\endhead
Name of Player: & Date: \\
\_\_\_\_\_\_\_\_\_\_\_\_\_\_\_\_\_\_\_\_\_\_\_\_\_ & \_\_\_\_\_\_\_\_\_\_\_\_\_\_\_\_\_\_\_\_\_\_\_\_\_ \\
Address of Player and Email Address of Player: & Name of New Team: \\
\_\_\_\_\_\_\_\_\_\_\_\_\_\_\_\_\_\_\_\_\_\_\_\_\_ & \\
\_\_\_\_\_\_\_\_\_\_\_\_\_\_\_\_\_\_\_\_\_\_\_\_\_ & \\
\_\_\_\_\_\_\_\_\_\_\_\_\_\_\_\_\_\_\_\_\_\_\_\_\_ & \_\_\_\_\_\_\_\_\_\_\_\_\_\_\_\_\_\_\_\_\_\_\_\_\_ \\
Name, Address and Email Address of Player's Representative Authorized to Act for Player: & Name of ROFR Team: \\
\_\_\_\_\_\_\_\_\_\_\_\_\_\_\_\_\_\_\_\_\_\_\_\_\_ & \_\_\_\_\_\_\_\_\_\_\_\_\_\_\_\_\_\_\_\_\_\_\_\_\_ \\
\_\_\_\_\_\_\_\_\_\_\_\_\_\_\_\_\_\_\_\_\_\_\_\_\_ & Address of ROFR Team: \\
\_\_\_\_\_\_\_\_\_\_\_\_\_\_\_\_\_\_\_\_\_\_\_\_\_ & \_\_\_\_\_\_\_\_\_\_\_\_\_\_\_\_\_\_\_\_\_\_\_\_\_ \\
\_\_\_\_\_\_\_\_\_\_\_\_\_\_\_\_\_\_\_\_\_\_\_\_\_ & \_\_\_\_\_\_\_\_\_\_\_\_\_\_\_\_\_\_\_\_\_\_\_\_\_ \\
\bottomrule()
\end{longtable}

Attached hereto is an unsigned Player Contract that the New Team has offered to the Player and that the Player desires to accept. The attached Player Contract separately specifies in its exhibits those Principal Terms that will be included in the Player Contract with the ROFR Team if that Team gives the Player a timely First Refusal Exercise Notice.

\begin{longtable}[]{@{}ll@{}}
\toprule()
\endhead
Player: & New Team: \\
& \\
By \_\_\_\_\_\_\_\_\_\_\_\_\_\_\_\_\_\_\_\_\_\_\_\_\_ & By \_\_\_\_\_\_\_\_\_\_\_\_\_\_\_\_\_\_\_\_\_\_\_\_\_ \\
\bottomrule()
\end{longtable}

\hypertarget{first-refusal-exercise-notice}{%
\chapter{FIRST REFUSAL EXERCISE NOTICE}\label{first-refusal-exercise-notice}}

\begin{longtable}[]{@{}
  >{\raggedright\arraybackslash}p{(\columnwidth - 2\tabcolsep) * \real{0.5000}}
  >{\raggedright\arraybackslash}p{(\columnwidth - 2\tabcolsep) * \real{0.5000}}@{}}
\toprule()
\endhead
Name of Player: & Date: \\
\_\_\_\_\_\_\_\_\_\_\_\_\_\_\_\_\_\_\_\_\_\_\_\_\_ & \_\_\_\_\_\_\_\_\_\_\_\_\_\_\_\_\_\_\_\_\_\_\_\_\_ \\
Address of Player: & Name of New Team: \\
\_\_\_\_\_\_\_\_\_\_\_\_\_\_\_\_\_\_\_\_\_\_\_\_\_ & \\
\_\_\_\_\_\_\_\_\_\_\_\_\_\_\_\_\_\_\_\_\_\_\_\_\_ & \\
\_\_\_\_\_\_\_\_\_\_\_\_\_\_\_\_\_\_\_\_\_\_\_\_\_ & \_\_\_\_\_\_\_\_\_\_\_\_\_\_\_\_\_\_\_\_\_\_\_\_\_ \\
Name and Address of Player's Representative Authorized to Act for Player & Name of ROFR Team: \\
\_\_\_\_\_\_\_\_\_\_\_\_\_\_\_\_\_\_\_\_\_\_\_\_\_ & \_\_\_\_\_\_\_\_\_\_\_\_\_\_\_\_\_\_\_\_\_\_\_\_\_ \\
\_\_\_\_\_\_\_\_\_\_\_\_\_\_\_\_\_\_\_\_\_\_\_\_\_ & Address of ROFR Team: \\
\_\_\_\_\_\_\_\_\_\_\_\_\_\_\_\_\_\_\_\_\_\_\_\_\_ & \_\_\_\_\_\_\_\_\_\_\_\_\_\_\_\_\_\_\_\_\_\_\_\_\_ \\
\_\_\_\_\_\_\_\_\_\_\_\_\_\_\_\_\_\_\_\_\_\_\_\_\_ & \_\_\_\_\_\_\_\_\_\_\_\_\_\_\_\_\_\_\_\_\_\_\_\_\_ \\
\bottomrule()
\end{longtable}

The undersigned member of the NBA hereby exercises its Right of First Refusal so as to create a binding agreement with the Player containing the Principal Terms set forth in the Player Contract annexed to the Player's Offer Sheet (a copy of which is attached hereto).

\begin{longtable}[]{@{}l@{}}
\toprule()
\endhead
ROFR Team: \\
By \_\_\_\_\_\_\_\_\_\_\_\_\_\_\_\_\_\_\_\_\_ \\
\bottomrule()
\end{longtable}

\hypertarget{section}{%
\chapter{}\label{section}}

\hypertarget{authorization-for-testing}{%
\section{AUTHORIZATION FOR TESTING}\label{authorization-for-testing}}

\begin{longtable}[]{@{}lc@{}}
\toprule()
\endhead
To: & \_\_\_\_\_\_\_\_\_\_\_\_\_\_\_\_\_\_\_ \\
& \\
Player & \_\_\_\_\_\_\_\_\_\_\_\_\_\_\_\_\_\_\_ \\
\bottomrule()
\end{longtable}

Please be advised that on \_\_\_\_\_\_\_\_\_\_\_\_\_\_\_\_\_\_\_\_\_\_\_\_\_\_, you were the subject of a meeting or conference call held pursuant to the Anti-Drug Program set forth in Article XXXIII of the Collective Bargaining Agreement between the NBA and the National Basketball Players Association, dated July 29, 2005 (the ``Agreement''). Following the meeting or conference call, I authorized the NBA to conduct the testing procedures set forth in the Agreement, and you are hereby directed to submit to those testing procedures, on demand, no more than four times during the next six weeks.

Please be advised that your failure to submit to these procedures may result in substantial penalties, including but not limited to your dismissal and disqualification from the NBA.

\begin{longtable}[]{@{}l@{}}
\toprule()
\endhead
\_\_\_\_\_\_\_\_\_\_\_\_\_\_\_\_\_\_\_\_\_\_\_\_ \\
Independent Expert \\
Dated: \\
\_\_\_\_\_\_\_\_\_\_\_\_\_\_\_\_\_\_\_\_\_\_\_\_ \\
\bottomrule()
\end{longtable}

\hypertarget{prohibited-substances}{%
\section{PROHIBITED SUBSTANCES}\label{prohibited-substances}}

\begin{enumerate}
\def\labelenumi{(\alph{enumi})}
\tightlist
\item
  Drugs of Abuse
\end{enumerate}

\begin{itemize}
\tightlist
\item
  Amphetamine and its analogs (including but not limited to methamphetamine and MDMA)
\item
  Cocaine
\item
  LSD
\item
  Opiates (Heroin, Codeine, Morphine)
\item
  Phencyclidine (``PCP'')
\end{itemize}

\begin{enumerate}
\def\labelenumi{(\alph{enumi})}
\setcounter{enumi}{1}
\item
  Marijuana and its by-products
\item
  Steroids, Performance Enhancing Drugs, and Masking Agents (SPEDs)
\end{enumerate}

\begin{itemize}
\tightlist
\item
  Amiphenazole
\item
  Androstanediol
\item
  Androstanedione
\item
  Androstenediol
\item
  Androstenedione
\item
  Bolasterone
\item
  Boldenone
\item
  Boldione
\item
  Bromantan
\item
  Calusterone
\item
  Clenbuterol
\item
  Clobenzorex
\item
  Clomiphene
\item
  Clostebol
\item
  Cyclofenil
\item
  Danazol
\item
  Dehydrochlormethyltestosterone
\item
  Dehydroepiandrosterone (DHEA)
\item
  Desoxymethyltestosterone (DMT)
\item
  Dihydrotestosterone
\item
  4-dihydrotestosterone
\item
  Dromostanolone
\item
  Drostanolone
\item
  18a-homo-17b-hydroxyestr-4-en-3-one
\item
  Ephedra (also called Ma Huang, Bishop's Tea and Chi Powder)
\item
  Ephedrine
\item
  Epitestosterone
\item
  13a-ethyl-17a-hydroxygon-4-en-one
\item
  Etilefrine
\item
  Ethylestrenol
\item
  Fenethylline
\item
  Fenfluramine
\item
  Fluoxymesterone
\item
  Formebolone
\item
  Fulvestrant
\item
  Furazabol
\item
  Gestrinone
\item
  Human Chorionic Gonadotropin
\item
  Human Growth Hormone
\item
  4-hydroxytestosterone
\item
  Mestanolone
\item
  Mesterolone
\item
  Methandienone
\item
  Methandriol
\item
  Methenolone
\item
  Methyldienolone
\item
  17a-methyl-3b, 17b-dihydroxy-5a-androstane
\item
  17a-methyl-3a, 17b-dihydroxy-5a-androstane
\item
  17a-methyl-3b, 17b-dihydroxyandrost-4-ene
\item
  17a-methyl-1-dihydrotestosterone
\item
  17a-methyl-4a-hydroxynandrolone
\item
  Methylephedrine
\item
  Methylphenidate
\item
  Methyltestosterone
\item
  Methyltrienolone
\item
  Mibolerone
\item
  Modafinil
\item
  Nandrolone (also called 19-nortestosterone)
\item
  Nikethamide
\item
  19-norandrostenediol
\item
  19-norandrostendione
\item
  Norbolethone
\item
  Norclostebol
\item
  Norethandrolone
\item
  Norfenfluramine
\item
  Normethandrolone
\item
  Norpseudoephedrine (also called cathine)
\item
  Oxabolone (also called 4-hydroxy-19-nortestosterone)
\item
  Oxandrolone
\item
  Oxymesterone
\item
  Oxymetholone
\item
  Pemoline
\item
  Phenmetrazine
\item
  Phentermine
\item
  Phenylpropanolamine (PPA)
\item
  Probenecid
\item
  Pseudoephedrine
\item
  Quinbolone
\item
  Stanozolol
\item
  Stenbolone
\item
  Strychnine
\item
  Testolactone
\item
  Testosterone
\item
  Tetrahydrogestrinone (THG)
\item
  Trenbolone
\item
  Zeranol
\item
  Zilpaterol
\end{itemize}

\begin{enumerate}
\def\labelenumi{(\alph{enumi})}
\setcounter{enumi}{3}
\tightlist
\item
  Diuretics
\end{enumerate}

\begin{itemize}
\tightlist
\item
  Acetazolamide
\item
  Amiloride
\item
  Bendroflumethiazide
\item
  Benzthiazide
\item
  Bumetanide
\item
  Canrenone
\item
  Chlorothiazide
\item
  Chlorthalidone
\item
  Clopamide
\item
  Cyclothiazide
\item
  Dichlorphenamide
\item
  Ethacrynic Acid
\item
  Flumethiazide
\item
  Furosemide
\item
  Hydrochlorothiazide
\item
  Hydroflumethiazide
\item
  Indapamide
\item
  Methylclothiazide
\item
  Metolazone
\item
  Polythiazide
\item
  Quinethazone
\item
  Spironolactone
\item
  Triamterene
\item
  Trichlormethiazide
\end{itemize}

\hypertarget{collection-procedures}{%
\section{COLLECTION PROCEDURES}\label{collection-procedures}}

When the player arrives at the collection site, the collector will ensure that the player is positively identified through presentation of photo ID or identification by a team representative. If the player's identity cannot be established, the collector shall not proceed with the collection.

The player will be asked to select a sealed urine specimen cup. The player will then provide his urine specimen under the direct observation of the collector.

The collector shall ensure that the player has provided a urine specimen of sufficient volume for accurate testing. If such a sample cannot immediately be provided by the player, he shall be instructed to remain at the testing site for a reasonable period of time until he can provide such a specimen. Once the specimen has been obtained, the player will select a sealed specimen kit, which contains two bottles. The collector, in the presence of the player, will pour the specimen into two bottles. One bottle will be used as the primary or ``A'' specimen and the other will be used as the split or ``B'' specimen. The specimen bottles will be sealed with tamper-proof seals in the presence of the player. The seals will contain a unique identification number that corresponds to the number on the chain of custody form.

The player and collector will complete the chain of custody form that documents the handling of the specimen. The collector will note any irregularities concerning the specimen on the chain of custody form. Both the player and collector will sign the chain of custody form. The chain of custody form along with the two specimen bottles will be placed back into the kit. The kit will be sealed and sent via overnight courier to the laboratory for testing.

Once the specimen arrives at the laboratory, the primary specimen will be analyzed. If the primary specimen tests positive, the split sample will be placed in frozen storage and will be available for testing by a different laboratory, if requested by the player.

\hypertarget{steroids-performance-enhancing-drugs-and-masking-agents-confirmatory-laboratory-analysis-levels}{%
\section{STEROIDS, PERFORMANCE-ENHANCING DRUGS AND MASKING AGENTS CONFIRMATORY LABORATORY ANALYSIS LEVELS}\label{steroids-performance-enhancing-drugs-and-masking-agents-confirmatory-laboratory-analysis-levels}}

All SPEDs, except those listed below, at any detectable level.

Ephedra/Ephedrine 10 mcg/ml
Methylephedrine 10 mcg/ml
Nandrolone 2 ng/ml
Norpseudoephedrine 5 mcg/ml
Phenylpropanolamine (PPA) 25 mcg/ml
Pseudoephedrine 25 mcg/ml
Testosterone 4:1 t/e ratio

\hypertarget{section-1}{%
\chapter{}\label{section-1}}

\hypertarget{exhibit-j-2}{%
\section{EXHIBIT J-2}\label{exhibit-j-2}}

\textbf{FORM OF CONFIDENTIALITY AGREEMENT}

{[}Date{]}

National Basketball Players Association
2 Penn Plaza
New York, NY 10121

Re: Confidentiality Agreement

Gentlemen:

This will confirm the agreement of the National Basketball Players Association (``Players Association'') (on behalf of itself and its employees, officers and outside advisors) to maintain the confidentiality of all Confidential Information (as defined in Paragraph 6 below) in connection with the audit, with respect to the 20\_\_-20\_\_ Salary Cap Year, of (i) the National Basketball Association (``NBA'') League Office, (ii) any NBA team that the parties agree is to be included in such audit with respect to such Salary Cap Year (the ``Team(s)''), and (iii) the expenses incurred in connection with proceeds that come within Article VII, Section 1(a)(1)(viii) of the Collective Bargaining Agreement (``CBA'') entered into July \_\_\_, 2005, between the Players Association and the NBA (collectively, the ``Audit''). Capitalized terms not defined herein shall have the meaning ascribed to such terms in the CBA.

\begin{enumerate}
\def\labelenumi{\arabic{enumi}.}
\item
  The NBA and the Team(s) shall make available Confidential Information for purposes of the Audit based on the representation of the Players Association that it (and its employees, officers and outside advisors shall comply with the terms of this Confidentiality Agreement at all times during and after the Audit. To that end, before any employee, officer or outside advisor of the Players Association may be permitted to review any Confidential Information, the Players Association shall require such employee, officer or outside advisor to agree, in writing (in the form of acknowledgment annexed hereto), to comply with the terms of this Confidentiality Agreement, and the Players Association shall promptly provide copies thereof to the NBA.
\item
  The Players Association shall maintain the absolute confidentiality of all Confidential Information at all times and shall not disclose to any person or entity, except as permitted herein, any Confidential Information. The Players Association agrees that it may use or refer to Confidential Information only during the course of the Audit and solely for the purpose of conducting the Audit in accordance with the terms and conditions of the CBA and this Confidentiality Agreement, and that Confidential Information may not be used or referred to by the Players Association (or any of its employees, officers and outside advisors), at any time, for any purpose other than to conduct the Audit in accordance with the terms and conditions of the CBA and this Confidentiality Agreement. The Players Association (and its employees, officers, NBA team player representatives (``Player Representatives'') and outside advisors) shall not disclose, disseminate or provide any Confidential Information to any other person or entity (including NBA players who are not officers of, or Player Representatives for, the Players Association and any representative of any player), at any time, for any purpose; provided, however, that the Players Association may only disclose, disseminate or provide a summary of Confidential Information to Player Representatives in aggregate form without identifying any specific information (e.g., by sponsor). Notwithstanding anything to the contrary in this Confidentiality Agreement, the Players Association shall not be deemed to have violated any provision herein if the Players Association discloses to such third party that the Audit is being undertaken and that the Players Association is subject to a confidentiality agreement and, therefore, not permitted to discuss the Audit. The foregoing shall not foreclose the Players Association from disclosing Confidential Information during the course of a proceeding before the System Arbitrator, an appeal to the Appeals Panel of an award of the System Arbitrator and a judicial action to enforce any such proceeding or award.
\item
  The Players Association shall adopt and implement such procedures to insure the confidentiality of Confidential Information as would be employed by a reasonable and prudent person to safeguard the confidentiality of his or her own most confidential information, or, if more stringent, such procedures as are employed for such purposes by the Players Association. Such procedures shall include, but not be limited to, steps to insure that such Confidential Information is disclosed only to those Players Association employees and officers and, subject to the restrictions set forth in Paragraph 2 above, Player Representatives who have a need to have access to such Confidential Information and only for the purpose of conducting the Audit in accordance with the terms of the CBA and this Confidentiality Agreement. The foregoing shall not foreclose the Players Association from disclosing Confidential Information during the course of a proceeding before the System Arbitrator, an appeal to the Appeals Panel of an award of the System Arbitrator and a judicial action to enforce any such proceeding or award.
\item
  The Players Association agrees that no copies of Confidential Information made available by the NBA and the Team(s) may be removed from the NBA's or the Team(s)' offices (as applicable).
\item
  If the Players Association is required by governmental or judicial authorities (by oral questions, interrogatories, requests for information or documents, subpoena, civil investigative demand or similar process) to disclose any Confidential Information, it shall provide the NBA and/or the Team(s) with prompt notice so that the NBA and/or the Team(s) may seek an appropriate protective order. If, in the absence of a protective order, the Players Association is, after giving notice in accordance with the preceding sentence, compelled to disclose Confidential Information or else stand liable for contempt or suffer other censure or penalty, the Players Association may disclose only such Confidential Information as is necessary to avoid such liability without incurring liability hereunder.
\item
  For purposes of this Confidentiality Agreement, ``Confidential Information'' shall mean all documents, materials and other information reviewed or made available (whether in written or oral form) in connection with the Audit (including, without limitation, all documents, materials and other information made available by Pricewaterhouse Coopers, LLP), and shall include all excerpts, extracts, summaries and contents thereof and notes taken during the Audit; provided, however that Confidential Information shall not include information that (a) is or becomes generally available to the public other than as a result of disclosure by the Players Association (including Players Association affiliates or representatives), (b) was available to the Players Association prior to its disclosure by the NBA or the Team(s) (as applicable), or (c) becomes available to the Players Association from a source other than the NBA or the Team(s), provided that such source is not bound by a confidentiality agreement with the NBA, the Team(s) or the Players Association.
\item
  The Players Association acknowledges that the terms and conditions contained in this Confidentiality Agreement are reasonable and necessary to protect the legitimate interests of the NBA and the Team(s), do not cause the Players Association undue hardship, and that any violation of the provisions of this Confidentiality Agreement or disclosure of any Confidential Information without the NBA's or the Team(s)' (as applicable) prior written consent will result in irreparable injury to the NBA and/or the Team(s) for which there is no adequate remedy. Accordingly, in the event of any such violation or disclosure, the NBA and/or the Team(s) shall be entitled to preliminary and permanent injunctive relief from any federal or state court of competent jurisdiction located in New York, New York, and the Players Association hereby consents to, and waives any objection to, venue and jurisdiction in such courts. In addition, the Players Association shall indemnify and hold harmless the NBA and its members teams and their respective affiliates, owners, directors, governors, officers and employees, and the successors, assigns and personal representatives of the foregoing parties (``NBA indemnified parties''), from and against all liability, damages and costs (including attorneys fees) arising out of any claim asserted against any NBA indemnified party relating to any use or disclosure of Confidential Information by, or the violation of this Confidentiality Agreement by, the Players Association (or any of its employees, officers, Player Representatives and outside advisors), provided that: (a) the Players Association is given prompt notice of any such claim, (b) has the right to approve counsel and/or has the opportunity to undertake the defense of such claim and (c) the indemnified party does not admit liability with respect to and does not settle such claim without the prior written consent of the Players Association. The Players Association also agrees that the relief provided for in this Paragraph 7 shall be cumulative and in addition to any other rights or remedies to which the NBA and the Team(s) may be entitled.
\item
  This Confidentiality Agreement is the final and complete agreement between the parties with respect to its subject matter. Any waiver of or modification to this Confidentiality Agreement must be in a writing and signed by each party. Any waiver in any particular instance of the rights and limitations contained herein shall not be deemed and is not intended to be a general waiver of any rights or limitations contained herein and shall not operate as a waiver beyond the particular instance.
\item
  This Confidentiality Agreement shall be governed by and construed and enforced in accordance with the laws of the State of New York, without giving effect to the principles of conflicts of law thereof.
\end{enumerate}

If the foregoing coincides with your understanding of our agreement, please sign the enclosed copy of this letter and return it to me.

Sincerely,

\begin{longtable}[]{@{}l@{}}
\toprule()
\endhead
NATIONAL BASKETBALL ASSOCIATION \\
By: \_\_\_\_\_\_\_\_\_\_\_\_\_\_\_\_\_\_\_\_\_\_\_\_\_\_\_\_\_\_\_\_\_\_\_ \\
 \\
AGREED TO AND ACCEPTED: \\
NATIONAL BASKETBALL PLAYERS ASSOCIATION \\
By: \_\_\_\_\_\_\_\_\_\_\_\_\_\_\_\_\_\_\_\_\_\_\_\_\_\_\_\_\_\_\_\_\_\_\_ \\
\bottomrule()
\end{longtable}

\newpage

\textbf{ACKNOWLEDGMENT}

The undersigned hereby acknowledges that I have read the Confidentiality Agreement between the National Basketball Players Association and the National Basketball Association dated \_\_\_\_\_\_\_\_\_\_\_\_\_, 20\_\_\_, and agree to comply with all of the terms and conditions contained therein.

Date: \_\_\_\_\_\_\_\_\_\_\_\_\_\_\_\_\_\_\_\_\_\_\_\_\\
By:\_\_\_\_\_\_\_\_\_\_\_\_\_\_\_\_\_\_\_\_\_\_\_\_\_\_\_\\
Name:\_\_\_\_\_\_\_\_\_\_\_\_\_\_\_\_\_\_\_\_\_\_\_\_\_\\
Title:\_\_\_\_\_\_\_\_\_\_\_\_\_\_\_\_\_\_\_\_\_\_\_\_\_\_

The National Basketball Players Association hereby acknowledges that the person identified above is an employee, officer, or outside advisor of the National Basketball Players Association.

National Basketball Players Association\\
Date:\_\_\_\_\_\_\_\_\_\_\_\_\_\_\_\_\_\_\_\_\_\_\_\_\_\_\_\\
By:\_\_\_\_\_\_\_\_\_\_\_\_\_\_\_\_\_\_\_\_\_\_\_\_\_\_\_\_\_\\
Name:\_\_\_\_\_\_\_\_\_\_\_\_\_\_\_\_\_\_\_\_\_\_\_\_\_\_

\newpage

\hypertarget{exhibit-j-2-1}{%
\section{EXHIBIT J-2}\label{exhibit-j-2-1}}

July 29, 2005\\
Mr.~G. William Hunter\\
Executive Director\\
National Basketball Players Association\\
2 Penn Plaza, Suite 2430\\
New York, New York 10121

Dear Billy:

This will confirm our agreement that the attached accounting procedures are the procedures that will be in effect for purposes of Article VII, Section 10 of the Collective Bargaining Agreement entered into on July 29, 2005, unless such procedures shall be modified by agreement of the parties.

If the foregoing coincides with your understanding of our agreement, please sign this letter in the space provided below.

Sincerely,\\
/s/ JOEL M. LITVIN\\
Joel M. Litvin

AGREED TO AND ACCEPTED:

NATIONAL BASKETBALL PLAYERS ASSOCIATION

By:\\
/s/ G. WILLIAM HUNTER\\
G. William Hunter\\
Executive Director

\newpage

\hypertarget{minimum-procedures-to-be-provided-by-the-accountants}{%
\subsection{Minimum Procedures To Be Provided By The Accountants}\label{minimum-procedures-to-be-provided-by-the-accountants}}

\textbf{General}

\begin{itemize}
\tightlist
\item
  The Audit Report (and any Interim Audit Report or Interim Escrow Audit Report) must be prepared in accordance with the relevant terms of the Collective Bargaining Agreement (``CBA''), which should be reviewed and understood by all auditors.
\item
  The Basketball Related Income Reporting Package and instructions should be reviewed and understood by all auditors.
\item
  All audit workpapers should be made available for review by representatives of the NBA and Players Association prior to issuance of the report.
\item
  A summary of all audit findings (including any unusual or non-recurring transactions) and proposed adjustments must be jointly reviewed with representatives of the NBA and Players Association prior to issuance of the report.
\item
  Any problems or questions raised during the audit should be resolved jointly with representatives of the NBA and Players Association (or by the Accountants, to the extent called for under the CBA).
\item
  All estimates should be reviewed in accordance with the CBA. Estimates are to be reviewed based upon the previous year's actual results and current year activity. All estimates should be confirmed with third parties when possible.
\item
  Revenue and expense amounts that have been estimated should be reconfirmed with the controller or other team representatives prior to the issuance of the Audit Report on or before the last day of the Moratorium Period.
\item
  Where appropriate, team and NBA revenues and expenses should be reconciled to audited financial statements.
\item
  All reporting packages and supporting schedules are to be completed in U.S. dollars.
\item
  The Auditors may consider, but are not bound by, the value attributed to or treatment of revenue or expense items in prior years.
\item
  Auditors should be aware of revenues excluded from BRI. The Teams should be instructed to make available to the Auditors all information necessary to determine categories of revenues they have excluded from BRI. Questions regarding whether revenues or expenses are includable or excludable from BRI should be reviewed with both parties to determine proper treatment. Auditors should perform a review for revenues improperly excluded from, or included in, BRI.
\end{itemize}

\textbf{Team Salaries}

\begin{itemize}
\tightlist
\item
  Trace amounts to the team's general ledger or other supporting documentation for agreement.
\item
  Foot all schedules and perform other clerical tests.
\item
  Examine an appropriate sample of player contracts, noting agreement of all salary amounts, in accordance with the definition of Salary in the CBA.
\item
  Compare player names with all player lists for the season in question.
\item
  Inquire of controller or other representative of each team if any additional compensation was paid to players and not included on the schedule, whether or not paid for basketball services. Also inquire if any business arrangements were entered into by the team or team affiliate with players or their affiliates, including with retired players who played for the team within the past five (5) years.
\item
  Review performance bonuses to determine whether such bonuses were actually earned for such season.
\item
  Review signing bonuses to determine if they have been properly allocated in accordance with the terms of the CBA.
\item
  Confirm that, where provided in the CBA, certain contracts have been averaged.
\end{itemize}

\textbf{Benefits}

\begin{itemize}
\tightlist
\item
  Trace amounts to the team's general ledger or other supporting documentation for agreement.
\item
  Foot all schedules and perform other clerical tests.
\item
  Investigate variations in amounts from the prior year through discussion with the controller or other representative of the team.
\item
  Review each team's insurance expenses for premium credits (refunds) received from Planet Insurance Ltd.~(owned by Teams) and the players' medical and dental insurance carriers (amounts can be obtained from League Office).
\item
  Review League Office supporting documentation with respect to Benefits.
\end{itemize}

\textbf{Basketball Related Income}

\begin{itemize}
\tightlist
\item
  Trace amounts to team's general ledger or other supporting documentation for agreement.
\item
  Foot all schedules and perform other clerical tests.
\item
  Trace gate receipts to general ledger and test supporting documentation where appropriate.
\item
  Gate receipts should be reviewed and reconciled to League Office gate receipts summary.
\item
  Verify amounts reported as luxury suite revenues with supporting documentation from the entity that sold, leased or licensed such luxury suites.
\item
  Verify amounts reported as complimentary tickets and tickets traded for goods or services with supporting documentation from the team.
\item
  Trace amounts reported for novelties and concessions, game parking, game programs, Team sponsorships and promotions, arena signage and arena club sales to general ledgers and test supporting documentation where appropriate.
\item
  Where reported amounts include proceeds received by a Related Party, verify the amounts reported with supporting documentation from the Related Party.
\item
  Examine the National Television and Cable contracts at the League Office, and agree to amounts reported.
\item
  Review, at League Office, expenses deducted from the National contracts in accordance with the terms of the CBA. Review supporting documentation and test where applicable.
\item
  Examine local television, local cable and local radio contracts. Verify to amounts reported by teams.
\item
  When local broadcast revenues are not verifiable by reviewing a contract, detailed supporting documentation should be reviewed and tested.
  -All loans, advances, bonuses, etc. received by the League Office or its teams should be noted in the report and included in BRI where appropriate.
\item
  Schedules of NBA Radio, NBA TV, international broadcast, NBA Media Ventures, copyright royalty revenues and expenses should be obtained from the NBA. Schedules should be verified by agreeing to general ledgers and examining supporting documentation where applicable.
\item
  Schedules of revenues and expenses reported by Properties for sponsorship, NBA related revenues from NBA Entertainment, and NBA Special Events should be obtained from the NBA. Schedules should be verified by agreeing to general ledgers and examining supporting documentation where applicable.
\item
  Net exhibition revenues and expenses should be verified to supporting documentation where appropriate.
\item
  All amounts of other revenues should be reviewed for proper inclusion/exclusion in BRI. Test appropriateness of balances where appropriate.
\item
  Determine the ratio of expenses to revenues for those categories of proceeds that come within the provisions of Article VII of the CBA and determine the extent to which expenses should be disallowed, if at all, pursuant to the provisions of that Section.
\end{itemize}

\textbf{Playoff Revenues}

\begin{itemize}
\tightlist
\item
  All sources of playoff revenues and expenses should be verified per the procedure outlined for Basketball Related Income.
\item
  Because of the late timing of the Playoffs, special attention should be given to revenue and expense estimates.
\item
  Playoff gate receipts should be recorded net of Taxes. Payments made to the Playoff Pool should not be deducted. Odd game payments should not be either deducted by the paying team or recorded by the receiving team.
\item
  Other playoff expenses should be reviewed in accordance with the terms of the CBA.
\item
  Team expenses paid by the League Playoff Pool, including travel expenses, should not be deducted by teams.
\item
  Review League Office supporting documentation as to expenses deducted from the Playoff Pool.
\end{itemize}

\textbf{Questions Concerning Related Party Transactions}

\begin{itemize}
\tightlist
\item
  Review with controller or other representatives of the team the answers to all questions on this schedule.
\item
  Review that appropriate details are provided where requested.
\item
  Prepare summary of all changes.
\end{itemize}

\textbf{List of Related Parties}

\begin{itemize}
\tightlist
\item
  Review with controller or other representatives of the team all information included on the schedule of related entities.
\item
  Prepare a summary of any changes, corrections or additions to the schedule.
\item
  Review supporting details of any changes.
\item
  Any revenue from a Related Party should be reviewed with both parties to determine proper treatment under the CBA.
\item
  Inquire of the controller or other representative of the team what, if any, Related Parties exist, and discuss with the parties what, if any, amounts should be included in BRI.
\end{itemize}

\end{document}
